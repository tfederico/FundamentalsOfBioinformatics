\chapter{Intracellular Compartments and Transport}

In the first section, we describe the principal membrane-enclosed com-
partments, or \textit{membrane-enclosed organelles}, of eucaryotic cells and
briefly consider their main functions. In the second section, we discuss
how the protein composition of the different compartments is set up and
maintained. Each compartment contains a unique set of proteins that
have to be transferred selectively from the cytosol, where they are made,
to the compartment in which they are used. This transfer process, called
\textit{protein sorting}, depends on signals built into the amino acid sequence of
the proteins. In the third section, we describe how certain membrane-enclosed 
compartments in a eucaryotic cell communicate with one
another by forming small membrane-enclosed sacs, or \textit{vesicles}. These
pinch off from one compartment, move through the cytosol, and fuse
with another compartment in a process called \textit{vesicular transport}. In the
last two sections, we discuss how this constant vesicular traffic also provides 
the main routes for releasing proteins from the cell by the process
of \textit{exocytosis} and for importing them by the process of endocytosis.

\section{Membrane-enclosed organelles}

Whereas a procaryotic cell usually consists of a single compartment, the
cytosol, enclosed by the plasma membrane, eucaryotic cells are elab-
orately subdivided by internal membranes. These membranes create
enclosed compartments in which sets of enzymes can operate without
interference from reactions occurring in other compartments.

\subsection{Eucaryotic Cells Contain a Basic Set of Membrane-enclosed Organelles}

The major membrane-enclosed organelles of an animal cell are summarized as follows:

\begin{itemize}
\item \textbf{Cytosol:} contains many metabolic pathways, protein synthesis;
\item \textbf{Nucleus:} contains main genome,DNA and RNA synthesis;
\item \textbf{Endoplasmatic Reticulum:} synthesis of most lipids, synthesis of proteins 
for distribution to many organelles and to the plasma membrane;
\item \textbf{Golgi apparatus:} modification, sorting, and packaging of proteins and 
lipids for either secretion or delivery to another organelle;
\item \textbf{Lysosomes:} intracellular degradation;
\item \textbf{Endosomes:} sorting of endocytosed material
\item \textbf{Mithocondria:} ATP synthesis by oxidative phosphorylation;
\item \textbf{Chloroplast:} ATP synthesis and carbon fixation by photosynthesis;
\item \textbf{Peroxisomes:} oxidation of toxic molecules.
\end{itemize}

These organelles are surrounded by the cytosol, which is enclosed by the
plasma membrane. The nucleus is generally the most prominent organelle
in eucaryotic cells. It is surrounded by a double membrane, known as the
nuclear envelope, and communicates with the cytosol via nuclear pores
that perforate the envelope. The outer nuclear membrane is continuous
with the membrane of the endoplasmic reticulum (ER), a system of interconnected 
sacs and tubes of membrane that often extends throughout
most of the cell. The ER is the major site of synthesis of new membranes
in the cell. Large areas of the ER have ribosomes attached to the cytosolic
surface and are designated rough endoplasmic reticulum (rough ER). The
ribosomes are actively synthesizing proteins that are delivered into the
ER lumen or ER membrane. The smooth endoplasmic reticulum (smooth
ER) lacks ribosomes. It is scanty in most cells but is highly developed
for performing particular functions in others.

The Golgi apparatus, which is usually situated near the nucleus, receives
proteins and lipids from the ER, modifies them, and then dispatches them
to other destinations in the cell. Small sacs of digestive enzymes called
lysosomes degrade worn-out organelles, as well as macromolecules and
particles taken into the cell by endocytosis. On their way to lysosomes,
endocytosed materials must first pass through a series of compartments
called endosomes, which sort the ingested molecules and recycle some of
them back to the plasma membrane. Peroxisomes are small organelles
enclosed by a single membrane. They contain enzymes used in a variety
of oxidative reactions that break down lipids and destroy toxic mole-
cules. Mitochondria and (in plant cells) chloroplasts are each surrounded
by a double membrane and are the sites of oxidative phosphorylation
and photosynthesis, respectively; both contain
membranes that are highly specialized for the production of ATP.

Many of the membrane-enclosed organelles, including the ER, Golgi apparatus, 
mitochondria, and chloroplasts, are held in their relative locations
in the cell by attachment to the cytoskeleton, especially to microtubules.
Cytoskeletal filaments provide tracks for moving the organelles around
and for directing the traffic of vesicles between them. These movements
are driven by motor proteins that use the energy of ATP hydrolysis to
propel the organelles and vesicles along the filaments.

On average, the membrane-enclosed organelles together occupy nearly
half the volume of a eucaryotic cell.

\subsection{Membrane-enclosed Organelles Evolved in Different Ways}

Membrane-enclosed organelles are thought to have arisen in evolution
in at least two ways. The nuclear membranes and the membranes of the
ER, Golgi apparatus, endosomes, and lysosomes are believed to have
originated by invagination of the plasma membrane. These
membranes, and the organelles they enclose, are all part of what is 
collectively called the \textit{endomembrane system}.

Mitochondria and chloroplasts are thought to have originated in a different 
way. They differ from all other organelles in that they possess their
own small genomes and can make some of their own proteins. 
The similarity of these genomes to those of bacteria
and the close resemblance of some of their proteins to bacterial proteins
strongly suggest that mitochondria and chloroplasts evolved from bacteria 
that were engulfed by primitive eucaryotic cells with which they
initially lived in symbiosis.

\section{Protein sorting}

Before a eucaryotic cell reproduces by dividing in two, it has to duplicate
its membrane-enclosed organelles. A cell cannot make these organelles
from scratch: it requires information and materials contained in the
organelle itself. Thus, most of the organelles are formed from preexisting
organelles, which grow and then divide.

For some organelles, including the mitochondria, chloroplasts, and the
interior of the nucleus, proteins are delivered directly from the cytosol.
For others, including the Golgi apparatus, lysosomes, endosomes, and
the nuclear membranes, proteins and lipids are delivered indirectly via
the ER, which is itself a major site of lipid and protein synthesis. Proteins
enter the ER directly from the cytosol: some are retained there, but most
are transported by vesicles to the Golgi apparatus and then onward to
other organelles or the plasma membrane.

\subsection{Proteins Are Imported into Organelles by Three Mechanisms}

The synthesis of virtually all proteins in the cell begins on ribosomes in
the cytosol. The exceptions are the few mitochondrial and chloroplast
proteins that are synthesized on ribosomes inside these organelles; most
mitochondrial and chloroplast proteins, however, are made in the cytosol
and subsequently imported. The fate of any protein molecule synthesized
in the cytosol depends on its amino acid sequence, which can contain
a \textit{sorting signal} that directs the protein to the organelle in which it is
required. Proteins that lack such signals remain as permanent residents
in the cytosol.

When a membrane-enclosed organelle imports proteins from the cytosol
or from another organelle, it faces a problem: how can it draw the
protein across membranes that are normally impermeable to hydrophilic
macromolecules? This task is accomplished in different ways for different
organelles.

\begin{itemize}
\item Proteins moving from the cytosol into the nucleus are transported
through the nuclear pores that penetrate the inner and outer
nuclear membranes. The pores function as selective gates that
actively transport specific macromolecules but also allow free diffusion 
of smaller molecules;
\item Proteins moving from the cytosol into the ER, mitochondria, or
chloroplasts are transported across the organelle membrane by
\textit{protein translocators} located in the membrane. Unlike transport
through nuclear pores, the transported protein molecule must usually 
unfold in order to snake through the membrane;
\item Proteins moving from the ER onward and from one compartment of
the endomembrane system to another are transported by a mechanism 
that is fundamentally different from the other two. These
proteins are ferried by \textit{transport vesicles}, which become loaded
with a cargo of proteins from the interior space, or \textit{lumen}, of one
compartment, as they pinch off from its membrane. The vesicles
subsequently discharge their cargo into a second compartment by
fusing with its membrane.
\end{itemize}

\subsection{Signal Sequences Direct Proteins to the Correct Compartment}

The typical sorting signal on proteins is a continuous stretch of amino
acid sequence, typically 15–60 amino acids long. This signal sequence is
often (but not always) removed from the finished protein once it has been
sorted.

Signal sequences are both necessary and sufficient to direct a protein
to a particular organelle. Deleting 
a signal sequence from an ER protein, for example, converts it into a
cytosolic protein, while placing an ER signal sequence at the beginning
of a cytosolic protein redirects the protein to the ER.

\subsection{Proteins Enter the Nucleus Through Nuclear Pores}

The \textbf{nuclear envelope} encloses the nuclear DNA and defines the nuclear
compartment. It is formed from two concentric membranes. The inner
nuclear membrane contains proteins that act as binding sites for the
chromosomes and provide anchorage for the
\textit{nuclear lamina}, a finely woven meshwork of protein filaments that lines
the inner face of this membrane and provides a structural support for
the nuclear envelope. The composition of the
outer nuclear membrane closely resembles the membrane of the ER, with
which it is continuous.

The \textbf{nuclear envelope} in all eucaryotic cells is perforated by nuclear
pores that form the gates through which all molecules enter or leave the
nucleus. Traffic occurs in both directions through the pores.

A nuclear pore is a large, elaborate structure composed of about 30 dif-
ferent proteins. Each pore contains water-filled passages
through which small water-soluble molecules can pass freely and non-
selectively between the nucleus and the cytosol.

Larger molecules (such as RNAs and proteins) and macromolecular complexes 
must display an appropriate sorting signal to pass through the
nuclear pore. The signal sequence that directs a protein from the cytosol
into the nucleus, called a \textit{nuclear localization signal}, typically consists of
one or two short sequences containing several positively charged lysines
or arginines.

Cytosolic proteins called \textit{nuclear transport receptors} bind to the nuclear
localization signal on newly synthesized proteins destined for the
nucleus. These receptors help direct the new protein to a nuclear pore
by interacting with the tentacle-like fibrils that extend from the rim of the
pore. During transport, the nuclear transport receptors grab
onto repeated amino acid sequences within the tangle of nuclear pore
proteins, pulling themselves from one to the next, to carry their cargo
protein into the nucleus. Once the protein has been delivered, the nuclear
transport receptor is returned to the cytosol via the nuclear pore for reuse.
Like any process that creates order, importing proteins
into the nucleus requires energy.

Nuclear pores transport proteins in their fully folded conformation and
transfer ribosomal components as assembled particles. This feature distinguishes 
the nuclear transport mechanism from the mechanisms that
transport proteins into other organelles. Proteins have to unfold during
their transport across the membranes of other organelles such as mitochondria, 
chloroplasts, and the ER.

\subsection{Proteins unfold to Enter Mitochondria and Chloroplasts}

Both mitochondria and chloroplasts are surrounded by inner and outer
membranes, and both organelles specialize in the synthesis of ATP.
Although both organelles contain their
own genomes and make some of their own proteins, most mitochondrial
and chloroplast proteins are encoded by genes in the nucleus and are
imported from the cytosol. These proteins usually have a signal sequence
at their N-terminus that allows them to enter their specific organelle.
Proteins destined for either organelle are translocated simultaneously
across both the inner and outer membranes at specialized sites where the
two membranes are in contact with each other. Each protein is unfolded
as it is transported, and its signal sequence is removed after translocation
is complete. Chaperone proteins
inside the organelles help to pull the protein across the two membranes
and to refold the protein once it is inside. Subsequent transport to a particular 
site within the organelle usually requires further sorting
signals in the protein, which are often only exposed after the first signal
sequence has been removed.

The growth and maintenance of mitochondria and chloroplasts require
not only the import of new proteins but also the incorporation of new
lipids into their membranes. Most of their membrane phospholipids are
thought to be imported from the ER, which is the main site of lipid synthesis 
in the cell. Phospholipids are transported individually to these
organelles by water-soluble lipid-carrying proteins that extract a phospholipid 
molecule from one membrane and deliver it into another.

\subsection{Proteins Enter the Endoplasmic Reticulum While Being Synthesized}

The \textbf{endoplasmatic reticulum (ER)} serves as an entry point for proteins destined 
for other organelles, as well as for the ER itself. Proteins destined for the Golgi
apparatus, endosomes, and lysosomes, as well as proteins destined for
the cell surface, all first enter the ER from the cytosol. Once inside the ER
or embedded in the ER membrane, individual proteins will not reenter
the cytosol during their onward journey.

Two kinds of proteins are transferred from the cytosol to the ER: 

\begin{enumerate}
\item water-soluble proteins are completely translocated across the ER membrane
and are released into the ER lumen;
\item prospective transmembrane proteins are only partly translocated across 
the ER membrane and become embedded in it.
\end{enumerate}

All of these proteins are initially directed to the ER by an ER signal sequence,
a segment of eight or more hydrophobic amino acids
that is also involved in the process of translocation across the
membrane.

Unlike the proteins that enter the nucleus, mitochondria, chloroplasts,
and peroxisomes, most of the proteins that enter the ER begin to be
threaded across the ER membrane before the polypeptide chain has been
completely synthesized. This requires that the ribosome synthesizing
the protein be attached to the ER membrane. These membrane-bound
ribosomes coat the surface of the ER, creating regions termed rough
\textbf{endoplasmic reticulum}.

There are, therefore, two separate populations of ribosomes in the cytosol.
\textit{Membrane-bound ribosomes} are attached to the cytosolic side of the
ER membrane (and outer nuclear membrane) and are making proteins
that are being translocated into the ER. \textit{Free ribosomes} are unattached
to any membrane and are making all of the other proteins encoded by
the nuclear DNA. Membrane-bound ribosomes and free ribosomes are
structurally and functionally identical; they differ only in the proteins they
are making at any given time. When a ribosome happens to be mak-
ing a protein with an ER signal sequence, the signal sequence directs
the ribosome to the ER membrane. As an mRNA molecule is translated,
many ribosomes bind to it, forming a \textbf{polyribosome}.

In the case of an mRNA molecule encoding a protein with an ER signal
sequence, the polyribosome becomes riveted to the ER membrane by the
growing polypeptide chains, which have become inserted into the membrane.

\subsection{Soluble proteins are released into the ER lumen}

The ER signal sequence is guided to the ER membrane with the aid of
at least two protein components:

\begin{enumerate}
\item a \textit{signal-recognition particle} (\textbf{SRP}), present in the cytosol, which binds 
to the ER signal sequence when it is exposed on the ribosome;
\item an \textit{SRP receptor}, embedded in the membrane of the ER, which recognizes the SRP.
\end{enumerate}

Binding of an SRP to a signal sequence causes protein synthesis by the ribosome to slow down,
until the ribosome and its bound SRP locate an SRP receptor on the ER.
After binding to its receptor, the SRP is released and protein synthesis
recommences, with the polypeptide now being threaded into the lumen
of the ER through a \textit{translocation channel} in the ER membrane.

Thus, the SRP and SRP receptor function as molecular match-makers, 
connecting ribosomes that are synthesizing proteins containing
ER signal sequences to available ER translocation channels.

In addition to directing proteins to the ER, the signal sequence - which for
soluble proteins is almost always at the N-terminus - functions to open
the translocation channel. The signal peptide remains bound to the channel 
while the rest of the protein chain is threaded through the membrane
as a large loop. At some stage during translocation, the signal sequence
is cleaved off by a signal peptidase located on the lumenal side of the
ER membrane; the signal peptide is then released from the translocation
channel and rapidly degraded. Once the C-terminus of the protein has
passed through the membrane, the protein is released into the ER lumen.

\subsection{Start and Stop Signals Determine the Arrangement of a Transmembrane Protein in the Lipid Bilayer}

Not all proteins that enter the ER are released into the ER lumen. Some
remain embedded in the ER membrane as transmembrane proteins.
The translocation process for such proteins is more complicated than it
is for soluble proteins, as some parts of the polypeptide chain must be
translocated clear across the lipid bilayer while others remain fixed in the
membrane.

In the simplest case, that of a transmembrane protein with a single
membrane-spanning segment, the N-terminal signal sequence initiates
translocation, just as for a soluble protein. But the transfer process is
halted by an additional sequence of hydrophobic amino acids, a \textit{stop-transfer 
sequence}, further into the polypeptide chain.

This second sequence is released from the translocation channel and drifts into
the plane of the lipid bilayer, where it forms an $\alpha$-helical membrane-spanning 
segment that anchors the protein in the membrane. Simultaneously,
the N-terminal signal sequence is also released from the channel into the
lipid bilayer and is cleaved off. As a result, the translocated protein ends
up as a transmembrane protein inserted in the membrane with a defined
orientation - the N-terminus on the lumenal side of the lipid bilayer and
the C-terminus on the cytosolic side.
Once inserted into the membrane, a transmembrane protein
does not change its orientation, which is retained throughout any subsequent 
vesicle budding and fusion events.

In some transmembrane proteins, an internal, rather than an N-terminal,
signal sequence is used to start the protein transfer; this internal signal 
sequence, called a s\textit{tart-transfer sequence}, is never removed from
the polypeptide. This arrangement occurs in some transmembrane proteins 
in which the polypeptide chain passes back and forth across the
lipid bilayer. In these cases, hydrophobic signal sequences are thought
to work in pairs: an internal start-transfer sequence serves to initiate
translocation, which continues until a stop-transfer sequence is reached;
the two hydrophobic sequences are then released into the bilayer, where
they remain as membrane-spanning $\alpha$ helices.
In complex multipass proteins, in which many hydrophobic a helices span the
bilayer, additional pairs of stop and start sequences come into play:
one sequence reinitiates translocation further down the polypeptide chain,
and the other stops translocation and causes polypeptide release, and
so on for subsequent starts and stops. Thus, multipass membrane proteins 
are stitched into the lipid bilayer as they are being synthesized, by a
mechanism resembling the workings of a sewing machine.

\section{Vesicular transport}

Entry into the ER is usually only the first step on a pathway to another
destination. That destination, initially at least, is the Golgi apparatus.
Transport from the ER to the Golgi apparatus and from the Golgi apparatus 
to other compartments of the endomembrane system is carried out
by the continual budding and fusion of \textbf{transport vesicles}. The transport
pathways mediated by these vesicles extend outward from the ER to the
plasma membrane, and inward from the plasma membrane to lysosomes,
and thus provide routes of communication between the interior of the
cell and its surroundings. As proteins and lipids are transported outward
along these pathways, many of them undergo various types of chemical
modification, such as the addition of carbohydrate side chains (to both
proteins and lipids) and the formation of disulfide bonds (in polypeptides)
that stabilize protein structure.

\subsection{Transport vesicles Carry Soluble Proteins and Membrane Between Compartments}

\textbf{Vesicular transport} between membrane-enclosed compartments of the
endomembrane system is highly organized. A major outward \textit{secretory
pathway} starts with the synthesis of proteins on the ER membrane and
their entry into the ER, and it leads through the Golgi apparatus to the
cell surface; at the Golgi apparatus, a side branch leads off through endosomes 
to lysosomes. A major inward \textit{endocytic pathway},
which is responsible for the ingestion and degradation of extracellular
molecules, moves materials from the plasma membrane, through endosomes, to lysosomes.

To function correctly, each transport vesicle that buds off from a compartment 
must take with it only the proteins appropriate to its destination and
must fuse only with the appropriate target membrane.
While participating in this constant flow of membrane components, each
organelle must maintain its own distinct identity, that is, its own distinctive 
protein and lipid composition. All of these recognition events depend
on proteins associated with the transport vesicle membrane.

\subsection{Vesicle Budding Is Driven by the Assembly of a Protein Coat}

Vesicles that bud from membranes usually have a distinctive protein coat
on their cytosolic surface and are therefore called coated vesicles. After
budding from its parent organelle, the vesicle sheds its coat, allowing its
membrane to interact directly with the membrane to which it will fuse.
The coat serves at least two functions: it shapes the membrane
into a bud, and it helps to capture molecules for onward transport.

The best-studied vesicles are those that have coats made largely of the
protein \textbf{clathrin}. These \textit{clathrin-coated} vesicles bud from the Golgi apparatus 
on the outward secretory pathway and from the plasma membrane
on the inward endocytic pathway. At the plasma membrane, for example,
each vesicle starts off as a \textit{clathrin-coated pit}. Clathrin molecules assemble 
into a basketlike network on the cytosolic surface of the membrane,
and it is this assembly process that starts shaping the membrane into
a vesicle. A small GTP-binding protein called \textit{dynamin}
assembles as a ring around the neck of each deeply invaginated coated
pit. Together with other proteins recruited to the neck of the vesicle, the
dynamin causes the ring to constrict, thereby pinching off the vesicle
from the membrane.

Clathrin itself plays no part in capturing specific molecules for transport.
This is the function of a second class of coat proteins called \textit{adaptins},
which both secure the clathrin coat to the vesicle membrane and help
select cargo molecules for transport. Molecules for onward transport
carry specific \textit{transport signals} that are recognized by cargo receptors in the
compartment membrane. Adaptins help capture specific cargo molecules
by trapping the cargo receptors that bind them. In this way, a selected set
of cargo molecules, bound to their specific receptors, is incorporated into
the lumen of each newly formed clathrin-coated vesicle.

Another class of coated vesicles, called \textit{COP-coated} vesicles (COP is short-hand 
for ‘coat protein’), is involved in transporting molecules between
the ER and the Golgi apparatus and from one part of the Golgi apparatus
to another.

\subsection{Vesicle Docking Depends on Tethers and SNAREs}

After a transport vesicle buds from a membrane, it must find its way to
its correct destination to deliver its contents. In most cases, the vesicle
is actively transported by motor proteins that move along cytoskeletal
fibers.

Once a transport vesicle has reached its target, it must recognize and
dock with the organelle. Only then can the vesicle membrane fuse with
the target membrane and unload the vesicle’s cargo. The impressive specificity 
of vesicular transport suggests that each type of transport vesicle
in the cell displays molecular markers on its surface that identify the
vesicle according to its origin and cargo. These markers must be recognized 
by complementary receptors on the appropriate target membrane,
including the plasma membrane. This identification process depends on
a family of proteins called \textbf{Rab proteins}. Rab proteins on the surface of
the vesicle are recognized by tethering proteins on the cytosolic surface of
the target membrane. Additional recognition is provided by a family of
related transmembrane proteins called \textbf{SNAREs}. Once the tethering protein 
has captured a vesicle by grabbing hold of its Rab protein, SNAREs
on the vesicle (called v-SNAREs) interact with complementary SNAREs
on the target membrane (called t-SNAREs), docking the vesicle in place.

\section{Secretory pathways}

Vesicular traffic is not confined to the interior of the cell. It extends to
and from the plasma membrane. Newly made proteins, lipids, and carbohydrates 
are delivered from the ER, via the Golgi apparatus, to the cell
surface by transport vesicles that fuse with the plasma membrane in a
process called \textbf{exocytosis}. Each molecule that travels along this route
passes through a fixed sequence of membrane-enclosed compartments
and is often chemically modified en route.

\subsection{Most Proteins Are Covalently Modified in the ER}

Most proteins that enter the ER are chemically modified there. \textit{Disulfide
bonds} are formed by the oxidation of pairs of cysteine side chains, 
a reaction catalyzed by an enzyme that resides in the ER
lumen. The disulfide bonds help to stabilize the structure of those proteins
that may encounter changes in pH and degradative enzymes outside the
cell - either after they are secreted or after they are incorporated into the
plasma membrane. Because of the reducing environment in the cytosol,
disulfide bonds do not form there.

Many of the proteins that enter the ER lumen or ER membrane are converted 
to glycoproteins in the ER by the covalent attachment of short
oligosaccharide side chains. This process of \textbf{glycosylation} is carried out
by glycosylating enzymes present in the ER but not in the cytosol. Very
few proteins in the cytosol are glycosylated, and those that are have only
a single sugar attached to them. The oligosaccharides on proteins serve
various functions, depending on the protein. They can protect the protein
from degradation, hold it in the ER until it is properly folded, or help guide
it to the appropriate organelle by serving as a transport signal for packaging 
the protein into appropriate transport vesicles. When displayed
on the cell surface, oligosaccharides form part of the cell’s carbohydrate
layer and can function in the recognition of one cell by another.

In the ER, individual sugars are not added one by one to the protein to
create the oligosaccharide side chain. Instead, a preformed, branched
oligosaccharide containing a total of 14 sugars is attached en bloc to all
proteins that carry the appropriate site for glycosylation. The oligosaccharide 
is originally attached to a specialized lipid, called \textbf{dolichol}, in the
ER membrane; it is then transferred to the amino ($NH_2$) group of an asparagine 
side chain (composed by \textbf{glucose}, \textbf{mannose} and \textbf{N-acetylglucosamine}) 
on the protein, immediately after the target asparagine
emerges in the ER lumen during protein translocation. The
addition takes place in a single enzymatic step that is catalyzed by a
membrane-bound enzyme (an \textbf{oligosaccharide protein transferase}) that
has its active site exposed on the lumenal side of the ER membrane - 
which explains why cytosolic proteins are not glycosylated in this way.
A simple sequence of three amino acids, of which the asparagine is
one, defines which asparagines in a protein receive the oligosaccharide.
Oligosaccharide side chains linked to an asparagine $NH_2$ group in a protein 
are said to be \textit{N-linked} and are by far the most common type of
linkage found on glycoproteins.

The addition of the 14-sugar oligosaccharide in the ER is only the first
step in a series of further modifications before the mature glycoprotein
emerges at the other end of the outward pathway. Despite their initial
similarity, the N-linked oligosaccharides on mature glycoproteins are
remarkably diverse. All of the diversity results from extensive modification 
of the original precursor structure. This
\textit{oligosaccharide processing} begins in the ER and continues in the Golgi
apparatus.

\subsection{Exit from the ER Is Controlled to Ensure Protein quality}

Some proteins made in the ER are destined to function there. They are
retained in the ER (and are returned to the ER when they escape to the
Golgi apparatus) by a C-terminal sequence of four amino acids called
an ER retention signal. This retention signal is
recognized by a membrane-bound receptor protein in the ER and Golgi
apparatus. Most proteins that enter the ER, however, are destined for
other locations; they are packaged into transport vesicles that bud from
the ER and fuse with the Golgi apparatus. Exit from the ER is highly selective. 
Proteins that fold incorrectly, and dimeric or multimeric proteins that
fail to assemble properly, are actively retained in the ER by binding to
chaperone proteins that reside there. Interaction with chaperones holds
the proteins in the ER until proper folding occurs; if this does not happen,
the proteins are eventually degraded. \textbf{Antibody} molecules,
for example, are composed of four polypeptide chains
that assemble into the complete antibody molecule in the ER. Partially
assembled antibodies are retained in the ER until all four polypeptide
chains have assembled; any antibody molecule that fails to assemble
properly is ultimately degraded. In this way, the ER controls the quality of
the proteins that it exports to the Golgi apparatus.

\subsection{The Size of the ER Is Controlled by the Amount of Protein that Flows Through It}

Although chaperones help proteins in the ER fold properly and retain those
that do not, when protein synthesis is vigorous, the system can become
overwhelmed. When a cell’s protein production exceeds the carrying - 
and folding - capacity of its ER, misfolded proteins begin to accumulate.
These aberrant proteins actually serve as a signal to direct the cell to make
more ER. They do so by activating a special set of receptors that reside in
the ER membrane, which in turn activate a vast transcriptional program
called the \textit{unfolded protein response} (UPR). The UPR program prompts
the cell to produce more ER, including all of the molecular machinery
required to restore proper protein folding and processing.

The UPR program allows cells to adjust the size of the ER according to
need, so that the load of proteins entering the secretory pathway will be
folded efficiently and properly. In some cases, however, even an expanded
ER can become overloaded. If a proper balance can not be reestablished - 
and misfolded proteins continue to accumulate - the UPR program can
direct the cell to self-destruct by undergoing apoptosis.

\subsection{Proteins Are Further Modified and Sorted in the Golgi Apparatus}

The \textbf{Golgi apparatus} is usually located near the cell nucleus, and in
animal cells it is often close to the centrosome, a small structure near
the cell center. The Golgi apparatus consists of a collection of flattened,
membrane-enclosed sacs (\textit{cisternae}), which are piled like stacks of plates.
Each stack contains 3-20 cisternae.

Each Golgi stack has two distinct faces: an entry, or \textit{cis}, face and an exit,
or \textit{trans}, face. The \textit{cis} face is adjacent to the ER, while the \textit{trans} face
points toward the plasma membrane. The outermost cisterna at each
face is connected to a network of interconnected membranous tubes and
vesicles. Soluble proteins and membrane enter the
\textit{cis Golgi network} via transport vesicles derived from the ER. The proteins
travel through the cisternae in sequence by means of transport vesicles
that bud from one cisterna and fuse with the next. Proteins exit from the
\textit{trans Golgi network} in transport vesicles destined for either the cell surface 
or another compartment. Both the \textit{cis} and \textit{trans}
Golgi networks are thought to be important for protein sorting: proteins
entering the \textit{cis} Golgi network can either move onward through the Golgi
stack or, if they contain an ER retention signal, be returned to the ER;
proteins exiting from the \textit{trans} Golgi network are sorted according to
whether they are destined for lysosomes or for the cell surface.

Many of the oligosaccharide groups that are added to proteins in the ER
undergo further modifications in the Golgi apparatus.

\subsection{Secretory Proteins Are Released from the Cell by Exocytosis}

In all eucaryotic cells, a steady stream of vesicles buds from the trans
Golgi network and fuses with the plasma membrane. This \textit{constitutive
exocytosis pathway} operates continually and supplies newly made lipids
and proteins to the plasma membrane; it is the pathway for
plasma membrane growth when cells enlarge before dividing. The constitutive 
pathway also carries proteins to the cell surface to be released
to the outside, a process called \textbf{secretion}. Some of the released proteins
adhere to the cell surface, where they become peripheral proteins of the
plasma membrane; some are incorporated into the extracellular matrix;
still others diffuse into the extracellular fluid to nourish or to signal other
cells. Because entry into this nonselective pathway does not require a
particular signal sequence (like those that direct proteins to lysosomes or
back to the ER), it is sometimes referred to as the \textit{default pathway}.

In addition to the constitutive exocytosis pathway, which operates continually 
in all eucaryotic cells, there is a \textit{regulated exocytosis pathway},
which operates only in cells that are specialized for secretion. Specialized
\textit{secretory cells} produce large quantities of particular products, such as
hormones, mucus, or digestive enzymes, which are stored in \textbf{secretory
vesicles} for later release. These vesicles bud off from the \textit{trans} Golgi network 
and accumulate near the plasma membrane. There they wait for
the extracellular signal that will stimulate them to fuse with the plasma
membrane and release their contents to the cell exterior.

Proteins destined for regulated secretion are sorted and packaged in the
\textit{trans} Golgi network. Proteins that travel by this pathway have special
surface properties that cause them to aggregate with one another under
the ionic conditions (acidic pH and high $Ca^{2+}$) that prevail in the \textit{trans}
Golgi network. The aggregated proteins are packaged into secretory vesicles, 
which pinch off from the network and await a signal instructing
them to fuse with the plasma membrane. Proteins secreted by the constitutive 
pathway, on the other hand, do not aggregate and are therefore
carried automatically to the plasma membrane by the transport vesicles
of the constitutive pathway.

When a secretory vesicle or transport vesicle fuses with the plasma mem-
brane and discharges its contents by exocytosis, its membrane becomes
part of the plasma membrane. Although this should greatly increase the
surface area of the plasma membrane, it does so only transiently because
membrane components are removed from other regions of the surface by
endocytosis almost as fast as they are added by exocytosis. This removal
returns both the lipids and the proteins of the vesicle membrane to the
Golgi network, where they can be used again.

\section{Encocytic pathways}

Eucaryotic cells are continually taking up fluid, as well as large and
small molecules, by the process of endocytosis. Specialized cells are also
able to internalize large particles and even other cells. The material to
be ingested is progressively enclosed by a small portion of the plasma
membrane, which first buds inward and then pinches off to form an intracellular 
\textit{endocytic vesicle}. The ingested material is ultimately delivered to
lysosomes, where it is digested. The metabolites generated by digestion
are transferred directly out of the lysosome into the cytosol, where they
can be used by the cell.

Two main types of \textit{endocytosis} are distinguished on the basis of
the size of the endocytic vesicles formed. \textit{Pinocytosis} (`cellular drinking') 
involves the ingestion of fluid and molecules via small vesicles. 
\textit{Phagocytosis} (‘cellular eating’) involves the ingestion 
of large particles, such as microorganisms and cell debris, via large
vesicles called phagosomes. Whereas all
eucaryotic cells are continually ingesting fluid and molecules by pinocytosis, 
large particles are ingested mainly by specialized phagocytic cells.

\subsection{Specialized Phagocytic Cells Ingest Large Particles}

The most dramatic form of endocytosis, \textbf{phagocytosis}, was first observed
more than a hundred years ago. In protozoa, phagocytosis is a form of
feeding: these unicellular eucaryotes ingest large particles such as bacteria 
by taking them up into phagosomes. Nevertheless, phagocytosis is important in most animals for purposes
other than nutrition. \textit{Phagocytic cells} - including macrophages, which are
widely distributed in tissues, and some other white blood cells - defend us
against infection by ingesting invading microorganisms. To be taken up
by a macrophage or other white blood cell, particles must first bind to the
phagocytic cell surface and activate one of a variety of surface receptors.
Some of these receptors recognize antibodies, the proteins that protect
us against infection by binding to the surface of microorganisms. Binding
of antibody-coated bacteria to these receptors induces the phagocytic
cell to extend sheetlike projections of the plasma membrane, called
\textit{pseudopods}, that engulf the bacterium and fuse at their
tips to form a phagosome. The phagosome then fuses with a lysosome,
and the microbe is digested.

Phagocytic cells also play an important part in scavenging dead and damaged 
cells and cell debris.

\subsection{Fluid and Macromolecules Are Taken up by Pinocytosis}

Eucaryotic cells continually ingest bits of their plasma membrane, along
with small amounts of extracellular fluid, in the form of small pinocytic
vesicles that are later returned to the cell surface. The rate at which
plasma membrane is internalized by \textbf{pinocytosis} varies from cell type to
cell type, but it is usually surprisingly large.
Because a cell’s total surface area and volume remain unchanged during
this process, as much membrane is being added to the cell surface by
vesicle fusion (exocytosis) as is being removed by endocytosis.

Pinocytosis is carried out mainly by the clathrin-coated pits and vesicles
that we discussed earlier. After they pinch
off from the plasma membrane, clathrin-coated vesicles rapidly shed their
coat and fuse with an endosome. Extracellular fluid is trapped in the coated
pit as it invaginates to form a coated vesicle, and so substances dissolved
in the extracellular fluid are internalized and delivered to endosomes. This
fluid intake is generally balanced by fluid loss during exocytosis.

\subsection{Receptor-mediated Endocytosis Provides a Specific Route into Animal Cells}

Pinocytosis, as just described, is indiscriminate. The endocytic vesicles
simply trap any molecules that happen to be present in the extracellular
fluid and carry them into the cell. In most animal cells, however, pinocytosis 
via clathrin-coated vesicles also provides an efficient pathway for
taking up specific macromolecules from the extracellular fluid. The macromolecules 
bind to complementary receptors on the cell surface and
enter the cell as receptor–macromolecule complexes in clathrin-coated
vesicles. This process, called \textbf{receptor-mediated endocytosis}, provides
a selective concentrating mechanism that increases the efficiency of
internalization of particular macromolecules more than 1000-fold compared 
with ordinary pinocytosis, so that even minor components of the
extracellular fluid can be taken up in large amounts without taking in a
correspondingly large volume of extracellular fluid.

Cholesterol is extremely insoluble and is transported in the bloodstream
bound to protein in the form of particles called low-density lipoproteins,
or \textbf{LDL}. The LDL binds to receptors located on cell surfaces, and the
receptor - LDL complexes are ingested by receptor-mediated endocytosis
and delivered to endosomes. The interior of endosomes is more acidic
than the surrounding cytosol or the extracellular fluid, and in this acidic
environment the LDL dissociates from its receptor: the receptors are
returned in transport vesicles to the plasma membrane for reuse, while
the LDL is delivered to lysosomes. In the lysosomes the LDL is broken
down by hydrolytic enzymes. The cholesterol is released and escapes
into the cytosol, where it is available for new membrane synthesis. The
LDL receptors on the cell surface are continually internalized and recy-
cled, whether they are occupied by LDL or not

This pathway for cholesterol uptake is disrupted in individuals who inherit
a defective gene encoding the LDL receptor protein. In some cases, the
receptors are missing; in others, they are present but nonfunctional.

Receptor-mediated endocytosis is also used to take up many other essen-
tial metabolites, such as vitamin $B_{12}$ and iron, that cells cannot take up
by the processes of membrane transport

\subsection{Endocytosed Macromolecules Are Sorted in Endosomes}

Because extracellular material taken up by pinocytosis is rapidly
transferred to \textbf{endosomes}, it is possible to visualize the endosomal compartment 
by incubating living cells in fluid containing an electron-dense
marker that will show up when viewed in an electron microscope. When
examined in this way, the endosomal compartment reveals itself to be
a complex set of connected membrane tubes and larger vesicles.

The endosomal compartment acts as the main sorting station in the
inward endocytic pathway, just as the trans Golgi network serves this
function in the outward secretory pathway. The acidic environment of
the endosome plays a crucial part in the sorting process by causing many
receptors to release their bound cargo. The routes taken by receptors once
they have entered an endosome differ according to the type of receptor:

\begin{enumerate}
\item Most are returned to the same plasma membrane domain from which
they came, as is the case for the LDL receptor discussed earlier; 
\item Some travel to lysosomes, where they are degraded; 
\item Some proceed to a different domain of the plasma membrane, thereby transferring their
bound cargo molecules across the cell from one extracellular space to
another, a process called \textit{transcytosis}.
\end{enumerate}

Cargo proteins that remain bound to their receptors share the fate of their
receptors. Those that dissociate from their receptors in the endosome
are doomed to destruction in lysosomes, along with most of the con-
tents of the endosome lumen.

\subsection{Lysosomes Are the Principal Sites of Intracellular Digestion}

Many extracellular particles and molecules ingested by cells end up in
lysosomes, which are membranous sacs of hydrolytic enzymes that
carry out the controlled intracellular digestion of both extracellular
materials and worn-out organelles. They contain about 40 types of
hydrolytic enzymes, including those that degrade proteins, nucleic acids,
oligosaccharides, and phospholipids. All of these enzymes are optimally
active in the acidic conditions (pH 5) maintained within lysosomes.
The membrane of the lysosome normally keeps these destructive enzymes
out of the cytosol (whose pH is about 7.2), but the enzymes’ acid dependence 
protects the contents of the cytosol against damage even if some of
them should escape.

Like all other intracellular organelles, the lysosome not only contains a
unique collection of enzymes but also has a unique surrounding membrane. 
The lysosomal membrane contains transporters that allow the
final products of the digestion of macromolecules, such as amino acids,
sugars, and nucleotides, to be transferred to the cytosol; from there, they
can be either excreted or utilized by the cell. The membrane also contains
an ATP-driven $H^+$ pump, which, like the ATPase in the endosome mem-
brane, pumps $H^+$ into the lysosome, thereby maintaining its contents at
an acidic pH.

The specialized digestive enzymes and membrane proteins of the lysosome 
are synthesized in the ER and transported through the Golgi
apparatus to the trans Golgi network. While in the ER and the cis Golgi
network, the enzymes are tagged with a specific phosphorylated sugar
group, so that when they arrive in the trans Golgi
network they can be recognized by an appropriate receptor. 
This tagging permits the enzymes to be sorted and
packaged into transport vesicles, which bud off and deliver their contents
to lysosomes via late endosomes.

Depending on their source, materials follow different paths to lysosomes.
We have seen that extracellular particles are taken up into phagosomes,
which fuse with lysosomes, and that extracellular fluid and macromolecules 
are taken up into smaller endocytic vesicles, which deliver their
contents to lysosomes via endosomes. But cells have an additional pathway 
for supplying materials to lysosomes; this pathway, called \textbf{autophagy},
is used for degrading obsolete parts of the cell itself. The process begins with the
enclosure of the organelle by a double membrane, creating an autophago-
some, which then fuses with lysosomes. It is not known
what marks an organelle for such destruction.

\section{Essential concepts}

\begin{itemize}
\item Eucaryotic cells contain many membrane-enclosed organelles,
including a nucleus, an endoplasmic reticulum (ER), a Golgi apparatus, 
lysosomes, endosomes, mitochondria, chloroplasts (in plant cells), and peroxisomes.
\item Most organelle proteins are made in the cytosol and transported into
the organelle where they function. Sorting signals in the amino acid
sequence guide the proteins to the correct organelle; proteins that
function in the cytosol have no such signals and remain where they
are made.
\item Nuclear proteins contain nuclear localization signals that help direct
their active transport from the cytosol into the nucleus through
nuclear pores, which penetrate the double-membrane nuclear envelope. 
Proteins enter the nucleus without being unfolded.
\item Most mitochondrial and chloroplast proteins are made in the cytosol
and are then actively transported into the organelles by protein
translocators in their membranes. Proteins must be unfolded to allow
them to snake through the translocators in the chloroplast or mitochondrial membrane.
\item The ER is the membrane factory of the cell; it makes most of the cell’s
lipids and many of its proteins. The proteins are made by ribosomes
bound to the surface of the rough ER.
\item Ribosomes in the cytosol are directed to the ER if the protein they
are making has an ER signal sequence, which is recognized by a
signal-recognition particle (SRP) in the cytosol; the binding of the
ribosome - SRP complex to a receptor on the ER membrane initiates 
the translocation process that threads the growing polypeptide
across the ER membrane through a translocation channel.
\item Soluble proteins destined for secretion or for the lumen of an organelle
pass completely into the ER lumen, while transmembrane proteins
destined for the ER membrane or for other cell membranes remain
anchored in the lipid bilayer by one or more membrane-spanning a
helices.
\item In the ER lumen, proteins fold up, assemble with other proteins, form
disulfide bonds, and become decorated with oligosaccharide chains.
\item Exit from the ER is an important quality-control step; proteins that
either fail to fold properly or fail to assemble with their normal partners 
are retained in the ER by chaperone proteins and are eventually
degraded.
\item An accumulation of misfolded proteins triggers a response that
expands the size of the ER, thus increasing its capacity to fold new
proteins properly.
\item Protein transport from the ER to the Golgi apparatus and from the
Golgi apparatus to other destinations is mediated by transport vesicles 
that continually bud off from one membrane and fuse with
another, a process called vesicular transport.
\item Budding transport vesicles have distinctive coat proteins on their
cytosolic surface; the assembly of the coat drives the budding process, 
and the coat proteins help incorporate receptors with their
bound cargo molecules into the forming vesicle.
\item Coated vesicles lose their protein coat soon after pinching off, enabling 
them to dock and then fuse with a particular target membrane;
docking and fusion are mediated by proteins on the vesicle and on
the target membranes, including Rab proteins and SNAREs.
\item The Golgi apparatus receives newly made proteins from the ER; it
modifies their oligosaccharides, sorts the proteins, and dispatches
them from the trans Golgi network to the plasma membrane, lysosomes, 
or secretory vesicles.
\item In all eucaryotic cells, transport vesicles continually bud from the
trans Golgi network and fuse with the plasma membrane, a process
called constitutive exocytosis; the process delivers plasma membrane
lipids and proteins to the cell surface and also releases molecules
from the cell in the process of secretion.
\item Specialized secretory cells also have a regulated exocytosis pathway,
where molecules stored in secretory vesicles are released from the
cell by exocytosis when the cell is signaled to secrete.
\item Cells ingest fluid, molecules, and sometimes even particles, by endocytosis, 
in which regions of plasma membrane invaginate and pinch
off to form endocytic vesicles.
\item Much of the material that is endocytosed is delivered to endosomes
and then to lysosomes, where it is degraded by hydrolytic enzymes;
most of the components of the endocytic vesicle membrane, however, 
are recycled in transport vesicles back to the plasma membrane
for reuse.
\end{itemize}















