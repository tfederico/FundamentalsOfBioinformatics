\chapter{Membrane structure}

A living cell is a self-reproducing system of molecules held inside a container. 
That container is the plasma membrane - a fatty film so thin and
transparent that it cannot be seen directly in the light microscope.

The plasma membrane is simple in form: its structure is based on a two-ply 
sheet of lipid molecules about 5 nm - or 50 atoms - thick.

The simplest bacteria have only a single membrane - the plasma membrane. 
Eucaryotic cells, however, also contain an abundance of internal
membranes that enclose intracellular compartments to form the various
organelles, including the endoplasmic reticulum, Golgi apparatus, and
mitochondria.

Regardless of their location, all cell membranes are composed of lipids
and proteins and share a common general structure. The
lipids are arranged in two closely apposed sheets, forming a lipid bilayer.

\section{The lipid layer}

The lipid bilayer has been firmly established as the universal basis of
membrane structure, and its properties are responsible for the general
properties of all cell membranes.

\subsection{Membrane lipids Form bilayers in Water}

The lipids in cell membranes combine two very different properties in a
single molecule: each lipid has a hydrophilic (“water-loving”) head and
one or two hydrophobic (“water-fearing”) hydrocarbon tails.
The most abundant lipids in cell membranes are the phospholipids,
molecules in which the hydrophilic head is linked to the rest of the lipid
through a phosphate group. The most common type of phospholipid
in most cell membranes is phosphatidylcholine, which has the small
molecule choline attached to a phosphate as its hydrophilic head and two
long hydrocarbon chains as its hydrophobic tails.
Phosphatidylcholine is the most common phospholipid in cell membranes.
This particular phospholipid is built from five parts: the hydrophilic head,
choline, is linked via a phosphate to glycerol, which in turn is linked to two 
hydrocarbon chains, forming the hydrophobic tail. The two hydrocarbon chains originate 
as fatty acids - that is, hydrocarbon chains with a -COOH group at one end - which become 
attached to glycerol via their -COOh groups. A kink in one of the hydrocarbon chains 
occurs where there is a double bond between two carbon atoms. The ‘phosphatidyl’ part of the n
ame of phospholipids refers to the phosphate-glycerol-fatty acid portion of the molecule

Molecules with both hydrophilic and hydrophobic properties are termed
amphipathic.

These non-polar atoms force adjacent water molecules to reorganize into a cagelike
structure around the hydrophobic molecule. Because the
cagelike structure is more highly ordered than the surrounding water,
its formation requires energy. The energy cost is minimized, however, if
the hydrophobic molecules cluster together, limiting their contact with
water to the smallest possible number of water molecules.

In contrast, amphipathic molecules, such as phospholipids, are subject to two conflicting forces: the hydrophilic head is
attracted to water, while the hydrophobic tail shuns water and seeks to
aggregate with other hydrophobic molecules. This conflict is beautifully
resolved by the formation of a lipid bilayer - an arrangement that satisfies 
all parties and is energetically most favorable. The hydrophilic heads
face the water from both surfaces of the bilayer sheet; the hydrophobic
tails are all shielded from the water as they lie next to one another in the
interior.

The same forces that drive the amphipathic molecules to form a bilayer
make the bilayer self-sealing. Any tear in the sheet will create a free edge
that is exposed to water. Because this situation is energetically unfavorable,
the molecules of the bilayer will spontaneously rearrange to eliminate
the free edge.

The prohibition on free edges has a profound consequence: the only way
a finite sheet can avoid having free edges is to bend and seal, forming a
boundary around a closed space. Therefore, amphipathic
molecules such as phospholipids necessarily assemble into self-sealing
containers that define closed compartments.

\subsection{The lipid bilayer is a two-dimensional Fluid}

The membrane therefore behaves as a two-dimensional fluid, which is crucial for membrane function and integrity. 
This property is distinct from flexibility, which is the ability of
the membrane to bend.

The fluidity of lipid bilayers can be studied using synthetic lipid bilayers,
which are easily produced by the spontaneous aggregation of amphipathic 
lipid molecules in water. Two types of synthetic lipid bilayers are
commonly used in experiments. Closed spherical vesicles, called liposomes, 
form if pure phospholipids are added to water. Alternatively, flat
phospholipid bilayers can be formed across a hole in a partition between
two aqueous compartments.

Thus, in synthetic lipid bilayers, phospholipid molecules very rarely tumble from one half of the bilayer,
or monolayer, to the other. Without proteins to facilitate the process
and under conditions similar to those in a cell, it is estimated that this
event, called ‘flip-flop,’ occurs less than once a month for any individual
lipid molecule. On the other hand, as the result of thermal motions, lipid
molecules within a monolayer continuously exchange places with their neighbors.

Similar results are obtained when one examines isolated cell membranes
and whole cells, indicating that the lipid bilayer of a cell membrane
also behaves as a two-dimensional fluid in which the constituent lipid
molecules are free to move within their own layer in any direction in the
plane of the membrane. These studies also show that lipid hydrocarbon
chains are flexible and that individual lipid molecules within a monolayer
rotate very rapidly about their long axis (lateral diffusion). In cells, as in synthetic bilayers,
individual phospholipid molecules are normally confined to their own
monolayer and do not flip-flop spontaneously.

\subsection{The Fluidity of a lipid bilayer depends on its Composition}

Just how fluid a lipid
bilayer is at a given temperature depends on its phospholipid composition 
and, in particular, on the nature of the hydrocarbon tails: the closer
and more regular the packing of the tails, the more viscous and less fluid
the bilayer will be. Two major properties of hydrocarbon tails affect how
tightly they pack in the bilayer: their length and the number of double
bonds they contain.

A shorter chain length reduces the tendency of the hydrocarbon tails
to interact with one another and therefore increases the fluidity of the
bilayer.
Most phospholipids contain one hydrocarbon tail that has one or more
double bonds between adjacent carbon atoms, and a second tail with
single bonds only. The chain that harbors a double bond
does not contain the maximum number of hydrogen atoms that could, in
principle, be attached to its carbon backbone; it is thus said to be unsaturated 
with respect to hydrogen. The fatty acid tail with no double bonds
has a full complement of hydrogen atoms; it is said to be saturated. Each
double bond in an unsaturated tail creates a small kink in the hydrocarbon tail.

In bacterial and yeast cells, which have to adapt to varying temperatures,
both the lengths and the unsaturation of the hydrocarbon tails in the
bilayer are constantly adjusted to maintain the membrane at a relatively
constant fluidity.

In animal cells, membrane fluidity is modulated by the inclusion of the
sterol cholesterol.

\subsection{The lipid bilayer is asymmetrical}

The lipid asymmetry is established and maintained as the membrane
grows. In eucaryotic cells, new phospholipids are manufactured by
enzymes bound to the part of the endoplasmic reticulum membrane that
faces the cytosol. These enzymes, which use free fatty acids as substrates, 
deposit all newly made phospholipids into the
cytosolic half of the bilayer. To enable the membrane as a whole to grow
evenly, half of the new phospholipid molecules then have to be transferred 
to the opposite monolayer. This transfer is catalyzed by enzymes
called flippases. In the plasma membrane, flippases transfer
specific phospholipids selectively, so that different types become concen-
trated in each monolayer.

Using selective flippases is not the only way to produce asymmetry in
lipid bilayers, however. In particular, a different mechanism operates for
glycolipids - the lipids that show the most striking and consistent asymmetric 
distribution in animal cells.

\subsection{Lipid asymmetry is preserved during Membrane transport}

Nearly all new membrane synthesis in eucaryotic cells occurs in the
membrane of one intracellular compartment - the endoplasmic reticulum.
The new membrane assembled there is exported
to the other membranes of the cell through a cycle of membrane budding
and fusion: bits of the bilayer pinch off from the ER to form small spheres
called vesicles, which then become incorporated into another membrane,
such as the plasma membrane, by fusing with it. The orientation 
of the bilayer relative to the cytosol is preserved during vesicle
formation and fusion. This preservation of orientation means that all cell
membranes, whether the external plasma membrane or an intracellular
membrane around an organelle, have distinct ‘inside’ and ‘outside’ faces
that are established at the time of membrane synthesis: the cytosolic face
is always adjacent to the cytosol, while the noncytosolic face is exposed
to either the cell exterior or the interior space of an organelle

Glycolipids are located mainly in the plasma membrane, and they are
found only in the noncytosolic half of the bilayer. Their sugar groups are
therefore exposed to the exterior of the cell, where they
form part of a continuous protective coat of carbohydrate that surrounds
most animal cells. The glycolipid molecules acquire their sugar groups
in the Golgi apparatus, the organelle to which proteins and membranes
made in the ER often go next.

\section{Membrane proteins}

Although the lipid bilayer provides the basic structure of all cell membranes 
and serves as a permeability barrier to the molecules on either
side of it, most membrane functions are carried out by membrane proteins.
Membrane proteins not only transport particular nutrients, metabolites,
and ions across the lipid bilayer; they serve many other functions.

\subsection{Membrane proteins associate with the lipid bilayer in Various Ways}

Proteins can be associated with the lipid bilayer of a cell membrane in
several ways:

\begin{enumerate}
\item Many membrane proteins extend through the bilayer, with part of
their mass on either side. Like their lipid neighbors, 
these transmembrane proteins have both hydrophobic and
hydrophilic regions. Their hydrophobic regions lie in the interior
of the bilayer, nestled against the hydrophobic tails of the lipid
molecules. Their hydrophilic regions are exposed to the aqueous
environment on either side of the membrane.
\item Other membrane proteins are located entirely in the cytosol, associated 
with the inner leaflet of the lipid bilayer by an amphipathic $\alpha$-helix exposed on the surface of the protein.
\item Some proteins lie entirely outside the bilayer, on one side or the
other, attached to the membrane only by one or more covalently
attached lipid groups.
\item Yet other proteins are bound indirectly to one or the other face of
the membrane, held in place only by their interactions with other
membrane proteins
\end{enumerate}

Proteins that are directly attached to a lipid bilayer - whether they are
transmembrane, monolayer-associated, or lipid-linked - can be removed
only by disrupting the bilayer with detergents, as discussed shortly. Such
proteins are known as integral membrane proteins. The remaining membrane p
roteins are known as peripheral membrane proteins; they can
be released from the membrane by more gentle extraction procedures
that interfere with protein-protein interactions but leave the lipid bilayer
intact.

\subsection{A polypeptide Chain usually Crosses the bilayer as an $\alpha$-helix}

The peptide bonds that join the successive amino acids in a protein are normally polar, making
the polypeptide backbone hydrophilic. Because water is
absent from the bilayer, atoms forming the backbone are driven to form
hydrogen bonds with one another. Hydrogen-bonding is maximized if the
polypeptide chain forms a regular a helix, and so the great majority of the
membrane-spanning segments of polypeptide chains traverse the bilayer
as $\alpha$ helices.

The proteins that form pores are more complicated, usually possessing a 
series of $\alpha$ helices that cross the bilayer a number of times. In many of these proteins,
one or more of the transmembrane regions are formed from $\alpha$ helices
that contain both hydrophobic and hydrophilic amino acid side chains.

Although the $\alpha$-helix is by far the most common form in which a polypeptide
chain crosses a lipid bilayer, the polypeptide chain of some transmembrane 
proteins crosses the lipid bilayer as a $\beta$ sheet that is curved into a
cylinder, forming an open-ended keglike structure called a $\beta$ barrel. As
expected, the amino acid side chains that face the inside of the barrel, and
therefore line the aqueous channel, are mostly hydrophilic, while those
on the outside of the barrel, which contact the hydrophobic core of the
lipid bilayer, are exclusively hydrophobic. The most striking example of a
$\beta$ barrel structure is found in the porin proteins, which form large, water-filled 
pores in mitochondrial and bacterial membranes.

Some examples of plasma membrane proteins and their functions are the following:

\begin{itemize}
\item Transporters: actively pumps $Na^{+}$ out of cells and $K^{+}$ in.
\item Anchors: link intracellular actin (i.e. family of globular multi-functional proteins) filaments to extracellular matrix proteins.
\item Receptors: binds extracellular platelet-derived growth factor (PDGF) and, as a consequence, generates intracellular signals
that cause the cell to grow and divide.
\item Enzymes: link intracellular actin filaments to extracellular matrix proteins.
\end{itemize}

\subsection{Membrane proteins Can be Solubilized in detergents and purified}

Before an individual protein can be studied in detail, it must be separated
from all the other cell proteins. For most membrane proteins, the first step
in this separation process involves solubilizing the membrane with agents
that destroy the lipid bilayer by disrupting hydrophobic associations. The
most widely used disruptive agents are detergents. These
are small, amphipathic, lipidlike molecules that have both a hydrophilic
and a hydrophobic region. Detergents differ from membrane 
phospholipids in that they have only a single hydrophobic tail and,
consequently, behave in a significantly different way. Because they have
one tail, detergent molecules are shaped like cones; in water, they tend to
aggregate into small clusters called micelles, rather than forming a bilayer
as do the phospholipids, which are more cylindrical in shape.

When mixed in great excess with membranes, the hydrophobic ends of
detergent molecules bind to the membrane-spanning hydrophobic region
of the transmembrane proteins, as well as to the hydrophobic tails of the
phospholipid molecules, thereby disrupting the lipid bilayer and separating 
the proteins from most of the phospholipids. Because the other
end of the detergent molecule is hydrophilic, this association brings the
membrane proteins into solution as protein-detergent complexes.
At the same time, the detergent solubilizes the phospholipids. The
protein-detergent complexes can then be separated from one another
and from the lipid-detergent complexes by a technique such as polyacrylamide-gel 
electrophoresis

\subsection{The Complete Structure is Known for relatively Few Membrane proteins}

With recent advances in crystallography, the X-ray structures of many membrane 
proteins have now been determined to high resolution, including
bacteriorhodopsin and a photosynthetic reaction center-membrane proteins 
that have important roles in the capture and use of energy from sunlight.
Bacteriorhodopsin is a type of transporter protein, a class of
transmembrane proteins that move molecules and ions into and out of
cells

\subsection{The plasma Membrane is reinforced by the Cell Cortex}

Most cell membranes are therefore strengthened
and supported by a framework of proteins, attached to the membrane
via transmembrane proteins. In particular, the shape of a cell and the
mechanical properties of its plasma membrane are determined by a
meshwork of fibrous proteins, called the cell cortex, that is attached to the
cytosolic surface of the membrane.

The cortex of human red blood cells is a relatively simple and regular
structure and is by far the best understood cell cortex. Red blood cells
are small and have a distinctive flattened shape.
The main
component of their cortex is the protein spectrin, a long, thin, flexible
rod about 100 nm in length. It forms a meshwork that provides support
for the plasma membrane and maintains the cell’s shape. The spectrin
meshwork is connected to the membrane through intracellular attachment 
proteins that link the spectrin to specific transmembrane proteins.
The individuals that have genetic abnormalities in spectrin structure are anemic.

\subsection{Cells Can restrict the Movement of Membrane proteins}

Because a membrane is a two-dimensional fluid, many of its proteins, like
its lipids, can move freely within the plane of the lipid bilayer.

The picture of a membrane as a sea of lipid in which all proteins float
freely is too simple, however. Cells have ways of confining particular
plasma membrane proteins to localized areas within the bilayer, thereby
creating functionally specialized regions, or membrane domains, on the
cell or organelle surface.

Plasma membrane proteins can be linked to fixed structures outside the cell
or to relatively immobile structures inside the cell, especially to the cell cortex.

Finally, cells can create barriers that restrict particular membrane components 
to one membrane domain.

\subsection{FLAP}

An essential feature of the lipid bilayer is its fluidity. This
vital molecular flow is crucial for cell membrane integrity and function. 
It allows the resident proteins to float
about the bilayer, coupling and uncoupling, engaging in
the molecular interactions on which cells depend. The
dynamic nature of cell membranes is so central to their
proper function that our working model of membrane
structure is commonly called the fluid-mosaic model.
Given its importance for membrane structure and func-
tion, how do we measure and study the fluidity of cell
membranes? The most common methods are visual:
simply label some of the molecules native to the membrane and then watch them move.

One such technique, called fluorescence recovery after
photobleaching (FRAP), involves uniformly labeling proteins 
across the cell surface, bleaching the label from a
small region in this fluorescent sea, and then seeing how
quickly the surrounding labeled proteins seep into this
bleached patch of membrane. To start, the membrane
protein of interest is tagged with a specific fluorescent
group. This labeling can be done either with a fluorescent 
antibody or by fusing the membrane protein with
a fluorescent protein such as green fluorescent protein
(GFP) using recombinant DNA techniques.
Once the cell has been labeled, it is placed under a
microscope and a small patch of its membrane is irradiated 
with an intense pulse from a sharply focused laser
beam. This treatment irreversibly bleaches the fluorescent 
groups in a spot, typically 1 mm square, on the cell surface.
The time it takes for fluorescent
proteins to migrate from the adjacent areas into the
bleached region of the membrane can then be measured. The rate of this ‘fluorescence recovery’ is a direct
measure of the rate at which the surrounding protein
molecules can diffuse within the membrane




\subsection{The Cell Surface is Coated with Carbohydrate}

We saw earlier that many of the lipids in the outer layer of the plasma
membrane have sugars covalently attached to them. The same is true
for most of the proteins in the plasma membrane. The great majority of
these proteins have short chains of sugars, called oligosaccharides, linked
to them; they are called glycoproteins. Other membrane proteins have
one or more long polysaccharide chains attached to them; they are called
proteoglycans. All of the carbohydrate on the glycoproteins, proteoglycans,
and glycolipids is located on one side of the membrane, the noncytosolic
side, where it forms a sugar coating called the carbohydrate layer.

By forming a layer of material covering the lipid bilayer, the carbohydrate
layer helps to protect the cell surface from mechanical and chemical
damage.

Cell-surface carbohydrates do more than just protect and lubricate the
cell, however. They have an important role in cell-cell recognition and
adhesion. Just as many proteins will recognize and bind to a particular
site on another protein, some proteins (called lectins) are specialized to
recognize particular oligosaccharide side chains and bind to them.

In a multicellular organism, the carbohydrate layer thus serves as a kind
of distinctive clothing, like a police officer’s uniform, that is character-
istic of cells specialized for a particular function and that is recognized
by other cells with which each must interact. They are also involved
in our responses to infection. In the early stages of a bacterial infection,
for instance, the carbohydrate on the surface of white blood cells called
neutrophils is recognized by a lectin on the cells lining the blood vessels 
at the site of infection.

\section{Essential concepts}

\begin{itemize}
\item Cell membranes enable a cell to create barriers that confine particular 
molecules to specific compartments.
\item Cell membranes consist of a continuous double layer - a bilayer - of
lipid molecules in which proteins are embedded.
\item The lipid bilayer provides the basic structure and barrier function of
all cell membranes.
\item Membrane lipid molecules have both hydrophobic and hydrophilic
regions. They assemble spontaneously into bilayers when placed in
water, forming closed compartments that reseal if torn.
\item There are three major classes of membrane lipid molecules: phospholipids, 
sterols, and glycolipids.
\item The lipid bilayer is fluid, and individual lipid molecules are able to
diffuse within their own monolayer; they do not, however, spontaneously 
flip from one monolayer to the other.
\item The two layers of the plasma membrane have different lipid compositions, 
reflecting the different functions of the two faces of a cell
membrane.
\item Some cells adjust their membrane fluidity by modifying the lipid composition 
of their membranes.
\item Membrane proteins are responsible for most of the functions of a
membrane, such as the transport of small water-soluble molecules
across the lipid bilayer.
\item Transmembrane proteins extend across the lipid bilayer, usually as
one or more $\alpha$ helices but sometimes as a $\beta$ sheet curved into the
form of a barrel.
\item Other membrane proteins do not extend across the lipid bilayer but
are attached to one or the other side of the membrane, either by non-covalent 
association with other membrane proteins or by covalent
attachment to lipids.
\item Most cell membranes are supported by an attached framework of
proteins. An example is the meshwork of fibrous proteins forming the
cell cortex underneath the plasma membrane.
\item Although many membrane proteins can diffuse rapidly in the plane of
the membrane, cells have ways of confining proteins to specific mem-
brane domains and of immobilizing particular proteins by attaching
them to intracellular or extracellular macromolecules.
\item Many of the proteins and some of the lipids exposed on the surface of
cells have attached sugars, which help protect and lubricate the cell
surface and are involved in cell–cell recognition.
\end{itemize}

\chapter{Membrane Transport}
