\chapter{Membrane structure}

A living cell is a self-reproducing system of molecules held inside a container.
That container is the plasma membrane - a fatty film so thin and
transparent that it cannot be seen directly in the light microscope.

The plasma membrane is simple in form: its structure is based on a two-ply
sheet of lipid molecules about 5 nm - or 50 atoms - thick.

The simplest bacteria have only a single membrane - the plasma membrane.
Eucaryotic cells, however, also contain an abundance of internal
membranes that enclose intracellular compartments to form the various
organelles, including the endoplasmic reticulum, Golgi apparatus, and
mitochondria.

Regardless of their location, all cell membranes are composed of lipids
and proteins and share a common general structure. The
lipids are arranged in two closely apposed sheets, forming a lipid bilayer.

\section{The lipid layer}

The \textbf{lipid bilayer} has been firmly established as the universal basis of
membrane structure, and its properties are responsible for the general
properties of all cell membranes.

\subsection{Membrane lipids Form bilayers in Water}

The lipids in cell membranes combine two very different properties in a
single molecule: each lipid has a \textbf{hydrophilic} (“water-loving”) head and
one or two \textbf{hydrophobic} (“water-fearing”) hydrocarbon tails.
The most abundant lipids in cell membranes are the \textit{phospholipids},
molecules in which the hydrophilic head is linked to the rest of the lipid
through a phosphate group. The most common type of phospholipid
in most cell membranes is phosphatidylcholine, which has the small
molecule choline attached to a phosphate as its hydrophilic head and two
long hydrocarbon chains as its hydrophobic tails.
Phosphatidylcholine is the most common phospholipid in cell membranes.
This particular phospholipid is built from five parts: the \textbf{hydrophilic head},
choline, is linked via a \textbf{phosphate} to \textbf{glycerol}, which in turn is linked to two
hydrocarbon chains, forming the \textbf{hydrophobic tail}. The two hydrocarbon chains originate
as fatty acids - that is, hydrocarbon chains with a -COOH group at one end - which become
attached to glycerol via their -COOh groups. A kink in one of the hydrocarbon chains
occurs where there is a double bond between two carbon atoms. The ‘phosphatidyl’ part of the 
name of phospholipids refers to the phosphate-glycerol-fatty acid portion of the molecule.

Molecules with both hydrophilic and hydrophobic properties are termed
amphipathic.

These non-polar atoms force adjacent water molecules to reorganize into a cagelike
structure around the hydrophobic molecule. Because the
cagelike structure is more highly ordered than the surrounding water,
its formation requires energy. The energy cost is minimized, however, if
the hydrophobic molecules cluster together, limiting their contact with
water to the smallest possible number of water molecules.

In contrast, amphipathic molecules, such as phospholipids, are subject to two conflicting forces: the hydrophilic head is
attracted to water, while the hydrophobic tail shuns water and seeks to
aggregate with other hydrophobic molecules. This conflict is beautifully
resolved by the formation of a lipid bilayer - an arrangement that satisfies
all parties and is energetically most favorable. The hydrophilic heads
face the water from both surfaces of the bilayer sheet; the hydrophobic
tails are all shielded from the water as they lie next to one another in the
interior.

The same forces that drive the amphipathic molecules to form a bilayer
make the bilayer self-sealing. Any tear in the sheet will create a free edge
that is exposed to water. Because this situation is energetically unfavorable,
the molecules of the bilayer will spontaneously rearrange to eliminate
the free edge.

The prohibition on free edges has a profound consequence: the only way
a finite sheet can avoid having free edges is to bend and seal, forming a
boundary around a closed space. Therefore, amphipathic
molecules such as phospholipids necessarily assemble into self-sealing
containers that define closed compartments.

\subsection{The lipid bilayer is a two-dimensional Fluid}

The membrane therefore behaves as a two-dimensional fluid, which is crucial for membrane function and integrity.
This property is distinct from flexibility, which is the ability of
the membrane to bend.

The fluidity of lipid bilayers can be studied using synthetic lipid bilayers,
which are easily produced by the spontaneous aggregation of amphipathic
lipid molecules in water. Two types of synthetic lipid bilayers are
commonly used in experiments. Closed spherical vesicles, called \textbf{liposomes},
form if pure phospholipids are added to water. Alternatively, flat
phospholipid bilayers can be formed across a hole in a partition between
two aqueous compartments.

Thus, in synthetic lipid bilayers, phospholipid molecules very rarely tumble from one half of the bilayer,
or monolayer, to the other. Without proteins to facilitate the process
and under conditions similar to those in a cell, it is estimated that this
event, called ‘\textbf{flip-flop},’ occurs less than once a month for any individual
lipid molecule. On the other hand, as the result of thermal motions, lipid
molecules within a monolayer continuously exchange places with their neighbors.

Similar results are obtained when one examines isolated cell membranes
and whole cells, indicating that the lipid bilayer of a cell membrane
also behaves as a two-dimensional fluid in which the constituent lipid
molecules are free to move within their own layer in any direction in the
plane of the membrane. These studies also show that lipid hydrocarbon
chains are flexible and that individual lipid molecules within a monolayer
rotate very rapidly about their long axis (\textbf{lateral diffusion}). In cells, as in synthetic bilayers,
individual phospholipid molecules are normally confined to their own
monolayer and do not flip-flop spontaneously.

\subsection{The Fluidity of a lipid bilayer depends on its Composition}

Just how fluid a lipid
bilayer is at a given temperature depends on its phospholipid composition
and, in particular, on the nature of the hydrocarbon tails: the closer
and more regular the packing of the tails, the more viscous and less fluid
the bilayer will be. Two major properties of hydrocarbon tails affect how
tightly they pack in the bilayer: their length and the number of double
bonds they contain.

A shorter chain length reduces the tendency of the hydrocarbon tails
to interact with one another and therefore increases the fluidity of the
bilayer.
Most phospholipids contain one hydrocarbon tail that has one or more
double bonds between adjacent carbon atoms, and a second tail with
single bonds only. The chain that harbors a double bond
does not contain the maximum number of hydrogen atoms that could, in
principle, be attached to its carbon backbone; it is thus said to be \textbf{unsaturated}
with respect to hydrogen. The fatty acid tail with no double bonds
has a full complement of hydrogen atoms; it is said to be \textbf{saturated}. Each
double bond in an unsaturated tail creates a small kink in the hydrocarbon tail.

In bacterial and yeast cells, which have to adapt to varying temperatures,
both the lengths and the unsaturation of the hydrocarbon tails in the
bilayer are constantly adjusted to maintain the membrane at a relatively
constant fluidity.

In animal cells, membrane fluidity is modulated by the inclusion of the
sterol \textbf{cholesterol}.

\subsection{The lipid bilayer is asymmetrical}

The lipid asymmetry is established and maintained as the membrane
grows. In eucaryotic cells, new phospholipids are manufactured by
enzymes bound to the part of the endoplasmic reticulum membrane that
faces the \textbf{cytosol}. These enzymes, which use free fatty acids as substrates,
deposit all newly made phospholipids into the
cytosolic half of the bilayer. To enable the membrane as a whole to grow
evenly, half of the new phospholipid molecules then have to be transferred
to the opposite monolayer. This transfer is catalyzed by enzymes
called \textbf{flippases}. In the plasma membrane, flippases transfer
specific phospholipids selectively, so that different types become concentrated
in each monolayer.

Using selective flippases is not the only way to produce asymmetry in
lipid bilayers, however. In particular, a different mechanism operates for
\textbf{glycolipids} - the lipids that show the most striking and consistent asymmetric
distribution in animal cells.

\subsection{Lipid asymmetry is preserved during Membrane transport}

Nearly all new membrane synthesis in eucaryotic cells occurs in the
membrane of one intracellular compartment - the endoplasmic reticulum.
The new membrane assembled there is exported
to the other membranes of the cell through a cycle of membrane budding
and fusion: bits of the bilayer pinch off from the ER to form small spheres
called vesicles, which then become incorporated into another membrane,
such as the plasma membrane, by fusing with it. The orientation
of the bilayer relative to the cytosol is preserved during vesicle
formation and fusion. This preservation of orientation means that all cell
membranes, whether the external plasma membrane or an intracellular
membrane around an organelle, have distinct ‘inside’ and ‘outside’ faces
that are established at the time of membrane synthesis: the cytosolic face
is always adjacent to the cytosol, while the noncytosolic face is exposed
to either the cell exterior or the interior space of an organelle

Glycolipids are located mainly in the plasma membrane, and they are
found only in the noncytosolic half of the bilayer. Their sugar groups are
therefore exposed to the exterior of the cell, where they
form part of a continuous protective coat of carbohydrate that surrounds
most animal cells. The glycolipid molecules acquire their sugar groups
in the Golgi apparatus, the organelle to which proteins and membranes
made in the ER often go next.

\section{Membrane proteins}

Although the lipid bilayer provides the basic structure of all cell membranes
and serves as a permeability barrier to the molecules on either
side of it, most membrane functions are carried out by membrane proteins.
Membrane proteins not only transport particular nutrients, metabolites,
and ions across the lipid bilayer; they serve many other functions.

\subsection{Membrane proteins associate with the lipid bilayer in Various Ways}

Proteins can be associated with the lipid bilayer of a cell membrane in
several ways:

\begin{enumerate}
\item Many membrane proteins extend through the bilayer, with part of
their mass on either side. Like their lipid neighbors,
these transmembrane proteins have both hydrophobic and
hydrophilic regions. Their hydrophobic regions lie in the interior
of the bilayer, nestled against the hydrophobic tails of the lipid
molecules. Their hydrophilic regions are exposed to the aqueous
environment on either side of the membrane.
\item Other membrane proteins are located entirely in the cytosol, associated
with the inner leaflet of the lipid bilayer by an \textbf{amphipathic $\alpha$-helix} exposed on the surface of the protein.
\item Some proteins lie entirely outside the bilayer, on one side or the
other, attached to the membrane only by one or more covalently
attached lipid groups.
\item Yet other proteins are bound indirectly to one or the other face of
the membrane, held in place only by their interactions with other
membrane proteins
\end{enumerate}

Proteins that are directly attached to a lipid bilayer - whether they are
\textbf{transmembrane, monolayer-associated, or lipid-linked} - can be removed
only by disrupting the bilayer with detergents, as discussed shortly. Such
proteins are known as integral membrane proteins. The remaining membrane p
roteins are known as \textbf{peripheral membrane proteins}; they can
be released from the membrane by more gentle extraction procedures
that interfere with protein-protein interactions but leave the lipid bilayer
intact.

\subsection{A polypeptide Chain usually Crosses the bilayer as an $\alpha$-helix}

The peptide bonds that join the successive amino acids in a protein are normally polar, making
the polypeptide backbone hydrophilic. Because water is
absent from the bilayer, atoms forming the backbone are driven to form
hydrogen bonds with one another. Hydrogen-bonding is maximized if the
polypeptide chain forms a regular a helix, and so the great majority of the
membrane-spanning segments of polypeptide chains traverse the bilayer
as $\alpha$ helices.

The proteins that form pores are more complicated, usually possessing a
series of $\alpha$ helices that cross the bilayer a number of times. In many of these proteins,
one or more of the transmembrane regions are formed from $\alpha$ helices
that contain both hydrophobic and hydrophilic amino acid side chains.

Although the $\alpha$-helix is by far the most common form in which a polypeptide
chain crosses a lipid bilayer, the polypeptide chain of some transmembrane
proteins crosses the lipid bilayer as a $\beta$ sheet that is curved into a
cylinder, forming an open-ended keglike structure called a $\beta$ barrel. As
expected, the amino acid side chains that face the inside of the barrel, and
therefore line the aqueous channel, are mostly hydrophilic, while those
on the outside of the barrel, which contact the hydrophobic core of the
lipid bilayer, are exclusively hydrophobic. The most striking example of a
$\beta$ barrel structure is found in the porin proteins, which form large, water-filled
pores in mitochondrial and bacterial membranes.

Some examples of plasma membrane proteins and their functions are the following:

\begin{itemize}
\item \textbf{Transporters}: actively pumps $Na^{+}$ out of cells and $K^{+}$ in.
\item \textbf{Anchors:} link intracellular actin (i.e. family of globular multi-functional proteins) filaments to extracellular matrix proteins.
\item \textbf{Receptors:} binds extracellular platelet-derived growth factor (PDGF) and, as a consequence, generates intracellular signals
that cause the cell to grow and divide.
\item \textbf{Enzymes:} link intracellular actin filaments to extracellular matrix proteins.
\end{itemize}

\subsection{Membrane proteins Can be Solubilized in detergents and purified}

Before an individual protein can be studied in detail, it must be separated
from all the other cell proteins. For most membrane proteins, the first step
in this separation process involves solubilizing the membrane with agents
that destroy the lipid bilayer by disrupting hydrophobic associations. The
most widely used disruptive agents are \textbf{detergents}. These
are small, amphipathic, lipidlike molecules that have both a hydrophilic
and a hydrophobic region. Detergents differ from membrane
phospholipids in that they have only a single hydrophobic tail and,
consequently, behave in a significantly different way. Because they have
one tail, detergent molecules are shaped like cones; in water, they tend to
aggregate into small clusters called micelles, rather than forming a bilayer
as do the phospholipids, which are more cylindrical in shape.

When mixed in great excess with membranes, the hydrophobic ends of
detergent molecules bind to the membrane-spanning hydrophobic region
of the transmembrane proteins, as well as to the hydrophobic tails of the
phospholipid molecules, thereby disrupting the lipid bilayer and separating
the proteins from most of the phospholipids. Because the other
end of the detergent molecule is hydrophilic, this association brings the
membrane proteins into solution as protein-detergent complexes.
At the same time, the detergent solubilizes the phospholipids. The
protein-detergent complexes can then be separated from one another
and from the lipid-detergent complexes by a technique such as polyacrylamide-gel
electrophoresis

\subsection{The Complete Structure is Known for relatively Few Membrane proteins}

With recent advances in crystallography, the X-ray structures of many membrane
proteins have now been determined to high resolution, including
bacteriorhodopsin and a photosynthetic reaction center-membrane proteins
that have important roles in the capture and use of energy from sunlight.
Bacteriorhodopsin is a type of transporter protein, a class of
transmembrane proteins that move molecules and ions into and out of
cells

\subsection{The plasma Membrane is reinforced by the Cell Cortex}

Most cell membranes are therefore strengthened
and supported by a framework of proteins, attached to the membrane
via transmembrane proteins. In particular, the shape of a cell and the
mechanical properties of its plasma membrane are determined by a
meshwork of fibrous proteins, called the \textbf{cell cortex}, that is attached to the
cytosolic surface of the membrane.

The cortex of human red blood cells is a relatively simple and regular
structure and is by far the best understood cell cortex. Red blood cells
are small and have a distinctive flattened shape.
The main
component of their cortex is the protein \textbf{spectrin}, a long, thin, flexible
rod about 100 nm in length. It forms a meshwork that provides support
for the plasma membrane and maintains the cell’s shape. The spectrin
meshwork is connected to the membrane through intracellular attachment
proteins that link the spectrin to specific transmembrane proteins.
The individuals that have genetic abnormalities in spectrin structure are anemic.

\subsection{Cells Can restrict the Movement of Membrane proteins}

Because a membrane is a two-dimensional fluid, many of its proteins, like
its lipids, can move freely within the plane of the lipid bilayer.

The picture of a membrane as a sea of lipid in which all proteins float
freely is too simple, however. Cells have ways of confining particular
plasma membrane proteins to localized areas within the bilayer, thereby
creating functionally specialized regions, or membrane domains, on the
cell or organelle surface.

Plasma membrane proteins can be linked to fixed structures outside the cell
or to relatively immobile structures inside the cell, especially to the cell cortex.

Finally, cells can create barriers that restrict particular membrane components
to one membrane domain.

\subsection{FLAP}

An essential feature of the lipid bilayer is its fluidity. This
vital molecular flow is crucial for cell membrane integrity and function.
It allows the resident proteins to float
about the bilayer, coupling and uncoupling, engaging in
the molecular interactions on which cells depend. The
dynamic nature of cell membranes is so central to their
proper function that our working model of membrane
structure is commonly called the fluid-mosaic model.
Given its importance for membrane structure and function,
how do we measure and study the fluidity of cell
membranes? The most common methods are visual:
simply label some of the molecules native to the membrane and then watch them move.

One such technique, called fluorescence recovery after
photobleaching (\textbf{FRAP}), involves uniformly labeling proteins
across the cell surface, bleaching the label from a
small region in this fluorescent sea, and then seeing how
quickly the surrounding labeled proteins seep into this
bleached patch of membrane. To start, the membrane
protein of interest is tagged with a specific fluorescent
group. This labeling can be done either with a fluorescent
antibody or by fusing the membrane protein with
a fluorescent protein such as green fluorescent protein
(GFP) using recombinant DNA techniques.
Once the cell has been labeled, it is placed under a
microscope and a small patch of its membrane is irradiated
with an intense pulse from a sharply focused laser
beam. This treatment irreversibly bleaches the fluorescent
groups in a spot, typically 1 mm square, on the cell surface.
The time it takes for fluorescent
proteins to migrate from the adjacent areas into the
bleached region of the membrane can then be measured. The rate of this ‘fluorescence recovery’ is a direct
measure of the rate at which the surrounding protein
molecules can diffuse within the membrane.

\subsection{The Cell Surface is Coated with Carbohydrate}

We saw earlier that many of the lipids in the outer layer of the plasma
membrane have sugars covalently attached to them. The same is true
for most of the proteins in the plasma membrane. The great majority of
these proteins have short chains of sugars, called oligosaccharides, linked
to them; they are called \textbf{glycoproteins}. Other membrane proteins have
one or more long polysaccharide chains attached to them; they are called
proteoglycans. All of the carbohydrate on the glycoproteins, proteoglycans,
and glycolipids is located on one side of the membrane, the noncytosolic
side, where it forms a sugar coating called the carbohydrate layer.

By forming a layer of material covering the lipid bilayer, the carbohydrate
layer helps to protect the cell surface from mechanical and chemical
damage.

Cell-surface carbohydrates do more than just protect and lubricate the
cell, however. They have an important role in cell-cell recognition and
adhesion. Just as many proteins will recognize and bind to a particular
site on another protein, some proteins (called lectins) are specialized to
recognize particular oligosaccharide side chains and bind to them.

In a multicellular organism, the carbohydrate layer thus serves as a kind
of distinctive clothing, like a police officer’s uniform, that is characteristic
of cells specialized for a particular function and that is recognized
by other cells with which each must interact. They are also involved
in our responses to infection. In the early stages of a bacterial infection,
for instance, the carbohydrate on the surface of white blood cells called
neutrophils is recognized by a lectin on the cells lining the blood vessels
at the site of infection.

\section{Essential concepts}

\begin{itemize}
\item Cell membranes enable a cell to create barriers that confine particular
molecules to specific compartments.
\item Cell membranes consist of a continuous double layer - a bilayer - of
lipid molecules in which proteins are embedded.
\item The lipid bilayer provides the basic structure and barrier function of
all cell membranes.
\item Membrane lipid molecules have both hydrophobic and hydrophilic
regions. They assemble spontaneously into bilayers when placed in
water, forming closed compartments that reseal if torn.
\item There are three major classes of membrane lipid molecules: phospholipids,
sterols, and glycolipids.
\item The lipid bilayer is fluid, and individual lipid molecules are able to
diffuse within their own monolayer; they do not, however, spontaneously
flip from one monolayer to the other.
\item The two layers of the plasma membrane have different lipid compositions,
reflecting the different functions of the two faces of a cell
membrane.
\item Some cells adjust their membrane fluidity by modifying the lipid composition
of their membranes.
\item Membrane proteins are responsible for most of the functions of a
membrane, such as the transport of small water-soluble molecules
across the lipid bilayer.
\item Transmembrane proteins extend across the lipid bilayer, usually as
one or more $\alpha$ helices but sometimes as a $\beta$ sheet curved into the
form of a barrel.
\item Other membrane proteins do not extend across the lipid bilayer but
are attached to one or the other side of the membrane, either by non-covalent
association with other membrane proteins or by covalent
attachment to lipids.
\item Most cell membranes are supported by an attached framework of
proteins. An example is the meshwork of fibrous proteins forming the
cell cortex underneath the plasma membrane.
\item Although many membrane proteins can diffuse rapidly in the plane of
the membrane, cells have ways of confining proteins to specific membrane
domains and of immobilizing particular proteins by attaching
them to intracellular or extracellular macromolecules.
\item Many of the proteins and some of the lipids exposed on the surface of
cells have attached sugars, which help protect and lubricate the cell
surface and are involved in cell-cell recognition.
\end{itemize}

\chapter{Membrane Transport}

A few \textbf{solutes} can simply diffuse across the lipid bilayer, but
the vast majority cannot. Instead, their transfer depends on specialized
membrane transport proteins that span the lipid bilayer, providing
private passageways across the membrane for select substances

Here, we begin by outlining some of the general principles that guide the passage
of small, water-soluble molecules through cell membranes. We then
examine, in turn, the two main classes of membrane proteins that mediate
this transfer. A \textbf{transporter}, which has moving parts, can shift small
molecules from one side of the membrane to the other by changing its
shape. Solutes transported in this way can be either small organic molecules
or inorganic ions. \textbf{Channels}, in contrast, form tiny hydrophilic \textbf{pores}
in the membrane through which solutes can pass by diffusion. Most channels
let through inorganic ions only and are therefore called ion channels.
Because these ions are electrically charged, their movements can create
powerful electric forces across the membrane.

\section{Princples of membrane transport}

\subsection{The Ion Concentrations Inside a Cell Are Very Different from Those Outside}

Living cells maintain an internal ion composition that is very different
from the ion composition in the fluid around them, and these differences
are crucial for a cell’s survival and function.

For a cell to avoid being torn apart by electrical forces, the quantity of positive charge inside the
cell must be balanced by an almost exactly equal quantity of negative
charge there, and the same is true for the charge in the surrounding fluid.
However, tiny excesses of positive or negative charge, concentrated in
the neighborhood of the plasma membrane, do occur, and they have
important electrical effects.

This differential distribution of ions inside and outside the cell is controlled
in part by the activity of membrane transport proteins and in part by the
permeability characteristics of the lipid bilayer itself.

\subsection{Lipid Bilayers Are Impermeable to Solutes and Ions}


Given enough time, virtually any molecule will diffuse across a lipid bilayer.
The rate at which it diffuses, however, varies
enormously depending on the size of the molecule and its solubility properties.
In general, the smaller the molecule and the more soluble it is in
oil (that is, the more hydrophobic, or nonpolar, it is), the more rapidly it
will diffuse across. Thus:

\begin{enumerate}
\item Small nonpolar molecules, such as molecular oxygen (molecular
mass 32 daltons) and carbon dioxide (44 daltons), readily dissolve
in lipid bilayers and therefore rapidly diffuse across them.
\item Uncharged polar molecules (molecules with an uneven distribution
of electric charge) also diffuse rapidly across a bilayer, if they
are small enough. Water (18 daltons) and ethanol (46 daltons), for
example, cross fairly rapidly; glycerol (92 daltons) crosses less rapidly;
and glucose (180 daltons) crosses hardly at all.
\item In contrast, lipid bilayers are highly impermeable to all ions and
charged molecules, no matter how small. The molecules’ charge
and their strong electrical attraction to water molecules inhibit
them from entering the hydrocarbon phase of the bilayer.
\end{enumerate}

Cell membranes allow water and small nonpolar molecules to permeate
by simple diffusion. But for cells to take up nutrients and release wastes,
membranes must also allow the passage of many other molecules, such
as ions, sugars, amino acids, nucleotides, and many cell metabolites.

\subsection{Membrane Transport Proteins Fall into Two Classes: Transporters and Channels}

Membrane transport proteins occur in many forms and in all types of biological
membranes. Each protein provides a private passageway across
the membrane for a particular class of molecule. Most of these protein portals are even more exclusive,
allowing entrance of only select members of a particular molecular class.
The set of membrane transport proteins present in the plasma
membrane or in the membrane of an intracellular organelle determines
exactly which solutes can pass into and out of that cell or organelle.

Membrane transport proteins can be divided into two main classes:
transporters and channels. The basic difference between transporters
and channels is the way they discriminate between solutes, transporting
some solutes but not others. Channels discriminate mainly
on the basis of size and electric charge: if a channel is open, an ion or a
molecule that is small enough and carries the appropriate charge can slip
through, as through a narrow trapdoor. A transporter, on the other hand,
allows passage only to those molecules or ions that fit into a binding
site on the protein; it then transfers these molecules across the membrane
one at a time by changing its own conformation, acting more like
a turnstile than an open door. Transporters bind their solutes with great
specificity in the same way that an enzyme binds its substrate, and it is
this requirement for specific binding that makes the transport selective.

\subsection{Solutes Cross Membranes by Passive or Active Transport}

Transporters and channels allow small molecules to cross the cell membrane,
but what controls whether these solutes move into the cell or out
of it? In many cases, the direction of transport depends on the relative
concentrations of the solute. Molecules will spontaneously flow ‘down-hill’
from a region of high concentration to a region of low concentration,
provided a pathway exists. Such movements are called \textbf{passive}, because
they need no other driving force. If, for example, a solute is present at
a higher concentration outside the cell than inside and an appropriate
channel or transporter is present in the plasma membrane, the solute
will move spontaneously across the membrane down its concentration
gradient into the cell by passive transport (sometimes called facilitated
diffusion), without expenditure of energy by its membrane transport protein.
All channels and many transporters act as conduits for such passive
transport.

To move a solute against its concentration gradient, however, a membrane
transport protein must do work: it has to drive the flow ‘uphill’
by coupling it to some other process that provides energy. Transmembrane solute movement
driven in this way is termed \textbf{active} transport, and it is carried out only
by special types of transporters that can harness some energy source to
the transport process. Because they drive the transport of
solutes against their concentration gradient, many of these transporters
are called pumps.

\section{Transporters and their functions}

\subsection{Concentration Gradients and Electrical Forces Drive Passive Transport}

Solutes can cross the membrane by passive or active transport - and
transporters are capable of facilitating both types of movement.
A simple example of a transporter that mediates passive transport
is the glucose transporter found in the plasma membrane of mammalian
liver cells and many other cell types. It is thought that the
transporter can adopt at least two conformations and switches reversibly
and randomly between them. In one conformation, the transporter
exposes binding sites for glucose to the exterior of the cell; in the other, it
exposes these sites to the interior of the cell.

When sugar is plentiful outside a liver cell, as it is after a meal, glucose
molecules bind to the transporter’s externally displayed binding sites;
when the protein switches conformation, it carries these molecules
inward and releases them into the cytosol, where the glucose concentration
is low. Conversely, when blood sugar levels are low - when you
are hungry - the hormone glucagon stimulates the liver cell to produce
large amounts of glucose by the breakdown of glycogen. As a result, the
glucose concentration is higher inside the cell than outside, and glucose
binds to any internally displayed binding sites on the transporter; when
the protein switches conformation in the opposite direction, the glucose
is transported out of the cell. The flow of glucose can thus go either way,
according to the direction of the glucose concentration gradient across
the membrane.

For glucose, which is an uncharged molecule, the direction of passive
transport is determined solely by its concentration gradient. For electrically
charged molecules, either small organic ions or inorganic ions, an
additional force comes into play. For reasons we explain later, most cell
membranes have a voltage across them, a difference in the electrical
potential on each side of the membrane, which is referred to as the membrane
potential. This difference in potential exerts a force on any molecule
that carries an electric charge. The cytoplasmic side of the plasma membrane
is usually at a negative potential relative to the outside.

The net force driving a charged solute across the membrane is therefore
a composite of two forces, one due to the concentration gradient and the
other due to the voltage across the membrane. This net driving force is
called the \textbf{electrochemical gradient} for the given solute. This gradient
determines the direction of passive transport across the membrane.

\subsection{Active Transport Moves Solutes Against Their Electrochemical Gradients}

Active transport of solutes against their electrochemical gradient is essential to maintain
the intracellular ionic composition of cells and to import solutes that are
at a lower concentration outside the cell than inside. Cells carry out active
transport in three main ways: (1) \textbf{Coupled transporters} couple
the uphill transport of one solute across the membrane to the downhill
transport of another. (2) ATP-driven pumps couple uphill transport to the
hydrolysis of ATP. (3) Light-driven pumps, which are found mainly in bacterial
cells, couple uphill transport to an input of energy from light, as
discussed for bacteriorhodopsin.
Because a substance has to be carried uphill before it can flow downhill,
the different forms of active transport are necessarily linked.

\subsection{Animal Cells Use the Energy of ATP Hydrolysis to Pump Out $Na^{+}$}

The ATP-driven $Na^{+}$ pump in animal cells hydrolyzes ATP to ADP to transport
$Na^{+}$ out of the cell; this pump is therefore not only a transporter,
but also an enzyme - an ATPase. At the same time, the protein couples
the outward transport of $Na^{+}$ to an inward transport of $K^{+}$. The pump is
therefore commonly known as the $Na^{+}$-$K^{+}$ ATPase, or the \textbf{$Na^{+}$-$K^{+}$ pump}.

This transporter plays a central part in the energy economy of animal
cells, typically accounting for 30\% or more of their total ATP consumption.
Like a bilge pump in a leaky ship, it operates ceaselessly to expel the
$Na^{+}$ that is constantly entering through other transporters and ion channels.
In this way, the pump keeps the $Na^{+}$ concentration in the cytosol
about 10-30 times lower than in the extracellular fluid and the $K^{+}$ concentration
about 10-30 times higher. Under normal
conditions, the interior of most cells is at a negative electric potential
compared with the exterior, so that positive ions tend to be pulled into
the cell. This means that the inward electrochemical driving force for $Na^{+}$
is large, as it includes the driving force due to the concentration gradient
and a driving force in the same direction due to the voltage gradient

\subsection{The $Na^{+}$-$K^{+}$ Pump Is Driven by the Transient Addition of a Phosphate Group}

The $Na^{+}$-$K^{+}$ pump provides a beautiful illustration of how a protein
couples one reaction to another. The pump works in a cycle.
$Na^{+}$ binds to the pump at sites exposed inside the cell (stage 1),
activating the ATPase activity. ATP is split, with the release of ADP and
the transfer of a phosphate group into a high-energy linkage to the pump
itself - that is, the pump phosphorylates itself (stage 2). Phosphorylation
causes the pump to switch its conformation so as to release $Na^{+}$ at the
exterior surface of the cell and, at the same time, to expose a binding
site for $K^{+}$ at the same surface (stage 3). The binding of extracellular $K^{+}$
(stage 4) triggers the removal of the phosphate group (stage 5), causing
the pump to switch back to its original conformation, discharging the $K^{+}$
into the cell interior (stage 6).

\subsection{The $Na^{+}$-$K^{+}$ Pump Helps Maintain the Osmotic Balance of Animal Cells}

The plasma membrane is permeable to water, and if
the total concentration of solutes is low on one side of the membrane
and high on the other, water will tend to move across it until the solute
concentrations are equal. The movement of water from a region of low
solute concentration (high water concentration) to a region of high solute
concentration (low water concentration) is called \textbf{osmosis}. Cells contain
specialized water channels (called \textbf{aquaporins}) in their plasma membrane
that facilitate this flow. The driving force for the water movement
is equivalent to a difference in water pressure and is called the \textbf{osmotic
pressure}. In the absence of any counteracting pressure, the osmotic
movement of water into a cell will cause it to swell.

This function is performed mainly by the $Na^{+}$-$K^{+}$ pump, which pumps
out the $Na^{+}$ that leaks in. At the same time, the $Na^{+}$-$K^{+}$ pump helps to
maintain a \textbf{membrane potential}. This membrane
potential tends to prevent the entry of $Cl^{-}$ , which is negatively charged
and would need to move against the electrical gradient generated by the
pump to enter the cell.

\subsection{Intracellular $Ca^{2+}$ Concentrations Are Kept Low by $Ca^{2+}$ Pumps}

$Ca^{2+}$, like $Na^{+}$, is also kept at a low concentration in the cytosol compared
with its concentration in the extracellular fluid, but it is much less plentiful
than $Na^{+}$, both inside and outside cells. The movement of $Ca^{2+}$ across
cell membranes, however, is crucially important because $Ca^{2+}$ can bind
tightly to a variety of proteins in the cell, altering their activities. An influx
of $Ca^{2+}$ into the cytosol through $Ca^{2+}$ channels, for example, is often used
as a signal to trigger other intracellular events, such as the secretion of
signal molecules and the contraction of muscle cells.


Like the $Na^{+}$-$K^{+}$ pump, the $Ca^{2+}$ pump is an ATPase that is phosphorylated and dephosphorylated during its pumping cycle.

\subsection{Coupled Transporters Exploit Gradients to Take Up Nutrients Actively}

A gradient of any solute across a membrane, like the $Na^{+}$ gradient generated
by the $Na^{+}$-$K^{+}$ pump, can be used to fuel the active transport of
a second molecule. The downhill movement of the first solute down its
gradient provides the energy to drive the uphill transport of the second.
The transporters that do this are called coupled transporters

If the transporter moves both solutes in the same direction across the membrane,
it is called a \textbf{symport}. If it moves them in opposite directions,
it is called an \textbf{antiport}. A transporter that ferries only one type of
solute across the membrane (and is therefore not a coupled transporter)
is called a \textbf{uniport}. The passive glucose transporter described earlier is a uniport.

In animal cells, an especially important role is played by symports that
use the inward flow of $Na^{+}$ down its steep electrochemical gradient to
drive the import of other solutes into the cell. The epithelial cells that line
the gut, for example, transfer glucose from the gut\footnote{intestine} across the gut epithelium.
If these cells had only the passive, uniport glucose transporters,
they would release glucose into the gut after a sugar-free meal as freely
as they take it up from the gut after a sugar-rich meal. But these epithelial
cells also possess a \textbf{glucose-$Na^{+}$ symport}, which they can use to take up
glucose from the \textbf{gut lumen} by active transport, even when the concentration of
glucose is higher inside the cell than in the gut. Because the
electrochemical gradient for $Na^{+}$ is steep, when $Na^{+}$ moves into the cell
down its gradient, the sugar is, in a sense, “dragged” into the cell with it.
Because the binding of $Na^{+}$ and glucose is cooperative -
the binding of one enhances the binding of the other - both molecules
must be present for coupled transport to occur.

If the gut epithelial cells had only this symport, however, they could never
release glucose for use by the other cells of the body. These cells, therefore,
have two types of glucose transporters. In the apical domain of the plasma
membrane, which faces the lumen of the gut, they have the glucose-$Na^{+}$
symports.

Cells in the lining of the gut and in many other organs, such as the kidney,
contain a variety of symports in their plasma membrane that are similarly
driven by the electrochemical gradient of $Na^{+}$; each of these transporters
specifically imports a small group of related sugars or amino acids into
the cell. But $Na^{+}$-driven antiports are also important for cell function. For
example, the $Na^{+}$-$H^{+}$ exchanger in the plasma membranes of many animal
cells uses the downhill influx of $Na^{+}$ to pump $H^{+}$ out of the cell and
is one of the main devices that animal cells use to control the pH in their
cytosol.

\subsection{$H^{+}$ Gradients Are Used to Drive Membrane Transport in Plants, Fungi, and Bacteria}

Plant cells, fungi (including yeasts), and bacteria do not have $Na^{+}$-$K^{+}$
pumps in their plasma membrane. Instead of an electrochemical gradient
of $Na^{+}$, they rely mainly on an electrochemical gradient of $H^{+}$ to drive the
transport of solutes into the cell. The gradient is created by \textbf{$H^{+}$ pumps} in
the plasma membrane, which pump $H^{+}$ out of the cell, thus setting up an
electrochemical proton gradient, with $H^{+}$ higher outside than inside; in the
process, the $H^{+}$ pump also creates an acid pH in the medium surrounding
the cell. The uptake of many sugars and amino acids into bacterial cells,
then, is driven by $H^{+}$ symports, which use the electrochemical gradient of
$H^{+}$ across the plasma membrane in much the same way that animal cells
use the electrochemical gradient of $Na^{+}$.

A different type of $H^{+}$ ATPase is found in the membranes of some intracellular
organelles, such as the lysosomes of animal cells and the central
vacuole of plant and fungal cells. Their function is to pump $H^{+}$ out of the
cytosol into the organelle, thereby helping to keep the pH of the cytosol
neutral and the pH of the interior of the organelle acidic. The acid environment
in many organelles is crucial to their function.

\section{Ion channels and the membrane potential}

In principle, the simplest way to allow a small water-soluble molecule to
cross from one side of a membrane to the other is to create a hydrophilic
channel through which the molecule can pass. Channels perform this
function in cell membranes, forming transmembrane aqueous pores that
allow the passive movement of small water-soluble molecules into or out
of the cell or organelle.

A few channels form relatively large pores: examples are the proteins
that form gap junctions between two adjacent cells and the porins that
form channels in the outer membrane of mitochondria and some bacteria. But such large, permissive
channels would lead to disastrous leaks if they directly connected the
cytosol of a cell to the extracellular space. Thus, most of the channels
in the plasma membrane of animal and plant cells have narrow, highly
selective pores. One specialized channel, called aquaporin, facilitates the
flow of water across the plasma membrane. The structure of this protein
allows the rapid passage of uncharged water molecules, while prohibiting
the movement of ions, including $H^{+}$ . But the bulk of the cell’s channels
enable the transport of inorganic ions, mainly $Na^{+}$, $K^{+}$ , $Cl^{-}$, and $Ca^{2+}$ . It is
these ion channels that we discuss next.

\subsection{Ion Channels Are Ion-selective and Gated}

Two important properties distinguish ion channels from simple holes in
the membrane. First, they show ion selectivity, permitting some inorganic
ions to pass but not others. Ion selectivity depends on the diameter and
shape of the ion channel and on the distribution of the charged amino acids that line it.

Each ion in aqueous solution is surrounded by a small shell of water molecules,
and the ions have to shed most of their associated water molecules in
order to pass, in single file, through the selectivity filter in the narrowest
part of the channel. There, ions make important but very transient contacts
with atoms in the amino acids that line the walls of the selectivity
filter. These precisely positioned atoms allow the channel
to discriminate between ions that differ only minutely in size. This
step in the transport process also limits the maximum rate of passage of
ions through the channel. Thus, as ion concentrations are increased, the
flow of ions through a channel at first increases proportionally but then
levels off (saturates) at a maximum rate.

The second important distinction between simple pores and ion channels
is that ion channels are not continuously open. Ion transport would be of
no value to the cell if there were no means of controlling the flow and if
all of the many thousands of ion channels in a cell membrane were open
all of the time. Instead, ion channels open briefly and then close again.
As we discuss later, most ion channels are gated: a specific
stimulus triggers them to switch between a closed and an open state
by a change in their conformation.

Because an open ion channel does not need to undergo conformational
changes with each ion it passes, ion channels have a large advantage
over transporters with respect to their maximum rate of transport. More
than a million ions can pass through an open channel each second, which
is a rate 1000 times greater than the fastest rate of transfer known for
any transporter. On the other hand, channels cannot couple the ion flow
to an energy source to carry out active transport. The function of most
ion channels is simply to make the membrane transiently permeable to
selected inorganic ions

Thanks to active transport by pumps and other transporters, most ion
concentrations are far from equilibrium across the membrane. When a
channel opens, therefore, ions rush through it. The rush of ions amounts
to a pulse of electric charge delivered either into the cell (as ions flow
in) or out of the cell (as ions flow out). The ion flow changes the voltage
across the membrane - the membrane potential - thus altering the
electrochemical driving forces for transmembrane movements of all the
other ions. It also forces other ion channels, which are specifically sensitive
to changes in the membrane potential, to open or close in a matter of
milliseconds. The resulting flurry of electrical activity can spread rapidly
from one region of the cell membrane to another, conveying an electrical
signal.

The membrane potential is the basis of all electrical activity in cells.

\subsection{Ion Channels Randomly Snap Between Open and Closed States}

Measuring changes in electrical current is the main method used to study
ion movements and ion channels in living cells. Amazingly, electrical
recording techniques have been refined to the point where it is now possible
to detect and measure the electric current flowing through a single
channel molecule. The procedure for doing this is known as \textbf{patch-clamp}
recording, and it provides a direct and surprising picture of how individual
ion channels behave.
In patch-clamp recording, a fine glass tube is used as a microelectrode to
isolate and make electrical contact with a small area of the membrane
at the surface of the cell. The technique makes it possible
to record the activity of ion channels in all sorts of cell types.
By varying the concentrations of ions in
the medium on either side of the membrane patch, one can test which
ions will go through its resident channels. With the appropriate electronic
circuitry, the voltage across the membrane patch - that is, the membrane
potential - can also be set and held clamped at any chosen value (hence
the term “patch-clamp”). The ability to expose the membrane to different
voltages makes it possible to see how changes in membrane potential
affect the opening and closing of the channels in the membrane.

If ion channels randomly snap between open and closed conformations
even when conditions on each side of the membrane are held constant,
how can their state be regulated by conditions inside or outside the cell?

The answer is that when the appropriate conditions change, the random
behavior continues but with a greatly changed probability: if the altered
conditions tend to open the channel, for example, the channel will now
spend a much greater proportion of its time in the open conformation,
although it will not remain open continuously.

\subsection{Different Types of Stimuli Influence the Opening and Closing of Ion Channels}

Ion channels differ from one another primarily with respect to their ion selectivity - the type of ions they allow to pass; and
gating - the conditions that influence their opening and closing. For a
\textbf{voltage-gated channel}, the probability of being open is controlled by
the membrane potential. For a \textbf{ligand-gated channel}, it
is controlled by the binding of some molecule (the ligand) to the channel.
For a \textbf{stress-gated channel}, opening is controlled
by a mechanical force applied to the channel.

\subsection{Voltage-gated Ion Channels Respond to the Membrane Potential}

To appreciate the function of voltage-gated ion channels in a living cell,
we have to consider what controls the membrane potential. The simple
answer is that ion channels themselves control it, and the opening and
closing of these channels is what makes it change. This control loop, from
ion channels $\rightarrow$ membrane potential $\rightarrow$ ion channels, is fundamental to
all electrical signaling in cells. Having seen how the membrane potential
can regulate ion channels, we now discuss how ion channels can control
the membrane potential.

\subsection{Membrane Potential Is Governed by Membrane Permeability to Specific Ions}

All cells have an electrical potential difference, or membrane potential,
across their plasma membrane. electricity in aqueous solutions is carried
by ions, which are either positively (cations) or negatively (anions)
charged. An ion flow across a cell membrane is detectable as an electric
current, and an accumulation of ions, if not exactly balanced by an accumulation
of oppositely charged ions, is detectable as an accumulation of
electric charge, or a membrane potential.

The resting membrane potential is the membrane potential in such steady-state
conditions, in which the flow of positive and negative ions across
the plasma membrane is precisely balanced, so that no further difference
in charge accumulates across the membrane. The membrane potential is
measured as a voltage difference across the membrane. In animal cells,
the \textbf{resting membrane potential} varies between -20 and -200 millivolts
(mV).

It is expressed as a negative value because the interior of the cell is negative with respect to the
exterior, as the negative charges inside the cell are in slight excess over
positive charges. The actual value of the resting membrane potential in
animal cells is chiefly a reflection of the $K^{+}$ concentration gradient across
the plasma membrane, because, at rest, this membrane is chiefly permeable
to $K^{+}$ , and $K^{+}$ is the main positive ion inside the cell. A simple
formula called the \textbf{Nernst equation} expresses the equilibrium quantitatively
and makes it possible to calculate the theoretical
resting membrane potential if the ratio of internal to external ion concentrations is known.

\begin{equation}
V = 62 \cdot log_{10} \frac{C_{o}}{C_{i}}
\end{equation}

where $V$ is the membrane potential in millivolts, and $C_o$ and $C_i$ are the outside and inside concentration
of the ion.

\section{Ion channels and signaling in nerve cells}

The fundamental task of a nerve cell, or \textbf{neuron}, is to receive, conduct,
and transmit signals. Neurons carry signals inward from sense organs
to the central nervous system - the brain and spinal cord.
From the central nervous system, neurons extend processes outward to
convey signals for action to muscles and glands. To perform these functions,
neurons are often extremely elongated.

Every neuron consists of a cell body (containing the nucleus) that has a
number of long, thin extensions radiating outward from it. Usually, a neuron
has one long \textbf{axon}, which conducts signals away from the cell body
toward distant target cells; it also usually has several shorter, branching
\textbf{dendrites}, which extend from the cell body like antennae and provide an
enlarged surface area to receive signals from the axons of other neurons.
The axon commonly divides at its far end into many
branches, each of which ends in a nerve terminal, so that the neuron’s
message can be passed simultaneously to many target cells.

\subsection{Action Potentials Provide for Rapid Long-Distance Communication}

A neuron is stimulated by a signal - typically from another neuron-delivered
to a localized site on its surface. This signal initiates a change in
the membrane potential at that site. To transmit the signal onward, however,
the change in membrane potential has to spread from this point,
which is usually on a dendrite or the cell body, to the axon terminals,
which relay the signal to the next cells in the pathway. Although a local
change in membrane potential will spread passively along an axon or a
dendrite to adjacent regions of the plasma membrane, it rapidly becomes
weaker with increasing distance from the source.

Neurons solve this long-distance communication problem by employing
an active signaling mechanism: a local electrical stimulus of sufficient
strength triggers an explosion of electrical activity in the plasma membrane
that is propagated rapidly along the membrane of the axon and
sustained by automatic renewal all along the way. This traveling wave of
electrical excitation, known as an \textbf{action potential}, or a nerve impulse,
can carry a message, without the signal weakening.

\subsection{Action Potentials Are Usually Mediated by Voltage-gated $Na^{+}$ Channels}

An action potential in a neuron is typically triggered by a sudden local
depolarization of the plasma membrane - that is, by a shift in the membrane
potential to a less negative value (that is, a shift towards zero). We
discuss later how such a depolarization is caused by the action of signal
molecules - \textbf{neurotransmitters} - released by another neuron. A stimulus
that causes a sufficiently large depolarization to pass a certain threshold
value, promptly causes \textbf{voltage-gated $Na^{+}$ channels} to open temporarily
at that site, allowing a small amount of $Na^{+}$ to enter the cell down
its electrochemical gradient. The influx of positive charge depolarizes
the membrane further (that is, it makes the membrane potential even
less negative), thereby opening more voltage-gated $Na^{+}$ channels, which
admit more $Na^{+}$ ions and cause still further depolarization. This process
continues in a self-amplifying fashion until, within about a millisecond,
the membrane potential in the local region of membrane has shifted from
its resting value of about -60 mV to about +40 mV. This
voltage is close to the membrane potential at which the electrochemical
driving force for movement of $Na^{+}$ across the membrane is zero - that
is, at which the effects of the membrane potential and the concentration
gradient for $Na^{+}$ are equal and opposite and $Na^{+}$ has no further tendency
to enter or leave the cell. If the channels continued to respond indefinitely
in this way to the altered membrane potential, the cell would get stuck
at this point with all of its voltage-gated $Na^{+}$ channels predominantly
open.

The cell is saved from this fate because the $Na^{+}$ channels have an automatic
inactivating mechanism, which causes them to rapidly adopt
(within a millisecond or so) a special inactive conformation, where the
channel is unable to open again. Even though the membrane is still depolarized,
the $Na^{+}$ channels will remain in this inactivated state until a few
milliseconds after the membrane potential returns to its initial negative
value. There are three distinct states of the voltage-gated
$Na^{+}$ channel: closed, open, and inactivated.

The membrane is further helped to return to its resting value by the opening
of voltage-gated $K^{+}$ channels. These also open in response to depolarization
of the membrane.

\subsection{Voltage-gated $Ca^{2+}$ Channels Convert Electrical Signals into Chemical Signals at Nerve Terminals}

When an action potential reaches the nerve terminals at the end of an
axon, the signal must somehow be relayed to the target cells that the
nerve terminals contact: usually neurons or muscle cells. The signal is
transmitted to the target cells at specialized junctions known as synapses.
At most synapses, the plasma membranes of the transmitting and
receiving cells - the presynaptic and the postsynaptic cells, respectively -
are separated from each other by a narrow synaptic cleft (typically 20 nm
across), which the electrical signal cannot cross. For the
message to be transmitted from one neuron to another, the electrical
signal is converted into a chemical signal, in the form of a small signal
molecule known as a neurotransmitter.

Neurotransmitters are stored ready-made in the nerve terminals, packaged
in membrane-enclosed synaptic vesicles.
When the action potential reaches the terminal, the neurotransmitters
are released from the nerve ending by exocytosis. This link between the
action potential and secretion involves the
activation of yet another type of voltage-gated cation channel. The
depolarization of the nerve-terminal plasma membrane caused by the
arrival of the action potential transiently opens voltage-gated $Ca^{2+}$ channels,
which are concentrated in the plasma membrane of the presynaptic nerve terminal.

\subsection{Transmitter-gated Channels in Target Cells Convert Chemical Signals Back into Electrical Signals}

The released neurotransmitter rapidly diffuses across the synaptic cleft
and binds to neurotransmitter receptors concentrated in the postsynaptic
membrane of the target cell. The binding of neurotransmitter to its
receptors causes a change in the membrane potential of the target cell,
which can trigger the cell to fire an action potential. The neurotransmitter
is then quickly removed from the synaptic cleft - either by enzymes that
destroy it, or by reuptake into the nerve terminals that released it or into
neighboring cells. This rapid removal of the neurotransmitter limits the
signal and ensures that, when the presynaptic cell falls quiet, the post-synaptic
cell will fall quiet as well.

Neurotransmitter receptors can be of various types; some mediate relatively
slow effects in the target cell, whereas others trigger more rapid
responses. Rapid responses - on a time scale of milliseconds - depend
on receptors that are transmitter-gated ion channels (also called ion-channel-coupled
receptors). These constitute a subclass of ligand-gated ion
channels, and their function is to convert the chemical signal carried by a
neurotransmitter back into an electrical signal.

\subsection{Neurons Receive Both Excitatory and Inhibitory Inputs}

The response produced by a neurotransmitter at a synapse can be either
\textbf{excitatory} or \textbf{inhibitory}. Excitatory neurotransmitters (delivered by axon
terminals of excitatory neurons) stimulate the postsynaptic cell, encouraging
it to fire an action potential. Inhibitory neurotransmitters (delivered
by axon terminals of inhibitory neurons) do the opposite, discouraging
the postsynaptic cell from firing.

Excitatory and inhibitory neurotransmitters bind to different receptors,
and it is the character of the receptor that makes the difference between
excitation and inhibition. The chief receptors for excitatory neurotransmitters,
mainly acetylcholine and glutamate, are ligand-gated cation
channels.

The receptors for inhibitory neurotransmitters, mainly $\gamma$-aminobutyric acid (GABA)
and glycine, by contrast, are ligand-gated $Cl^{-}$ channels.

\subsection{Transmitter-gated Ion Channels Are Major Targets for Psychoactive Drugs}

Most drugs used in the treatment of insomnia, anxiety, depression, and
schizophrenia exert their effects at synapses in the brain, and many of
them act by binding to transmitter-gated ion channels. The barbiturates
and tranquilizers such as Valium, Ambien, and temazepam, for example,
bind to GABA-gated $Cl^{-}$ channels. Their binding makes the channels easier
to open by GABA, thus making the cell more sensitive to GABA’s inhibitory
action. By contrast, the antidepressant Prozac blocks the reuptake
of an excitatory neurotransmitter, serotonin, increasing the amount of
serotonin available at those synapses that use this transmitter. Why this
should relieve depression is still a mystery.

The number of distinct types of neurotransmitter receptors is very large,
although they fall into a small number of families. With such a large variety of receptors, it
may be possible to design a new generation of psychoactive drugs that
will act more selectively on specific sets of neurons to alleviate the mental
illnesses that devastate so many people’s lives.

\subsection{Synaptic Connections Enable you to Think, Act, and Remember}

Each of the hundreds of types of neurons in your brain has its own characteristic set
of receptors and ion channels that enables the cell to respond in a particular
way to a certain set of inputs and thus to perform its specialized
task. Furthermore, the ion channels and other components at a synapse
can also undergo lasting modifications according to the usage they have
experienced, thereby preserving traces of past events. In this way, memories
are stored. Ion channels, therefore, are at the heart of the machinery
that enables you to act, think, feel, speak, and - perhaps most important
of all - to remember.

\section{Essential concepts}

\begin{itemize}
\item The lipid bilayer of cell membranes is permeable to small nonpolar
molecules such as oxygen and carbon dioxide and to very small
polar molecules such as water. It is highly impermeable to most
large, water-soluble molecules and all ions. Transfer of nutrients,
metabolites, and ions across the plasma membrane and internal cell
membranes is carried out by membrane transport proteins.
\item Cell membranes contain a variety of transport proteins, each of which
is responsible for transferring a particular type of solute across the
membrane. There are two classes of membrane transport proteins:
transporters and channels.
\item The electrochemical gradient represents the net driving force on an
ion due to its concentration gradient and the electric field.
\item In passive transport, an uncharged solute moves spontaneously
down its concentration gradient, a charged solute (an ion) moves
spontaneously down its electrochemical gradient, and water moves
down its osmotic gradient. In active transport, an uncharged solute
or an ion is transported against its concentration or electrochemical
gradient in a process that requires energy.
\item Transporters bind specific solutes (inorganic ions, small organic
molecules, or both) and transfer them across the lipid bilayer by
undergoing conformational changes that expose the solute-binding
site first on one side of the membrane and then on the other.
\item Transporters can act as pumps to move a solute uphill against its
electrochemical gradient, using energy provided by ATP hydrolysis,
by a downhill flow of $Na^{+}$ or $H^{+}$ ions, or by light.
\item The $Na^{+}$-$K^{+}$ pump in the plasma membrane of animal cells is an ATPase
that actively transports $Na^{+}$ out of the cell and $K^{+}$ in, maintaining the
steep $Na^{+}$ gradient across the plasma membrane that is used to drive
other active transport processes and to convey electrical signals.
\item Channels form aqueous pores across the lipid bilayer through which
solutes can diffuse. Whereas solute transfer carried out by transporters
can be active or passive, transport by channels is always
passive.
\item Most channels are selective ion channels, which allow inorganic ions
of appropriate size and charge to cross the membrane down their
electrochemical gradients. Transport through ion channels is at least
1000 times faster than movement through any known transporter.
Other channels conduct water or other small metabolites.
\item Most ion channels are gated; they open transiently in response to a
specific stimulus, such as a change in membrane potential (voltage-gated
channels) or the binding of a ligand (ligand-gated channels).
\item Even when opened by their specific stimulus, ion channels do not
remain continuously open: they flicker randomly between open and
closed conformations. An activating stimulus increases the proportion
of time that the channel spends in the open state.
\item The membrane potential is determined by the unequal distribution
of electric charge on the two sides of the plasma membrane and is
altered when ions flow through open channels. In most animal cells,
$K^{+}$ -selective leak channels hold the resting membrane potential at
a negative value, close to the value at which the driving force for
movement of $K^{+}$ across the membrane is almost zero.
\item Neurons propagate signals in the form of action potentials, which
can travel long distances along an axon without weakening. Action
potentials are usually mediated by voltage-gated $Na^{+}$ channels that
open in response to depolarization of the plasma membrane.
\item Voltage-gated $Ca^{2+}$ channels in nerve terminals couple electrical
signals to transmitter release at synapses. Transmitter-gated ion
channels convert these chemical signals back into electrical signals
in the postsynaptic target cell.
\item Excitatory neurotransmitters open transmitter-gated channels that
are permeable to $Na^{+}$ and thereby depolarize the postsynaptic cell
membrane toward the threshold potential for firing an action potential.
Inhibitory neurotransmitters open transmitter-gated $Cl^{-}$ channels
and thereby suppress firing by keeping the postsynaptic cell membrane polarized.
\end{itemize}
