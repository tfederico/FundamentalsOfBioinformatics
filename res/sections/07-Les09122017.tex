\chapter{Data}

\section{todo}

\subsection{Use case: RNAseq}

\begin{itemize}
\item Our data is composed by all the RNA sequences.
\item We want to have a human reference genome to identify the human RNA in the database.
\item We also need a reference genomes of bacteria to retrieve the bacterias' DNA.
\item Functional database that groups proteins (DNA) sequences.
\item Network annotation
\end{itemize}

\subsection{Use case: tumor mutations}

Check the mutations if are due a tumor or the single subject.

\subsection{Reasons to need data}

Mine literature, create models, benchmark analysis/prediction method.

\section{Data in the life science}

%todo copy image

All of these molecular levels and connections may be measured, predicted and annotaded.

\subsection{Data publicly available}

Sequences (gene and genomes, proteins, structures), clusters of sequences (clusters of orthologous groups, homologous proteins, clusters of structures).

Sequences are experimental data, clusters are data derived from the experimental one.

\subsection{How to represent knowledge}

Knowledge: literature, public knowledge and observations.

Relations: taxonomic classification, metabolic pathways and protein-protein interactions.

Knowladge is human readable, relations are machine readable.

\subsubsection{Data curation}

Is a term used to indicate management activities required to maintain research 
data long-term such that it is available for reuse and preservation.

\subsection{Collection of Databases \& Integration}

EBI/EMBL: genomes, nucleotide, genome-phenome, proteins, etc.

NCBI: nucleotides collection.

NAR: nucleic acids research database

\subsection{Gene Ontology (GO)}

Collecting genes using an ontology (function).

The aim of GO is describing gene products in terms of their associated: 
biological processes, cellular components and molecular function. These
three structured ontologies are a species-indipendent manner to describe 
genes.

\subsubsection{What is an ontology?}

It is a formal naming and definition of types, properties, and interrelationships
of the entities.
An ontology compartmentalizes the variables needed for some set of computations
and establishes the relationships between them.

\subsubsection{Applying GO}

Evidence code: each annotation must include an evidence code that indicates how...

\paragraph{Evidence Codes Ontology}

todo

\subsection{Provenance in Bioinformatics}

It is the chrnology of the ownership, custody or location of historical object.
In data science it is a historical record of data and its origins.

\subsection{Pfam database}

It is a large collection of protein families.
%todo continue

Database that classifies proteins based on sequence. Protein sequences are split 
into domains to be classified. HMMs are used to annotate new sequences. The seed
allignment for the HMMs are curated.

\subsubsection{Clans in Pfam}

Families that have significant sequence similarity (are homologous), but cannot 
be put together in a single HMM profile, are put in separate families, which form 
a clan.

\subsubsection{Pfam A vs Pfam B}

todo

\subsection{Structural Classification of Proteins (SCOP)}

The aim of this database is to provide a detailed...%todo

\subsubsection{Structural Classification}

Provides solutions for two major challenges:
\begin{itemize}
\item groups similar protein structures together...
\item todo
\end{itemize}

\subsection{STRING}

todo