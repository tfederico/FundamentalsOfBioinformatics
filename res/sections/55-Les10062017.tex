\chapter{How cells obtain energy from food}

In this chapter we trace the major steps in the breakdown - or \textit{catabolism} -
of sugars and show how this oxidation produces ATP, NADH, and
other activated carrier molecules in cells. We concentrate on the break-
down of glucose because these reactions dominate energy production in
most animal cells. A very similar pathway operates in plants, fungi, and
many bacteria. Other molecules, such as fatty acids and proteins, can
also serve as energy sources if they are funneled through appropriate
enzymatic pathways. We also see how many of the molecules generated
from the breakdown of sugars and fats can be used to build the macro-
molecules in cells.

\section{The breakdown and utilization of sugars and fats}

Living cells use enzymes to carry out the oxidation of sugars in
a tightly controlled series of reactions. A
glucose molecule is degraded step by step, paying out energy in small
packets to activated carrier molecules by means of coupled reactions. In
this way, much of the energy released by oxidizing glucose is saved in
the high-energy bonds of ATP and other activated carrier molecules and
made available to do useful work for the cell.


Animal cells make ATP in two ways. First, certain steps in a series of
enzyme-catalyzed reactions are directly coupled to the energetically
unfavorable reaction $ADP + P_{i} \rightarrow ATP$. The oxidation of food molecules
provides the energy that allows this unfavorable reaction to proceed. Most
ATP synthesis, however, takes place in mitochondria and uses the energy
from activated carrier molecules to drive ATP production; this process
involves the mitochondrial membrane.

\subsection{Food Molecules are Broken down in Three Stages}

The proteins, lipids, and polysaccharides that make up most of the food
we eat must be broken down into smaller molecules before our cells can
use them - either as a source of energy or as building blocks for other
molecules. This breakdown process - which uses enzymes to degrade
complex molecules into simpler ones - is dubbed \textbf{catabolism}. Catabolism
must act on food taken in from outside, but not on the macromolecules
inside our own cells. Therefore stage 1 in the enzymatic breakdown of
food molecules - \textit{digestion} - occurs either outside cells (in our intestine) or
in a specialized organelle within cells called the lysosome. A membrane
that surrounds the lysosome keeps its digestive enzymes separated from
the cytosol.

Digestive enzymes reduce the large polymeric molecules in food into
their monomeric subunits. After digestion, the
small organic molecules derived from food enter the cytosol of a cell,
where their gradual oxidation begins. This
oxidation occurs in two further stages: stage 2 starts in the cytosol and
ends in mitochondria, while stage 3 is confined to the mitochondria.

In stage 2 of cellular catabolism, a chain of reactions called \textit{glycolysis}
converts each molecule of glucose into two smaller molecules of \textit{pyruvate}.
During the formation of pyruvate, two types of activated carrier molecules are produced:
ATP and NADH. The pyruvate is then transported from the cytosol into
the mitochondrion’s large, internal compartment or \textit{matrix}. There a giant
enzyme complex converts each pyruvate molecule into $CO_{2}$ plus \textbf{acetyl
CoA}, another of the activated carrier molecule. Large amounts of acetyl CoA are also produced by the
stepwise breakdown and oxidation of fatty acids derived from fats.

Stage 3 of the oxidative breakdown of food molecules takes place entirely
in mitochondria. The acetyl group in acetyl CoA is transferred to a molecule
called oxaloacetate to form citrate, which enters a series of reactions
called the \textit{citric acid cycle}. The transferred acetyl group is oxidized to $CO_2$
in these reactions, and large amounts of the high-energy electron carrier
NADH are generated. Finally, the high-energy electrons from NADH are
passed along a series of enzymes within the mitochondrial inner membrane
called an \textit{electron-transport chain}, where the energy released by
their transfer is used to drive a process that produces ATP and consumes
molecular oxygen ($O_2$ gas). It is in these final steps that most of the energy
released by oxidation is harnessed to produce most of the cell’s ATP.

Through the production of ATP, the energy derived from the breakdown
of sugars and fats is redistributed as packets of chemical energy in a
form convenient for use in the cell.

In total, nearly half of the energy that could, in theory, be derived from
the oxidation of glucose or fatty acids to $H_{2}O$ and $CO_2$ is captured and
used to drive the energetically unfavorable reaction $P_{i} + ADP \rightarrow ATP$.
The remaining energy is released as heat, which in animals
helps to make our bodies warm.

\subsection{Glycolysis is a Central ATP-producing Pathway}

The central process in stage 2 of the breakdown of food molecules is
the degradation of \textit{glucose} in the sequence of reactions known as \textbf{glycolysis}.
Glycolysis produces ATP without the involvement of $O_2$. It occurs in the cytosol of
most cells, including many anaerobic microorganisms.

During glycolysis, a glucose molecule, which has six carbon atoms, is
cleaved into two molecules of \textbf{pyruvate}, each of which contains three
carbon atoms. For each molecule of glucose, two molecules of ATP are
consumed to provide energy to drive the early steps, but four molecules
of ATP are produced in the later steps. Thus, at the end of glycolysis, there
is a net gain of two molecules of ATP for each glucose molecule broken
down.

Glycolysis involves a sequence of 10 separate reactions, each producing
a different sugar intermediate and each catalyzed by a different enzyme.
Like most enzymes, the enzymes that catalyze glycolysis all have names ending in \textit{-ase} - like
isomerase and dehydrogenase - which specify the type of reaction they
catalyze.

Although no molecular oxygen is involved in glycolysis, oxidation occurs:
electrons are removed from some of the carbons derived from glucose
by $NAD^{+}$, producing $NADH$. The stepwise nature of the process allows
the energy of oxidation to be released in small packets, so that much of
rather it can be stored in carrier molecules than all of it being released
as heat. Some of the energy released by this oxidation
drives the synthesis of ATP molecules from ADP and $P_{i}$. The synthesis
of ATP in glycolysis is known as \textit{substrate-level phosphorylation} because
it occurs by the transfer of a phosphate group directly from a substrate
molecule - a sugar intermediate - to ADP. The remainder of the energy
harnessed during glycolysis is stored in the electrons in NADH.

Two molecules of NADH are formed per molecule of glucose in the
course of glycolysis. In aerobic organisms these NADH molecules donate
their electrons to the electron-transport chain.
The electrons are passed along this chain to $O_2$, forming
water, and the $NAD^{+}$ formed from the NADH is used again for glycolysis.

\subsection{Fermentations allow ATP to Be Produced in the absence of oxygen}

For many anaerobic microorganisms,
which do not use $O_2$ and can grow and divide in its absence, glycolysis
is the principal source of ATP. In these anaerobic conditions, the pyruvate and the NADH stay in the
cytosol. The pyruvate is converted into products that are excreted from
the cell: for example, lactate in muscle or ethanol and $CO_2$ in the yeasts
used in brewing and breadmaking. In the process, the NADH gives up its
electrons and is converted back into $NAD^+$. This regeneration of $NAD^+$ is
required to maintain the reactions of glycolysis. Anaerobic
energy-yielding pathways like these are called \textbf{fermentations}.

\subsection{Glycolysis illustrates how enzymes Couple oxidation to energy Storage}

These reactions - steps 6 and 7 - convert the three-carbon sugar
intermediate glyceraldehyde 3-phosphate (an aldehyde) into
3-phosphoglycerate (a carboxylic acid). This conversion entails the oxidation
of an aldehyde group to a carboxylic acid group, which occurs in
two steps. The overall reaction releases enough free energy to convert a
molecule of ADP to ATP and to transfer two electrons from the aldehyde
to $NAD^+$ to form NADH, while still releasing enough heat to the environment
to make the overall reaction energetically favorable.

The chemical reactions are precisely guided by two enzymes to which the
sugar intermediates are tightly bound. In fact,
the first enzyme (glyceraldehyde 3-phosphate dehydrogenase) forms a
short-lived covalent bond to the aldehyde through a reactive -SH group
on the enzyme, and catalyzes its oxidation in this attached state. The
reactive enzyme - substrate bond is then displaced by an inorganic phosphate
ion to produce a high-energy phosphate intermediate, which is
released from the enzyme. The intermediate, 1,3-bisphosphoglycerate,
binds to the second enzyme (phosphoglycerate kinase). This enzyme
then catalyzes the energetically favorable transfer of the intermediate’s
high-energy phosphate to ADP, forming ATP and completing the process
of oxidizing an aldehyde to a carboxylic acid.

These reactions (steps 6 and 7) are the only ones in glycolysis that create a high-energy phosphate
linkage directly from inorganic phosphate. As such, they account for the
net yield of two ATP molecules and two NADH molecules per molecule
of glucose. As we have mentioned, this NADH must be reoxidized to the
$NAD^+$ required for these coupled reactions. If $NAD^+$ is not available, gly-
colysis will stop

\subsection{Sugars and fats are Both degraded to acetyl Coa in Mitochondria}

In aerobic metabolism in eucaryotic cells, the pyruvate produced by
glycolysis is actively pumped into the mitochondrial matrix, the major
internal compartment of this organelle. There it is
rapidly decarboxylated by a giant complex of three enzymes, called the
\textit{pyruvate dehydrogenase complex}. The products of pyruvate decarboxylation
are a molecule of $CO_2$ (a waste product), a molecule of NADH, and a
molecule of acetyl CoA.

Fatty acids, derived from \textit{fat}, are an alternative fuel to sugars for energy
generation. Like the pyruvate derived from glycolysis, fatty acids are converted
into acetyl CoA in mitochondria. Each long molecule of fatty acid
(in the form of the activated molecule, fatty acyl CoA) is broken down
completely by a cycle of reactions that trims two carbons at a time from
its carboxyl end, generating one molecule of acetyl CoA for each turn of
the cycle. A molecule of NADH and a molecule of another electron carrier
$FADH_2$, are also produced in this process.

Sugars and fats provide the major energy sources for most nonphotosynthetic
organisms, including humans. In the course of their processing to
acetyl CoA, only a small part of the useful energy stored in these foodstuffs
is extracted and converted into ATP or NADH. Most of the energy is
still locked up in acetyl CoA. The next stage in respiration, in which the
acetyl group in acetyl CoA is oxidized to $CO_2$ and $H_{2}O$ in the citric acid
cycle, is therefore central to the energy metabolism of aerobic organisms.
In eucaryotes the citric acid cycle takes place in mitochondria, the
organelles to which pyruvate and fatty acids are directed for acetyl CoA
production.

In addition to pyruvate and fatty acids, some amino acids are transported
from the cytosol into mitochondria, where they are also converted into
acetyl CoA or one of the other intermediates of the citric acid cycle.
In aerobic bacteria, which have no mitochondria, all of these reactions
- glycolysis, acetyl CoA production, and the citric acid cycle - take
place in the single compartment of the cytosol.

\subsection{The Citric acid Cycle generates nadh by oxidizing acetyl groups to $CO_2$}

The third and final stage in the oxidative breakdown of food molecules
to generate energy requires abundant $O_2$.

In the nineteenth century, biologists noticed that in the absence of air
(anaerobic conditions) cells produce lactic acid (for example, in muscle)
or ethanol (for example, in yeast), while in the presence of air (aerobic
conditions) these cells consume $O_2$ and produce $CO_2$ and $H_{2}O$. Intensive
efforts to define the pathways of aerobic metabolism eventually focused
on the oxidation of pyruvate and led in 1937 to the discovery of the \textbf{citric
acid cycle}, also known as the \textit{tricarboxylic acid cycle} or the
\textbf{Krebs cycle}.

The citric acid cycle accounts for about
two-thirds of the total oxidation of carbon compounds in most cells, and
its major end products are $CO_2$ and high-energy electrons in the form of
NADH. The $CO_2$ is released as a waste product, while the high-energy
electrons from NADH are passed to a series of membrane-bound enzymes
known collectively as the \textit{electron-transport chain}. At the end of the chain,
these electrons combine with $O_2$ to produce $H_{2}O$. The citric acid cycle
itself does not use $O_2$. However, it requires $O_2$ to proceed because the
electron-transport chain allows NADH to get rid of its electrons and thus
regenerate the $NAD^+$ that is needed to keep the cycle going.

The citric acid cycle, which takes place in the mitochondrial matrix, catalyzes
the complete oxidation of the carbon atoms of the acetyl groups in
acetyl CoA, converting them into $CO_2$. The acetyl group is not oxidized
directly, however. Instead, it is transferred from acetyl CoA to a larger
four-carbon molecule, \textit{oxaloacetate}, to form the six-carbon tricarboxylic
acid \textit{citric acid}, for which the subsequent cycle of reactions is named.
The citric acid molecule is then gradually oxidized, and the energy of this
oxidation is harnessed to produce energy-rich carrier molecules, in much
the same manner as we described for glycolysis. The chain of eight reactions
forms a cycle, because the oxaloacetate that began the process is
regenerated at the end.

We have so far discussed only one of the three types of activated carrier
molecules that are produced by the citric acid cycle - NADH. In addition
to three molecules of NADH, each turn of the cycle also produces one
molecule of $FADH_2$ (reduced flavin adenine dinucleotide) from FAD and
one molecule of the ribonucleotide GTP (guanosine triphosphate) from
GDP. GTP is a close relative of ATP, and
the transfer of its terminal phosphate group to ADP produces one ATP
molecule in each cycle. Like NADH, $FADH_2$ is a carrier of high-energy
electrons and hydrogen. As we discuss shortly, the energy that is stored
in the readily transferred high-energy electrons of NADH and $FADH_2$ are
subsequently used to produce ATP through the process of \textit{oxidative phosphorylation},
which occurs in the mitochondrial membrane. Oxidative
phosphorylation is the only step in the oxidative catabolism of foodstuffs
that directly requires $O_2$ from the atmosphere.

\subsection{Many Biosynthetic Pathways Begin with glycolysis or the Citric acid Cycle}

So far we have emphasized
energy production rather than the provision of starting materials for biosynthesis.
But many of the intermediates formed in glycolysis and the
citric acid cycle are siphoned off by biosynthetic, or \textit{anabolic}, pathways,
where they are converted by series of enzyme-catalyzed reactions into
amino acids, nucleotides, lipids, and other small organic molecules that
the cell needs.

\subsection{Electron Transport drives the Synthesis of the Majority of the ATP in Most Cells}

We now return to the last stage in the oxidation of a food molecule -
the stage in which most of its chemical energy is released. In this final
process, the electron carriers NADH and $FADH_2$ transfer the electrons
they have gained by oxidizing other molecules to the \textbf{electron-transport chain}.
This specialized chain of electron carriers is embedded in
the inner membrane of the mitochondrion in eucaryotic cells (in the
plasma membrane of bacteria). As the electrons pass through the series
of electron acceptor and donor molecules that form the chain, they fall to
successively lower energy states. The energy released is used to drive $H^+$
ions (protons) across the membrane, from the inner mitochondrial compartment to the outside.
This generates a transmembrane gradient of $H^+$
ions that serves as a source of energy (like a battery) that can be tapped
to drive a variety of energy-requiring reactions. In mitochondria,
the most prominent of these reactions is the phosphorylation
of ADP to generate ATP.

At the end of the transport chain, the electrons are added to molecules
of $O_2$ that have diffused into the mitochondrion; the resulting reduced
$O_2$ molecules simultaneously combine with protons ($H^+$) from the surrounding
solution to produce water. The electrons have now reached
their lowest energy level, and all the available energy has been extracted
from the food molecule being oxidized. The oxygen-requiring generation
of ATP is termed \textbf{oxidative phosphorylation}.

In total, the complete oxidation of a molecule of glucose to $H_{2}O$ and $CO_2$
produces about 30 molecules of ATP. In contrast, only two molecules
of ATP are produced per molecule of glucose by glycolysis alone.

\section{Regulation of Metabolism}

Many sets of reactions need to be carefully controlled. For example, to
maintain order within their cells, all organisms need to constantly replenish
their ATP pools through sugar or fat oxidation.
Yet animals have only periodic access to food, and plants need to survive
overnight without sunlight, when they are unable to produce sugar
through photosynthesis. Plants and animals have evolved several ways
to get around this problem. One is to synthesize food reserves in times of
plenty that can be later consumed when other energy sources are scarce.
Thus, a cell must control whether key metabolites will be routed into
anabolic or catabolic pathways - in other words, whether they will be
commissioned to build other molecules or burned to provide immediate
energy.

\subsection{Catabolic and anabolic reactions are organized and regulated}

the metabolic balance of a cell is amazingly stable. Whenever the balance is perturbed,
the cell reacts so as to restore the initial state: cells
can adapt and continue to function during starvation or disease. This
resilience is made possible by an elaborate network of \textit{control mechanisms}
that act on enzymes to regulate and coordinate the rates of the
many metabolic reactions in a cell.

\subsection{Feedback regulation allows Cells to Switch from glucose degradation to glucose Biosynthesis}

The body needs a continuous supply of glucose to meet its metabolic
needs. One way to replenish blood glucose is to synthesize it from
small non-carbohydrate organic molecules such as lactate, pyruvate, or
amino acids in a process called \textbf{gluconeogenesis}. An intricate pattern of
feedback regulation enables cells to switch from breaking down glucose
through glycolysis to synthesizing it through gluconeogenesis.

Most of the reactions involved in the breakdown of glucose to pyruvate
are readily reversible. However, three of the reactions - steps 1, 3, and 10
are effectively irreversible. In fact, it is the large negative
free-energy change that occurs in these reactions that normally drives the
breakdown of glucose. For the pathway to go in the opposite direction - to
make glucose from pyruvate - these three reactions must be bypassed.
This detour is achieved by substituting a set of alternative, enzyme-catalyzed
“bypass reactions” that require an input of chemical energy.
The reactions that synthesize 	a molecule of glucose in gluconeogenesis thus require the hydrolysis of
four ATP and two GTP molecules, compared with the overall generation
of two molecules of ATP for each molecule of glucose consumed during
glycolysis.

One of the key control points in the breakdown of glucose lies in step 3
of glycolysis, the production of fructose 1,6-bisphosphate by the enzyme
\textit{phosphofructokinase}. This is one of the reactions that must be bypassed
in gluconeogenesis. Phosphofructokinase is allosterically activated by AMP, ADP, and
inorganic phosphate - the byproducts of ATP hydrolysis; it is allosterically
inhibited by ATP, citrate, and alternative fuels for respiration, such as
fatty acids, which can be liberated from stored fat when glucose is not
available. Thus, when energy reserves are low and the products of ATP
hydrolysis accumulate, phosphofructokinase is activated and glycolysis
proceeds. On the other hand, when ATP or fuel sources - represented
by citrate and fatty acids - are abundant, phosphofructokinase is turned
off, favoring gluconeogenesis and, ultimately, the storage of food molecules.
To add an additional level of control, the enzyme that catalyzes
the reverse reaction (fructose 1, 6-bisphosphatase)
is regulated by the same molecules - but in the opposite
direction. Thus, this enzyme is activated when phosphofructokinase
is turned off.

\subsection{Cells Store food Molecules in Special reservoirs to Prepare for Periods of need}

To compensate for long periods when food
is unavailable, animals store food reserves within their cells. Glucose is
stored as the subunits of the large, branched polysaccharide \textbf{glycogen},
which is present as small granules in the cytoplasm of many cells, mainly
liver and muscle. The syn thesis and degradation of glycogen occur by quite separate metabolic
pathways, which can be rapidly and coordinately regulated according
to need. When more ATP is needed than can be generated from food
molecules taken in from the bloodstream, cells break down glycogen in
a reaction that produces \textit{glucose 1-phosphate}, which is then converted to
the glucose 6-phosphate that feeds into the glycolytic pathway.

The glycogen synthetic and degradative pathways are coordinated by
enzymes in each pathway that are allosterically regulated by glucose
6-phosphate, but in opposite directions: \textit{glycogen synthase} in the synthetic
pathway is activated by glucose 6-phosphate, whereas the \textit{glycogen
phosphorylase} that catalyzes the breakdown of glycogen is inhibited by
both glucose 6-phosphate and ATP. This regulation helps to prevent the
breakdown of glycogen when ATP is plentiful and favors its synthesis
when glucose 6-phosphate concentration is high. The balance between
glycogen synthesis and breakdown is also regulated by intracellular signaling
pathways that are controlled by the hormones insulin, adrenaline,
and glucagon.

Quantitatively, fat is a far more important storage material than glycogen,
in part because the oxidation of a gram of fat releases about twice as
much energy as the oxidation of a gram of glycogen. Moreover, glycogen
binds a great deal of water, producing a sixfold difference in the actual
mass of glycogen required to store the same amount of energy as fat.
An average adult human stores enough glycogen for only about a day
of normal activity, but enough fat to last nearly a month.

Most of our fat is stored as droplets of water-insoluble triacylglycerols in
specialized adipose tissue.
In response to hormonal signals, fatty acids can be released from these
depots into the bloodstream for other cells to use as required. Such a
need arises after a period of not eating; even a normal overnight fast
results in the mobilization of fat. In the morning, most of the acetyl CoA
that enters the citric acid cycle is derived from fatty acids rather than from
glucose. After a meal, however, most of the acetyl CoA entering the citric
acid cycle comes from glucose derived from food, and any excess glucose
is used to replenish depleted glycogen stores or to synthesize fats.

The food reserves in both animals and plants form a vital part of the
human diet. Plants convert some of the sugars that they make through
photosynthesis during daylight into fats and into \textit{starch}, a branched polymer
of glucose very similar to the glycogen of animals. The fats in plants
are triacylglycerols, just like the fats in animals, and they differ only in the
types of fatty acids that predominate.

\section{Essential concepts}

\begin{itemize}
\item Glucose and other food molecules are broken down by controlled
stepwise oxidation to provide useful chemical energy in the form of
the activated carriers ATP and NADH.
\item Sugars derived from food are broken down by distinct sets of reactions:
glycolysis (which occurs in the cytosol), the citric acid cycle
(in the mitochondrial matrix), and oxidative phosphorylation (in the
inner mitochondrial membrane).
\item The reactions of glycolysis degrade the six-carbon sugar glucose to
two molecules of the three-carbon sugar pyruvate, producing a relatively
small amount of ATP and NADH.
\item In the presence of oxygen, pyruvate is converted to acetyl CoA plus
$CO_2$. The citric acid cycle then converts the acetyl group in acetyl
CoA to $CO_2$ and $H_{2}O$. Much of the energy released in these oxidation
reactions is stored as high-energy electrons in the activated carriers
NADH and $FADH_2$. In eucaryotic cells, all these reactions occur in
mitochondria.
\item The other major energy source in foods is fat. The fatty acids produced
from the digestion of fats are imported into mitochondria and
converted to acetyl CoA molecules. These acetyl CoA molecules are
then further oxidized through the citric acid cycle, producing NADH
and $FADH_2$, just like the acetyl CoA derived from pyruvate.
\item NADH and $FADH_2$ pass their high-energy electrons to an electron-transport
chain in the inner mitochondrial membrane, where a series
of electron transfers is used to drive the formation of ATP. Most of
the energy captured during the breakdown of food molecules is harvested
during this process of oxidative phosphorylation (described in
detail in Chapter 14).
\item The food we eat is not only a source of metabolic energy but also of
raw materials for biosynthesis. Many intermediates of glycolysis and
the citric acid cycle are starting points for pathways that lead to the
synthesis of proteins, nucleic acids, and the many other specialized
molecules of the cell.
\item The thousands of different reactions carried out simultaneously by a
cell are closely coordinated, enabling the cell to adapt and continue
to function under a wide range of external conditions.
\item During periods when food is scarce, regulation of the activities of a
few key enzymes allows the cell to switch from glucose breakdown
to glucose biosynthesis (gluconeogenesis).
\item Cells store food molecules in special reserves. Glucose subunits are
stored as glycogen in animals and as starch in plants; both animals
and plants store fatty acids as fats. The food reserves stored by plants
are major sources of food for animals, including humans.
\end{itemize}

\chapter{Energy generation in Mithocondria and Chloroplast}

The main chemical energy currency in cells is ATP. In
eucaryotic cells, small amounts of ATP are generated during glycolysis
in the cytosol, but most ATP is produced by oxidative phosphorylation
in mitochondria. The mechanism by which
the bulk of ATP is generated in the mitochondria differs from the way
in which ATP is produced by glycolysis in that it involves a membrane:
oxidative phosphorylation depends on electron transport within the mitochondrial
membrane and the transport of ions across it. The same type of
ATP-generating process occurs in the plasma membrane of bacteria.

The membrane-based process for making ATP consists of two linked
stages; both are carried out by protein complexes in the membrane.

\begin{enumerate}
\item Electrons derived from the oxidation of food molecules or from other sources
are transferred along a series of electron carriers - called an
\textbf{electron-transport chain} - embedded in the membrane. These
electron transfers release energy that is used to pump protons
($H^{+}$), derived from the water that is ubiquitous in cells, across
the membrane and thus generate an electrochemical proton
gradient. An ion gradient across a membrane is
a form of stored energy that can be harnessed to do useful work
when the ions are allowed to flow back across the membrane
down their gradient.
\item $H^{+}$ flows back down its electrochemical gradient through a
protein complex called \textit{ATP synthase}, which catalyzes the
energy-requiring synthesis of ATP from ADP and inorganic
phosphate ($P_i$). This ubiquitous enzyme serves the role of a turbine,
permitting the proton gradient to drive the production of ATP
\end{enumerate}

The linkage of electron transport, proton pumping, and ATP synthesis
was called the \textit{chemiosmotic hypothesis} when it was first proposed in the
1960s, because of the link between the chemical bond-forming reactions
that synthesize ATP (“chemi-”) and the membrane transport processes
(“osmotic,” from the Greek osmos, “to push”). It is now known as \textit{chemi-
osmotic coupling}. Chemiosmotic mechanisms allow cells to harness the
energy of electron transfers in much the same way that the energy stored
in a battery can be harnessed to do useful work.

Chemiosmotic coupling first evolved in bacteria.

\section{Mithocondria and oxidative phosphorylation}

Mitochondria are present in nearly all eucaryotic cells - in plants, animals,
and most eucaryotic microorganisms - and most of a cell’s ATP
is produced in these organelles.

Defects in mitochondrial function can have serious repercussions for an
organism.

The same metabolic reactions that occur in mitochondria also take place
in aerobic bacteria, which do not possess these organelles; in these
organisms the plasma membrane carries out the chemiosmotic coupling.

\subsection{A Mitochondrion Contains an Outer Membrane, an Inner Membrane, and Two Internal Compartments}

Mitochondria are generally similar in size and shape to bacteria, although
these attributes can vary depending on the cell type. They contain their
own DNA and RNA, and a complete transcription and translation system
including ribosomes, which allows them to synthesize some of their
own proteins. Time-lapse movies of living cells reveal mitochondria as
remarkably mobile organelles, constantly changing shape and position.
Present in large numbers these organelles can form long, moving chains in
association with the microtubules of the cytoskeleton. In other cells, they
remain fixed in one location to target ATP directly to a site of unusually
high ATP consumption. The number of mitochondria present in different cell
types varies dramatically, and can change with the energy needs of the cell.

An individual mitochondrion is bounded by two highly specialized
membranes - one surrounding the other—that play a crucial part in its
activities. The outer and inner mitochondrial membranes create two mitochondrial
compartments: a large internal space called the matrix and the much narrower
intermembrane space.

The outer membrane contains many molecules of a transport protein called
porin, which forms wide aqueous channels through the lipid bilayer.
As a result, the outer membrane is like a sieve\footnote{setaccio}
that is permeable to all molecules of 5000 daltons or less, including small
proteins. This makes the intermembrane space chemically equivalent to
the cytosol with respect to the small molecules it contains. In contrast,
the inner membrane, like other membranes in the cell, is impermeable
to the passage of ions and most small molecules, except where a path
is provided by membrane transport proteins. The mitochondrial matrix
therefore contains only molecules that can be selectively transported
into the matrix across the inner membrane, and its contents are highly
specialized.

The inner mitochondrial membrane is the site of electron transport and
proton pumping, and it contains the ATP synthase. Most of the proteins
embedded in the inner mitochondrial membrane are components of the
electron-transport chains required for oxidative phosphorylation. This
membrane also contains a variety of transport proteins that allow the
entry of selected small molecules, such as pyruvate and fatty acids, into
the matrix.

The inner membrane is usually highly convoluted, forming a series of
infoldings, known as cristae, that project into the matrix space to greatly
increase the surface area of the inner membrane. These
folds provide a large surface on which ATP synthesis can take place.

\subsection{The Citric Acid Cycle generates High-Energy Electrons}

Mitochondria use both pyruvate and fatty acids as fuel, the pyruvate
coming mainly from glucose and other sugars, and the fatty acids from
fats. These fuel molecules are transported across the inner mitochon-
drial membrane and then converted to the crucial metabolic intermediate
acetyl CoA by enzymes located in the mitochondrial matrix.
The acetyl groups in acetyl CoA are then oxidized in the matrix
via the citric acid cycle. The cycle converts
the carbon atoms in acetyl CoA to $CO_2$, which is released from the cell as
a waste product. In addition, the cycle generates high-energy electrons,
carried by the activated carrier molecules $NADH$ and $FADH_2$.

Although the citric acid cycle is considered to be part of aerobic metabolism,
it does not itself use molecular oxygen ($O_2$). Oxygen is directly
consumed only in the final catabolic reactions that take place on the
inner mitochondrial membrane.

\subsection{A Chemiosmotic Process Converts the Energy from Activated Carrier Molecules into ATP}

Nearly all the energy available from burning carbohydrates, fats, and
other foodstuffs in the earlier stages of their oxidation is initially saved
in the form of the activated carrier molecules generated during glycoly-
sis and the citric acid cycle - $NADH$ and $FADH_2$. These carrier molecules
donate their high-energy electrons to the electron-transport chain in
the mitochondrial membrane, and thus become oxidized to $NAD^+$ and
$FAD$. The electrons are quickly passed along the chain to molecular oxygen
($O_2$) to form water ($H_{2}O$). The passage of the high-energy electrons
along the electron-transport chain releases energy that is harnessed to
pump protons across the inner mitochondrial membrane.

The resulting proton gradient in turn drives the synthesis of ATP.
The inner mitochondrial membrane thus serves as a device that converts
the energy contained in the high-energy electrons of NADH into the
high-energy phosphate bond of ATP. This chemiosmotic mechanism of ATP synthesis
is called \textbf{oxidative phosphorylation}, because it involves both the consumption
of $O_2$ and the addition of a phosphate group to ADP to form
ATP.

The source of the electrons that power the proton pumping differs widely
between different organisms and different processes. In aerobic respiration
in mitochondria and aerobic bacteria, the electrons are ultimately
derived from glucose or fatty acids. In photosynthesis, the required
electrons are derived from the action of light on the green pigment \textit{chlorophyll}.

\subsection{The Electron-Transport Chain Pumps Protons Across the Inner Mitochondrial Membrane}

The electron-transport chain—or respiratory chain—that carries out
oxidative phosphorylation is present in many copies in the inner mitochondrial
membrane. Each chain contains over 40 proteins, most of
which are embedded in the lipid bilayer and function only in the intact
membrane. Most of the proteins involved in the mitochondrial
electron-transport chain are grouped into three large \textit{respiratory enzyme
complexes}, each containing multiple individual proteins. Each complex
includes transmembrane proteins that hold the entire complex firmly in
the inner mitochondrial membrane.

The three respiratory enzyme complexes, in the order in which they
receive electrons, are:

\begin{enumerate}
\item the NADH dehydrogenase complex;
\item the cytochrome b-c 1 complex;
\item the cytochrome oxidase complex.
\end{enumerate}

Each contains metal ions and other chemical groups that form a pathway for
the pass-age of electrons through the complex. The respiratory complexes
are the sites of proton pumping, and each can be thought of as a
protein machine that pumps protons across the membrane as electrons
are transferred through it.

Electron transport begins when a hydride ion ($H^-$) is removed from NADH
and is converted into a proton and two high-energy electrons: $H^{-} \rightarrow
H^{+} + 2e^-$. This reaction is catalyzed by the first of the respiratory enzyme
complexes, the NADH dehydrogenase, which accepts electrons from NADH.
The electrons are then passed along the chain to each of the other enzyme complexes in turn,
using mobile electron carriers to ferry electrons between complexes. The
transfer of electrons along the chain is energetically favorable: the electrons
start out at very high energy and lose energy at each transfer step,
eventually entering cytochrome oxidase where they combine with a molecule
of $O_2$ to form water. This is the oxygen-requiring step of cellular
respiration, and it consumes nearly all of the oxygen that we breathe.

\subsection{Proton Pumping Creates a Steep Electrochemical Proton gradient Across the Inner Mitochondrial Membrane}

Without a mechanism for harnessing the energy released by electron
transfers, this energy would simply be liberated as heat. But cells utilize
much of the energy of electron transfer by having the transfers take
place within proteins that are capable of pumping protons. In this way,
the energetically favorable flow of electrons along the electron-transport
chain results in the pumping of protons across the membrane out of the
mitochondrial matrix and into the space between the inner and outer
mitochondrial membranes.

Later in the chapter we review the detailed molecular mechanisms that
couple electron transport to the movement of protons. For now, we focus
on the consequences of this nifty biological maneuver. First, the active
pumping of protons generates a gradient of $H^+$ concentration - a pH gradient
- across the inner mitochondrial membrane, where the pH is about
0.5 unit higher in the matrix (around pH 7.5) than in the intermembrane
space (which is close to 7, the same pH as the cytosol). Second, proton
pumping generates a membrane potential across the inner mitochondrial
membrane, with the inside (the matrix side) negative and the outside
positive as a result of the net outflow of $H^+$.

The force driving the passive flow of an ion
across a membrane is proportional to the electrochemical gradient for
the ion across the membrane. This in turn depends on the voltage across
the membrane, which is measured as the membrane potential, and on
the concentration gradient of the ion. Because protons
are positively charged, they will move more readily across a membrane if
the membrane has an excess of negative electrical charges on the other
side. In the case of the inner mitochondrial membrane, the pH gradient
and membrane potential work together to create a steep electrochemical
proton gradient that makes it energetically very favorable for $H^+$ to
flow back into the mitochondrial matrix. In the energy-producing membranes
we discuss in this chapter, the membrane potential adds to the
driving force pulling $H^+$ back across the membrane, which is called the
\textit{proton-motive force}; hence the membrane potential increases the amount
of energy stored in the proton gradient

\subsection{The Electrochemical Proton gradient Drives ATP Synthesis}

As explained previously, the electrochemical proton gradient across the
inner mitochondrial membrane is used to drive ATP synthesis. The device
that makes this possible is a large enzyme called \textbf{ATP synthase}, which
is also embedded in the inner mitochondrial membrane. ATP synthase
creates a hydrophilic pathway across the inner mitochondrial membrane
that allows protons to flow back across the membrane down their
electrochemical gradient. As these ions thread their way
through the enzyme, they are used to drive the energetically unfavorable
reaction between ADP and $P_i$ that makes ATP.

ATP synthase is a large, multisubunit protein. A large
enzymatic portion, shaped like a lollipop head, projects into the matrix
and carries out the phosphorylation reaction. This enzymatic structure is
attached through a thinner multisubunit “stalk” to a transmembrane proton
carrier. As protons pass through a narrow channel within the carrier,
their movement causes the stalk to spin rapidly within the head, inducing
the head to make ATP. The synthase essentially
acts as an energy-generating molecular motor, converting the energy of
proton flow down a gradient into the mechanical energy of two sets of
proteins rubbing against one another - rotating stalk proteins pushing
against stationary head proteins. The movement of the stalk changes the
conformation of subunits within the head. This mechanical deformation
gets converted into chemical bond energy as the subunits produce ATP.

The ATP synthase is a reversible coupling device. It can either harness
the flow of protons down their electrochemical gradient to make ATP
(its normal role in mitochondria and the plasma membrane of bacteria
growing aerobically) or use the energy of ATP hydrolysis to pump protons
across a membrane. In the latter mode, ATP synthase
functions like the $H^+$ pumps. Whether the ATP
synthase primarily makes or consumes ATP depends on the magnitude
of the electrochemical proton gradient across the membrane in which it
sits. In many bacteria that can grow either aerobically or anaerobically,
the direction in which the ATP synthase works is routinely reversed when
the bacterium runs out of $O_2$. At this point, the ATP synthase uses some of
the ATP generated inside the cell by glycolysis to pump protons out of the
cell, creating the proton gradient that the bacterial cell needs to import its
essential nutrients by coupled transport

\subsection{Coupled Transport Across the Inner Mitochondrial Membrane Is Also Driven by the Electrochemical Proton gradient}

The synthesis of ATP is not the only process driven by the electrochemical
proton gradient. In mitochondria, many charged molecules, such as
pyruvate, ADP, and $P_i$, are pumped into the matrix from the cytosol, while
others, such as ATP, must be moved in the opposite direction. Carrier
proteins that bind these molecules can couple their transport to the energetically
favorable flow of $H^+$ into the mitochondrial matrix. Pyruvate and
inorganic phosphate ($P_i$), for example, are individually co-transported
inward with $H^+$ as the latter moves down its electrochemical gradient,
into the matrix.

Other transporters take advantage of the fact that the electrochemical
proton gradient generates a membrane potential, such that the matrix
side of the inner mitochondrial membrane is more negatively charged
than the intermembrane space on the other side. An antiport carrier
protein exploits this voltage gradient to expel ATP from - and import
ADP to - the mitochondrial matrix. Because an ATP molecule has one
more negative charge than ADP, swapping these nucleotides results in
the movement of one negative charge out of the mitochondrion. This
nucleotide exchange - which sends ATP to the cytosol - is thus driven by
the charge difference across the inner mitochondrial membrane.

In eucaryotic cells, therefore, the electrochemical proton gradient is used
to drive both the formation of ATP and the transport of certain metabolites
across the inner mitochondrial membrane. In bacteria, the proton
gradient across the bacterial plasma membrane serves all of these functions.
But, in bacteria, this gradient is also an important source of directly
usable energy: in motile bacteria, a flow of protons into the cell drives
the rapid rotation of the bacterial flagellum, which propels the bacterium
along.

\subsection{Oxidative Phosphorylation Produces Most of the Cell’s ATP}

Glycolysis on its own produces a net yield of two
molecules of ATP for every molecule of glucose, whereas the complete
oxidation of glucose - which includes glycolysis and oxidative
phosphorylation - generates about 30 ATPs. In glycolysis, it is obvious where those
ATP molecules come from: two molecules of ATP are consumed early
in the process and four molecules of ATP are produced toward the end.
But for oxidative phosphorylation, the accounting is
less straightforward, because the ATPs are not produced directly, as they
are in glycolysis. Instead, they are produced from the energy carried by
NADH and $FADH_2$, which are generated during glycolysis and the citric
acid cycle. These activated carrier molecules donate their electrons to the
electron transport chain that lies in the inner mitochondrial membrane.
These movement of these electrons along the respiratory chain fuels the
formation of the proton gradient, which in turn powers the production of
ATP.

How much ATP each carrier molecule ultimately produces depends on
several factors, including where its electrons enter the respiratory chain.
The NADH molecules produced during the citric acid cycle, which takes
place inside the mitochondria, pass their electrons to NADH dehydrogenase -
the first respiratory enzyme complex in the chain. These electrons
then pass from one enzyme complex to the next, promoting the pumping
of protons across the inner mitochondrial membrane at each step along
the way. These NADH molecules provide energy for the net formation of
about 2.5 molecules of ATP

The $FADH_2$ generated during the citric acid cycle, on the other hand,
produces a net of only 1.5 molecules of ATP. This is because $FADH_2$ molecules
bypass the NADH dehydrogenase complex and pass their electrons
to the membrane-embedded mobile carrier ubiquinone.
These electrons enter further down the respiratory chain, and they therefore
promote the pumping of fewer protons and generate less ATP.

The oxidation of fatty acids also produces large amounts of NADH and
$FADH_2$, which in turn produce large amounts of ATP via oxidative
phosphorylation.

\subsection{The Rapid Conversion of ADP to ATP in Mitochondria Maintains a High ATP/ADP Ratio in Cells}

As a result of the co-transport process discussed earlier, ADP molecules
produced by ATP hydrolysis in the cytosol are rapidly drawn back into
mitochondria for recharging, and the bulk of the ATP molecules formed in
the mitochondrial matrix by oxidative phosphorylation are pumped into
the cytosol where they are needed. A small amount of ATP is used within
the mitochondrion itself to power the replication of its DNA, protein synthesis,
and other energy-consuming reactions.

Biosynthetic enzymes often drive energetically
unfavorable reactions by coupling them to the energetically favorable
hydrolysis of ATP. The ATP pool is thus used to drive
cellular processes in much the same way that a battery can drive an electric
engine. If the activity of the mitochondria were halted, ATP levels
would fall and the cell’s battery would run down; eventually, energetically
unfavorable reactions could no longer take place and the cell would die.

\section{Molecular mechanisms of electron transport and proton pumping}

\subsection{Protons Are Readily Moved by the Transfer of Electrons}

Although protons resemble other positive ions such as $Na^+$ and $K^+$ in the
way they move across membranes, in some respects they are unique.
Hydrogen atoms are by far the most abundant type of atom in living
organisms and are plentiful not only in all carbon-containing biological
molecules but also in the water molecules that surround them. The protons
in water are highly mobile, flickering through the hydrogen-bonded
network of water molecules by rapidly dissociating from one water
molecule in order to associate with its neighbor. Thus, water, which is
everywhere in cells, serves as a ready reservoir for donating and accepting
protons.

Whenever a molecule is reduced by acquiring an electron, the electron
($e^{-}$) brings with it a negative charge. In many cases, this charge is rapidly
neutralized by the addition of a proton from water, so that the net effect
of the reduction is to transfer an entire hydrogen atom, $H^{+} + e^{-}$.
Similarly, when a molecule is oxidized, the hydrogen atom can
be readily dissociated into its constituent electron and proton, allowing
the electron to be transferred separately to a molecule that accepts electrons,
while the proton is passed to the water. Therefore, in a membrane
in which electrons are being passed along an electron-transport chain,
it is a relatively simple matter, in principle, to pump protons from one
side of the membrane to another. All that is required is that the electron
carrier be arranged in the membrane in a way that causes it to pick up a
proton from one side of the membrane when it accepts an electron, while
releasing the proton on the other side of the membrane as the electron is
passed on to the next carrier molecule in the chain.

\subsection{The Redox Potential Is a Measure of Electron Affinities}

In biochemical reactions, any electrons removed from one molecule are
always passed to another, so that whenever one molecule is oxidized,
another is reduced. Like any other chemical reaction, the tendency of such
oxidation-reduction reactions, or \textbf{redox reactions}, to proceed spontane-
ously depends on the free-energy change ($\Delta G$) for the electron transfer,
which in turn depends on the relative affinities of the two molecules for
electrons.


\subsection{Metals Tightly Bound to Proteins form Versatile Electron Carriers}

Within each of the three respiratory enzyme complexes, electrons move
mainly between metal atoms that are tightly bound to the proteins,
traveling by skipping from one metal ion to another one with a greater
affinity for electrons. In contrast, electrons are carried between the different
respiratory complexes by molecules that diffuse along the lipid bilayer,
picking up electrons from one complex and delivering them to another in
an orderly sequence. In both the respiratory and photosynthetic electron-transport
chains, one of these carriers is a \textbf{quinone}, a small hydrophobic
molecule that dissolves in the lipid bilayer; in the mitochondrial respiratory
chain, the quinone is called \textit{ubiquinone}. Quinones are the only
electron carriers in electron-transport chains that can function without
being tightly bound to a protein.

Ubiquinone picks up electrons from the NADH dehydrogenase complex
and delivers them to the cytochrome $b-c_1$ complex.
Ubiquinone can pick up or donate either one or two electrons, and it
picks up one $H^+$ from the surroundings with each electron that it carries.
Its redox potential of +30 mV places ubiquinone about
one-quarter of the way down the chain from NADH in terms of energy
loss. Ubiquinone can also receive electrons directly from
the $FADH_2$ generated by the citric acid cycle or by fatty acid oxidation.
Because these electrons bypass NADH hydrogenase - which is one of the
proton pumps in the electron transport chain - they cause less proton
pumping than do the two electrons transported from NADH.

The rest of the electron carriers in the electron-transport chain are either
small molecules or metal-containing groups that are all tightly bound to
proteins. To get from NADH to ubiquinone, for example, the electrons are
passed inside the NADH dehydrogenase complex between a flavin group
bound to one of the proteins and a set of \textbf{iron-sulfur centers} of
increasing redox potentials. The final iron-sulfur center in the dehydrogenase
donates its electrons to ubiquinone.

Iron-sulfur centers have relatively low affinities for electrons and thus
are prominent in the early part of the electron-transport chain.

Later, in the pathway from ubiquinone to $O_2$, iron atoms in heme groups that are
tightly bound to cytochrome proteins are commonly used as electron
carriers, as in the cytochrome $b-c_1$ and cytochrome oxidase complexes.
The cytochromes constitute a family of colored proteins; each contains one or more heme
groups whose iron atom changes from the ferric ($Fe^{3+}$) to the ferrous
($Fe^{2+}$) state whenever it accepts an electron. As one would expect, the
various cytochromes increase in redox potential as one progresses down
the mitochondrial electron-transport chain towards $O_2$.

At the very end of the respiratory chain, just before $O_2$, the electron carriers
are those in the cytochrome oxidase complex. The carriers here are
either iron atoms in heme groups or copper atoms that are tightly bound
to the complex in specific ways that give them a high redox potential.

\subsection{Cytochrome Oxidase Catalyzes the Reduction of Molecular Oxygen}

Cytochrome oxidase is a protein complex that receives electrons from
cytochrome \textit{c}, thus oxidizing it (hence the name cytochrome oxidase).
It then donates these electrons to $O_2$. In brief, four electrons from cytochrome
c and four protons from the aqueous environment are added to
each $O_2$ molecule in the reaction $4e^{-} + 4H^{+} + O_{2} \rightarrow 2H_{2}O$.
In addition to the protons that couple with $O_2$, four other protons are
pumped across the membrane during electron transfer, further increasing
the electrochemical proton gradient.

Of course, for proton pumping to occur, it must be coupled in some way
to energetically favorable reactions. In the case of cytochrome oxidase,
the energy comes from the transfer of a series of four electrons to an
$O_2$ molecule that is bound tightly to the protein; these electron transfers
drive allosteric changes in the conformation of the protein that move
protons out of the mitochondrial matrix. At its active site, where $O_2$ is
bound, cytochrome oxidase contains a complex of a heme iron atom juxtaposed
with a tightly bound copper atom. It is here that
nearly all of the oxygen we breathe is used, serving as the final repository
for the electrons that NADH donated at the start of the electron-transport
chain.

Oxygen is useful as an electron sink because of its very high affinity for
electrons. However, once $O_2$ picks up one electron, it forms the superoxide
radical $O^{2-}$; this radical is dangerously reactive and will avidly take up
another three electrons wherever it can find them, a tendency that can
cause serious damage to nearby DNA, proteins, and lipid membranes.
One of the roles of cytochrome oxidase is to hold on tightly to an oxygen
molecule until all four electrons needed to convert it to two $H_{2}O$
molecules are in hand, thereby preventing a random attack on cellular
macromolecules by superoxide radicals—damage that has been postulated
to be a cause of human aging.

\subsection{Respiration Is Amazingly Efficient}

The free-energy changes for burning fats and carbohydrates directly to
$CO_2$ and $H_{2}O$ can be compared with the total amount of energy generated
and stored in the phosphate bonds of ATP during the corresponding
biological oxidations. When this is done, one finds that the efficiency
with which oxidation energy is converted into ATP bond energy is often
greater than 40\%.

\section{Chloroplasts and photosynthesis}

Virtually all of the organic material required by present-day living cells
is produced by \textbf{photosynthesis} - the series of light-driven reactions
that creates organic molecules from atmospheric carbon dioxide.

In plants, photosynthesis is carried out in a specialized intracellular
organelle - the \textbf{chloroplast}, which contains light-capturing pigments
such as the green pigment chlorophyll. All green parts of a plant contain
chloroplasts, but for most plants the leaves are the major sites of photosynthesis.
Chloroplasts perform photosynthesis during the daylight hours.
The process produces ATP and NADPH, which in turn are used to convert
$CO_2$ into sugar inside the chloroplast.

\subsection{Chloroplasts Resemble Mitochondria but Have an Extra Compartment}

Chloroplasts carry out their energy interconversions by means of proton
gradients in much the same way that mitochondria do. Although
chloroplasts are larger, they are organized on the same
principles as mitochondria. Chloroplasts have a highly permeable outer
membrane and a much less permeable inner membrane, in which membrane
transport proteins are embedded. Together these membranes - and
the narrow, intermembrane space that separates them - form the chloroplast
envelope. The inner membrane surrounds a large
space called the \textbf{stroma}, which is analogous to the mitochondrial matrix
and contains many metabolic enzymes.

There is, however, an important difference between the organization
of mitochondria and that of chloroplasts. The inner membrane of the
chloroplast does not contain the electron-transport chains. Instead, the
light-capturing systems, the electron-transport chains, and ATP synthase
are all contained in the \textit{thylakoid membrane}, a third membrane that forms
a set of flattened disclike sacs, called the \textit{thylakoids}. These
are arranged in stacks, and the space inside each thylakoid is thought to
be connected with that of other thylakoids, thereby defining a continuous
third internal compartment that is separated from the stroma by the
thylakoid membrane.

\subsection{Chloroplasts Capture Energy from Sunlight and use it to fix Carbon}

The overall equation that summarizes the net
result of photosynthesis can be written as follows:

$$
light energy + CO_{2} + H_{2}O \rightarrow sugars + O_{2} + energy
$$

Although the equation is quite simple, the reactions that allow the process
to occur are fairly elaborate. Generally speaking, however, the many
reactions that make up photosynthesis in plants take place in two stages:

\begin{enumerate}
\item In the first stage, which is dependent on light, energy from sunlight
is captured and stored transiently in the high-energy bonds
of ATP and the activated carrier molecule NADPH. These energy-producing
\textit{photosynthetic electron-transfer reactions}, also called the
`light reactions,' occur entirely within the thylakoid membrane of
the chloroplast. In this series of reactions, energy derived from
sunlight energizes an electron in the green organic pigment \textbf{chlorophyll},
enabling the electron to move along an electron-transport
chain in the thylakoid membrane in much the same way that an
electron moves along the respiratory chain in mitochondria. The
electron that chlorophyll donates to the electron-transport chain
is ultimately replaced by an electron extracted from water. This
electron shuffle splits a molecule of water ($H_{2}O$), producing $O_2$ as
a by-product. During the electron-transport process, $H^+$ is pumped
across the thylakoid membrane, and the resulting electrochemical
proton gradient drives the synthesis of ATP in the stroma. As
the final step in this series of reactions, high-energy electrons are
loaded (together with $H^+$) onto $NADP^+$, converting it to NADPH.
\item In the second, light-independent, stage of photosynthesis, the ATP
and the NADPH produced by the photosynthetic electron-transfer
reactions serve as the source of energy and reducing power,
respectively, to drive the manufacture of sugars from $CO_2$.
These carbon-fixation reactions, also called the `dark
reactions,' begin in the chloroplast stroma and continue in the
plant cell cytosol. They produce sucrose and many other organic
molecules in the leaves of the plant. The sucrose is exported to
other tissues as a source of both organic molecules and energy for
growth.
\end{enumerate}

Thus, the formation of ATP, NADPH, and $O_2$ (which requires light energy
directly) and the conversion of $CO_2$ to carbohydrate (which requires light
energy only indirectly) are separate processes, although elaborate feedback
mechanisms interconnect the two sets of reactions.

\subsection{Sunlight is Absorbed by Chlorophyll Molecules}

When sunlight is absorbed by a molecule of chlorophyll, electrons in the
molecule interact with photons of light and are raised to a higher energy
level. The electrons in the extensive network of alternating single and
double bonds in the chlorophyll molecule absorb red light
most strongly, which is why chlorophyll looks green to us.

\subsection{Excited Chlorophyll Molecules funnel Energy into a Reaction Center}

An isolated molecule of chlorophyll is incapable of converting the light
it absorbs to a form of energy useful to living systems. It can accomplish
this feat only when it is associated with the appropriate proteins
and embedded in a membrane. In plant thylakoid membranes and in the
membranes of photosynthetic bacteria, the light-absorbing chlorophylls
are held in large multiprotein complexes called \textbf{photosystems}. Each
photosystem consists of an \textbf{antenna complex} that captures light energy
and a \textbf{reaction center} that enables this light energy to be converted
into chemical energy. The antenna portion of a photosystem consists of
hundreds of chlorophyll molecules that capture light energy in the form
of excited (high-energy) electrons. These chlorophylls are arranged so
that the energy of an excited electron can be passed from one molecule
to another, until finally the energy is funneled into two chlorophyll
molecules called the \textit{special pair}. These two chlorophyll
molecules are located in the reaction center, a protein complex that sits
adjacent to the antenna complex in the membrane. There the energy is
trapped and used to energize one electron in the special pair of chlorophyll
molecules.

The reaction center is a transmembrane complex of proteins and organic
pigments that lies at the heart of photosynthesis. The reaction center
acts as an irreversible trap for an excited electron, because the special
pair of chlorophylls are poised to pass the high-energy electron to a precisely
positioned neighboring molecule in the same protein complex.
By moving the energized electron rapidly away from the chlorophylls,
a process known as \textit{charge separation}, the reaction center transfers this
high-energy electron to an environment where it is much more stable.

When the chlorophyll molecule in the reaction center loses an electron,
it becomes positively charged; it then rapidly regains an electron from an
adjacent electron donor to return to its unexcited, uncharged state.
Then, in slower reactions, the electron donor has its missing
electron replaced with an electron removed from water. The high-energy
electron that was generated by the excited chlorophyll is then transferred
to the electron-transport chain. This transfer leaves the reaction center
ready to receive the next high-energy electron excited by sunlight.

\subsection{Light Energy Drives the Synthesis of Both ATP and NADPH}

In mitochondria, the electron-transport chain functions solely to generate
ATP. But in the chloroplast, and in free-living photosynthetic organisms
such as cyanobacteria, electron transport has an additional role: it also
produces the activated carrier molecule NADPH. NADPH
is needed because photosynthesis is ultimately a biosynthetic process. To
build organic molecules from $CO_2$, a cell requires a huge input of energy,
in the form of ATP, and a very large amount of reducing power, in the
form of NADPH. To produce this NADPH from $NADP^+$, the cell uses energy
captured from sunlight to convert the low-energy electrons in water to
the high-energy electrons in NADPH.

To produce both ATP and NADPH, plant cells and cyanobacteria use two
photons of light: ATP is made after the first photon is absorbed, NADPH
after the second. These photons are absorbed by two different photosystems
that operate in series. Working together, these photosystems impart
to an electron a high enough energy to produce NADPH. Along the way,
a proton gradient is generated, allowing ATP to be made.

In outline, the process works as follows: the first photon of light is
absorbed by one photosystem (which is paradoxically called photosystem
II for historical reasons). As we have seen, that photon is used to produce
a high-energy electron that is handed off to an electron-transport chain.
While traveling down the electron-transport chain,
the electron drives an $H^+$ pump in the thylakoid membrane and creates
a proton gradient in the manner described previously for oxidative phosphorylation.
An ATP synthase in the thylakoid membrane then uses this
proton gradient to drive the synthesis of ATP on the stromal side of the
membrane.

In the meantime, the electron-transport chain delivers the electron generated
by photosystem II to the second photosystem in the pathway (called
photosystem I). There the electron fills the positively charged `hole' that
was left in the reaction center of photosystem I when it absorbed the
second photon of light. Because photosystem I starts at a higher energy
level than photosystem II, it is able to boost electrons to the very high
energy level needed to make NADPH from $NADP^+$.

In the overall process described thus far, we have seen that an electron
removed from a chlorophyll molecule at the reaction center of photosystem
II travels all the way through the electron-transport chain in the
thylakoid membrane until it winds up being donated to NADPH. This initial
electron must be replaced to return the system to its unexcited state.
The replacement electron comes from a low-energy electron donor,
which, in plants and many photosynthetic bacteria, is water.
The reaction center of photosystem II includes a water-splitting
enzyme that holds the oxygen atoms of two water molecules bound to a
cluster of manganese atoms in the protein.
This enzyme removes electrons one at a time from the water to
fill the holes created by light in the chlorophyll molecules of the reaction
center. When four electrons have been removed from two water molecules
(which requires four photons of light), $O_2$ is released.

\subsection{Chloroplasts Can Adjust their ATP Production}

In addition to carrying out the photosynthetic process outlined so far,
chloroplasts can also generate ATP without making NADPH. To produce
this extra ATP, chloroplasts can switch photosystem I into a cyclic mode
so that it produces ATP instead of NADPH. In this process, called \textbf{cyclic
photophosphorylation}, the high-energy electrons produced by light
activation of photosystem I are transferred back to the cytochrome $b_{6}-f$
complex rather than being passed on to $NADP^+$. From the $b_{6}-f$ complex,
the electrons are handed back to photosystem I at low energy.
The net result, aside from the conversion of some light energy
to heat, is that $H^+$ is pumped across the thylakoid membrane by the $b_{6}-f$
complex as electrons pass through it. This cycle increases the electro-chemical
proton gradient that drives the synthesis of ATP. Cells adjust
the relative amounts of cyclic photophosphorylation (which involves
only photosystem I) and the standard, noncyclic form of phosphorylation
(which involves both photosystems I and II) depending on their relative
need for reducing power (in the form of NADPH) and high-energy phosphate
bonds (in the form of ATP).

\subsection{Carbon fixation uses ATP and NADPH to Convert $CO_2$ into Sugars}

The light reactions of photosynthesis generate ATP and NADPH in the
chloroplast stroma. But the inner membrane of the chloroplast is impermeable
to both of these compounds, which means that they cannot be
exported directly to the cytosol. To provide reducing power and energy
for the rest of the cell, ATP and NADPH are instead used within the chloroplast
stroma to produce sugars that can then be directly exported. This
sugar production, which occurs during the dark reactions of photosynthesis,
is called \textbf{carbon fixation}.
$CO_2$ from the atmosphere combines with the five-carbon sugar derivative
\textit{ribulose 1,5-bisphosphate} plus water to give two
molecules of the three-carbon compound \textit{3-phosphoglycerate}.
This carbon-fixing reaction is catalyzed in the chloroplast stroma by a
large enzyme called \textit{ribulose bisphosphate carboxylase} (also called
\textit{ribulose bisphosphate carboxylase/oxygenase} or \textit{Rubisco}).

When carbohydrates are broken down and oxidized to $CO_2$ and $H_{2}O$
by cells, a large amount of free energy is released. Clearly, the reverse
overall reaction - in which $CO_2$ and $H_{2}O$ combine to make carbohydrate
during photosynthesis - must be energetically very unfavorable. For this
process to occur, it must be coupled to an energetically favorable reaction
that drives it. The reaction in which $CO_2$ is fixed by Rubisco is in fact
energetically favorable, but only because it receives a continuous supply
of the energy-rich compound ribulose 1,5-bisphosphate, to which each
molecule of $CO_2$ is added. The energy and reducing
power required for the elaborate metabolic pathway by which this compound
is regenerated are provided by the ATP and NADPH produced by
the photosynthetic light reactions.

The series of reactions that allows cells to incorporate $CO_2$ into sugars
forms a cycle that begins and ends with ribulose 1,5-bisphosphate.
For every three molecules of $CO_2$ that enter the cycle, one
new molecule of \textit{glyceraldehyde 3-phosphate} is produced - the three-carbon
sugar that is the net product - and three molecules of ATP and two
molecules of NADPH are consumed. Glyceraldehyde 3-phosphate then
provides the starting material for the synthesis of many other sugars and
organic molecules. The \textit{carbon-fixation cycle} (or Calvin cycle) was worked
out in the 1940s and 1950s in one of the first successful applications of
radioisotopes as tracers in biochemistry.

\subsection{Sugars generated by Carbon fixation Can Be Stored As Starch or Consumed to Produce ATP}

The glyceraldehyde 3-phosphate generated by carbon fixation in the chloroplast
can be used in a number of different ways, depending on the needs
of the plant. During periods of excess photosynthetic activity, glyceraldehyde
3-phosphate is retained in the chloroplast, where it is mainly
converted to \textit{starch} in the stroma. Like glycogen in animal of
cells, starch is a large polymerglucose that serves as a carbohydrate
reserve. Starch is stored as large grains in the chloroplast stroma,
and at night it is broken down to sugars to help support
the metabolic needs of the plant.

\section{Essential concepts}

\begin{itemize}
\item Mitochondria, chloroplasts, and many bacteria produce ATP by a
membrane-based mechanism known as chemiosmotic coupling.
\item Mitochondria produce most of an animal cell’s ATP, using energy
derived from oxidation of sugars and fatty acids.
\item Mitochondria have an inner and an outer membrane. The inner
membrane encloses the mitochondrial matrix, a compartment that
contains many enzymes, including those of the citric acid cycle.
These enzymes produce large amounts of NADH and $FADH_2$ from the
oxidation of acetyl CoA.
\item In the inner mitochondrial membrane, high-energy electrons donated
by NADH and $FADH_2$ pass along an electron-transport chain - the respiratory
chain - eventually combining with molecular oxygen ($O_2$) in
an energetically favorable reaction.
\item Much of the energy released by electron transfers along the respiratory
chain is harnessed to pump $H^+$ out of the matrix, thereby creating
a transmembrane proton ($H^+$ ) gradient. The proton pumping is carried
out by three large respiratory enzyme complexes embedded in
the membrane.
\item The resulting electrochemical proton gradient across the inner mitochondrial
membrane is harnessed to make ATP when $H^+$ ions flow
back into the matrix through ATP synthase, an enzyme located in the
inner mitochondrial membrane.
\item The proton gradient also drives the active transport of metabolites
into and out of the mitochondrion.
\item In photosynthesis in chloroplasts and photosynthetic bacteria,
high-energy electrons are generated when sunlight is absorbed by
chlorophyll; this energy is captured by protein complexes known as
photosystems, which in plant cells are located in the thylakoid membranes
of chloroplasts.
\item Electron-transport chains associated with photosystems transfer
electrons from water to $NADP^+$ to form NADPH. $O_2$ is generated as a
by-product.
\item The electron-transport chains in chloroplasts also generate a proton
gradient across the thylakoid membrane. As in mitochondria, this
electrochemical proton gradient is used by an ATP synthase embedded
in the membrane to generate ATP.
\item The ATP and the NADPH made by photosynthesis are used within
the chloroplast to drive the carbon-fixation cycle in the chloroplast
stroma, thereby producing carbohydrate from $CO_2$.
\item Carbohydrate is exported to the cell cytosol, where it is metabolized
to provide organic carbon, ATP (mostly via mitochondria), and reducing
power for the rest of the cell.
\item Both mitochondria and chloroplasts are thought to have evolved
from bacteria that were endocytosed by primitive eucaryotic cells.
Each retains its own genome and divides by processes that resemble
a bacterial cell division.
\item Chemiosmotic coupling mechanisms are widespread and of ancient
origin. Modern microorganisms that live in environments similar
to those thought to have been present on the early Earth also use
chemiosmotic coupling to produce ATP.
\end{itemize}
