\chapter{Protein Structure and Function}

\section{The Shape and Structure of proteins}

Understanding protein
structure at the atomic level will allow us to describe how the precise
shape of each protein determines its function in the cell.

\subsection{The Shape of a protein is Specified by its amino acid Sequence}

Proteins are assembled from a set of 20 different amino acids, each with different chemical properties. A protein
molecule is made from a long chain of these amino acids, each linked
to its neighbor through a covalent peptide bond. Proteins
are therefore referred to as polypeptides or polypeptide chains. In each
type of protein, the amino acids are present in a unique order, called the
amino acid sequence, which is exactly the same from one molecule
of that protein to the next.

Each polypeptide chain consists of a backbone that supports the different
amino acid side chains. The polypeptide backbone is made from
the repeating sequence of the core atoms of the amino acids that form
the chain. Projecting from this repetitive backbone are any of the 20 different
amino acid side chains - the parts of the amino acids that are not
involved in forming the peptide bond.

Long polypeptide chains are very flexible: many of the covalent bonds
that link carbon atoms in an extended chain of amino acids allow free
rotation of the atoms they join. Thus proteins can in principle fold in an
enormous number of ways. Each folded chain is constrained by many different
sets of weak noncovalent bonds that form within proteins. These
bonds involve atoms in the polypeptide backbone as well as atoms in
the amino acid side chains. The noncovalent bonds that help proteins
maintain their shape include hydrogen bonds, electrostatic attractions,
and van der Waals attractions.

An important factor governing the folding of any protein is the distribution
of its polar and nonpolar amino acids.

\subsection{Proteins fold into a conformation of lowest energy}

Each type of protein has a particular three-dimensional structure, which
is determined by the order of the amino acids in its chain. The final folded
structure, or conformation, adopted by any polypeptide chain is determined
by energetic considerations: a protein generally folds into the
shape in which the free energy (G) is minimized. Protein folding
has been studied in the laboratory using highly purified proteins. A
protein can be unfolded, or denatured, by treatment with solvents that
disrupt the noncovalent interactions holding the folded chain together.
This treatment converts the protein into a flexible polypeptide chain that
has lost its natural shape. When the denaturing solvent is removed, the
protein often refolds spontaneously, or renatures, into its original conformation.
All the information necessary to specify the three-dimensional shape of a protein
is cointained in its amino acid sequence.

Each protein normally folds into a single stable conformation. This conformation,
however, often changes slightly when the protein interacts
with other molecules in the cell. This change in shape is crucial to the
function of the protein.

When proteins fold incorrectly, they sometimes form aggregates that can
damage cells and even whole tissues.

Although a protein chain can fold into its correct conformation without
outside help, protein folding in a living cell is generally assisted by special
proteins called molecular chaperones. These proteins bind to partly folded
chains and help them to fold along the most energetically favorable pathway.

\subsection{Proteins come in a wide Variety of complicated Shapes}

Proteins can be globular or fibrous; they can form filaments, sheets,
rings, or spheres.

Resolving a protein’s structure often begins with determining its amino
acid sequence, a task that can be accomplished in several ways. For
many years, protein sequencing was accomplished by directly analyzing
the amino acids in the purified protein; the first protein sequenced
was the hormone insulin, in 1955. Today we can determine the order of
amino acids in a protein much more easily by sequencing the gene that
encodes it. Once the order of the nucleotides
in the DNA that encodes a protein is known, this information can be
converted into an amino acid sequence by applying the genetic code.

\subsection{The $\alpha$-helix and the $\beta$ sheet are common folding patterns}

When the three-dimensional structures of many different protein molecules
are compared, it becomes clear that, although the overall
conformation of each protein is unique, two regular folding patterns are
often present. Both were discovered more than 50 years ago from studies
of hair and silk. The first folding pattern to be discovered, called the
$\alpha$-helix, was found in the protein $\alpha$-keratin, which is abundant in skin and
its derivatives - such as hair, nails, and horns. Within a year of the discovery
of the a helix, a second folded structure, called a $\beta$ sheet, was found
in the protein fibroin, the major constituent of silk.

These two folding patterns are particularly common because they result
from hydrogen bonds that form between the $N-H$ and $C=O$ groups in
the polypeptide backbone. Because the amino acid side chains are not
involved in forming these hydrogen bonds, $\alpha$ helices and $\beta$ sheets can
be generated by many different amino acid sequences.

\subsection{Helices Form Readily in Biological Structures}

A helix is a regular structure that resembles a spiral staircase.
Depending on the twist of the staircase, a helix is said to be either
right-handed or left-handed.

An $\alpha$-helix is generated when a single polypeptide chain turns around
itself to form a structurally rigid cylinder. A hydrogen bond is made
between every fourth amino acid, linking the $C=O$ of one peptide bond to
the $N-H$ of another. This gives rise to a regular helix
with a complete turn every 3.6 amino acids.
Short regions of $\alpha$-helix are especially abundant in the proteins located
in cell membranes, such as transport proteins and receptors.
The polypeptide backbone, which is hydrophilic, is hydrogen-bonded to itself in
the $\alpha$-helix, and it is shielded
from the hydrophobic lipid environment of the membrane by its protruding nonpolar side chains.

Sometimes two (or three) $\alpha$ helices will wrap around one another to
form a particularly stable structure known as a coiled-coil. This structure
forms when the a helices have most of their nonpolar (hydrophobic)
side chains on one side, so that they can twist around each other with
these side chains facing inward - minimizing their contact with the aqueous cytosol.

\subsection{$\beta$ Sheets form rigid Structures at the core of Many proteins}

$\beta$ sheets are made when hydrogen bonds form between segments of polypeptide
chains lying side by side. When the structure consists of neighboring polypeptide chains that
run in the same orientation (say, from the N-terminus to the C-terminus),
it is considered a parallel $\beta$ sheet; when it forms from a polypeptide chain
that folds back and forth upon itself - with each section of the chain running
in the direction opposite to that of its immediate neighbors - the
structure is an antiparallel $\beta$ sheet. Both types of $\beta$ sheet
produce a very rigid, pleated structure, and they form the core of many
proteins.

\subsection{Proteins have Several levels of organization}

A protein’s structure begins with its
amino acid sequence, which is thus considered its primary structure. The
next level of organization includes the a helices and b sheets that form
within certain segments of a polypeptide chain; these folds are elements
of the protein’s secondary structure. The full, three-dimensional conformation
formed by an entire polypeptide chain - including the $\alpha$ helices,
$\beta$ sheets, random coils, and any other loops and folds that form between
the N- and C-termini - is sometimes referred to as the tertiary structure.
Finally, if a particular protein molecule is formed as a complex of more than one polypeptide
chain, then the complete structure is designated its quaternary structure.
Studies of the conformation, function, and evolution of proteins have
also revealed the importance of a level of organization distinct from
those just described. This organizational unit is the protein domain,
which is defined as any segment of a polypeptide chain that can fold
independently into a compact, stable structure.
The different domains of a protein are ofter associated with different functions.

\subsection{Few of the many possible polypeptide chains will be useful}

For a polypeptide that is $n$ amino acids long, $20^n$ different chains are possible.
However, only a very small fraction of this unimaginably large number
of polypeptide chains would fold into a stable, well-defined three-dimensional conformation.

\subsection{Proteins can be classified into families}

Once a protein had evolved a stable conformation with useful properties,
its structure could be modified over time to enable it to perform
new functions. We know that this occurred quite often during evolution,
because many present-day proteins can be grouped into protein families,
in which each family member has an amino acid sequence and a
three-dimensional conformation that closely resembles that of the other
family members.

\subsection{Large protein Molecules often contain More Than one polypeptide chain}

The same weak noncovalent bonds that enable a polypeptide chain to
fold into a specific conformation also allow proteins to bind to each other
to produce larger structures in the cell. Any region on a protein’s surface
that interacts with another molecule through sets of noncovalent bonds
is termed a binding site.
If a binding site recognizes the surface
of a second protein, the tight binding of two folded polypeptide chains
at this site will create a larger protein whose quaternary structure has a
precisely defined geometry. Each polypeptide chain in such a protein is
called a subunit.

In the simplest case, two identical folded polypeptide chains form a symmetrical
complex of two protein subunits (called a dimer) that is held
together by interactions between two identical binding sites.
Other proteins contain two or more different types of polypeptide chains.

\subsection{Proteins can assemble into Filaments, Sheets, or Spheres}

A chain of identical protein molecules can be formed if the
binding site on one protein molecule is complementary to another region
on the surface of another protein molecule of the same type. Because
each protein molecule is bound to its neighbor in an identical way, the
molecules will often be arranged in a helix that can be extended indefinitely.
This type of arrangement can produce an extended protein filament.
Other sets of proteins associate to form extended sheets or tubes,
as in the microtubules of the cell cytoskeleton, or cagelike spherical shells,
as in the protein coats of virus particles.

Many large structures, such as viruses and ribosomes, are built from a
mixture of one or more types of protein plus RNA or DNA molecules.

\subsection{Some Types of proteins have elongated fibrous Shapes}

Most of the proteins we have discussed so far are globular proteins, in
which the polypeptide chain folds up into a compact shape like a ball
with an irregular surface. Enzymes tend to be globular proteins: even
though many are large and complicated, with multiple subunits, most
have a quaternary structure with an overall rounded shape.
In contrast, other proteins have roles in the cell that require them to
span a large distance. These proteins generally have a relatively simple,
elongated three-dimensional structure and are commonly referred to as
fibrous proteins.
An $\alpha$-keratin molecule is a dimer of two identical subunits, with the
long a helices of each subunit forming a coiled-coil.
These coiled-coil regions are capped at either end by globular domains
containing binding sites that allow them to assemble into ropelike intermediate
filaments - a component of the cell cytoskeleton that creates a
structural scaffold for the cell’s interior.
Fibrous proteins are especially abundant outside the cell, where they
form the gel-like extracellular matrix that helps cells bind together to form
tissues.

\subsection{Extracellular proteins are often Stabilized by covalent cross-linkages}

Many protein molecules are either attached to the outside of a cell’s
plasma membrane or secreted as part of the extracellular matrix. All such
proteins are directly exposed to extracellular conditions. To help maintain
their structures, the polypeptide chains in such proteins are often
stabilized by covalent cross-linkages.
The most common covalent cross-links
in proteins are sulfur-sulfur bonds. These disulfide bonds (also called
S-S bonds) form as proteins are being exported from cells.
Disulfide bonds do not change the
conformation of a protein, but instead act as a sort of “atomic staple”
to reinforce its most favored conformation.

\section{How proteins work}

\subsection{All proteins bind to other molecules}

All proteins stick, or bind, to other molecules. In some cases this binding is
very tight; in others it is weak and short-lived.
Regardless of its strength, the binding
of proteins to other biological molecules always shows great specificity:
each protein molecule can bind to just one or a few molecules out of
the many thousands of different molecules it encounters. Any substance
that is bound by a protein - whether it is an ion, a small molecule, or a
macromolecule - is referred to as a ligand for that protein.

When molecules have poorly matching surfaces, few noncovalent bonds
are formed and the two molecules dissociate as rapidly as they come
together.

The region of a protein that associates with a ligand, known as its binding site,
usually consists of a cavity in the protein surface formed by a
particular arrangement of amino acids. Other regions on the surface often provide
binding sites for different ligands, allowing the protein’s
activity to be regulated, as we shall see later. Yet other parts of the protein
may be required to attract or attach the protein to a particular location in
the cell - for example, the hydrophobic $\alpha$-helix of a membrane-spanning
protein allows it to be inserted into the lipid bilayer of a cell membrane.

\subsection{The Binding Sites of antibodies are especially Versatile}

Antibodies, or immunoglobulins, are proteins produced by the immune
system in response to foreign molecules, such as those on the surface
of an invading microorganism. Each antibody binds to a particular target
molecule extremely tightly, either inactivating the target directly or
marking it for destruction. An antibody recognizes its target (called an
antigen) with remarkable specificity, and because there are potentially
billions of different antigens that a person might encounter, we have to
be able to produce billions of different antibodies.

Antibodies are Y-shaped molecules with two identical binding sites that
are each complementary to a small portion of the surface of the antigen
molecule.

\subsection{Enzymes are powerful and highly Specific catalysts}

There are proteins for which ligand binding is simply a necessary first step
in their function. This is the case for the large and very important class
of proteins called enzymes. Enzymes bind to one or more ligands, called substrates,
and convert them into chemically modified products, doing this over and over again with amazing rapidity.
They speed up reactions, often by a factor of a
million or more, without themselves being changed - that is, enzymes act
as catalysts that permit cells to make or break covalent bonds at will.

Enzymes can be grouped into functional classes based on the chemical
reactions they catalyze. Each type of enzyme is highly specific,
catalyzing only a single type of reaction.

\subsection{Lysozyme illustrates how an enzyme works}

To explain how enzymes catalyze chemical reactions, we will use the
example of lysozyme - an enzyme that acts as a natural antibiotic in egg
white, saliva, tears, and other secretions. Lysozyme severs the polysaccharide
chains that form the cell walls of bacteria.

The reaction catalyzed by lysozyme is a hydrolysis: the enzyme adds a
molecule of water to a single bond between two adjacent sugar groups in
the polysaccharide chain, thereby causing the bond to break.
For a colliding water molecule to break a bond linking two sugars, the
polysaccharide molecule has to be distorted into a particular shape - the
transition state - in which the atoms around the bond have an altered
geometry and electron distribution. To distort the molecule in this way
requires a large input of energy (energy barrier) from random molecular collisions.

This is where the enzyme comes in. Like all enzymes, lysozyme has a
special binding site on its surface, termed an active site, that cradles the
contours of its substrate molecule. Here the catalysis of the chemical reaction
occurs. Because its substrate is a polymer, lysozyme’s active site is a
long groove that holds six linked sugars at the same time. As soon as the
polysaccharide binds to form an enzyme - substrate complex, the enzyme
cuts the polysaccharide by catalyzing the addition of a water molecule to
one of its sugar-sugar bonds.

The chemistry that underlies the binding of lysozyme to its substrate is
the same as that for antibody binding - the formation of multiple non-covalent bonds.

The overall chemical reaction, from the initial binding of the polysaccharide
on the surface of the enzyme to the final release of the severed
chains, occurs many millions of times faster than it would in the absence
of enzyme.

In reactions involving two or more
substrates, the active site also acts like a template or mold that brings
the reactants together in the proper orientation for chemistry to occur between them.
Binding to the enzyme also changes the shapes of substrates,
bending bonds so as to drive a substrate toward a particular
transition state. Subsequent
steps in the reaction restore the side chain to its original state, so that the
enzyme remains unchanged after the reaction and can go on to catalyze
many more reactions.

\subsection{Most drugs inhibit enzymes}

Many of the drugs we take to treat or prevent illness work by blocking
the activity of a particular enzyme.


\subsection{Tightly Bound Small Molecules add extra functions to proteins}

Proteins often employ small nonprotein molecules to perform functions that would be
difficult or impossible using amino acids alone. When these small molecules are attached covalently and permanently
to their protein, they become an integral part of the protein molecule
itself.

Enzymes, too, make use of nonprotein molecules: they frequently have
a small molecule or metal atom associated with their active site that
assists with their catalytic function. Carboxypeptidase, an enzyme that
cuts polypeptide chains, carries a tightly bound zinc ion in its active site.

\section{How proteins are controlled}

The activities of cellular proteins are regulated in
a coordinated fashion so that the cell can maintain itself in an optimal
state, generating only those molecules it requires to thrive under the current conditions.
By coordinating when - and how vigorously - proteins
function, the cell ensures that it does not deplete its energy reserves by
accumulating molecules it does not need or waste its stockpiles of critical substrates.

The regulation of protein activity occurs at many levels. At one level, the
cell controls how many molecules of each enzyme it makes by regulating
the expression of the gene that encodes that protein.
At another level, the cell controls enzymatic activities by confining s
ets of enzymes to particular subcellular compartments, often - but
not always - enclosed by distinct membranes.
But the most rapid and general process used to adjust reaction
rates operates at the level of the enzyme itself. Although proteins can be
switched on - or switched off - by a variety of mechanisms, as we see
next, all in some way cause the protein to alter its shape, and therefore
its function.

\subsection{The catalytic activities of enzymes are often regulated by other Molecules}

By their catalytic action, enzymes generate a complex web of metabolic pathways
composed of chains of chemical reactions in which the product of
one enzyme becomes the substrate of the next.

The most common type of control occurs when a molecule other than a
substrate specifically binds to an enzyme at a special regulatory site outside
of the active site, altering the rate at which the enzyme converts its
substrates to products. In feedback inhibition, an enzyme acting early in
a reaction pathway is inhibited by a late product of that pathway.

Feedback inhibition is a negative regulation: it prevents an enzyme from
acting. Enzymes can also be subject to positive regulation, in which the
enzyme’s activity is stimulated by a regulatory molecule rather than being
shut down.

\subsection{Allosteric enzymes have Binding Sites That influence one another}

The regulatory molecule often has a shape that is totally
different from the shape of the enzyme’s preferred substrate. Indeed,
when this form of regulation was discovered in the 1960s, it was termed
allostery (from the Greek allo, “other,” and stere, “solid” or “shape”). As
more was learned about feedback inhibition, researchers realized that
many enzymes must have at least two different binding sites on their
surface: the active site that recognizes the substrates and one or more
sites that recognize regulatory molecules. Furthermore, the substrate and
regulatory sites must somehow “communicate” in a way that allows the
catalytic events at the active site to be influenced by the binding of the
regulatory molecule at its separate site.
The interaction between sites that are located on separate regions of a
protein molecule is now known to depend on conformational changes
in the protein: binding at one of the sites causes a shift in the protein’s
structure from one folded shape to a slightly different folded shape.

Many - if not most - protein molecules are allosteric: they can adopt
two or more slightly different conformations, and by a shift from one to
another, their activity can be regulated.

\subsection{Phosphorylation can control protein activity by Triggering a conformational change}

Another method commonly used by eucaryotic cells to regulate protein
activity involves attaching a phosphate group covalently to one of
its amino acid side chains. Because each phosphate group carries two
negative charges, the enzyme-catalyzed addition of a phosphate group
to a protein can cause a major conformational change.
Removal of the phosphate group by a second enzyme returns the protein to its original
conformation and restores its initial activity.

This reversible protein phosphorylation controls the activity of many
different types of proteins in eucaryotic cells.
The addition and removal of phosphate groups from specific proteins
often occurs in response to signals that specify some change in a cell’s
state. And many of the signals generated by hormones and neurotransmitters are
carried from the plasma membrane to the nucleus by a cascade of protein
phosphorylation events.

Protein phosphorylation involves the enzyme-catalyzed transfer of the
terminal phosphate group of ATP to the hydroxyl group on a serine, threonine,
or tyrosine side chain of the protein. This reaction is catalyzed by a
protein kinase. The reverse reaction - removal of the phosphate group,
or dephosphorylation - is catalyzed by a protein phosphatase.

For many proteins, a phosphate group is added to a particular side
chain and then removed in a continuous cycle. Phosphorylation cycles
of this kind allow proteins to switch rapidly from one state to another.

\subsection{GTP-Binding proteins are also regulated by the cyclic Gain and loss of a phosphate Group}

Eucaryotic cells have a second way to regulate protein activity by phosphate
addition and removal. In this case, instead of being enzymatically
transferred from ATP to the protein, the phosphate is part of a guanine
nucleotide - either guanosine triphosphate (GTP) or guanosine diphosphate
(GDP) - that is bound tightly to the protein. Such GTP-binding
proteins are in their active conformations with GTP bound; the protein
itself then hydrolyzes this GTP to GDP-releasing a phosphate - and flips
to an inactive conformation. As with protein phosphorylation, this process is reversible.

The GTP-binding proteins bind to other proteins to
control their activities, and their crucial role in intracellular signaling
pathways will be discussed in detail later.

\subsection{Nucleotide hydrolysis allows Motor proteins to produce large Movements in cells}

Conformational changes also play another important role in the operation of the cell: they enable
proteins whose major function is to move other molecules, the motor
proteins, to generate the forces responsible for muscle contraction and
many of the dramatic movements of cells.

How are shape changes in proteins used to generate orderly movements
in cells?

To make the series of conformational changes unidirectional - and force
the entire cycle to proceed in one direction - it is enough to make any
one of the steps irreversible. For most proteins that are able to walk in
a single direction for long distances, this irreversibility is achieved by
coupling one of the conformational changes to the hydrolysis of an ATP
molecule bound to the protein

Many motor proteins generate directional movement in this general way,
including the muscle motor protein myosin - which “runs” along actin
filaments to generate muscle contraction - and
the kinesin protein involved in chromosome movements at mitosis.

\subsection{Proteins often form large complexes That function as protein Machines}

It is clear that each central process in
a cell - such as DNA replication, protein synthesis, vesicle budding, and
transmembrane signaling - is catalyzed by a highly coordinated, linked
set of 10 or more proteins. In most such protein machines the hydrolysis
of bound nucleoside triphosphates (ATP or GTP) drives an ordered series
of conformational changes in some of the individual protein subunits,
enabling the ensemble of proteins to move coordinately.

\subsection{Covalent Modification controls the location and assembly of protein Machines}

It has recently become clear that
most protein machines form at specific sites in the cell and are activated
only when and where they are needed. This mobilization is generally
accomplished by the covalent addition of a modifying group to one or
more specific amino acid side chains on the participating proteins.

The set of covalent modifications that a protein contains at any moment
constitutes an important combinatorial regulatory protein code. The
attachment or removal of these modifying groups controls the behavior
of a protein, changing its activity or stability, its binding partners, or its
location inside the cell.

\section{Essential concepts}

\begin{itemize}
\item Living cells contain an enormously diverse set of protein molecules,
each made as a linear chain of amino acids covalently linked
together.
\item Each type of protein has a unique amino acid sequence that determines
both its three-dimensional shape and its biological activity.
\item The folded structure of a protein is stabilized by noncovalent interactions
between different parts of the polypeptide chain.
\item Hydrogen bonds between neighboring regions of the polypeptide
backbone often give rise to regular folding patterns, known as $\alpha$ helices and $\beta$ sheets.
\item The structure of many proteins can be subdivided into smaller globular
regions of compact three-dimensional structure, known as protein
domains.
\item The biological function of a protein depends on the detailed chemical
properties of its surface and how it binds to other molecules, called
ligands.
\item When a protein catalyzes the formation or breakage of covalent
bonds in a ligand, the protein is called an enzyme and the ligand is
called a substrate.
\item At the active site of an enzyme, the amino acid side chains of the
folded protein are precisely positioned so that they favor the formation
of the high-energy transition states that the substrates must
pass through to be converted to product.
\item The three-dimensional structure of many proteins has evolved so
that the binding of a small ligand can induce a significant change in
protein shape.
\item Most enzymes are allosteric proteins that can exist in two conformations
that differ in catalytic activity, and the enzyme can be turned
on or off by ligands that bind to a distinct regulatory site to stabilize
either the active or the inactive conformation.
\item The activities of most enzymes within the cell are strictly regulated.
One of the most common forms of regulation is feedback inhibition,
in which an enzyme early in a metabolic pathway is inhibited by its
binding to one of the pathway’s end products.
\item Many thousands of proteins in a typical eucaryotic cell are regulated
either by cycles of phosphorylation and dephosphorylation, or by the
binding and hydrolysis of GTP by a GTP-binding protein.
\item The hydrolysis of ATP to ADP by motor proteins produces directed
movements in the cell.
\item Highly efficient protein machines are formed by assemblies of allosteric
proteins in which conformational changes are coordinated to
perform complex cellular functions.
\item A regulatory protein code based on the covalent modification of multiple
amino acid side chains allows each cell to control the location
and assembly of its protein complexes.
\item Starting from crude cell homogenates, individual proteins can be
obtained in pure form by using a series of chromatography steps.
Purification allows the detailed properties of a protein to be revealed
by biochemical techniques and its exact three-dimensional structure
to be determined.
\end{itemize}

\chapter{DNA and Chromosomes}

\section{The structure and Function of DNA}

Well before biologists understood the structure of DNA, they had recognized
that inherited traits and the genes that determine them were
associated with the chromosomes.
We now know that the DNA carries the hereditary information of the
cell, and that the protein components of chromosomes function largely
to package and control the enormously long DNA molecules.

\subsection{A DNA molecule consists of Two complementary chains of nucleotides}

A molecule of deoxyribonucleic acid (DNA) consists of two long polynucleotide
chains. Each of these DNA chains, or DNA strands, is composed
of four types of nucleotide subunits, and the two chains are held together
by hydrogen bonds between the base portions of the nucleotides.
Nucleotides are composed of a five-carbon sugar to which are attached one or more phosphate
groups and a nitrogen-containing base. For the nucleotides in DNA, the
sugar is deoxyribose attached to a single phosphate group (hence the
name deoxyribonucleic acid); the base may be either adenine (A), cytosine (C),
guanine (G), or thymine (T). The nucleotides are covalently linked
together in a chain through the sugars and phosphates, which thus form
a “backbone” of alternating sugar-phosphate-sugar-phosphate.
These same symbols (A, C, G, and T) are also commonly
used to denote the four different nucleotides - that is, the bases with their
attached sugar and phosphate groups.

The way in which the nucleotide subunits are linked together gives a
DNA strand a chemical polarity. If we imagine that each nucleotide has a
knob (the phosphate) and a hole, each chain, formed by
interlocking knobs with holes, will have all of its subunits lined up in the
same orientation. Moreover, the two ends of the chain can be easily distinguished,
as one will have a hole (the 3` hydroxyl) and the other a knob
(the 5` phosphate). This polarity in a DNA chain is indicated by referring
to one end as the 3` end and the other as the 5` end.

All the bases are therefore on the inside of the helix, with the sugar-phosphate
backbones on the outside. The bases not pair at random, however: A always pairs with T,
and G always pairs with C. In each case, a bulkier two-ring base (a purine is paired
with a single-ring base (a pyrimidine). Each purine-pyrimidine pair is called a base pair,
and this complementary base-pairing
enables the base pairs to be packed in the energetically most favorable
arrangement in the interior of the double helix.

The members of each base pair can fit together within the double helix because the two
strands of the helix run antiparallel to each other - that is, they are oriented
with opposite polarities.
A consequence of the double helix base-pairing requirements is that
each strand of a DNA molecule contains a sequence of nucleotides
that is exactly complementary to the nucleotide sequence of its partner
strand - an A always matches a T on the opposite strand, and a C
always matches a G.

\subsection{The Structure of DNA Provides a mechanism for heredity}

Organisms differ
from one another because their respective DNA molecules have different
nucleotide sequences and, consequently, carry different biological messages.
It had already been established some time before the structure of DNA
was determined that genes contain the instructions for producing proteins.
The DNA messages, therefore, must somehow encode proteins. The function of a protein is
determined by its three-dimensional structure, and its structure in turn is
determined by the sequence of the amino acids in its polypeptide chain.
The linear sequence of nucleotides in a gene must therefore somehow
spell out the linear sequence of amino acids in a protein.
We will describe this code in detail in the course of explaining
the process, known as gene expression, through which a cell transcribes
the nucleotide sequence of a gene into the nucleotide sequence of an
RNA molecule, and then translates that information into the amino acid
sequence of a protein

The complete set of information in an organism’s DNA is called its genome
(the term is also used to refer to the DNA that carries this information).
The exact correspondence between the 4-letter nucleotide alphabet
of DNA and the 20-letter amino acid alphabet of proteins - the genetic
code - is not obvious from the structure of the DNA molecule.

\section{The structure of eucaryotic chromosomes}

In eucaryotic cells, very long double-stranded DNA molecules are packaged
into structures called chromosomes, which not only fit readily inside the
nucleus but can be easily apportioned between the two daughter cells at
each cell division. The complex task of packaging DNA is accomplished
by specialized proteins that bind to and fold the DNA, generating a series
of coils and loops that provide increasingly higher levels of organization
and prevent the DNA from becoming an unmanageable tangle.

\subsection{Eucaryotic DNA is Packaged into multiple chromosomes}

In eucaryotes, such as ourselves, the DNA in the nucleus is distributed
among a set of different chromosomes.
Each chromosome consists of a single, enormously long,
linear DNA molecule associated with proteins that fold and pack the fine
thread of DNA into a more compact structure. The complex of DNA and
protein is called chromatin. In addition to the proteins involved in packaging
the DNA, chromosomes are also associated with many other proteins
involved in gene expression, DNA replication, and DNA repair.

With the exception of the germ cells (sperm and eggs) and highly specialized
cells that lack DNA entirely (such as the mature red blood cell),
human cells each contain two copies of each chromosome, one inherited
from the mother and one from the father. The maternal and paternal
chromosomes of a pair are called homologous chromosomes (homologs).
The only nonhomologous chromosome pairs are the sex chromosomes
in males, where a Y chromosome is inherited from the father and an X
chromosome from the mother.

In addition to being different sizes, human chromosomes can be distinguished
from one another by a variety of techniques. Each chromosome
can be “painted” a different color using sets of chromosome-specific
DNA molecules coupled to different fluorescent dyes. This
involves the technique of DNA hybridization.

A display of the full set of 46 human chromosomes is called the human
karyotype. If parts of a chromosome are lost, or switched between chromosomes,
these changes can be detected by changes in the banding
patterns. Cytogeneticists use alterations in banding patterns to detect
chromosomal abnormalities.

\subsection{Chromosomes contain long Strings of genes}

The most important function of chromosomes is to carry the genes - the
functional units of heredity. A gene is usually defined as a
segment of DNA that contains the instructions for making a particular
protein (or, in some cases, a set of closely related proteins). Although
this definition fits the majority of genes, some genes direct the production
of an RNA molecule, instead of a protein, as their final product. Like
proteins, these RNA molecules perform a diverse set of structural and
catalytic functions in the cell, as we will see in later chapters.
As might be expected, some correlation exists between the complexity
of an organism and the number of genes in its genome.

In general, the more complex an organism, the larger is its genome.

\subsection{Chromosomes exist in different States Throughout the life of a cell}

To form a functional chromosome, a DNA molecule must do more than
simply carry genes: it must be able to be replicated, and the replicated
copies must be separated and partitioned reliably into daughter cells at
each cell division. These processes occur through an ordered series of
events, known collectively as the cell cycle.
Only two broad stages of the cell cycle need concern
us in this chapter: interphase, when chromosomes are duplicated; and
mitosis, when they are distributed to the two daughter nuclei.

During interphase, the chromosomes are extended as long, thin, tangled
threads of DNA in the nucleus and cannot be easily distinguished
in the light microscope. We refer to chromosomes in this extended state
as interphase chromosomes. Specialized DNA sequences found in all
eucaryotes ensure that the interphase chromosomes replicate efficiently
One type of nucleotide sequence acts as a replication origin, where duplication of the DNA begins.
Eucaryotic chromosomes contain many replication origins to ensure that
the entire chromosome can be replicated rapidly. Another DNA sequence
forms the telomeres found at each of the two ends of a chromosome.
Telomeres contain repeated nucleotide sequences that enable the ends
of chromosomes to be replicated. They also
cap the end of the chromosome, preventing it from being mistaken by the
cell as a broken DNA molecule in need of repair.

When the cell cycle reaches M phase, the DNA coils up, adopting a more
and more compact structure, ultimately forming highly compacted, or
condensed, mitotic chromosomes. In this highly condensed state, duplicated
chromosomes can be readily separated when the cell divides.
Once the chromosomes have condensed, it is the presence of the
third specialized DNA sequence, the centromere, that allows one copy
of each duplicated chromosome to be apportioned to each daughter cell.

\subsection{Interphase chromosomes are organized Within the nucleus}

The nucleus is surrounded by a nuclear envelope formed by two concentric
membranes. The nuclear envelope is punctured at intervals by
nuclear pores, which actively transport selected molecules to and from
the cytosol, and is supported by the
nuclear lamina, a network of protein filaments that forms a thin layer
underlying and the inner nuclear membrane.

Inside the nucleus, the interphase chromosomes - although longer and
finer than mitotic chromosomes - are nonetheless organized in various
ways. First, each interphase chromosome tends to occupy a particular
region of the nucleus, and so different chromosomes do not become
extensively entangled with one another.

The most obvious example of chromosome organization in the interphase
nucleus is the nucleolus. The nucleolus is where the
parts of the different chromosomes carrying genes for ribosomal RNA.
Here, ribosomal RNAs are synthesized and combined with proteins to
form ribosomes, the cell’s protein-synthesizing machines.

\subsection{Nucleosomes are the Basic units of eucaryotic chromosome Structure}

The proteins that bind to the DNA to form eucaryotic chromosomes
are traditionally divided into two general classes: the histones and the
nonhistone chromosomal proteins.
The complex of both classes of protein with nuclear DNA is called chromatin.
Histones are responsible for the first and most fundamental level of chromatin
packing, the nucleosome. When interphase nuclei are broken open very gently
and their contents examined under the electron microscope, most of the chromatin is in the form
of fibers. If this chromatin is subjected
to treatments that cause it to unfold partially, it can
then be seen under the electron microscope as a series of “beads on
a string”. The string is DNA, and each bead is a nucleosome
core particle that consists of DNA wound around a core of proteins formed from histones.

The structure of the nucleosome core particle was determined after first
isolating nucleosomes from the unfolded chromatin by digestion with
particular enzymes (called nucleases) that break down DNA by cutting
between the nucleotides. After digestion for a short period, only the
exposed DNA between the core particles, the linker DNA, is degraded.
An individual nucleosome core particle consists of a complex of eight
histone proteins - two molecules each of histones H2A, H2B, H3, and
H4 - and the double-stranded DNA, 147 nucleotide pairs long, that winds
around this histone octamer.

The formation of nucleosomes converts a DNA molecule
into a chromatin thread approximately one-third of its initial length, and
it provides the first level of DNA packing.

All four of the histones that make up the nucleosome core are relatively
small proteins with a high proportion of positively charged amino acids
(lysine and arginine). The positive charges help the histones bind tightly
to the negatively charged sugar–phosphate backbone of DNA.
Each of the core histones also has a long
N-terminal amino acid “tail,” which extends out from the nucleosome
core particle. These histone tails are subject to several
types of covalent chemical modifications that control many aspects of
chromatin structure.

The histones that form the nucleosome core are among the most highly
conserved of all known eucaryotic proteins.

\subsection{Chromosome Packing occurs on multiple levels}

Although long strings of nucleosomes form on most chromosomal DNA,
chromatin in the living cell rarely adopts the extended beads-on-a-string
form. Instead, the nucleosomes are further packed
upon one another to generate a more compact structure, the 30-nm
fiber.

Several models have been proposed to explain how nucleosomes are packed in the 30-nm
chromatic fiber; the one most consistent with the available data is a series of structural
variations known collectively as the zigzag model.

\section{The regulation of chromosome structure}

We discuss how a cell can alter its chromatin structure to
expose localized regions of DNA and allow access to specific proteins,
particularly those involved in gene expression and in DNA replication
and repair. We then discuss how chromatin structure is established and
maintained - and how a cell can pass on some forms of this structure
to its descendants.

\subsection{Changes in nucleosome Structure allow access to DNA}

Eucaryotic cells have several ways to adjust the local structure of their
chromatin rapidly. One way takes advantage of chromatin-remodeling
complexes, protein machines that use the energy of ATP hydrolysis to
change the position of the DNA wrapped around nucleosomes. By pushing
on the tightly bound DNA as they move along, these complexes can
loosen (decondense) the underlying DNA, making it more accessible to
other proteins in the cell. During mitosis, at least some of
the chromatin-remodeling complexes are inactivated, which may help
mitotic chromosomes maintain their tightly packed structure.

Another way of altering chromatin structure relies on the reversible chemical
modification of the histones. The tails of all four of the core histones
are particularly subject to these covalent modifications. For example,
acetyl, phosphate, or methyl groups can be added to and removed from
the assembled nucleosome by enzymes that reside in the nucleus. These
modifications of the histone tails have little direct effect on the stability of
an individual nucleosome. But some seem to directly affect the stability
of the 30-nm chromatin fiber and the higher-order structures discussed
earlier.

Most importantly, these modifications affect the ability of the histone tails
to bind specific proteins and thereby recruit them to particular stretches
of chromatin.

Like the chromatin-remodeling complexes, the enzymes that modify histone
tails are tightly regulated.

\subsection{Interphase chromosomes contain Both condensed and more extended Forms of chromatin}

Regions of the chromosome that contain
genes that are being expressed are generally more extended, while those
that contain quiescent genes are more compact. Thus, the detailed structure
of an interphase chromosome can differ from one cell type to the
next, helping to determine which genes are expressed.

The most highly condensed form of interphase chromatin is called heterochromatin
(from the Greek heteros, “different,” plus chromatin).

Heterochromatin typically makes up about 10\% of an interphase chromosome, and in mammalian
chromosomes, it is concentrated around the centromere region
and in the telomeres at the ends of the chromosomes.

Most DNA that is permanently folded into heterochromatin in the cell
does not contain genes. Because heterochromatin is so compact, genes
that accidentally become packaged into heterochromatin usually fail to
be expressed. Such inappropriate packaging of genes in
heterochromatin can cause disease.

Perhaps the most striking example of the use of heterochromatin to keep
genes shut down, or silenced, is found in the interphase X chromosomes of
female mammals.

The rest of the interphase chromatin is called euchromatin (from the
Greek eu, “true” or “normal,” plus chromatin).

\subsection{Changes in chromatin Structure can Be inherited}

The ability to inherit localized chromatin structure helps eucaryotic cells
to “remember” whether a gene was active in its parental cell, a process
that appears to be critical for the establishment and maintenance of different
cell types, tissues, and organs during the development and growth
of a complex multicellular organism. This type of inheritance does not
involve passing along specific DNA sequences from one cell generation to
the next, but instead depends on passing along specifically modified histone
proteins. It is an example of epigenetic inheritance (from the Greek
epi-, “on”), because it is superimposed on genetic inheritance based on
DNA.

\section{Essential concepts}

\begin{itemize}
\item Life depends on the stable and compact storage of genetic
information.
\item Genetic information is carried by very long DNA molecules and
encoded in the linear sequence of nucleotides A, T, G, and C.
\item Each molecule of DNA is a double helix composed of a pair of complementary
strands of nucleotides held together by hydrogen bonds
between G-C and A-T base pairs.
\item A strand of DNA has a chemical polarity due to the linkage of alternating
sugars and phosphates in its backbone. The two strands of the
DNA double helix run antiparallel - that is, in opposite orientations.
\item The genetic material of a eucaryotic cell is contained in a set of chromosomes,
each formed from a single, enormously long DNA molecule
that contains many genes.
\item When a protein-coding gene is expressed, part of its nucleotide
sequence is copied into RNA, which then directs the synthesis of a
specific protein.
\item The DNA that forms each eucaryotic chromosome contains, in addition
to genes, many replication origins, one centromere, and two
telomeres. These sequences ensure that the chromosome can be replicated
efficiently and passed on to daughter cells.
\item Chromosomes in eucaryotic cells consist of DNA tightly bound to a
mass of specialized proteins. These proteins fold the DNA into a compact form.
The complex of DNA and protein in chromosomes is called chromatin.
\item The most abundant chromosomal proteins are the histones, which
pack DNA into a repeating array of DNA-protein particles called
nucleosomes.
\item Nucleosomes pack together, with the aid of histone H1 molecules, to
form a 30-nm fiber. This fiber is generally coiled and folded, producing
more compact chromatin structures.
\item Chromatin structure is dynamic: by temporarily decondensing its
structure - using chromatin remodeling complexes and enzymes that
covalently modify histone tails - the cell can ensure that proteins
involved in gene expression, replication, and repair have rapid, localized
access to the necessary DNA sequences.
\item Some forms of chromatin have a pattern of histone tail modification
that causes the DNA to become so highly compacted that the packaged
genes cannot be expressed to produce RNA and protein.
\item Chromatin structure can be transmitted from one cell generation to
the next, producing a form of epigenetic inheritance that helps a cell
to remember the state of gene expression in its parent cell.
\end{itemize}

\chapter{DNA Replication, Repair and Recombination}

\section{DNA replication}

\subsection{Base-Pairing Enables DNA Replication}

We saw that each strand of the DNA double
helix contains a sequence of nucleotides that is exactly complementary to
the nucleotide sequence of its partner strand. Each strand can therefore
act as a template, or mold, for the synthesis of a new complementary
strand.

The ability of each strand of a DNA molecule to act as a template for
producing a complementary strand enables a cell to copy, or replicate,
its genes before passing them on to its descendants. But the task is awe-inspiring,
as it can involve copying billions of nucleotide pairs every time
a cell divides. This feat is performed by a cluster of proteins that together form
a replication machine.

DNA replication produces two complete double helices from the original
DNA molecule, each new DNA helix identical (except for rare copying
errors) in nucleotide sequence to the parental DNA double helix.
Because each parental strand serves as the template for one
new strand, each of the daughter DNA double helices ends up with one
of the original (old) strands plus one strand that is completely new; this
style of replication is said to be semiconservative.

\subsection{DNA Synthesis Begins at Replication Origins}

The process of DNA replication is begun by initiator proteins that bind
to the DNA and pry the two strands apart, breaking the hydrogen bonds
between the bases. Although the hydrogen bonds collectively
make the DNA helix very stable, individually each hydrogen bond is weak.

The positions at which the DNA is first opened are called replication
origins, and they are usually marked by a particular sequence of nucleotides.
An A-T base pair is held together by fewer hydrogen bonds than is a G-C base
pair. Therefore, DNA rich in A-T base pairs is relatively easy to pull apart,
and A-T-rich stretches of DNA are typically found at replication origins.
Once an initiator protein binds to DNA at the replication origin and
locally opens up the double helix, it attracts a group of proteins that carry
out DNA replication. These proteins form a protein machine, with each
member of the group carrying out a specific function.

\subsection{New DNA Synthesis Occurs at Replication Forks}

DNA molecules in the process of being replicated contain Y-shaped junctions
called replication forks. Two replication forks are formed at each replication origin, and
they move away from the origin in opposite directions, unzipping the DNA
as they go. DNA replication in bacterial and eucaryotic chromosomes
is therefore termed bidirectional.

At the heart of the replication machine is an enzyme called DNA
polymerase, which synthesizes new DNA using one of the old strands
as a template. This enzyme catalyzes the addition of nucleotides to the 3`
end of a growing DNA strand by forming a phosphodiester bond between
this end and the 5`-phosphate group of the incoming nucleotide.
Nucleotides enter the reaction initially as nucleoside triphosphates,
which provide the energy for polymerization. The hydrolysis of one high-energy
bond in the nucleoside triphosphate fuels the reaction that links
the nucleotide monomer to the chain and releases pyrophosphate ($PP_{i}$).
The DNA polymerase couples the release of this energy to the polymerization
reaction. Pyrophosphate is further hydrolyzed to inorganic
phosphate $P_i$), which makes the polymerization reaction effectively irreversible.

DNA polymerase does not dissociate from the DNA each time it adds a chain
new nucleotide to the growing rather, it stays associated with the
DNA and moves along the template strand stepwise for many cycles of
the polymerization reaction.

\subsection{The Replication Fork Is asymmetrical}

The 5`-to-3` direction of the DNA polymerization mechanism poses a
problem at the replication fork.
The two strands in the double helix run in
opposite orientations. As a consequence, at the replication fork, one new
DNA strand is being made on a template that runs in one direction (3`
to 5`), whereas the other new strand is being made on a template that
runs in the opposite direction (5` to 3`). The replication fork is therefore
asymmetrical.

DNA polymerase, however, can catalyze the growth of the DNA chain in
only one direction: it can add new subunits only to the 3` end of the chain.

As a result, a new DNA chain can be synthesized only
in a 5`-to-3` direction. One
might have expected a second type of DNA polymerase to synthesize the
other DNA strand - one that works by adding subunits to the 5` end of
a DNA chain. However, no such enzyme exists. Instead, the problem is
solved by the use of a ‘backstitching’ maneuver. The DNA strand whose
5` end must grow is made discontinuously, in successive separate small
pieces, with the DNA polymerase moving backward with respect to the
direction of the replication fork, as each new piece is made in the 5`-to-3`
direction.
The resulting small DNA pieces - called Okazaki fragments after the
biochemist who discovered them - are later joined together to form a
continuous new strand. The DNA strand that is synthesized
discontinuously in this way is called the lagging strand; the other strand,
which is synthesized continuously, is called the leading strand.
Although they differ in subtle details, the replication forks of all cells,
procaryotic and eucaryotic, have leading and lagging strands

\subsection{DNA Polymerase Is self-correcting}

Although A-T and C-G are by far the most stable base pairs, other, less
stable base pairs - for example, G-T and C-A - can also be formed. Such
incorrect base pairs are formed much less frequently than correct ones,
but if allowed to remain, they would kill the cell through an accumulation
of mutations. This catastrophe is avoided because DNA polymerase
has two special qualities that greatly increase the accuracy of DNA replication.
First, the enzyme carefully monitors the base-pairing between
each incoming nucleotide and the template strand.
When DNA polymerase makes a rare mistake and adds
the wrong nucleotide, it can correct the error through an activity called
proofreading.

Proofreading takes place at the same time as DNA synthesis. Before
the enzyme adds the next nucleotide to a growing DNA chain, it checks
whether the previous nucleotide is correctly base-paired to the template
strand. This proofreading mechanism explains why DNA polymerases synthesize
DNA only in the 5`-to-3` direction, despite the need that this imposes for a
cumbersome backstitching mechanism at the replication fork.

A hypothetical DNA polymerase that synthesized in the
3`-to-5` direction (and would thereby circumvent the need for backstitching)
would be unable to proofread: if it removed an incorrectly paired
nucleotide, the polymerase would create a chain end that is chemically
dead, in the sense that it could no longer be elongated.

\subsection{Short Lengths of Rna act as Primers for dna Synthesis}

Because the polymerase can join a nucleotide
only to a base-paired nucleotide in a DNA double helix, it cannot start
a completely new DNA strand. A different enzyme is needed - one that
an begin a new polynucleotide chain simply by joining two nucleotides
together without the need for a base-paired end. This enzyme does not,
however, synthesize DNA. It makes a short length of a closely related type
of nucleic acid - RNA (ribonucleic acid) - using the DNA strand as a template.
This short length of RNA, about 10 nucleotides long, is base-paired
to the template strand and provides a base-paired 3` end as a starting
point for DNA polymerase. It thus serves as a primer for DNA synthesis,
and the enzyme that synthesizes the RNA primer is known as primase.

Primase is an example of an RNA polymerase, an enzyme that synthesizes
RNA using DNA as a template. A strand of RNA is very similar chemically
to a single strand of DNA except that it is made of ribonucleotide
subunits, in which the sugar is ribose, not deoxyribose; RNA also differs
from DNA in that it contains the base uracil (U) instead of thymine (T)

To produce a continuous new DNA strand from the many separate pieces
of nucleic acid made on the lagging strand, three additional enzymes are
needed. These act quickly to remove the RNA primer, replace it with DNA,
and join the DNA fragments together. Thus, a nuclease breaks apart the
RNA primer, a DNA polymerase called a repair polymerase then replaces
this RNA with DNA (using the end of the adjacent Okazaki fragment as
a primer), and the enzyme DNA ligase joins the 5`-phosphate end of one
new DNA fragment to the adjacent 3`-hydroxyl end of the next.

Primase can begin new polynucleotide chains, but this activity is possible
because the enzyme does not proofread its work. As a result, primers
contain a high frequency of mistakes.

\subsection{Proteins at a Replication Fork Cooperate to Form a Replication Machine}

As mentioned earlier, DNA replication requires a variety of proteins acting in concert.

At the very front of the replication
machine is the helicase, a protein that uses the energy of ATP hydrolysis
to pry apart the double helix as it speeds along the DNA.
The single-strand binding protein clings to the single-stranded
DNA exposed by the helicase, transiently preventing it from re-forming
base pairs and keeping it in an elongated form so that it can readily serve
as a template for DNA polymerase.

An additional replication protein, called a sliding clamp, keeps the
polymerase firmly attached to the template while it is synthesizing new
strands of DNA. Left on their own, most DNA polymerase molecules will
synthesize only a short string of nucleotides before falling off the DNA
template. The sliding clamp forms a ring around the DNA helix and, by
tightly gripping the polymerase, allows it to move along the template
strand without falling off as it synthesizes new DNA

Assembly of the clamp around DNA requires the activity of another
replication protein, the clamp loader, which hydrolyzes
ATP each time it locks a clamp around the DNA.

Most of the proteins involved in DNA replication are held together in a
large multienzyme complex that moves as a unit along the DNA, enabling
DNA to be synthesized on both strands in a coordinated manner.
This complex can be likened to a miniature sewing machine composed of
protein parts and powered by nucleoside triphosphate hydrolysis.

\subsection{Telomerase Replicates the Ends of Eucaryotic Chromosomes}

We now turn to the special problem of replicating the very ends of chromosomes. As we discussed previously,
the fact that DNA is synthesized only in the 5`-to-3` direction means that
the lagging strand of the replication fork is synthesized in the form of discontinuous
DNA fragments, each of which is primed with an RNA primer laid down by a separate enzyme.
When the replication fork approaches the end of a chromosome, however, the replication
machinery encounters a serious problem: there is no place to lay down
the RNA primer needed to start the Okazaki fragment at the very tip of
the linear DNA molecule. Without a strategy to deal with this problem,
some DNA will inevitably be lost from the ends of a DNA molecule each
time it is replicated.

Eucaryotes solve it by having special nucleotide
sequences at the ends of their chromosomes which are incorporated
into telomeres. These telomeric DNA sequences attract an enzyme called
telomerase to the chromosome. Using an RNA template that is part of
the enzyme itself, telomerase replenishes the nucleotides that are lost
each time a eucaryotic chromosome is duplicated by adding multiple
copies of the same short DNA sequence to the chromosome ends. This
extended, repetitive DNA sequence then acts as a template that allows
replication of the lagging strand to be completed by conventional DNA
replication


\section{DNA repair}

\subsection{Mutations Can Have Severe Consequences for a Cell or Organism}

Only rarely do the cell’s DNA replication and repair processes fail and
allow a permanent change in the DNA. Such permanent changes are
called mutations, and they can have profound consequences. A mutation
that affects just a single nucleotide pair can severely compromise
an organism’s fitness if the change occurs in a vital position in the DNA
sequence. Because the structure and activity of each protein depend on
its amino acid sequence, a protein with an altered sequence may function
poorly or not at all.

This illustrates the importance of protecting reproductive cells (germ cells) against
mutation. A mutation in a germ cell will be passed on to all the cells in
the body of the multicellular organism that develops from it, including the
germ cells for production of the next generation.
The many other cells in a multicellular organism (its somatic cells) must
be protected from the genetic changes that arise during the life of an
individual.

\subsection{A DNA Mismatch Repair System Removes Replication Errors That Escape the Replication Machine}

In the first part of this chapter, we saw that the high fidelity of the cell’s
replication machinery generally prevents mistakes in copying. Despite
these safeguards, however, such errors do occur. Fortunately, the cell
has a backup system - called DNA mismatch repair - which is dedicated
to correcting these rare mistakes.

Whenever the replication machinery makes a copying mistake, it leaves
a mispaired nucleotide (commonly called a mismatch) behind. If left
uncorrected, the mismatch will result in a permanent mutation in the
next round of DNA replication. A complex of mismatch
repair proteins recognizes these DNA mismatches, removes (excises) one
of the two strands of DNA involved in the mismatch, and resynthesizes
the missing strand. To be effective in correcting replication
mistakes, this mismatch repair system must always excise only the newly
synthesized DNA strand: excising the other strand (the old strand) would
double the mistake instead of correcting it.

\subsection{DNA Is Continually Suffering damage in Cells}
DNA can be damaged in many ways, and these require different mechanisms for their repair. Just
like any other molecule in the cell, DNA is continually undergoing thermal collisions with other molecules. These often result in major chemical
changes in the DNA. For example, during the time it takes to read this
sentence, a total of about a trillion ($10^12$) purine bases (A and G) will be
lost from DNA in the cells of your body by a spontaneous reaction called
depurination.
Another common reaction is the spontaneous loss of an amino
group (deamination) from cytosine in DNA to produce the base uracil.

Some chemically reactive by-products of cell metabolism
also occasionally react with the bases in DNA, altering them in such a
way that their base-pairing properties are changed. The ultraviolet radiation
in sunlight is also damaging to DNA; it promotes covalent linkage
between two adjacent pyrimidine bases, forming, for example, the thymine dimer.

\subsection{The Stability of Genes depends on DNA Repair}

The basic pathway for repairing damage to DNA involves three steps:

\begin{enumerate}
\item The damaged DNA is recognized and removed by one of a variety
of different mechanisms. These involve nucleases, which cleave
the covalent bonds that join the damaged nucleotides to the rest of
the DNA molecule, and they leave a small gap on one strand of the
DNA double helix in the region.
\item A repair DNA polymerase binds to the 3`-hydroxyl end of the cut
DNA strand. It then fills in the gap by making a complementary
copy of the information stored in the undamaged strand. Although
different from the DNA polymerase that replicates DNA, repair
DNA polymerases synthesize DNA strands in the same way. For
example, they elongate chains in the 5`-to-3` direction and have
the same type of proofreading activity to ensure that the template
strand is copied accurately. In many cells, this is the same enzyme
that fills in the gap left after the RNA primers are removed during
the normal DNA replication process.
\item When the repair DNA polymerase has filled in the gap, a break
remains in the sugar-phosphate backbone of the repaired strand.
This nick in the helix is sealed by DNA ligase, the same enzyme that
joins the lagging-strand DNA fragments during DNA replication.
\end{enumerate}

Cells also have an alternative, error-free strategy for repairing double-strand
breaks, particularly those that occur in newly replicated DNA. This
mechanism, called homologous recombination, is discussed in the next
section of the chapter.

\section{Homologous recombination}

What happens to genetic information when both members of a
nucleotide pair are damaged simultaneously - for example, when a
double-strand break occurs? As we saw earlier, one approach to fixing
such damage is to use nonhomologous end-joining to rapidly heal the
wound. However, that mechanism usually sacrifices the information
contained at the site of the injury. A more elegant solution is to use the
genetic information provided by an entirely separate DNA duplex to repair
the break accurately. This strategy is carried out by a set of reactions
collectively known as homologous recombination. Its central feature
is the exchange of genetic information between a pair of homologous
DNA molecules: that is, DNA duplexes that are similar or identical
in nucleotide sequence. In this process, the information present in an
intact, undamaged DNA duplex is used as a template to accurately repair
a broken DNA double helix.

In addition to its role in repair, homologous recombination is also responsible
for generating genetic diversity during meiosis, the specialized form
of cell division by which sexually reproducing organisms make germ cells.
In this case, homologous recombination physically swaps genetic information
between the homologous maternal and paternal chromosomes to
produce chromosomes with novel DNA sequences.

\subsection{Homologous Recombination Requires Extensive Regions of Sequence Similarity}

Whether the end product is DNA repair or the exchange of nucleotide
sequences during meiosis, the hallmark of homologous recombination
is that it takes place only between DNA duplexes that contain extensive
regions of sequence similarity (homology). A pair of DNA molecules can
assess this homology by ‘sampling’ each other’s nucleotide sequences
when a single strand from one DNA duplex engages in an extensive bout
of base-pairing with the complementary strand from the other duplex.
The match need not be perfect for homologous recombination to succeed, but it must be very close.


\subsection{Homologous Recombination Can Flawlessly Repair DNA double-strand Breaks}

Homologous recombination is often initiated when a double-strand break
occurs shortly after a stretch of DNA has been replicated; at that time,
the duplicated helices are still in close proximity to one another.
To begin the repair, a nuclease generates single-stranded ends
at the break by chewing back one of the complementary DNA strands.
With the help of specialized enzymes, one of these single strands then ‘invades’
the homologous DNA duplex by forming base
pairs with its complementary strand. If this sampling results in extensive
base pairing, a branch point is created where the two DNA strands - one
from each duplex-cross. At this point, the invading strand
is elongated by a repair DNA polymerase, using the complementary
strand as a template. The branch point then ‘migrates’
as the base pairs holding together the duplexes break, and new ones
form. Repair is completed by additional DNA synthesis,
followed by DNA ligation. The net result is two intact DNA
helices, where the genetic information from one was used as a template
to repair the other.

\subsection{Homologous Recombination Exchanges Genetic Information during Meiosis}

Homologous recombination during meiosis begins with a bold stroke:
a specialized enzyme deliberately slices through both strands of one
of the recombining chromosomes, creating a double-strand break. At
this point, some of the same proteins that function in recombinational
double-strand break repair converge on the ‘damage.’ However, these
recombination proteins are now directed by meiosis-specific proteins to
perform their tasks differently, producing - via one or more crossovers -
two molecules with novel DNA sequences.

Crossing-over during meiosis ensures that each of our chromosomes
contains a combination of DNA sequences from our two parents.

\section{Mobile genetic elements and viruses}

genomes are also subject to more dramatic forms of genetic variation
- changes that alter the order of genes on a chromosome or even
add new information. It is this more radical form of genetic resculpturing
we turn to next. The agents underlying these drastic genetic changes
are mobile genetic elements, sometimes known informally as jumping these
genes. Found in virtually all cells,
elements are short, specialized
segments of DNA that can move from one position in the cell’s genome to
another. Although they can insert themselves into virtually any sequence
within the genome, most mobile genetic elements lack the ability to leave
the cell in which they reside. Their movement is therefore restricted to a
single cell and its descendants.

\subsection{Mobile Genetic Elements Encode the Components They need for Movement}

Unlike homologous recombination, the movement of mobile genetic elements
does not require DNA sequence similarity. Instead, each element
typically encodes a specialized recombination enzyme that mediates its
movement. These enzymes recognize and act on unique
DNA sequences that are present on each mobile genetic element. Many
mobile genetic elements also carry other genes.

Mobile genetic elements are also called transposons and are typically
classified according to the mechanism by which they move or transpose.
In bacteria, the most common mobile genetic elements are the DNA-only
transposons. The name is derived from the fact that, during their movement,
the element remains as DNA, rather than being converted to RNA as
for other elements that we discuss below.

\subsection{The Human Genome Contains Two Major Families of Transposable Sequences}

Amazingly, nearly half of the human genome is made up of millions of
copies of various mobile genetic elements, which form a very large part
of our DNA. Some of these elements have moved from place to place
within the human genome using the cut-and-paste mechanism discussed
above for DNA-only transposons. However, most have
moved not as DNA but via an RNA intermediate. These are called retrotransposons
and are, as far as is known, unique to eucaryotes.
One abundant human retrotransposon, the L1 element (sometimes referred
to as LINE-1), is transcribed into RNA by the host cell’s RNA polymerases.
A DNA copy of this RNA is then made using an enzyme called reverse
transcriptase, an unusual DNA polymerase that can use RNA as a template.
The reverse transcriptase is encoded by the L1 element itself.

L1 elements constitute about 15\% of the human genome. Although most
copies are immobile due to the accumulation of deleterious mutations, a
few still retain the ability to transpose. Their movement can sometimes
result in human disease

Another type of retrotransposon, the Alu sequence, is present in about
1 million copies in our genome. Alu elements do not encode their own
reverse transcriptase and thus depend on enzymes already present in the
cell to help them move.

\subsection{Viruses are Fully Mobile Genetic Elements That Can Escape from Cells}

Viruses were first noticed as disease-causing agents that, by virtue of
their tiny size, passed through ultrafine filters that can hold back even the
smallest bacterial cell. We now know that viruses are essentially genes
encoled by a protective coat.
However, these genes must enter a
cell and use the cell’s machinery to express their genes to make proteins
and to replicate their chromosomes. Virus reproduction is often lethal to
the cells in which it occurs; in many cases the infected cell breaks open
(lyses), releasing the progeny viruses, which can then attack neighboring
cells.

Viral genomes can be made of DNA or RNA and can be single-stranded
or double-stranded. The amount of DNA or RNA that can be
packaged inside a protein shell is limited, and it is too small to encode the
many different enzymes and other proteins that are required to replicate
even the simplest virus. For this reason, viruses must hijack their host’s
biochemical machinery to reproduce themselves.

\subsection{Retroviruses Reverse the normal Flow of Genetic Information}

Although there are many similarities between bacterial and eucaryotic
viruses, one important type of virus - the retrovirus - is found only in
eucaryotic cells. In many respects, retroviruses resemble the retrotransposons
we discussed earlier. A key feature of the life cycle of both these
genetic elements is a step in which DNA is synthesized using RNA as a
template. The retro prefix refers to the reversal of the usual flow of DNA
information to RNA and the enzyme that carries out this
step is a reverse transcriptase. The retroviral genome (which is single-stranded
RNA) encodes this enzyme, and a few molecules of the enzyme
are packaged along with the RNA genome in each virus particle.

When the single stranded RNA genome of the retrovirus enters a cell, the reverse
transcriptase brought in with it makes a complementary DNA strand
to form a DNA/RNA hybrid double helix. The RNA strand is removed,
and the reverse transcriptase (which can use either DNA or RNA as a
template) now synthesizes a complementary strand to produce a DNA
double helix. This DNA is then inserted, or integrated, into a randomly
selected site in the host genome by a virally encoded integrase enzyme.
In this state, the virus is latent: each time the host cell divides, it passes
on a copy of the integrated viral genome, which is known as a provirus,
to its progeny cells.

The next step in the replication of a retrovirus - which can take place
long after its integration into the host genome - is the copying of the
integrated viral DNA into RNA by a host-cell RNA polymerase, which
produces large numbers of single-stranded RNAs identical to the original
infecting genome. The viral genes are then expressed by the host
cell machinery to produce the protein shell, the envelope proteins, and
reverse transcriptase - all of which are assembled with the RNA genome
into new virus particles.

\section{Essential concepts}

\begin{itemize}
\item The ability of a cell to maintain order in a chaotic environment
depends on the accurate duplication of the vast quantity of genetic
information carried in its DNA.
\item Each of the two DNA strands can act as a template for the synthesis
of the other strand. A DNA double helix thus carries the same information
in each of its strands.
\item A DNA molecule is duplicated (replicated) by the polymerization of
new complementary strands using each of the old strands of the DNA
double helix as a template. Two identical DNA molecules are formed,
enabling the genetic information to be copied and passed on from
cell to daughter cell and from parent to offspring.
\item As a DNA molecule replicates, its two strands are pulled apart to form
one or more Y-shaped replication forks. DNA polymerase enzymes,
situated at the fork, produce a new complementary DNA strand on
each parental strand.
\item DNA polymerase replicates a DNA template with remarkable fidelity,
making less than one error in every 10 7 bases read. This accuracy
is made possible, in part, by a proofreading process in which the
enzyme removes its own polymerization errors as it moves along the
DNA.
\item Because DNA polymerase synthesizes new DNA in only one direction,
only the leading strand at the replication fork can be synthesized in
a continuous fashion. On the lagging strand, DNA is synthesized in
a discontinuous backstitching process, producing short fragments of
DNA that are later joined by the enzyme DNA ligase to complete that
DNA strand.
\item DNA polymerase is incapable of starting a new DNA chain from
scratch. Instead, DNA synthesis is primed by an RNA polymerase
called primase, which makes short lengths of RNA (primers) that are
elongated by DNA polymerase. The primers are subsequently erased
and replaced with DNA.
\item DNA replication requires the cooperation of many proteins; these form
a multienzyme replication machine that catalyzes DNA synthesis.
\item In eucaryotes, a special enzyme called telomerase replicates the DNA
at the ends of the chromosomes.
\item The rare copying mistakes that slip through the DNA replication
machinery are dealt with by the mismatch repair proteins. The overall
accuracy of DNA replication, including mismatch repair, is one
mistake per 10 9 nucleotides copied.
\item DNA damage caused by unavoidable chemical reactions is repaired
by a variety of enzymes that recognize damaged DNA and excise a
short stretch of the DNA strand that contains it. The missing DNA is
resynthesized by a repair DNA polymerase that uses the undamaged
strand as a template.
\item Nonhomologous end-joining allows the rapid repair of double-strand
DNA breaks; the process often alters the DNA sequence at the site of
the repair.
\item Homologous recombination can faithfully repair double-strand DNA
breaks using a homologous chromosome sequence as a guide. During
meiosis, a related homologous recombination process causes a shuffling
of genetic information that creates DNA molecules with novel
sequences.
\item Mobile genetic elements, or transposons, move from place to place in
the genomes of their hosts, providing a source of genetic variation.
\item Nearly half of the human genome consists of mobile genetic elements.
Two classes of these elements have multiplied to especially
high copy numbers.
\item Viruses are little more than genes packaged in protective protein
coats. They require host cells to reproduce themselves.
\item Some viruses have RNA instead of DNA as their genomes. One group
of RNA viruses - the retroviruses - must copy their RNA genomes into
DNA to replicate.
\end{itemize}
