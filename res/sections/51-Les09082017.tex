\chapter{Introduction to Cells}

All living things are made of cells: small, membrane-enclosed units
filled with a concentrated aqueous solution of chemicals and endowed
with the extraordinary ability to create copies of themselves by growing
and dividing in two.
Cells, therefore, are the fundamental units of life, and it is to cell biology
that we must look for an answer to the question of what life is and how it
works.

\section{Unity and Diversity of Cells}

Cells are not all alike; in fact, they can be wildly different.

\subsection{Cells vary enormously in appearance and function}

A bacterial cell is a few micrometers in length. Cells vary no less widely
their shapes and functions.
Some cells are clad only in a flimsy membrane; others augment this delicate
cover by cloaking themselves in an outer layer of slime, building
themselves rigid cell walls, or surrounding themselves with a hard, mineralized
material, such as that found in bone.
Cells are also enormously diverse in their chemical requirements and
activities.
Some modifications specialize a cell so much that they spoil its chances
of leaving any descendants. Such specialization would be senseless for a
cell that lived a solitary life. In a multicellular organism, however, there is
a division of labor among cells, allowing some cells to become specialized
to an extreme degree for particular tasks and leaving them dependent on
their fellow cells for many basic requirements.

\subsection{living Cells all have a similar Basic Chemistry}

While it seemed easy enough to recognize life, it was remarkably difficult to
say in what sense all living things were alike.
We now know that cells resemble one another to an astonishing
degree in the details of their chemistry, sharing the same machinery
for the most basic functions. All cells are composed of the same sorts of
molecules that participate in the same types of chemical reactions.
In all living things, genetic instructions - genes - are
stored in DNA molecules, written in the same chemical code, constructed
out of the same chemical building blocks, interpreted by essentially the
same chemical machinery, and duplicated in the same way to allow the
organism to reproduce. Thus, in every cell, the long DNA polymer chains
are made from the same set of four monomers, called nucleotides, strung
together in different sequences like the letters of an alphabet to convey
different information. In every cell, the instructions in the DNA are read
out, or transcribed, into a chemically related set of polymers called RNA.
RNA molecules have a variety of functions, but the major
class serve as messenger RNA: the messages carried by these molecules
are in turn translated into yet another type of polymer called a protein.

Protein molecules dominate the behavior of the cell, serving as structural
supports, chemical catalysts, molecular motors, and so on. Proteins
are built from amino acids, and every living thing uses the same set of
20 amino acids to make its proteins. But the amino acids are linked in
different sequences, giving each type of protein molecule a different
three-dimensional shape, or conformation, just as different sequences
of letters spell different words.

If cells are the fundamental unit of living matter, then nothing less than a
cell can truly be called living.

\subsection{All present-day Cells have apparently evolved from the same ancestor}

A cell reproduces by duplicating its DNA and then dividing in two, passing
a copy of the genetic instructions encoded in its DNA to each of its daughter
cells. That is why daughter cells resemble the parent cell. However,
the copying is not always perfect, and the instructions are occasionally
corrupted by mutations that change the DNA. That is why daughter cells
do not always match the parent cell exactly.
Mutations can create offspring that are changed for the worse (in that
they are less able to survive and reproduce), changed for the better (in
that they are better able to survive and reproduce), or changed neutrally
(in that they are genetically different but equally viable). The struggle
for survival eliminates the first, favors the second, and tolerates the
third. The genes of the next generation will be the genes of the survivors.
Intermittently, the pattern of descent may be complicated by sexual
reproduction, in which two cells of the same species fuse, pooling their
DNA; the genetic cards are then shuffled, re-dealt, and distributed in new
combinations to the next generation, to be tested again for their survival
value.
These simple principles of genetic change and selection, applied repeatedly
over billions of cell generations, are the basis of evolution - the
process by which living species become gradually modified and adapted
to their environment in more and more sophisticated ways.

\subsection{Genes provide the instructions for Cellular form, function, and Complex Behavior}

A cell’s genome - that is, the entire library of genetic information in its
DNA - provides a genetic program that instructs the cell how to function,
and, for plant and animal cells, how to grow into an organism with
hundreds of different cell types.
Yet all these differentiated cell types are generated during
embryonic development from a single fertilized egg cell, and all contain
identical copies of the DNA of the species.
Different
cells express different genes - that is, they use their genes to produce
some proteins and not others, depending on the cues that they and their
ancestor cells have received from their surroundings.
Each cell is capable of carrying out a variety of biological
tasks, depending on its environment and its history, using the information
encoded in its DNA to guide its activities.

\section{Cells under the microscope}

Cells, in general, are very small - too small to be seen with the naked
eye. They were not made visible until the seventeenth century, when the
microscope was invented. For hundreds of years afterward, all that was
known about cells was discovered using this instrument. Light microscopes,
which use visible light to illuminate specimens, are still vital
pieces of equipment in the cell biology laboratory.
Although these instruments now incorporate many sophisticated
improvements, the properties of light itself set a limit to the fineness of
detail they can reveal. Electron microscopes go
beyond this limit by using beams of electrons instead of beams of light
as the source of illumination, greatly extending our ability to see the fine
details of cells and even making some of the larger molecules visible
individually.

\subsection{The invention of the light Microscope led to the discovery of Cells}

Schleiden and Schwann documented the results
of a systematic investigation of plant and animal tissues with the light
microscope, showing that cells were the universal building blocks of all
living tissues. Their work, and that of other nineteenth-century microscopists,
slowly led to the realization that all living cells are formed by
the division of existing cells - a principle sometimes referred to as the
cell theory. The implication that living organisms do not arise
spontaneously but can be generated only from existing organisms was
hotly contested, but it was finally confirmed by experiments performed in
the 1860s by Louis Pasteur.

\subsection{Cells, organelles, and even Molecules Can Be seen Under the Microscope}

If you cut a very thin slice of a suitable plant or animal tissue and place
it under a light microscope, you will see that the tissue is divided into
thousands of small cells. These may be either closely packed or separated
from one another by an extracellular matrix, a dense material often made
of protein fibers embedded in a polysaccharide gel.
If you have taken care
to keep your specimen under the right conditions, you will see that the
cells show signs of life: particles move around inside them, and if you
watch patiently you may see a cell slowly change shape and divide into
two.

The cell thus revealed has a distinct anatomy. It has a sharply
defined boundary, indicating the presence of an enclosing membrane.
A large, round body, the nucleus, is prominent in the middle of the cell.
Around the nucleus and filling the cell’s interior lies the cytoplasm, a
transparent substance crammed with what seems at first to be a jumble
of miscellaneous objects.

When thin sections are cut, stained, and placed in the electron microscope,
much of the jumble of cell components becomes sharply resolved
into distinct organelles-separate, recognizable substructures that are
only hazily defined under the light microscope. A delicate membrane,
about 5 nm thick, is visible enclosing the cell, and similar membranes
form the boundary of many of the organelles inside. The
external membrane is called the plasma membrane, while the membranes
surrounding organelles are called internal membranes.

\section{The Procaryotic Cell}

Of all the types of cells revealed by the microscope, bacteria have the
simplest structure and come closest to showing us life stripped down to
its essentials. Indeed, a bacterium contains essentially no organelles - not
even a nucleus to hold its DNA. This property - the presence or absence of
a nucleus - is used as the basis for a simple but fundamental classification
of all living things. Organisms whose cells have a nucleus are called
eucaryotes (from the Greek words eu, meaning “well” or “truly,” and
karyon, a “kernel” or “nucleus”). Organisms whose cells do not have a
nucleus are called procaryotes (from pro, meaning “before”). The terms
“bacterium” and “procaryote” are often used interchangeably, although
we shall see that the category of procaryotes also includes another class
of cells, the archaea (singular archaeon), which are so remotely related to
bacteria that they are given a separate name.
They often have a tough protective coat, called a cell wall, surrounding the plasma
membrane, which encloses a single compartment containing the cytoplasm and the DNA.

\subsection{Procaryotes are the Most diverse of Cells}

In shape and structure, procaryotes may seem simple and limited,
but in terms of chemistry, they are the most diverse and inventive class
of cells.
As we discuss later in this chapter, mitochondria - the organelles that
generate energy for the eucaryotic cell - are thought to have evolved from
aerobic bacteria that took to living inside the anaerobic ancestors of
today’s eucaryotic cells.

Virtually any organic material, from wood to petroleum, can be used as
food by one sort of bacterium or another. Even more remarkably, some
procaryotes can live entirely on inorganic substances.
Some of these procaryotic cells, like plant cells, perform photosynthesis,
getting energy from sunlight; others derive energy from the chemical
reactivity of inorganic substances in the environment.

It is almost certain that the organelles in the plant cell that
perform photosynthesis - the chloroplasts - have evolved from photosynthetic
bacteria that found a home inside the plant cell’s cytoplasm.

\subsection{The World of procaryotes is divided into two domains: Bacteria and archaea}

There is a gulf within the class of procaryotes, dividing it into two distinct
domains called the bacteria (or sometimes eubacteria) and the archaea.
Remarkably, at a molecular level, the members of these two domains differ as
much from one another as either does from the eucaryotes.
Archaea are found not only in these habitats, but also
in environments hostile to most other cells.

\section{The Eucaryotic Cell}

Eucaryotic cells, in general, are bigger and more elaborate than bacteria
and archaea. All of the more complex multicellular organisms - including
plants, animals, and fungi - are formed from eucaryotic cells.

By definition, all eucaryotic cells have a nucleus.

\subsection{The nucleus is the information store of the Cell}

The nucleus is usually the most prominent organelle in a eucaryotic cell.
It is enclosed within two concentric membranes that form
the nuclear envelope, and it contains molecules of DNA-extremely long
polymers that encode the genetic information of the organism. In the
light microscope, these giant DNA molecules become visible as individual
chromosomes when they become more compact as a cell prepares to
divide into two daughter cells.

\subsection{Mitochondria generate Usable energy from food to power the Cell}

Mitochondria are present in essentially all eucaryotic cells, and they are
among the most conspicuous organelles in the cytoplasm.
These organelles have a very distinctive structure when seen with an
electron microscope: each mitochondrion appears sausage - or wormshaped,
from one to many micrometers long; and each is enclosed in
two separate membranes. The inner membrane is formed into folds that
project into the interior of the mitochondrion. Mitochondria
contain their own DNA and reproduce by dividing in two.

This revealed that mitochondria are generators of
chemical energy for the cell. They harness the energy from the oxidation
of food molecules, such as sugars, to produce adenosine triphosphate,
or ATP - the basic chemical fuel that powers most of the cell’s activities.
Because the mitochondrion consumes oxygen and releases carbon
dioxide in the course of this activity, the entire process is called cellular
respiration - essentially, breathing on a cellular level.

\subsection{Chloroplasts Capture energy from sunlight}

Chloroplasts are large green organelles that are found only in the cells
of plants and algae, not in the cells of animals or fungi. These organelles
have an even more complex structure than mitochondria: in addition to
their two surrounding membranes, chloroplasts possess internal stacks
of membranes containing the green pigment chlorophyll.

plants can get their energy directly from sunlight, and chloroplasts are
the organelles that enable them to do so.
From the standpoint of life on Earth, chloroplasts carry out an even more
essential task than mitochondria: they perform photosynthesis - that is,
they trap the energy of sunlight in chlorophyll molecules and use this
energy to drive the manufacture of energy-rich sugar molecules. In the
process they release oxygen as a molecular by-product. Plant cells can
then extract this stored chemical when they need it, by oxidizing
energy these sugars in their mitochondria, just as animal cells do.
Chloroplasts thus generate both the food molecules and the oxygen that all
mitochondria use.

\subsection{Internal Membranes Create intracellular Compartments with different functions}

The cytoplasm contains a profusion of other organelles - most of them surrounded
by single membranes - that perform many distinct functions. Most of these structures
are involved with the cell’s ability to import raw materials and to export
manufactured substances and waste products.

The endoplasmic reticulum (ER) - an irregular maze of interconnected
spaces enclosed by a membrane - is the site where most
cell membrane components, as well as materials destined for export
from the cell, are made. Stacks of flattened membrane-enclosed sacs
constitute the Golgi apparatus, which receives and often
chemically modifies the molecules made in the endoplasmic reticulum
and then directs them to the exterior of the cell or to various locations
inside the cell. Lysosomes are small, irregularly shaped organelles in
which intracellular digestion occurs, releasing nutrients from food particles
and breaking down unwanted molecules for recycling or excretion.
And peroxisomes are small, membrane-enclosed vesicles that provide a
contained environment for reactions in which hydrogen peroxide, a dangerously
reactive chemical, is generated and degraded. Membranes also
form many different types of small vesicles involved in the transport of
materials between one membrane-enclosed organelle and another.

A continual exchange of materials takes place between the endoplasmic
reticulum, the Golgi apparatus, the lysosomes, and the outside of the
cell. The exchange is mediated by small vesicles that pinch off from the
membrane of one organelle and fuse with another, like tiny soap bubbles
budding from and rejoining larger bubbles. At the surface of the cell, for
example, portions of the plasma membrane tuck inward and pinch off
to form vesicles that carry material captured from the external medium
into the cell. These vesicles fuse with membrane-enclosed
endosomes, which mature into lysosomes, where the imported material
is digested. Animal cells can engulf very large particles, or even entire
foreign cells, by this process of endocytosis. The reverse process, exocytosis,
whereby vesicles from inside the cell fuse with the plasma membrane
and release their contents into the external medium, is also a common
cellular activity.

\subsection{The Cytosol is a Concentrated aqueous gel of large and
small Molecules}

If we were to strip the plasma membrane from a eucaryotic cell and then
remove all of its membrane-enclosed organelles, including nucleus,
endoplasmic reticulum, Golgi apparatus, mitochondria, chloroplasts, and
so on, we would be left with the cytosol. The cytosol is the site of many chemical
reactions that are fundamental to the cell’s existence. The early steps
in the breakdown of nutrient molecules take place in the cytosol, for
example, and it is here that the cell performs one of its key synthetic processes
- the manufacture of proteins. Ribosomes, the molecular machines
that make the protein molecules, are visible with the electron microscope
as small particles in the cytosol, often attached to the cytosolic face of the
endoplasmic reticulum

\subsection{The Cytoskeleton is responsible for directed Cell Movements}

In eucaryotic cells the cytosol is criss-crossed by long, fine filaments of protein. Frequently the
filaments are seen to be anchored at one end to the plasma membrane
or to radiate out from a central site adjacent to the nucleus. This system of
filaments is called the cytoskeleton. The thinnest
of the filaments are actin filaments, which are present in all eucaryotic
cells but occur in especially large numbers inside muscle cells, where
they serve as part of the machinery that generates contractile forces. The
thickest filaments are called microtubules, because they have the form of
minute hollow tubes. In dividing cells they become reorganized into a
spectacular array that helps pull the duplicated chromosomes in opposite
directions and distribute them equally to the two daughter cells.
Intermediate in thickness between actin filaments and microtubules are the
intermediate filaments, which serve to strengthen the cell
mechanically.

\subsection{The cytoplasm is Far from Static}

The cell interior is in constant motion. The cytoskeleton is a dynamic
jungle of ropes and rods that are continually being strung together and
taken apart; its filaments can assemble and then disappear in a matter
of minutes. Along these tracks and cables, organelles and
vesicles hurry to and from, racing across the width of the cell in a second or so. The
endoplasmic reticulum and the molecules that fill every free space are in
frantic thermal commotion - with unattached proteins buzzing around
so fast that, even though they move at random, they visit every corner
of the cell within a few seconds, constantly colliding with an even more
tumultuous dust storm of smaller organic molecules.

\section{Model organisms}

\subsection{Comparing genome sequences reveals life’s Common heritage}

DNA sequencing has made it easy to detect family resemblances between
genes: if two genes from different organisms have closely similar DNA
sequences, it is highly probable that both genes descended from a common
ancestral gene. Genes (and gene products) related in this way are
said to be homologous.

\section{Essential concepts}

Cells are the fundamental units of life. All present-day cells are
believed to have evolved from an ancestral cell that existed more
than 3 billion years ago.
\begin{itemize}
\item All cells, and hence all living things, grow, convert energy from one
form to another, sense and respond to their environment, and reproduce themselves.
\item All cells are enclosed by a plasma membrane that separates the inside
of the cell from the environment.
\item All cells contain DNA as a store of genetic information and use it to
guide the synthesis of RNA molecules and of proteins.
\item Cells in a multicellular organism, though they all contain the same
DNA, can be very different. They turn on different sets of genes
according to their developmental history and to cues they receive
from their environment.
\item Cells of animal and plant tissues are typically 5–20 m m in diameter
and can be seen with a light microscope, which also reveals some of
their internal components, or organelles.
\item The electron microscope permits the smaller organelles and even
individual large molecules to be seen, but specimens require elaborate preparation and cannot be viewed alive.
\item The simplest of present-day living cells are procaryotes: although
they contain DNA, they lack a nucleus and other organelles and probably resemble most closely the ancestral cell.
\item Different species of procaryotes are diverse in their chemical
capabilities and inhabit an amazingly wide range of habitats. Two
fundamental evolutionary subdivisions are recognized: bacteria and
archaea.
\item Eucaryotic cells possess a nucleus and other organelles not found in
procaryotes. They probably evolved in a series of stages. An important step appears to have been the acquisition of mitochondria, which
are thought to have originated from bacteria engulfed by an ancestral
eucaryotic cell.
\item The nucleus is the most prominent organelle in most plant and animal cells. It contains the genetic information of the organism, stored
in DNA molecules. The rest of the cell’s contents, apart from the
nucleus, constitute the cytoplasm.
\item The cytoplasm includes all of the cell’s contents outside the nucleus.
It contains a variety of membrane-enclosed organelles with specialized chemical functions. Mitochondria carry out the oxidation of food
molecules. In plant cells, chloroplasts perform photosynthesis. The
endoplasmic reticulum, the Golgi apparatus, and lysosomes permit cells to synthesize complex molecules for export from the cell
and for insertion in cell membranes, and to import and digest large
molecules.
\item Outside the membrane-enclosed organelles in the cytoplasm is the
cytosol, a concentrated mixture of large and small molecules that
carry out many essential biochemical processes.
\item The cytoskeleton extends throughout the cytoplasm. This system of
protein filaments is responsible for cell shape and movement and
for the transport of organelles and molecules from one location to
another in the cytoplasm.
\item Free-living, single-celled eucaryotic microorganisms include some of
the most complex eucaryotic cells known, and they are able to swim,
mate, hunt, and devour food.
\item An animal, plant, or fungus consists of diverse eucaryotic cell types
all derived from a single fertilized egg cell; the number of such cells
cooperating to form a large multicellular organism such as a human
runs into thousands of billions.
\item Biologists have chosen a small number of model organisms to study
closely. These include the bacterium E. coli, brewer’s yeast, a nematode worm, a fly, a small plant, a fish, a mouse, and the human
species itself.
\item Although the minimum number of genes needed for a viable cell is
less than 400, most cells contain significantly more. Yet even such a
complex organism as a human has only about 24,000 protein-coding
genes - twice as many as a fly and seven times as many as E. coli.
\end{itemize}

\chapter{Chemical Components of Cells}

We now know that there is nothing in living organisms that disobeys
chemical or physical laws. However, the chemistry of life is indeed a
special kind. First, it is based overwhelmingly on carbon compounds,
the study of which is known as organic chemistry. Second, it depends
almost exclusively on chemical reactions that take place in a watery,
or aqueous, solution and in the relatively narrow range of temperatures
experienced on Earth. Third, it is enormously complex: even the simplest
cell is vastly more complicated in its chemistry than any other chemical
system known. Fourth, it is dominated and coordinated by collections of
enormous polymeric molecules - those formed from chains of chemical
subunits linked end-to-end - whose unique properties enable cells and
organisms to grow and reproduce and to do all the other things that are
characteristic of life. Finally, it is tightly regulated: cells deploy a variety
of mechanisms to make sure that all their chemical reactions occur at the
proper place and time.

\section{Chemical Bonds}

Matter is made of combinations of elements - substances such as hydrogen
or carbon that cannot be broken down or converted into other
substances by chemical means. The smallest particle of an element that
still retains its distinctive chemical properties is an atom. The characteristics
of substances other than pure elements - including the materials
from which living cells are made - depend on which atoms they contain,
and the way these atoms are linked together in groups to form molecules.

\subsection{Cells are made of relatively Few Types of atoms}

Each atom has at its center a dense, positively charged nucleus, which
is surrounded at some distance by a cloud of negatively charged electrons,
held in orbit by electrostatic attraction to the nucleus.
The nucleus consists of two kinds of subatomic particles: protons, which
are positively charged, and neutrons, which are electrically neutral. The
number of protons present in an atomic nucleus determines its atomic
number. The electric charge carried by each proton is exactly
equal and opposite to the charge carried by a single electron. Because
the whole atom is electrically neutral, the number of negatively charged
electrons surrounding the nucleus is equal to the number of positively
charged protons that the nucleus contains; thus the number of electrons
in an atom also equals the atomic number.

Neutrons are uncharged subatomic particles with essentially the same
mass as protons. They contribute to the structural stability of the nucleus -
if there are too many or too few, the nucleus may disintegrate by
radioactive decay - but they do not alter the chemical properties of the
atom. Thus an element can exist in several physically distinguishable but
chemically identical forms, called isotopes, each isotope having a different
number of neutrons but the same number of protons.

The atomic weight of an atom, or the molecular weight of a molecule,
is its mass relative to that of a hydrogen atom. This is essentially equal to
the number of protons plus neutrons that the atom or molecule contains,
because the electrons are so light that they contribute almost nothing to
the total mass. The mass of an atom or a molecule is generally specified in
daltons, one dalton being an atomic mass unit approximately equal to the mass
of a hydrogen atom.

One proton or neutron weighs approximately $\frac{1}{6 \cdot 10^23}$ gram. Hydrogen has
only one proton, with an atomic weight of one, so 1 gram of hydrogen
contains $6 \cdot 10^23$ atoms. For carbon, with an atomic weight of twelve,
12 grams of carbon contain $6 \cdot 10^23$ atoms. This huge number ($6 \cdot 10^23$ ,
called Avogadro’s number) is the key scale factor describing the relationship
between everyday quantities and numbers of individual atoms
or molecules. If a substance has a molecular weight of M, a mass of M
grams of the substance will contain $6 \cdot 10^23$ molecules. This quantity is
called one mole of the substance.

There are 92 naturally occurring elements, each differing from the others
in the number of protons and electrons in its atoms. Living organisms,
however, are made of only a small selection of these elements, four of
which - carbon (C), hydrogen (H), nitrogen (N), and oxygen (O) - make up
96.5\% of an organism’s weight.


\subsection{The outermost electrons determine how atoms interact}

In living tissues, only the electrons of an atom undergo rearrangements.
They form the accessible part of the atom and specify the rules of
chemistry by which atoms combine to form molecules.

Electrons are in continuous motion around the nucleus, but motions on
this submicroscopic scale obey different laws from those we are familiar
with in everyday life. These laws dictate that electrons in an atom can exist
only in certain discrete regions of movement - roughly speaking, discrete
orbits - and that there is a strict limit to the number of electrons that can
be accommodated in an orbit of a given type, a so-called electron shell.
The electrons closest on average to the positive nucleus are attracted
most strongly to it and occupy the inner, most tightly bound shell. This
innermost shell can hold a maximum of two electrons. The second shell
is farther away from the nucleus, and its electrons are less tightly bound.
This second shell can hold up to eight electrons. The third shell contains
electrons that are even less tightly bound; it can also hold up to eight
electrons. The fourth and fifth shells can hold 18 electrons each. Atoms
with more than four shells are very rare in biological molecules.

With certain exceptions in the larger atoms, the electrons of an atom fill
the shells in order - the first before the second, the second before the third,
and so on. An atom whose outermost shell is entirely filled with electrons
is especially stable and therefore chemically unreactive.

Electron exchange can be achieved either by transferring electrons from one
atom to another or by sharing electrons between two atoms. These two strategies generate the two
types of chemical bonds that bind atoms to one another: an ionic bond
is formed when electrons are donated by one atom to another, whereas a
covalent bond is formed when two atoms share a pair of electrons.
In the case of the covalent bond, the pair of electrons is often shared
unequally, with one atom attracting the shared electrons more than the
other; this results in a polar covalent bond.

The number of electrons an atom must acquire or lose (either by sharing or
by transfer) to attain a filled outer shell determines the number of bonds
the atom can make. This is known as its valence.

\subsection{Ionic Bonds Form by the Gain and loss of electrons}

Ionic bonds are most likely to be formed by atoms that have just one or
two electrons in their unfilled outer shell or are just one or two electrons
short of acquiring a filled outer shell. These atoms can generally attain
a completely filled outer electron shell most easily by giving electrons
to - or accepting electrons from - another atom, rather than by sharing
them.

When an electron jumps, for example, from Na to Cl, both atoms become
electrically charged ions.
Positive ions are called cations, and negative ions anions. Ions can be
further classified according to how many electrons are lost or gained.

Because of their opposite charges, $N^{+}$ and $Cl^{-}$ are attracted to each other
and are thereby held together in an ionic bond.
Ionic bonds are a type of electrostatic attraction - an attractive force
that occurs between oppositely charged atoms.
We discuss electrostatic attractions - and the other noncovalent bonds that
can exist between atoms - later in the chapter.

\subsection{Covalent Bonds Form by the sharing of electrons}

A molecule is a cluster of atoms held together by covalent bonds, in which
electrons are shared rather than transferred between atoms.
The attractive and repulsive forces are in balance when the nuclei are separated
by a characteristic distance, called the bond length.

When one atom forms covalent bonds with several others, these multiple
bonds have definite orientations in space relative to one another, reflecting
the orientations of the orbits of the shared electrons. Covalent bonds
between multiple atoms are therefore characterized by specific bond
angles as well as bond lengths and bond energies.

\subsection{Covalent Bonds Vary in strength}

A further crucial property of any bond-covalent or noncovalent - is its strength. Bond
strength is measured by the amount of energy that must be supplied to
break a bond, usually expressed in units of either kilocalories per mole
(kcal/mole) or kilojoules per mole (kJ/mole). A kilocalorie is the amount
of energy needed to raise the temperature of 1 liter of water by one degree centigrade. Thus
if 1 kilocalorie of energy must be supplied to break $6 \cdot 10^{23}$ bonds of a
specific type (that is, 1 mole of these bonds), then the strength of that
bond is 1 kcal/mole. One kilocalorie is equal to about 4.2 kJ.

The making and breaking of covalent bonds are violent events, and in living
cells these events are carefully controlled by highly specific catalysts,
called enzymes.

\subsection{There are different Types of Covalent Bonds}

Most covalent bonds involve the sharing of two electrons, one donated
by each participating atom; these are called single bonds. Some covalent
bonds, however, involve the sharing of more than one pair of electrons.
Four electrons can be shared, for example, two coming from each participating
atom; such a bond is called a double bond. Double bonds are
shorter and stronger than single bonds and have a characteristic effect
on the three-dimensional geometry of molecules containing them.

Some molecules contain atoms that share electrons in a way that produces
bonds that are intermediate in character between single and double
bonds.

When the atoms joined by a single covalent bond belong to different elements,
the two atoms usually attract the shared electrons to different degrees.

By definition, a polar structure (in the electrical sense) is one in which the
positive charge is concentrated toward
one end of the molecule (the positive pole) and the negative charge is
concentrated toward the other end (the negative pole). Covalent bonds
in which the electrons are shared unequally in this way are therefore
known as polar covalent bonds.

\subsection{Electrostatic attractions help Bring molecules Together in Cells}

Polar covalents bonds are extremely important in biology because they allow
molecules to interact through electrical forces.

\subsection{Water is held Together by hydrogen Bonds}

In each molecule of water ($H_{2}O$) the two H atoms are linked to the O
atom by covalent bonds. The two bonds are highly polar because the O is
strongly attractive for electrons, whereas the H is only weakly attractive.

When a positively charged
region of one water molecule (that is, one of its H atoms) comes close to
a negatively charged region (that is, the O) of a second water molecule,
the electrical attraction between them can establish a weak bond called a
hydrogen bond. These bonds are much weaker than covalent bonds and
are easily broken by the random thermal motions due to the heat energy
of the molecules, so each bond lasts only an exceedingly short time.

Not all hydrogen atoms form hydrogen bonds. In general, a hydrogen
bond can form whenever a positively charged H held in one molecule by
a polar covalent linkage comes close to a negatively charged atom - typically
an oxygen or a nitrogen - belonging to another molecule.

Molecules, such as alcohols, that contain polar bonds and that can form
hydrogen bonds mix well with water. As mentioned previously, molecules
carrying positive or negative charges (ions) likewise dissolve
readily in water. Such molecules are termed hydrophilic, meaning that
they are ‘water-loving.’ A large proportion of the molecules in the aqueous
environment of a cell necessarily fall into this category, including
sugars, DNA, RNA, and a majority of proteins. Hydrophobic ‘water-fearing’
molecules, by contrast, are uncharged and form few or no hydrogen
bonds, and so do not dissolve in water.
Because
they do not dissolve in water, the hydrophobic hydrocarbons can form
the thin membrane barriers that keep the aqueous interior of the cell
separate from the surrounding, also aqueous, environment

\subsection{Some Polar molecules Form acids and Bases in Water}

Substances that release protons when they dissolve in water, thus forming
$H_{3}O^{+}$, are termed acids. The higher the concentration of $H_{3}O^{+}$, the
more acidic the solution.
By tradition, the $H_{3}O^{+}$ concentration is usually referred to as the $H^{+}$ concentration, even though
most protons in an aqueous solution are present as $H_{3}O^{+}$. To avoid the
use of unwieldy numbers, the concentration of $H_{3}O^{+}$ is expressed using
a logarithmic scale called the pH scale.
Acids are characterized as being strong or weak, depending on how
readily they give up their protons to water. Strong acids lose their protons quickly.

Because the proton of a hydronium ion can be passed readily to many
types of molecules in cells, altering their character, the concentration of
$H_{3}O^{+}$ inside a cell (the acidity) must be closely regulated. Acids - especially
weak acids - will give up their protons more readily if the concentration
of $H_{3}O^{+}$ in solution is low and will tend to receive them back if the concentration
in solution is high.

The opposite of an acid is a base. Any molecule capable of accepting a
proton is called a base.
the defining property of a base is that it raises the concentration
of hydroxyl ($OH^{-}$) ions by removing a proton from a water molecule.
The term alkaline is also used.
Because NaOH dissociates readily in water, it is called a strong base.
More important in living cells, however, are the weak bases - those that
have a weak tendency to reversibly accept a proton from water.
Because an $OH^{-}$ ion combines with a $H_{3}O^{+}$ ion to form two water molecules,
an increase in the $OH^{-}$ concentration forces a decrease in the
concentration of $H_{3}O^{+}$, and vice versa. A pure solution of water thus contains
an equal concentration ($10^{-7}$ M) of both ions, rendering it neutral.

\section{Molecules in Cells}

\subsection{A Cell is Formed from Carbon Compounds}

Because carbon is small
and has four electrons and four vacancies in its outermost shell, a carbon
atom can form four covalent bonds with other atoms. Most importantly,
one carbon atom can join to other carbon atoms through highly stable
covalent C-C bonds to form chains and rings and hence generate large
and complex molecules with no obvious upper limit to their size.
The small and large carbon compounds made by
cells are called organic molecules. All other molecules, including water,
are said to be inorganic by contrast.

\subsection{Cells Contain Four major Families of small organic molecules}

The small organic molecules of the cell are carbon compounds with
molecular weights in the range 100-1000 that contain up to 30 or so carbon
atoms. They are usually found free in solution in the cytoplasm and
have many different fates. Some are used as monomer subunits to construct
the giant polymeric macromolecules - the proteins, nucleic acids,
and large polysaccharides - of the cell.

All organic molecules are synthesized from - and are broken down
into - the same set of simple compounds. Both their synthesis and their
breakdown occur through sequences of simple chemical changes that
are limited in variety and follow definite step-by-step rules. As a consequence,
the compounds in a cell are chemically related and most can be
classified into a small number of distinct families. Broadly speaking, cells
contain four major families of small organic molecules: the sugars, the
fatty acids, the amino acids, and the nucleotides.

\subsection{Sugars are energy sources for Cells and subunits of Polysaccharides}

The simplest sugars - the monosaccharides - are compounds with the
general formula $(CH_{2}O)_{n}$, where n is usually 3, 4, 5, or 6. Sugars, and
the molecules made from them, are also called carbohydrates because
of this simple formula. Glucose, for example, has the formula $C_{6}H_{12}O_{6}$.
The formula, however, does not fully define the molecule:
the same set of carbons, hydrogens, and oxygens can be joined together
by covalent bonds in a variety of ways, creating structures with different
shapes.
Each of these sugars, moreover, can exist in either of two forms, called
the D-form and the L-form, which are mirror images of each other. Sets
of molecules with the same chemical formula but different structures are
called isomers, and mirror-image pairs of molecules are called optical isomers.

Monosaccharides can be linked by covalent bonds - called glycosidic
bonds - to form larger carbohydrates.
Larger sugar polymers range from the oligosaccharides (trisaccharides, tetrasaccharides, and so on) up to giant
polysaccharides, which can contain thousands of monosaccharide units.
In most cases, the prefix “oligo-” is used to refer to macromolecules made
of a small number of monomers, between 3 and 50 or so. Polymers, in
contrast, can contain hundreds or thousands of subunits.

The way sugars are linked together illustrates some common features of
biochemical bond formation. A bond is formed between an -OH group
on one sugar and an -OH group on another by a condensation reaction,
in which a molecule of water is expelled as the bond is formed.
The bonds created by all of these condensation reactions can
be broken by the reverse process of hydrolysis, in which a molecule of
water is consumed.

It is much more
difficult to determine the arrangement of sugars in a polysaccharide than
to determine the nucleotide sequence of a DNA molecule, where each
unit is joined to the next in exactly the same way.

The monosaccharide glucose has a central role as an energy source for
cells. It is broken down to smaller molecules in a series of reactions,
releasing energy that the cell can harness to do useful work.
Cells use simple polysaccharides composed only
of glucose units - principally glycogen in animals and starch in plants - as
long-term stores of glucose, held in reserve for energy production.

Sugars do not function exclusively in the production and storage of
energy. They are also used, for example, to make mechanical supports.
The most abundant organic molecule on Earth - the cellulose that forms
plant cell walls - is a polysaccharide of glucose.

Smaller oligosaccharides can be covalently linked to proteins to form
glycoproteins, or to lipids to form glycolipids,
which are both found in cell membranes.

\subsection{Fatty acids are Components of Cell membranes}

A fatty acid molecule, has two
chemically distinct regions. One is a long hydrocarbon chain, which is
hydrophobic and not very reactive chemically. The other is a carboxyl
(-COOH) group, which behaves as an acid (carboxylic acid): it is ionized
in solution ($–COO^{-}$), extremely hydrophilic, and chemically reactive.
Almost all the fatty acid molecules in a cell are covalently linked to other
molecules by their carboxylic acid group.
Molecules such as fatty acids, which possess both hydrophobic and
hydrophilic regions, are termed amphipathic.

The hydrocarbon tail of palmitic acid is saturated: it has no double bonds
between its carbon atoms and contains the maximum possible number
of hydrogens. Some other fatty acids, such as oleic acid, have
unsaturated tails, with one or more double bonds along their length. The
double bonds create kinks in the molecules, interfering with their ability
to pack together in a solid mass, and it is the absence or presence of
these double bonds that accounts for the difference between hard (saturated)
and soft (polyunsaturated) margarine. Fatty acids are also found in
cell membranes, where the tightness of their packing affects the fluidity
of the membrane.

Fatty acids serve as a concentrated food reserve in cells: they can be
broken down to produce about six times as much usable energy, weight
for weight, as glucose. Fatty acids are stored in the cytoplasm of many cells
in the form of droplets of triacylglycerol molecules - compounds made of
three fatty acid chains joined to a glycerol molecule. When a cell needs
energy, the fatty acid chains can be released from triacylglycerols and
broken down into two-carbon units. These two-carbon units are identical
to those derived from the breakdown of glucose, and they enter the same
energy-yielding reaction pathways.

Fatty acids and their derivatives, including triacylglycerols, are examples
of lipids. This class of biological molecules is loosely defined, with the
common feature that the molecules in the class are insoluble in water and
soluble in fat and organic solvents such as benzene. Lipids typically contain
long hydrocarbon chains, as in the fatty acids and isoprenes - or multiple
linked aromatic rings, as in the steroids.

The most important function of fatty acids in cells is in the formation of
membranes. These thin sheets enclose all cells and surround their internal
organelles. They are composed largely of phospholipids, which are small
molecules that, like triacylglycerols, are constructed mainly from fatty
acids and glycerol. Phospholipids are strongly amphipathic: each phospholipid
molecule has a hydrophobic tail, composed of the two fatty acid chains,
and a hydrophilic head, where the phosphate is located.

The membrane-forming property of phospholipids results from their
amphipathic nature. Phospholipids will spread over the surface of water to
form a monolayer of phospholipid molecules, with the hydrophobic tails
facing the air and the hydrophilic heads in contact with the water. Two
such molecular layers can readily combine tail-to-tail in water to make a
phospholipid sandwich, or lipid bilayer, which forms the structural basis
of all cell membranes.

\subsection{Amino acids are the subunits of Proteins}

Amino acids are a varied class of molecules with one defining property:
they all possess a carboxylic acid group and an amino group, both
linked to the same carbon atom called the $\alpha$-carbon.
Cells use amino acids to build proteins, which are polymers of
amino acids joined head-to-tail in a long chain that is then folded into a
three-dimensional structure unique to each type of protein.

The covalent linkage between two adjacent amino acids in a protein
chain is called a peptide bond; the chain of amino acids is also known as
a polypeptide. Peptide bonds are formed by condensation
reactions that link one amino acid to the next. Regardless of the specific
amino acids from which it is made, the polypeptide always has an amino
($NH_{2}$) group at one end (its N-terminus) and a carboxyl (COOH) group at
its other end (its C-terminus). This gives a protein or polypeptide a definite
directionality - a structural (as opposed to electrical) polarity.

Twenty types of amino acids are commonly found in proteins, each with a
different side chain attached to the $\alpha$-carbon atom.
The same 20 amino acids occur over and over again in all proteins,
whether they hail from bacteria, plants, or animals.

Like sugars, all amino acids (except glycine) exist as optical isomers in D- 
and L-forms. But only L-forms are ever found in proteins
(although D-amino acids occur as part of bacterial cell walls and in some
antibiotics). The origin of this exclusive use of L-amino acids to make
proteins is another evolutionary mystery.

The chemical versatility that the 20 standard amino acids provide is
vitally important to the function of proteins. Five of the 20 amino acids
have side chains that can form ions in solution and can therefore carry
a charge. The others are uncharged.

\subsection{Nucleotides are the subunits of DNA and RNA}

A nucleoside is a molecule made of a nitrogen-containing ring compound
linked to a five-carbon sugar, which can be either ribose or deoxyribose.
A nucleoside sporting one or more phosphate
groups attached to its sugar is called a nucleotide. Nucleotides containing
ribose are known as ribonucleotides, and those containing deoxyribose as
deoxyribonucleotides.

The nitrogen-containing rings are generally referred to as bases for historical
reasons: under acidic conditions they can each bind a $H^{+}$ (proton)
and thereby increase the concentration of $OH^{-}$ ions in aqueous solution.
There is a strong family resemblance between the different nucleotide
bases. Cytosine (C), thymine (T), and uracil (U) are called pyrimidines
because they all derive from a six-membered pyrimidine ring; guanine (G)
and adenine (A) are purine compounds, which bear a second, five-membered
ring fused to the six-membered ring. Each nucleotide is named
after the base it contains.

Nucleotides can act as short-term carriers of chemical energy. Above all
others, the ribonucleotide adenosine triphosphate, or ATP,
participates in the transfer of energy in hundreds of cellular reactions.
ATP is formed through reactions that are driven by the energy released by
the breakdown of foodstuffs. Its three phosphates are linked in series by
two phosphoanhydride bonds

The most fundamental role of nucleotides in the cell is in the storage and
retrieval of biological information. Nucleotides serve as building blocks
for the construction of nucleic acids - long polymers in which nucleotide
subunits are covalently linked by the formation of a phosphodiester bond
between the phosphate group attached to the sugar of one nucleotide
and a hydroxyl group on the sugar of the next nucleotide.

There are two main types of nucleic acids, which differ in the type of
sugar they use in their sugar–phosphate backbone. Those based on the
sugar ribose are known as ribonucleic acids, or RNA, and contain the
bases A, G, C, and U. Those based on deoxyribose (in which the hydroxyl
at the 2' position of the ribose carbon ring is replaced by a hydrogen)
are known as deoxyribonucleic acids, or DNA, and contain the bases A, G, C, and T
(T is chemically similar to the U in RNA).
RNA usually occurs in cells in the form of a single-stranded polynucleotide chain,
but DNA is virtually always in the form
of a double-stranded molecule: the DNA double helix is composed of
two polynucleotide chains running antiparallel to each other, being held
together by hydrogen-bonding between the bases of the two chains.

DNA, with its more stable, hydrogen-bonded
helices, acts as a long-term repository for hereditary information, while
single-stranded RNA is usually a more transient carrier of molecular
instructions. The ability of the bases in different nucleic acid molecules
to recognize and pair with each other by hydrogen-bonding (called base-pairing)
- G with C, and A with either T or U - underlies all of heredity and
evolution.

\section{Macromolecules in Cells}

Macromolecules are by far the most abundant
of the carbon-containing molecules in a living cell. They
are the principal building blocks from which a cell is constructed and
also the components that confer the most distinctive properties on living things.
Intermediate in size and complexity between small molecules
and cell organelles, macromolecules are polymers that are constructed
simply by covalently linking small organic molecules (called monomers,
or subunits) into long chains, or polymers.

\subsection{Macromolecules Contain a specific sequence of subunits}

The stepwise polymerization of monomers into a long chain is a simple
way to manufacture a large, complex molecule, because the subunits are
added by the same reaction performed over and over again by the same
set of enzymes. Most important, the polymer chain is
not assembled at random from these subunits; instead the subunits are
added in a particular order, or sequence.

\section{Essential concepts}

\begin{itemize}
\item Living cells obey the same chemical and physical laws as nonliving
things. Like all other forms of matter, they are composed of atoms,
which are the smallest units of a chemical element that retains the
distinctive chemical properties of that element.
\item Atoms are made up of smaller particles. The nucleus of an atom
contains protons, which are positively charged, and uncharged neutrons.
The nucleus is surrounded by a cloud of negatively charged
electrons.
\item The number of electrons in an atom is equal to the number of protons
in its nucleus. The nuclei of different isotopes of the same
element contain the same number of protons but different numbers
of neutrons.
\item Living cells are made up of a limited number of elements, four of
which - C, H, N, O - make up 96.5\% of their mass.
\item The chemical properties of an atom are determined by the number
and arrangement of its electrons. An atom is most stable when all of
its electrons are at their lowest possible energy level and when each
electron shell is completely filled.
\item Chemical bonds form between atoms as electrons move to reach
a more stable arrangement. Clusters of two or more atoms held
together by chemical bonds are known as molecules.
\item When an electron jumps from one atom to another, two ions of opposite
charge are generated; ionic bonds can then arise by the mutual
attraction of these charged atoms.
\item A covalent bond consists of a pair of electrons shared between two
adjacent atoms. If two pairs of electrons are shared, a double bond is
formed.
\item Living organisms contain a distinctive and restricted set of small carbon-based
molecules that are essentially the same for every living
species. The main categories are sugars, fatty acids, amino acids,
and nucleotides.
\item Sugars are a primary source of chemical energy for cells and can be
incorporated into polysaccharides for energy storage.
\item Fatty acids are also important for energy storage, but their most
essential function is in the formation of cell membranes.
\item The vast majority of the dry mass of a cell consists of macromolecules,
formed as polymers of sugars, amino acids, or nucleotides.
\item Macromolecules are intermediate in both size and complexity between
small molecules and cell organelles. They have many remarkable
properties that are not easily deduced from the subunits from which
they are made.
\item The remarkably diverse and versatile class of macromolecules known
as proteins are polymers formed from amino acids.
\item Nucleotides play a central part in energy transfer and are the subunits
from which the informational macromolecules, RNA and DNA, are made.
\item Protein, RNA, and DNA molecules are synthesized from subunits by
repetitive condensation reactions. Each of these biological macromolecules
has a unique sequence of subunits.
\item Weak noncovalent bonds form between different regions of a macromolecule.
These can cause the macromolecule to fold into a unique
three-dimensional shape (conformation) with a special chemistry, as
seen most conspicuously in proteins.
\end{itemize}

\chapter{Energy, Catalysis and Biosynthesis}

\section{The Use of energy by Cells}

Nonliving things left to themselves eventually become disordered: buildings
crumble and dead organisms decay. Living cells, by contrast, not
only maintain, but actually generate, order at every level.

\subsection{Biological order is Made Possible by the release of heat energy from Cells}

The universal tendency of things to become disordered is expressed in a
fundamental law of physics, the second law of thermodynamics. This law
states that in the universe, or in any isolated system (a collection of matter
that is completely isolated from the rest of the universe), the degree
of disorder can only increase.

We can express the second law in terms of probability by stating that
systems will change spontaneously toward those arrangements that have
the greatest probability.

The measure of a system’s disorder is called the entropy of the system,
and the greater the disorder, the greater the entropy. Thus, another way
to express the second law of thermodynamics is to say that systems will
change spontaneously toward arrangements with greater entropy.
Living cells - by surviving, growing, and forming complex communities and
even whole organisms - are generating order and thus might appear to
defy the second law of thermodynamics. This is not the case, however,
because a cell is not an isolated system.
In the course of performing the chemical reactions that generate
order, chemical bond energy is converted into heat. Heat is energy in its
most disordered form - the random jostling of molecules.

The amount of heat released by a cell must be great enough that the order
generated inside the cell is more than compensated for by the decrease
in order in the environment.

According to the first law of thermodynamics, energy can be converted from one form to
another, but it cannot be created or destroyed.
The amount of energy
present in different forms will change as a result of the chemical reactions
inside the cell, but the first law tells us that the total amount of
energy in the universe must always be the same.

The cell cannot derive any benefit from the heat energy it produces,
however, unless the heat-generating reactions inside the cell are directly
linked to processes that maintain molecular order.

\subsection{Photosynthetic organisms Use sunlight to synthesize organic Molecules}

The energy animals obtain by eating plants came originally from the sun.
Solar energy enters the living world through photosynthesis, a process
that converts the electromagnetic energy in sunlight into chemical bond
energy in cells. Photosynthetic organisms - including plants, algae, and
some bacteria - are able to obtain all of the atoms they need from inorganic sources.
They use the energy they derive from sunlight
to form chemical bonds between these atoms, linking them into small
chemical building blocks such as sugars, amino acids, nucleotides, and
fatty acids. These small molecules in turn are converted into the macro-molecules

The reactions of photosynthesis take place in two stages: one that
depends on light and another that does not. In the first, light-dependent
stage, energy from sunlight is captured and transiently stored
as chemical bond energy in specialized small molecules that carry energy
in their reactive chemical groups.
In the second stage of photosynthesis, the molecules that serve as energy
carriers are used to help drive a carbon-fixation process in which sugars
are manufactured from carbon dioxide gas ($CO_{2}$) and water ($H_{2}O$).

The net result of both stages of photosynthesis, as far as the green plant
is concerned, can be summarized simply in the equation

\begin{equation}
light energy + CO_{2} + H_{2}O \rightarrow sugars + O_{2} + heat energy
\end{equation}

\subsection{Cells obtain energy by the oxidation of organic Molecules}

In both plants and
animals, energy is extracted from food molecules by a process of gradual
oxidation, or controlled burning.

A cell is therefore able to obtain energy from sugars or other
organic molecules by allowing the carbon and hydrogen atoms in these
molecules to combine with oxygen - that is, become oxidized - to produce
$CO_{2}$ and $H_{2}O$, respectively - a process known as cellular respiration.
Photosynthesis and respiration are complementary processes.

\subsection{Oxidation and reduction involve electron Transfers}

The term oxidation literally means the addition of oxygen atoms to a
molecule. More generally, though, oxidation is said to occur in any
reaction in which electrons are transferred from one atom to another.
Oxidation, in this sense, refers to the removal of electrons. The converse
reaction, called reduction, involves the addition of electrons.
Because the
number of electrons is conserved in a chemical reaction (there is no net
loss or gain), oxidation and reduction always occur simultaneously: that
is, if one molecule gains an electron in a reaction (reduction), a second
molecule must lose the electron (oxidation).

The terms oxidation and reduction apply even when there is only a partial
shift of electrons between atoms linked by a covalent bond. When a
carbon atom becomes covalently bonded to an atom with a strong affinity
for electrons - oxygen, chlorine, or sulfur, for example - it gives up more
than its equal share of electrons and forms a polar covalent bond.
When a molecule in a cell picks up an electron ($e^{-}$), it often picks up a
proton ($H^{+}$) at the same time (protons being freely available in water).
The net effect in this case is to add a hydrogen atom to the molecule:

\begin{equation}
A + e^{-} +  H^{+} \rightarrow AH
\end{equation}

Even though a proton plus an electron is involved (instead of just an
electron), such hydrogenation reactions are reductions, and the reverse,
dehydrogenation, reactions are oxidations. An easy way to tell whether an
organic molecule is being oxidized or reduced is to count its C-H bonds:
reduction occurs when the number of C-H bonds increases, whereas oxidation
occurs when the number of C-H bonds decreases.

Cells use enzymes to catalyze the oxidation of organic molecules in small steps,
through a sequence of reactions that allows useful energy to be harvested.

\section{Free energy and Catalysis}

\subsection{Enzymes lower the energy barriers That Prevent Chemical reactions from occurring}

\begin{equation}
paper + O_{2} \rightarrow smoke + ashes + heat + CO_{2} + H_{2}O
\end{equation}

But this occurs in only one direction: smoke and ashes never spontaneously
gather carbon dioxide and water from the heated atmosphere and
reconstitute themselves into paper. When paper burns, its chemical
energy is dissipated as heat. The atoms
and molecules of the paper become dispersed and disordered. In the
language of thermodynamics, there has been a release of free energy -
that is, of energy that can be harnessed to do work or drive chemical
reactions. Chemical reactions proceed only in the direction that leads to
a loss of free energy. In other words, the spontaneous direction for any
reaction is the direction that goes ‘downhill.’ A ‘downhill’ reaction in this
sense is said to be energetically favorable.
In other words, a molecule requires a boost over an energy barrier
before it can undergo a chemical reaction that moves it to a lower-energy
(more stable) state.

At the temperatures in living cells, the push over the energy barrier is
greatly aided by specialized proteins called enzymes. Each enzyme binds
tightly to one or two molecules, called substrates, and holds them in
a way that greatly reduces the activation energy needed to facilitate a
specific chemical interaction between them. A substance
that can lower the activation energy of a reaction is termed a catalyst;
catalysts increase the rate of chemical reactions because they allow a
much larger proportion of the random collisions with surrounding molecules
to kick the substrates over the energy barrier.
Unlike temperature, enzymes are highly selective. Each enzyme usually
speeds up only one particular reaction out of the several possible reactions
that its substrate molecules could undergo.

Each enzyme has a unique shape containing an active site, a pocket or only
groove in the enzyme into which particular substrates will fit.
Like all catalysts, enzyme molecules themselves remain unchanged
after participating in a reaction and therefore can function over and over
again.

\subsection{The free-energy Change for a reaction determines whether it can occur}

According to the second law of thermodynamics, a chemical reaction
can proceed only if it results in a net (or overall) increase in the disorder
of the universe. Disorder increases when useful
energy that could be harnessed to do work is dissipated as heat. The
criterion for an increase of disorder can be expressed most conveniently
in terms of the free energy, G, of a system. The value of G is of most
interest when a system undergoes a change, so the free-energy change,
denoted $\Delta G$ (“Delta G”), is the term we most often see.
Energetically favorable reactions, by definition, are those that create disorder
by decreasing the free energy of the system to which they belong; in
other words, they have a negative $\Delta G$. Conversely, energetically unfavorable reactions, with a positive
$\Delta G$ - such as those in which two amino acids are joined together to form
a peptide bond - by themselves create order in the universe. These reactions
cannot occur spontaneously. Energetically unfavorable reactions
can take place only if they are coupled to a second reaction with a negative
$\Delta G$ so large that the net DG of the entire process is negative.

\subsection{The Concentration of reactants influences the free-energy Change and a reaction’s direction}

a reaction Y $\Longleftrightarrow$ X will go in the direction Y $\rightarrow$ X
when the associated free-energy change, $\Delta G$, is negative, just as a tensed
spring left to itself will relax and lose its stored energy to its surroundings
as heat. For a chemical reaction, however, DG depends not only on the
energy stored in each individual molecule, but also on the concentrations
of the molecules in the reaction mixture. Remember that $\Delta G$ reflects the
degree to which a reaction creates a more disordered - in other words, a
more probable - state of the universe.
The same is true for a chemical reaction. For the reversible reaction Y $\Longleftrightarrow$
X, a large excess of Y over X will tend to drive the reaction in the direction
Y $\rightarrow$ X; that is, there will be a tendency for there to be more molecules
making the transition Y $\rightarrow$ X than there are molecules making the transition
X $\rightarrow$ Y. Thus, as the ratio of Y to X increases, the DG becomes more negative
for the transition Y $\rightarrow$ X (and more positive for the transition X $\rightarrow$ Y).

\subsection{The standard free-energy Change Makes it Possible to Compare the energetics of different reactions}

Because $\Delta G$ depends on the concentrations of the molecules in the reaction
mixture at any given time, it is not a particularly useful value for
comparing the relative energies of different types of reactions. Such comparisons
are necessary, for example, to predict whether an energetically
favorable reaction is likely to have a $\Delta G$ negative enough to drive an
energetically unfavorable reaction. To level the playing field and place
reactions on a comparable basis, we need to turn to the standard free-energy
change of a reaction, $\Delta G^{o}$. The $\Delta G^{o}$ is independent of concentration; it
depends only on the intrinsic characters of the reacting molecules, based
on their behavior under ideal conditions where the concentrations of all
the reactants are set to the same fixed value of 1 mole/liter.

\subsection{Cells exist in a state of Chemical disequilibrium}

Chemical reactions will generally proceed until they reach a state of
equilibrium. At that point, the rates of the forward and reverse reactions
are equal, and there is no further net change in the concentrations of
substrate or product. Because the maintenance of order within the cell
requires a continuous input of energy, any cell whose reactions have all
reached chemical equilibrium is dead.
Living cells avoid reaching a state of equilibrium because they are constantly
exchanging materials with their environment.

\subsection{The equilibrium Constant is directly Proportional to $\Delta G^{o}$}

As we have seen, at chemical equilibrium, when the forward and reverse
reaction rates are equal, the ratio of substrate to product will remain constant.
This state makes it possible to calculate a reaction’s equilibrium
constant, K

\begin{equation}
K = \frac{[X]}{[Y]}
\end{equation}

where [X] is the concentration of the product and [Y] is the concentration
of the reactant at equilibrium. This expression describes the situation at
the point at which the concentration effect just balances the push given
to the reaction by $\Delta G^{o}$, so that $\Delta G = 0$ and there is no net change of free
energy to drive the reaction in either direction.
This equation reveals how the equilibrium ratio of Y to X (expressed as
the equilibrium constant, K) depends on the intrinsic character of the molecules.

%\subsection{In Complex reactions, the equilibrium Constant depends on the Concentrations of all reactants and Products}

\subsection{The equilibrium Constant indicates the strength of Molecular interactions}

Two molecules will bind to each other if the $\Delta G^{o}$ of the interaction is
negative; that is, the free energy of the resulting complex is lower than
the sum of the free energies of the two partners when unbound. Because
the equilibrium constant of a reaction is related directly to $\Delta G^{o}$, K is
commonly employed as a measure of the binding strength of a noncovalent
interaction between two molecules. The binding strength is a very useful
quantity to know because it also indicates how specific the interaction is
between the two molecules.

%\subsection{For sequential reactions, the Changes in free energy are additive}

\subsection{Rapid diffusion allows enzymes to find Their substrates}

Enzymes and their substrates are both present in relatively small amounts
in a cell, yet a typical enzyme can capture and process about a thousand
substrate molecules every second. This means that an enzyme must be
able to release its product and bind a new substrate in a fraction of a
millisecond.

Rapid binding is possible because motions are enormously fast at the
molecular level. Because of heat energy, molecules are in constant
motion and consequently will explore the space inside the cell very efficiently
by wandering randomly through it - a process called diffusion. In
this way, every molecule in a cell collides with a huge number of other
molecules each second. As the molecules in a liquid collide and bounce
off one another, an individual molecule moves first one way and then
another, its path constituting a random walk. In such a walk,
the average distance that each molecule travels (as the crow flies) from
its starting point is proportional to the square root of the time it takes.
Thus, diffusion only works well for very short distances. To move molecules
quickly over larger distances, cells need to rely on more active
and directed methods of transport - processes that inevitably require an
expenditure of cellular energy.

Enzymes and other macromolecules, however, diffuse through the cytoplasm
much more slowly than do small molecules. In some cases, enzymes
that interact with other proteins are actually held in close proximity to
their partners by scaffold proteins that draw together sets of interacting
proteins at specific locations in the cell. But even if they’re not physically
sequestered in one place, enzymes move much more slowly than
small molecules. A random encounter between the surface of an enzyme and the matching
surface of its substrate molecule often leads immediately to the formation of
an enzyme-substrate complex that is ready to react.

When an enzyme and substrate have collided and snuggled together
properly at the active site, they form multiple weak bonds with each
other that persist until random thermal motion causes the molecules to
dissociate again.

\subsection{$V_{max}$ and $K_{M}$ Measure enzyme Performance}

To catalyze a reaction, an enzyme must first bind its substrate. The substrate
then undergoes a reaction to form the product, which initially
remains bound to the enzyme. Finally, the product is released and diffuses
away, leaving the enzyme free to bind another substrate molecule
and catalyze another reaction. The rates of the different
steps vary widely from one enzyme to another, and they can be measured
by mixing purified enzymes and substrates together under carefully
defined conditions.

In such experiments, if the concentration of the substrate is increased
progressively from a very low value, the concentration of the enzyme-substrate
complex - and therefore the rate at which product is formed - initially increases
in a linear fashion in direct proportion to substrate concentration.
However, as more and more enzyme molecules become occupied by substrate,
this rate increase tapers off, until at a very high concentration of substrate
it reaches a maximum value, termed $V_{max}$.

The concentration of substrate needed to make the enzyme work efficiently
is often measured by a different parameter, the Michaelis’ constant,
$K_M$, named after one of the biochemists who worked out this relationship.
An enzyme’s $K_M$ is the concentration of substrate at which the enzyme
works at half its maximum speed. In general, a low
value of $K_M$ indicates that a substrate binds very tightly to the enzyme,
and a large value corresponds to weak binding.

It is important to recognize that when an enzyme (or any catalyst) lowers
the activation energy for the reaction Y $\rightarrow$ X, it also lowers the activation
energy for the reverse reaction X $\rightarrow$ Y by exactly the same amount.
The forward and backward reactions will therefore be accelerated by the same factor
by an enzyme, and the equilibrium point for the reaction (and thus $\Delta G^{o}$)
remains unchanged.

\section{Activated Carrier Molecules and Biosynthesis}

Activated carriers store energy in an easily exchangeable form, either
as a readily transferable chemical group or as high-energy electrons,
and they can serve a dual role as a source of both energy and chemical
groups for biosynthetic reactions.

\subsection{The formation of an activated Carrier is Coupled to an energetically favorable reaction}

When a fuel molecule such as glucose is oxidized in a cell, enzyme-catalyzed
reactions ensure that a large part of the free energy that is released
by oxidation is captured in a chemically useful form, rather than being
released wastefully as heat. In living systems, this energy capture is achieved by means of a
coupled reaction, in which an energetically favorable reaction is used to
drive an energetically unfavorable one that produces an activated carrier
molecule or some other useful molecule.

Enzymes couple an energetically favorable reaction, such
as the oxidation of foodstuffs, to an energetically unfavorable reaction,
such as the generation of an activated carrier molecule. As a result, the
amount of heat released by the oxidation reaction is reduced
by exactly the amount of energy that is stored in the energy-rich covalent bonds
of the activated carrier molecule.

\subsection{ATP is the Most Widely Used activated Carrier Molecule}

The most important and versatile of the activated carriers in cells is
ATP (adenosine 5'-triphosphate). ATP is synthesized in an energetically unfavorable
phosphorylation reaction in which a phosphate group is added to ADP
(adenosine 5'-diphosphate).
When required, ATP gives up this energy
packet in an energetically favorable hydrolysis to ADP and inorganic
phosphate ($P_{i}$). The regenerated ADP is then available to be used for
another round of the phosphorylation reaction that forms ATP, creating
an ATP cycle in the cell.

Any reaction that involves the transfer of a phosphate group to a molecule is termed a
phosphorylation reaction. Phosphorylation reactions are examples of condensation reactions
and are involved in many important cellular functions: they
activate substrates, they facilitate the exchange of chemical energy, and
they help to control cell-signaling processes.

\subsection{Energy stored in ATP is often harnessed to Join Two Molecules Together}

A common type of reaction that is needed for biosynthesis is one in which
two molecules, A and B, are joined together to produce A–B in the energetically
unfavorable condensation reaction. ATP hydrolysis can be coupled indirectly to this
reaction to make it go forward.

\subsection{NADH and NADPH are important electron Carriers}

Other important activated carrier molecules participate in oxidation-reduction
reactions and are commonly part of coupled reactions in cells.
These activated carriers are specialized to carry both high-energy electrons
and hydrogen atoms. The most important of these electron carriers
are $NAD^{+}$ (nicotinamide adenine dinucleotide) and the closely related
molecule $NADP^{+}$ (nicotinamide adenine dinucleotide phosphate).
$NAD^{+}$ and $NADP^{+}$ each pick up a “packet of energy” in the form of two high-energy
electrons plus a proton ($H^{+}$), becoming NADH (reduced nicotinamide
adenine dinucleotide) and NADPH (reduced nicotinamide adenine dinucleotide
phosphate), respectively.

Like ATP, NADPH is an activated carrier that participates in many
important biosynthetic reactions that would otherwise be energetically
unfavorable.

Why should there be this division of labor? The answer lies in the need
to regulate two sets of electron-transfer reactions independently. NADPH
operates chiefly with enzymes that catalyze anabolic reactions, supplying
the high-energy electrons needed to synthesize energy-rich biological
molecules. NADH, by contrast, has a special role as an intermediate in
the catabolic system of reactions that generate ATP through the oxidation
of food molecules.

\subsection{Cells Make Use of Many other activated Carrier Molecules}

Other activated carriers also pick up and carry a chemical group in an
easily transferred, high-energy linkage. For example, $FADH_{2}$,
like NADH and NADPH, also carries hydrogen and high-energy electrons.
Coenzyme A, on the other hand, can carry an acetyl
group in a readily transferable linkage. This activated molecule, called
acetyl CoA (acetyl coenzyme A).

\section{Essential concepts}

\begin{itemize}
\item Living organisms are able to exist because of a continual input
of energy. Part of this energy is used to carry out essential functions
- reactions that support cellular metabolism, growth, and
reproduction - and the remainder is lost in the form of heat.
\item The primary source of energy for most living organisms is the sun.
Plants, algae, and photosynthetic bacteria use solar energy to produce
organic molecules from carbon dioxide. Animals obtain food by
eating plants or by eating animals that feed on plants.
\item Each of the many hundreds of chemical reactions that occur in a cell
is specifically catalyzed by an enzyme. Large numbers of different
enzymes work in sequence to form chains of reactions, called metabolic
pathways, each performing a different function in the cell.
\item Catabolic reactions break down food molecules through oxidative
pathways and release energy. Anabolic reactions generate the
many complex molecules needed by the cell, and they require an
energy input. In animal cells, both the building blocks and the energy
required for the anabolic reactions are obtained by catabolism.
\item Enzymes catalyze reactions by binding to particular substrate molecules
in a way that lowers the activation energy required for making
and breaking specific covalent bonds.
\item The rate at which an enzyme catalyzes a reaction depends on how
rapidly it finds its substrates and how quickly the product forms and
then diffuses away. These rates vary widely from one enzyme to
another, and they can be measured after mixing purified enzymes
and substrates together under a set of defined conditions.
\item The only chemical reactions possible are those that increase the
total amount of disorder in the universe. The free-energy change for
a reaction, DG, measures this disorder, and it must be less than zero
for a reaction to proceed spontaneously.
\item The free-energy change for a chemical reaction, DG, depends on the
concentrations of the reacting molecules, and it may be calculated
from these concentrations if the equilibrium constant (K) of the reaction
(or the standard free-energy change, $DG^{o}$, for the reactants) is
known.
\item Equilibrium constants govern all of the associations (and dissociations)
that occur between macromolecules and small molecules in
the cell. The larger the binding energy between two molecules, the
larger the equilibrium constant and the more likely that these molecules
will be found bound to each other.
\item By creating a reaction pathway that couples an energetically favorable
reaction to an energetically unfavorable one, enzymes can make
otherwise impossible chemical transformations occur.
\item A small set of activated carrier molecules, particularly ATP, NADH,
and NADPH, plays a central part in these coupling events. ATP carries
high-energy phosphate groups, whereas NADH and NADPH carry
high-energy electrons.
\item Food molecules provide the carbon skeletons for the formation of
larger molecules. The covalent bonds of these larger molecules are
typically produced in reactions that are coupled to energetically favorable
bond changes in activated carrier molecules such as ATP and NADPH.
\end{itemize}
