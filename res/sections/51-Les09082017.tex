\chapter{Introduction to Cells}

All living things are made of cells: small, membrane-enclosed units
filled with a concentrated aqueous solution of chemicals and endowed
with the extraordinary ability to create copies of themselves by growing
and dividing in two. 
Cells, therefore, are the fundamental units of life, and it is to cell biology
that we must look for an answer to the question of what life is and how it
works.

\section{Unity and Diversity of Cells}

Cells are not all alike; in fact, they can be wildly different.

\subsection{Cells vary enormously in appearance and function}

A bacterial cell is a few micrometers in length. Cells vary no less widely
their shapes and functions.
Some cells are clad only in a flimsy membrane; others augment this deli-
cate cover by cloaking themselves in an outer layer of slime, building
themselves rigid cell walls, or surrounding themselves with a hard, min-
eralized material, such as that found in bone.
Cells are also enormously diverse in their chemical requirements and
activities.
Some modifications specialize a cell so much that they spoil its chances
of leaving any descendants. Such specialization would be senseless for a
cell that lived a solitary life. In a multicellular organism, however, there is
a division of labor among cells, allowing some cells to become specialized
to an extreme degree for particular tasks and leaving them dependent on
their fellow cells for many basic requirements.

\subsection{living Cells all have a similar Basic Chemistry}

While it seemed easy enough to recognize life, it was remarkably difficult to 
say in what sense all living things were alike.
We now know that cells resemble one another to an aston-
ishing degree in the details of their chemistry, sharing the same machinery
for the most basic functions. All cells are composed of the same sorts of
molecules that participate in the same types of chemical reactions (dis-
cussed in Chapter 2). In all living things, genetic instructions—genes—are
stored in DNA molecules, written in the same chemical code, constructed
out of the same chemical building blocks, interpreted by essentially the
same chemical machinery, and duplicated in the same way to allow the
organism to reproduce. Thus, in every cell, the long DNA polymer chains
are made from the same set of four monomers, called nucleotides, strung
together in different sequences like the letters of an alphabet to convey
different information. In every cell, the instructions in the DNA are read
out, or transcribed, into a chemically related set of polymers called RNA. 
RNA molecules have a variety of functions, but the major
class serve as messenger RNA: the messages carried by these molecules
are in turn translated into yet another type of polymer called a protein.

Protein molecules dominate the behavior of the cell, serving as struc-
tural supports, chemical catalysts, molecular motors, and so on. Proteins
are built from amino acids, and every living thing uses the same set of
20 amino acids to make its proteins. But the amino acids are linked in
different sequences, giving each type of protein molecule a different
three-dimensional shape, or conformation, just as different sequences
of letters spell different words.

If cells are the fundamental unit of living matter, then nothing less than a
cell can truly be called living.

\subsection{All present-day Cells have apparently evolved from the same ancestor}

A cell reproduces by duplicating its DNA and then dividing in two, passing
a copy of the genetic instructions encoded in its DNA to each of its daugh-
ter cells. That is why daughter cells resemble the parent cell. However,
the copying is not always perfect, and the instructions are occasionally
corrupted by mutations that change the DNA. That is why daughter cells
do not always match the parent cell exactly.
Mutations can create offspring that are changed for the worse (in that
they are less able to survive and reproduce), changed for the better (in
that they are better able to survive and reproduce), or changed neutrally
(in that they are genetically different but equally viable). The struggle
for survival eliminates the first, favors the second, and tolerates the
third. The genes of the next generation will be the genes of the survi-
vors. Intermittently, the pattern of descent may be complicated by sexual
reproduction, in which two cells of the same species fuse, pooling their
DNA; the genetic cards are then shuffled, re-dealt, and distributed in new
combinations to the next generation, to be tested again for their survival
value.
These simple principles of genetic change and selection, applied repeat-
edly over billions of cell generations, are the basis of evolution—the
process by which living species become gradually modified and adapted
to their environment in more and more sophisticated ways.

\subsection{Genes provide the instructions for Cellular form, function, and Complex Behavior}

A cell’s genome—that is, the entire library of genetic information in its
DNA—provides a genetic program that instructs the cell how to func-
tion, and, for plant and animal cells, how to grow into an organism with
hundreds of different cell types.
Yet all these differentiated cell types are generated during
embryonic development from a single fertilized egg cell, and all contain
identical copies of the DNA of the species.
Different
cells express different genes—that is, they use their genes to produce
some proteins and not others, depending on the cues that they and their
ancestor cells have received from their surroundings.
Each cell is capable of carrying out a variety of biological
tasks, depending on its environment and its history, using the informa-
tion encoded in its DNA to guide its activities.

\section{Cells under the microscope}

Cells, in general, are very small—too small to be seen with the naked
eye. They were not made visible until the seventeenth century, when the
microscope was invented. For hundreds of years afterward, all that was
known about cells was discovered using this instrument. Light micro-
scopes, which use visible light to illuminate specimens, are still vital
pieces of equipment in the cell biology laboratory.
Although these instruments now incorporate many sophisticated
improvements, the properties of light itself set a limit to the fineness of
detail they can reveal. Electron microscopes, invented in the 1930s, go
beyond this limit by using beams of electrons instead of beams of light
as the source of illumination, greatly extending our ability to see the fine
details of cells and even making some of the larger molecules visible
individually.

\subsection{The invention of the light Microscope led to the discovery of Cells}

Schleiden and Schwann documented the results
of a systematic investigation of plant and animal tissues with the light
microscope, showing that cells were the universal building blocks of all
living tissues. Their work, and that of other nineteenth-century micro-
scopists, slowly led to the realization that all living cells are formed by
the division of existing cells - a principle sometimes referred to as the
cell theory. The implication that living organisms do not arise
spontaneously but can be generated only from existing organisms was
hotly contested, but it was finally confirmed by experiments performed in
the 1860s by Louis Pasteur.

\subsection{Cells, organelles, and even Molecules Can Be seen Under the Microscope}

If you cut a very thin slice of a suitable plant or animal tissue and place
it under a light microscope, you will see that the tissue is divided into
thousands of small cells. These may be either closely packed or separated
from one another by an extracellular matrix, a dense material often made
of protein fibers embedded in a polysaccharide gel.
If you have taken care
to keep your specimen under the right conditions, you will see that the
cells show signs of life: particles move around inside them, and if you
watch patiently you may see a cell slowly change shape and divide into
two.

The cell thus revealed has a distinct anatomy. It has a sharply
defined boundary, indicating the presence of an enclosing membrane.
A large, round body, the nucleus, is prominent in the middle of the cell.
Around the nucleus and filling the cell’s interior lies the cytoplasm, a
transparent substance crammed with what seems at first to be a jumble
of miscellaneous objects.

When thin sections are cut, stained, and placed in the electron micro-
scope, much of the jumble of cell components becomes sharply resolved
into distinct organelles—separate, recognizable substructures that are
only hazily defined under the light microscope. A delicate membrane,
about 5 nm thick, is visible enclosing the cell, and similar membranes
form the boundary of many of the organelles inside. The
external membrane is called the plasma membrane, while the membranes
surrounding organelles are called internal membranes.

\section{The Procaryotic Cell}

Of all the types of cells revealed by the microscope, bacteria have the
simplest structure and come closest to showing us life stripped down to
its essentials. Indeed, a bacterium contains essentially no organelles - not
even a nucleus to hold its DNA. This property - the presence or absence of
a nucleus - is used as the basis for a simple but fundamental classification
of all living things. Organisms whose cells have a nucleus are called
eucaryotes (from the Greek words eu, meaning “well” or “truly,” and
karyon, a “kernel” or “nucleus”). Organisms whose cells do not have a
nucleus are called procaryotes (from pro, meaning “before”). The terms
“bacterium” and “procaryote” are often used interchangeably, although
we shall see that the category of procaryotes also includes another class
of cells, the archaea (singular archaeon), which are so remotely related to
bacteria that they are given a separate name.
They often have a tough protective coat, called a cell wall, surrounding the plasma
membrane, which encloses a single compartment containing the cytoplasm and the DNA.

\subsection{Procaryotes are the Most diverse of Cells}

In shape and structure, procaryotes may seem simple and limited,
but in terms of chemistry, they are the most diverse and inventive class
of cells.
As we discuss later in this chapter, mitochondria - the organelles that 
generate energy for the eucaryotic cell - are thought to have evolved from 
aerobic bacteria that took to living inside the anaerobic ancestors of 
today’s eucaryotic cells.

Virtually any organic material, from wood to petroleum, can be used as
food by one sort of bacterium or another. Even more remarkably, some
procaryotes can live entirely on inorganic substances.
Some of these procaryotic cells, like plant cells, perform photosynthesis, 
getting energy from sunlight; others derive energy from the chemical 
reactivity of inorganic substances in the environment.

It is almost certain that the organelles in the plant cell that
perform photosynthesis - the chloroplasts - have evolved from photosynthetic 
bacteria that found a home inside the plant cell’s cytoplasm.

\subsection{The World of procaryotes is divided into two domains: Bacteria and archaea}

There is a gulf within the class of procaryotes, dividing it into two distinct 
domains called the bacteria (or sometimes eubacteria) and the archaea. 
Remarkably, at a molecular level, the members of these two domains differ as 
much from one another as either does from the eucaryotes.
Archaea are found not only in these habitats, but also
in environments hostile to most other cells.

\section{The Eucaryotic Cell}

Eucaryotic cells, in general, are bigger and more elaborate than bacteria
and archaea. All of the more complex multicellular organisms - including
plants, animals, and fungi - are formed from eucaryotic cells.

By definition, all eucaryotic cells have a nucleus.

\subsection{The nucleus is the information store of the Cell}

The nucleus is usually the most prominent organelle in a eucaryotic cell. 
It is enclosed within two concentric membranes that form
the nuclear envelope, and it contains molecules of DNA-extremely long
polymers that encode the genetic information of the organism. In the
light microscope, these giant DNA molecules become visible as individual
chromosomes when they become more compact as a cell prepares to
divide into two daughter cells.

\subsection{Mitochondria generate Usable energy from food to power the Cell}

Mitochondria are present in essentially all eucaryotic cells, and they are
among the most conspicuous organelles in the cytoplasm.
These organelles have a very distinctive structure when seen with an
electron microscope: each mitochondrion appears sausage - or wormshaped, 
from one to many micrometers long; and each is enclosed in
two separate membranes. The inner membrane is formed into folds that
project into the interior of the mitochondrion. Mitochondria
contain their own DNA and reproduce by dividing in two.

This revealed that mitochondria are generators of
chemical energy for the cell. They harness the energy from the oxidation
of food molecules, such as sugars, to produce adenosine triphosphate,
or ATP - the basic chemical fuel that powers most of the cell’s activities.
Because the mitochondrion consumes oxygen and releases carbon
dioxide in the course of this activity, the entire process is called cellular
respiration - essentially, breathing on a cellular level.

\subsection{Chloroplasts Capture energy from sunlight}

Chloroplasts are large green organelles that are found only in the cells
of plants and algae, not in the cells of animals or fungi. These organelles
have an even more complex structure than mitochondria: in addition to
their two surrounding membranes, chloroplasts possess internal stacks
of membranes containing the green pigment chlorophyll.

plants can get their energy directly from sunlight, and chloroplasts are 
the organelles that enable them to do so.
From the standpoint of life on Earth, chloroplasts carry out an even more
essential task than mitochondria: they perform photosynthesis - that is,
they trap the energy of sunlight in chlorophyll molecules and use this
energy to drive the manufacture of energy-rich sugar molecules. In the
process they release oxygen as a molecular by-product. Plant cells can
then extract this stored chemical when they need it, by oxidizing
energy these sugars in their mitochondria, just as animal cells do. 
Chloroplasts thus generate both the food molecules and the oxygen that all 
mitochondria use.

\subsection{Internal Membranes Create intracellular Compartments with different functions}

The cytoplasm contains a profusion of other organelles - most of them surrounded 
by single membranes - that perform many distinct functions. Most of these structures
are involved with the cell’s ability to import raw materials and to export
manufactured substances and waste products.

The endoplasmic reticulum (ER) - an irregular maze of interconnected
spaces enclosed by a membrane - is the site where most
cell membrane components, as well as materials destined for export
from the cell, are made. Stacks of flattened membrane-enclosed sacs
constitute the Golgi apparatus, which receives and often
chemically modifies the molecules made in the endoplasmic reticulum
and then directs them to the exterior of the cell or to various locations
inside the cell. Lysosomes are small, irregularly shaped organelles in
which intracellular digestion occurs, releasing nutrients from food particles 
and breaking down unwanted molecules for recycling or excretion.
And peroxisomes are small, membrane-enclosed vesicles that provide a
contained environment for reactions in which hydrogen peroxide, a dangerously 
reactive chemical, is generated and degraded. Membranes also
form many different types of small vesicles involved in the transport of
materials between one membrane-enclosed organelle and another.

A continual exchange of materials takes place between the endoplasmic 
reticulum, the Golgi apparatus, the lysosomes, and the outside of the
cell. The exchange is mediated by small vesicles that pinch off from the
membrane of one organelle and fuse with another, like tiny soap bubbles
budding from and rejoining larger bubbles. At the surface of the cell, for
example, portions of the plasma membrane tuck inward and pinch off
to form vesicles that carry material captured from the external medium
into the cell. These vesicles fuse with membrane-enclosed
endosomes, which mature into lysosomes, where the imported material
is digested. Animal cells can engulf very large particles, or even entire
foreign cells, by this process of endocytosis. The reverse process, exocytosis, 
whereby vesicles from inside the cell fuse with the plasma membrane
and release their contents into the external medium, is also a common
cellular activity.

\subsection{The Cytosol is a Concentrated aqueous gel of large and
small Molecules}

If we were to strip the plasma membrane from a eucaryotic cell and then
remove all of its membrane-enclosed organelles, including nucleus,
endoplasmic reticulum, Golgi apparatus, mitochondria, chloroplasts, and
so on, we would be left with the cytosol. The cytosol is the site of many chemical 
reactions that are fundamental to the cell’s existence. The early steps
in the breakdown of nutrient molecules take place in the cytosol, for
example, and it is here that the cell performs one of its key synthetic processes 
- the manufacture of proteins. Ribosomes, the molecular machines
that make the protein molecules, are visible with the electron microscope
as small particles in the cytosol, often attached to the cytosolic face of the
endoplasmic reticulum

\subsection{The Cytoskeleton is responsible for directed Cell Movements}

In eucaryotic cells the cytosol is criss-crossed by long, fine filaments of protein. Frequently the
filaments are seen to be anchored at one end to the plasma membrane
or to radiate out from a central site adjacent to the nucleus. This system of 
filaments is called the cytoskeleton. The thinnest
of the filaments are actin filaments, which are present in all eucaryotic
cells but occur in especially large numbers inside muscle cells, where
they serve as part of the machinery that generates contractile forces. The
thickest filaments are called microtubules, because they have the form of
minute hollow tubes. In dividing cells they become reorganized into a
spectacular array that helps pull the duplicated chromosomes in opposite
directions and distribute them equally to the two daughter cells. 
Intermediate in thickness between actin filaments and microtubules are the 
intermediate filaments, which serve to strengthen the cell
mechanically.

\subsection{The cytoplasm is Far from Static}

The cell interior is in constant motion. The cytoskeleton is a dynamic
jungle of ropes and rods that are continually being strung together and
taken apart; its filaments can assemble and then disappear in a matter
of minutes. Along these tracks and cables, organelles and
vesicles hurry to and from, racing across the width of the cell in a second or so. The
endoplasmic reticulum and the molecules that fill every free space are in
frantic thermal commotion - with unattached proteins buzzing around
so fast that, even though they move at random, they visit every corner
of the cell within a few seconds, constantly colliding with an even more
tumultuous dust storm of smaller organic molecules.

\section{Model organisms}

\subsection{Comparing genome sequences reveals life’s Common heritage}

DNA sequencing has made it easy to detect family resemblances between
genes: if two genes from different organisms have closely similar DNA
sequences, it is highly probable that both genes descended from a com-
mon ancestral gene. Genes (and gene products) related in this way are
said to be homologous.

\section{Essential concepts}

Cells are the fundamental units of life. All present-day cells are
believed to have evolved from an ancestral cell that existed more
than 3 billion years ago.
\begin{itemize}
\item All cells, and hence all living things, grow, convert energy from one
form to another, sense and respond to their environment, and reproduce themselves.
\item All cells are enclosed by a plasma membrane that separates the inside
of the cell from the environment.
\item All cells contain DNA as a store of genetic information and use it to
guide the synthesis of RNA molecules and of proteins.
\item Cells in a multicellular organism, though they all contain the same
DNA, can be very different. They turn on different sets of genes
according to their developmental history and to cues they receive
from their environment.
\item Cells of animal and plant tissues are typically 5–20 m m in diameter
and can be seen with a light microscope, which also reveals some of
their internal components, or organelles.
\item The electron microscope permits the smaller organelles and even
individual large molecules to be seen, but specimens require elaborate preparation and cannot be viewed alive.
\item The simplest of present-day living cells are procaryotes: although
they contain DNA, they lack a nucleus and other organelles and probably resemble most closely the ancestral cell.
\item Different species of procaryotes are diverse in their chemical
capabilities and inhabit an amazingly wide range of habitats. Two
fundamental evolutionary subdivisions are recognized: bacteria and
archaea.
\item Eucaryotic cells possess a nucleus and other organelles not found in
procaryotes. They probably evolved in a series of stages. An important step appears to have been the acquisition of mitochondria, which
are thought to have originated from bacteria engulfed by an ancestral
eucaryotic cell.
\item The nucleus is the most prominent organelle in most plant and animal cells. It contains the genetic information of the organism, stored
in DNA molecules. The rest of the cell’s contents, apart from the
nucleus, constitute the cytoplasm.
\item The cytoplasm includes all of the cell’s contents outside the nucleus.
It contains a variety of membrane-enclosed organelles with specialized chemical functions. Mitochondria carry out the oxidation of food
molecules. In plant cells, chloroplasts perform photosynthesis. The
endoplasmic reticulum, the Golgi apparatus, and lysosomes permit cells to synthesize complex molecules for export from the cell
and for insertion in cell membranes, and to import and digest large
molecules.
\item Outside the membrane-enclosed organelles in the cytoplasm is the
cytosol, a concentrated mixture of large and small molecules that
carry out many essential biochemical processes.
\item The cytoskeleton extends throughout the cytoplasm. This system of
protein filaments is responsible for cell shape and movement and
for the transport of organelles and molecules from one location to
another in the cytoplasm.
\item Free-living, single-celled eucaryotic microorganisms include some of
the most complex eucaryotic cells known, and they are able to swim,
mate, hunt, and devour food.
\item An animal, plant, or fungus consists of diverse eucaryotic cell types
all derived from a single fertilized egg cell; the number of such cells
cooperating to form a large multicellular organism such as a human
runs into thousands of billions.
\item Biologists have chosen a small number of model organisms to study
closely. These include the bacterium E. coli, brewer’s yeast, a nematode worm, a fly, a small plant, a fish, a mouse, and the human
species itself.
\item Although the minimum number of genes needed for a viable cell is
less than 400, most cells contain significantly more. Yet even such a
complex organism as a human has only about 24,000 protein-coding
genes—twice as many as a fly and seven times as many as E. coli.
\end{itemize}

\chapter{Chemical Components of Cells}






















\chapter{Energy, Catalysis and Biosynthesis}