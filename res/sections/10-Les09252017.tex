\chapter{Sequencing}

%todo slides 1-19

{Single cycle: washing ingredients}

Multiple washing steps for read of single nucleotide (single cycle). Depending 
on manufacturer. Typical ingredients:

\begin{itemize}
\item polymerase;
\item nucleotides;
\item fluorescence;
\item "temporary blockage";
\item light to excite fluorescence.
\end{itemize}

{NGS output}

Millions of sequenced reads (we do not know their order). Two computational 
solutions: alignment to a reference genome (or shorter sequence) of the reads, 
or de novo assembly of a genome.

{Alignment to reference (read mapping)}

Input: millions of sequenced framents, a reference genome sequence.

Process: String matching of the sequence reads against the reference (allow for 
small variations). Reference and sequenced organism need to be closely related 
(at least the same species).

Output: alignment and sequence of new strain.

{Reference alignment}

Math sequence to a given genome. Same species different individual, closely 
related species. Red area is the goodness of the coverage.

%todo copy image

Match sequence to a `similar' genome

%todo copy image

{How to search between millions of genomes}

%todo

{Mapping reads - Software}

General alignment programs: BLAST (and variants)

Specialised programs: %todo

Choice of software depends what you want to use it for (e.g. how does the 
aligner handle split reads, mismatches, indels).

{Question - Mismatches}

What may be the cause of any mismatches alignment between the sequence read 
and the reference genome? Sequencing errors, contamination or errors induced by 
duplication of DNA


Would it be useful if we could differentiate between different sources of 
noise? How could this be done? Sequencing errors.

{Depth of coverage}

Usage of sequencing ``depth'' (avg on the whole genome) and sequencing ``coverage'': 
\begin{itemize}
\item sequencing depth: average number of reads per base (ofter over entire 
sequencing sample);
\item coverage: number of reads per base (ofter on specific region).
\end{itemize}

{NGS output}

%tood

{Assembly questions}

Given a pile of reads, you do not know the order, you do not have a good 
reference genome. What would you do?
Looking for overlaps between my reads and my genome and between reads.

Do you think the sequencing depth matters? Would it be easier to solve this 
problem with a low or high sequecing depth?
Easier with high sequencing depth (lot of reads make it easier).

{De novo assembly}

%todo copy
Output: contigs %todo check slides, this is the right output

{Assembly: puzzling reads into a genome}

%todo copy slides
