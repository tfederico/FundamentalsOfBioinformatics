\chapter{From DNA to Protein: How Cells Read the Genome}

\section{From DNA to RNA}

Transcription and translation are the means by which cells read out, or
express, their genetic instructions - their genes. Many identical RNA copies
can be made from the same gene, and each RNA molecule can direct
the synthesis of many identical protein molecules. Because each cell
contains only one or two copies of any particular gene, this successive
amplification enables cells to rapidly synthesize large amounts of protein
whenever necessary. At the same time, each gene can be transcribed
and translated with a different efficiency, providing the cell with a way
to make vast quantities of some proteins and tiny quantities of others.

\subsection{Portions of DNA sequence Are Transcribed into RNA}

The first step a cell takes in reading out one of its many thousands of
genes is to copy the nucleotide sequence of that gene into RNA. The process
is called transcription because the information, though copied into
another chemical form, is still written in essentially the same language -
the language of nucleotides. Like DNA, RNA is a linear polymer made of
four different types of nucleotide subunits linked together by phosphodiester
bonds. It differs from DNA chemically in two respects:
(1) the nucleotides in RNA are ribonucleotides—that is, they contain the
sugar ribose (hence the name ribonucleic acid) rather than deoxyribose;
(2) although, like DNA, RNA contains the bases adenine (A), guanine (G),
and cytosine (C), it contains uracil (U) instead of the thymine (T) found
in DNA.

Whereas DNA always occurs in cells as a
double-stranded helix, RNA is single-stranded. This difference has important
functional consequences. Because an RNA chain is single-stranded,
it can fold up into a variety of shapes, just as a polypeptide chain folds
up to form the final shape of a protein; double-stranded DNA cannot fold in this fashion.

\subsection{Transcription produces RNA Complementary to One strand of DNA}

All of the RNA in a cell is made by transcription, a process that has certain
similarities to DNA replication.
The RNA chain produced by transcription - the transcript - is therefore
elongated one nucleotide at a time and has a nucleotide sequence exactly
complementary to the strand of DNA used as the template.

Transcription, however, differs from DNA replication in several crucial
features. Unlike a newly formed DNA strand, the RNA strand does not
remain hydrogen-bonded to the DNA template strand.

The enzymes that carry out transcription are called RNA polymerases.
RNA polymerases catalyze the formation of the phosphodiester bonds that
link the nucleotides together and form the sugar-phosphate
backbone of the RNA chain.

The almost immediate release of the RNA strand from the DNA as it is
synthesized means that many RNA copies can be made from the same
gene in a relatively short time; the synthesis of the next RNA is usually
started before the first RNA has been completed.

Although RNA polymerase catalyzes essentially the same chemical reaction
as DNA polymerase, there are some important differences between
the two enzymes. First, and most obviously, RNA polymerase catalyzes
the linkage of ribonucleotides, not deoxyribonucleotides. Second, unlike
the DNA polymerase involved in DNA replication, RNA polymerases can
start an RNA chain without a primer. This difference may exist because
transcription need not be as accurate as DNA replication; unlike DNA,
RNA is not used as the permanent storage form of genetic information in
cells, so mistakes in RNA transcripts have relatively minor consequences.

\subsection{Several Types of RNA Are produced in Cells}

The vast majority of genes carried in a cell’s DNA specify the amino
acid sequence of proteins, and the RNA molecules that are copied from
these genes (and that ultimately direct the synthesis of proteins) are collectively
called messenger RNA (mRNA). In eucaryotes, each mRNA
typically carries information transcribed from just one gene, coding for a
single protein; in bacteria, a set of adjacent genes is often transcribed as
a single mRNA that therefore carries the information for several different
proteins.

As we shall see in later sections of this chapter, these nonmessenger RNAs,
like proteins, serve as regulatory, structural, and enzymatic components
of cells, and they play key parts in translating the genetic message into
protein. Ribosomal RNA (rRNA) forms the core of the ribosomes, on which
mRNA is translated into protein, and transfer RNA (tRNA) forms the adaptors
that select amino acids and hold them in place on a ribosome for
their incorporation into protein. Other small RNAs, called microRNAs
(miRNAs), serve as key regulators of eucaryotic gene expression.

In the broadest sense, the term gene expression refers to the process
by which the information encoded in a DNA sequence is translated into
a product that has some effect on a cell or organism. In cases where
the final product of the gene is a protein, gene expression includes both
transcription and translation. When an RNA molecule is the gene’s final
product, however, gene expression does not require translation.

\subsection{Signals in DNA Tell RNA polymerase Where to start and Finish}

To begin transcription, RNA polymerase must be able to recognize the start of
a gene and bind firmly to the DNA at this site.

When an RNA polymerase collides randomly with a piece of DNA,
it sticks weakly to the double helix and then slides rapidly along. The
enzyme latches on tightly only after it has encountered a region called a
promoter, which contains a specific sequence of nucleotides indicating
the starting point for RNA synthesis. Once the RNA polymerase reaches
the promoter and binds tightly to the DNA, it opens up the double helix
immediately in front of it to expose the nucleotides on each strand of a
short stretch of DNA. One of the two exposed DNA strands
then acts as a template for complementary base-pairing with incoming
ribonucleotides, two of which are joined together by the polymerase to
begin the RNA chain. Chain elongation then continues until the enzyme
encounters a second signal in the DNA, the terminator (or stop site),
where the polymerase halts and releases both the DNA template and the
newly made RNA chain.

A subunit of the bacterial polymerase, called sigma ($\sigma$) factor, is primarily
responsible for recognizing the promoter sequence on DNA. Once the
polymerase has latched onto the DNA at this site and has synthesized
approximately 10 nucleotides of RNA, the sigma factor disengages, enabling
the polymerase to move forward and continue transcribing without it.

The promoter is asymmetrical and binds the polymerase in only one orientation; thus, once properly
positioned on a promoter, the RNA polymerase has no option but to tran-
scribe the appropriate DNA strand, since transcription can proceed only
in the 5`-to-3` direction.

\subsection{Initiation of eucaryotic gene Transcription is a Complex process}

Many of the principles outlined above for bacterial transcription also
apply to eucaryotes. However, transcription initiation in eucaryotes differs
in several important ways from that in bacteria:

\begin{itemize}
\item The first difference lies in the RNA polymerases themselves. While
bacteria contain a single type of RNA polymerase, eucaryotic
cells have three - RNA polymerase I, RNA polymerase II, and RNA
polymerase III. These polymerases are responsible for transcrib-
ing different types of genes. RNA polymerases I and III transcribe
the genes encoding transfer RNA, ribosomal RNA, and small RNAs
that play structural and catalytic roles in the cell.
RNA polymerase II transcribes the vast majority of eucaryotic genes,
including all those that encode proteins.
\item A second difference is that whereas the bacterial RNA polymerase
(along with its sigma subunit) is able to initiate transcription
on its own, eucaryotic RNA polymerases require the assistance of
a large set of accessory proteins. Principal among these are the
general transcription factors, which must assemble at each pro-
moter along with the polymerase before the polymerase can begin
transcription.
\item A third distinctive feature of transcription in eucaryotes is that
the mechanisms that control its initiation are much more elaborate
than those in prokaryotes. In bacteria, genes tend to lie very close to one another
in the DNA, with only very short lengths of nontranscribed DNA
between them. But in plants and animals, including humans, individual
genes are spread out along the DNA, with stretches of up
to 100,000 nucleotide pairs between one gene and the next. This
architecture allows a single gene to be controlled by an almost
unlimited number of regulatory sequences scattered along the
DNA, and enables eucaryotes to engage in more complex forms of
transcriptional regulation than bacteria do.
\item Last but not least, eucaryotic transcription initiation must take into
account the packing of DNA into nucleosomes and more compact
forms of chromatin structure.
\end{itemize}

\subsection{Eucaryotic RNA polymerase Requires general Transcription Factors}

The initial finding that, unlike bacterial RNA polymerase, purified eucaryotic
RNA polymerase II could not on its own initiate transcription in vitro
led to the discovery and purification of the general transcription factors.
These accessory proteins assemble on the promoter, where they position
the RNA polymerase, pull apart the double helix to expose the template
strand, and launch the RNA polymerase, to begin transcribing.

The assembly process typically
begins with the binding of the general transcription factor TFIID to a short
double-helical DNA sequence primarily composed of T and A nucleotides;
because of its composition, this sequence is known as the TATA sequence,
or TATA box. Upon binding to DNA, TFIID causes a dramatic local distortion in the DNA,
which helps to serve as a landmark for the
subsequent assembly of other proteins at the promoter. The TATA box is
a key component of many promoters used by RNA polymerase II, and it
is typically located 25 nucleotides upstream from the transcription start
site. Once the first general transcription factor has bound to this DNA site,
the other factors are assembled, along with RNA polymerase II, to form a
complete transcription initiation complex.


\subsection{Eucaryotic RNAs Are Transcribed and processed simultaneously in the Nucleus}

Bacterial DNA lies directly exposed to the cytoplasm,
which contains the ribosomes on which protein synthesis takes place.
As mRNA molecules in bacteria are transcribed, ribosomes immediately
attach to the free 5` end of the RNA transcript and protein synthesis
starts.

In eucaryotic cells, by contrast, DNA is enclosed within the nucleus.
Transcription takes place in the nucleus, but protein synthesis takes place
on ribosomes in the cytoplasm. So, before a eucaryotic mRNA can be
translated, it must be transported out of the nucleus through small pores
in the nuclear envelope. Before a eucaryotic RNA exits the
nucleus, however, it must go through several different RNA processing
steps. These reactions take place as the RNA is being transcribed. The
enzymes responsible for RNA processing ride on the ‘tail’ of the eucaryo-
tic RNA polymerase as it transcribes an RNA, processing the transcript as
it emerges from the RNA polymerase.

Depending on which type of RNA is being produced - an mRNA or some
other type - the transcripts are processed in various ways before leaving
the nucleus. Two processing steps that occur only on transcripts destined
to become mRNA molecules are capping and polyadenylation:

\begin{enumerate}
\item RNA capping involves a modification of the 5` end of the mRNA
transcript, the end that is synthesized first during transcription. The
RNA is capped by the addition of an atypical nucleotide - a guanine
(G) nucleotide with a methyl group attached. This capping occurs
after the RNA polymerase has produced about 25 nucleotides of
RNA, long before it has completed transcribing the whole gene.
\item Polyadenylation provides newly transcribed mRNAs with a special
structure at their 3` ends. In contrast with bacteria, where the 3`
end of an mRNA is simply the end of the chain synthesized by the
RNA polymerase, the 3` ends of eucaryotic RNAs are first trimmed
by an enzyme that cuts the RNA chain at a particular sequence of
nucleotides and are then finished off by a second enzyme that adds
a series of repeated adenine (A) nucleotides to the cut end. This
poly-A tail is generally a few hundred nucleotides long.
\end{enumerate}

\subsection{Eucaryotic genes Are interrupted by Noncoding sequences}

Most eucaryotic RNAs have to undergo an additional processing step
before they are functional.

Most eucaryotic genes, in contrast, have their coding sequences interrupted
by long, noncoding intervening sequences, called introns. The scattered pieces
of coding sequences, or expressed sequences, called exons, are usually shorter than
the introns, and the coding portion of a eucaryotic gene is often only a
small fraction of the total length of the gene.

\subsection{Introns Are Removed by RNA splicing}

To produce an mRNA in a eucaryotic cell, the entire length of the gene,
introns as well as exons, is transcribed into RNA. After capping, and as
the RNA polymerase continues to transcribe the gene, the process of RNA
splicing begins, in which the intron sequences are removed from the
newly synthesized RNA and the exons are stitched together. Each transcript
ultimately receives a poly-A tail.
Once a transcript has been spliced and its 5` and
3` ends have been modified, the RNA is a functional mRNA molecule that
can now leave the nucleus and be translated into protein.

Unlike the other steps of mRNA production we have discussed, RNA
splicing is carried out largely by RNA molecules instead of proteins. RNA
molecules recognize intron–exon boundaries (through complementary
base-pairing) and participate intimately in the chemistry of splicing.
These RNA molecules, called small nuclear RNAs (snRNAs), are packaged
with additional proteins to form small nuclear ribonucleoprotein
particles (snRNPs, pronounced “snurps”). The snRNPs form the core of
the spliceosome, the large assembly of RNA and protein molecules that
carries out RNA splicing in the cell.

The intron-exon type of gene arrangement in eucaryotes at first seems
wasteful, but it does have positive consequences. First, the transcripts of
many eucaryotic genes can be spliced in different ways, each of which
can produce a distinct protein. Such alternative splicing thereby allows
many different proteins to be produced from the same gene.
Thus RNA splicing enables eucaryotes to increase the already enormous
coding potential of their genomes.

The intron-exon structure of genes is thought to have speeded up the emergence of new and
useful proteins. The inclusion of long introns makes genetic recombination
between exons of different genes much more likely. This means that
genes for new proteins could have evolved quite rapidly by the combination
of parts of preexisting genes, a mechanism resembling the assembly
of a new type of machine from a kit of preexisting functional components.
Indeed, many proteins in present-day cells resemble patchworks
composed from a common set of protein pieces, called protein domains.

\subsection{Mature eucaryotic mRNAs Are selectively exported from the Nucleus}

We have seen how eucaryotic mRNA synthesis and processing takes
place in an orderly fashion within the cell nucleus. However, these events
create a special problem for eucaryotic cells: of the total mRNA that is
synthesized, only a small fraction -t he mature mRNA - is useful to the
cell. The remaining RNA fragments-excised introns, broken RNAs, and
aberrantly spliced transcripts - are not only useless but could be dangerous
to the cell if not destroyed. How, then, does the cell distinguish
between the relatively rare mature mRNA molecules it needs to keep and
the overwhelming amount of debris generated by RNA processing?

The answer is that the transport of mRNA from the nucleus to the
cytoplasm, where it is translated into protein, is highly selective: only
correctly processed RNAs are allowed to pass. This reliance on proper
processing for RNA transport is mediated by the nuclear pore complex,
which recognizes and exports only completed mRNAs. These aqueous
pores connect the nucleoplasm with the cytosol.
They act as gates that control which macromolecules can
enter or leave the nucleus.

The ‘waste RNAs’ that remain behind in the nucleus are degraded, and the building
blocks are reused for transcription.

\subsection{mRNA Molecules Are eventually Degraded by the Cell}

Because a single mRNA molecule can be translated many times, the length of time
that a mature mRNA molecule persists in the cell affects the amount of protein it produces.
Each mRNA molecule is eventually degraded into nucleotides by RNases, but the lifetimes
of mRNA molecules differ considerably - epending on the nucleotide
sequence of the mRNA and the type of cell in which the mRNA is produced.

These different lifetimes are in part controlled by nucleotide sequences
that are present in the mRNA itself, most often in the portion of RNA
called the 3` untranslated region, that lies between the 3` end of the cod-
ing sequence and the poly-A tail.

\subsection{The earliest Cells May have had introns in Their genes}

The process of transcription is universal: all cells use RNA polymerase,
coupled with complementary base-pairing, to synthesize RNA from DNA.

\section{From RNA to Protein}

How is the information in a linear
sequence of nucleotides in RNA translated into the linear sequence of a
chemically quite different set of subunits—the amino acids in proteins?

\subsection{An mRNA sequence is Decoded in sets of Three Nucleotides}

The conversion of the information in RNA into protein represents a translation
of the information into another language that uses
quite different symbols. Because there are only 4 different nucleotides in
mRNA but 20 different types of amino acids in a protein, this translation
cannot be accounted for by a direct one-to-one correspondence between
a nucleotide in RNA and an amino acid in protein. The rules by which the
nucleotide sequence of a gene, through the medium of mRNA, is translated
into the amino acid sequence of a protein are known as the genetic
code.

Because RNA is a linear polymer made of four different 
nucleotides, there are thus $4 \cdot 4 \cdot 4 = 64$ possible combinations of
three nucleotides. However, only 20 different amino acids are commonly
found in proteins. Either some nucleotide triplets are never used, or
the code is redundant and some amino acids are
specified by more than one triplet. The second possibility is, in fact, correct.
Each group of three consecutive nucleotides in RNA is called a codon,
and each specifies one amino acid.

In principle, an RNA sequence can be translated in any one of three different
reading frames, depending on where the decoding process begins. However,
only one of the three possible reading frames in an mRNA specifies the correct protein.

\subsection{tRNA Molecules Match Amino Acids to Codons in mRNA}

The codons in an mRNA molecule do not directly recognize the amino
acids they specify: the group of three nucleotides does not, for example,
bind directly to the amino acid. Rather, the translation of mRNA into
protein depends on adaptor molecules that can recognize and bind to a
codon at one site on their surface and to an amino acid at another site.
These adaptors consist of a set of small RNA molecules known as transfer RNAs (tRNAs).

We saw earlier that an RNA molecule will generally fold into a three-dimensional
structure by forming base pairs between different regions
of the molecule. If the base-paired regions are sufficiently extensive, they
will form a double-helical structure, like that of double-stranded DNA.
The tRNA molecule provides a striking example of this. Four short segments
of the folded tRNA are double-helical, producing a molecule that
looks like a cloverleaf when drawn schematically.
Two regions of unpaired nucleotides situated at either end of the L-shaped
molecule are crucial to the function of tRNA in protein synthesis. One
of these regions forms the anticodon, a set of three consecutive nucle-
otides that through base-pairing bind the complementary codon in an
mRNA molecule. The other is a short single-stranded region at the 3` end
of the molecule; this is the site where the amino acid that matches the
codon is attached to the tRNA.

Some amino acids have more than one tRNA and some tRNAs are constructed so that
they require accurate base-pairing only at the fi rst two positions of the
codon and can tolerate a mismatch (or wobble) at the third position. This
wobble base-pairing explains why so many of the alternative codons
for an amino acid differ only in their third nucleotide.

\subsection{Specific enzymes Couple tRNAs to the Correct Amino Acid}

We have seen that in order to read the genetic code in DNA, cells make
many different tRNAs. We now must consider how each tRNA molecule
becomes charged—linked to the one amino acid in 20 that is its right partner.
Recognition and attachment of the correct amino acid depend on
enzymes called aminoacyl-tRNA synthetases, which covalently couple
each amino acid to its appropriate set of tRNA molecules. In most organisms,
there is a different synthetase enzyme for each amino acid. The
synthetases are equal in importance to the tRNAs in the decoding process,
because it is the combined action of synthetases and tRNAs that
allows each codon in the mRNA molecule to associate with its proper
amino acid.

The synthetase-catalyzed reaction that attaches the amino acid to the 3`
end of the tRNA is one of many reactions coupled to the energy-releasing
hydrolysis of ATP, and it produces a high-energy bond
between the charged tRNA and the amino acid.

\subsection{The RNA Message is Decoded on Ribosomes}

The recognition of a codon by the anticodon on a tRNA molecule depends
on the same type of complementary base-pairing used in DNA replication
and transcription. However, accurate and rapid translation of mRNA
into protein requires a large molecular machine that moves along the
mRNA, captures complementary tRNA molecules, holds them in position,
and covalently links the amino acids that they carry so as to form a
protein chain. This protein-manufacturing machine is the ribosome - a
large complex made from more than 50 different proteins (the ribosomal
proteins) and several RNA molecules called ribosomal RNAs (rRNAs). A
typical living cell contains millions of ribosomes in its cytoplasm.

Eucaryotic and procaryotic ribosomes are very similar in structure and
in function. Both are composed of one large and one small subunit that
fit together to form a complete ribosome with a mass of several million daltons.
The small subunit matches the tRNAs to
the codons of the mRNA, while the large subunit catalyzes the forma-
tion of the peptide bonds that covalently link the amino acids together
into a polypeptide chain.

How does the ribosome choreograph all the movements required for
translation? Each ribosome contains a binding site for an mRNA mol-
ecule and three binding sites for tRNA molecules, called the A-site, the
P-site, and the E-site.

\subsection{The Ribosome is a Ribozyme}

Not only are the three tRNA-binding sites (the A-, P-, and E-sites) on the
ribosome formed primarily by the rRNAs, but the catalytic site for peptide
bond formation is formed by the 23S RNA of the large subunit.

RNA molecules that possess catalytic activity are called ribozymes.

\subsection{Codons in mRNA signal Where to start and to stop protein synthesis}

The translation of an mRNA begins with the codon AUG, and a special
tRNA is required to initiate translation. This initiator tRNA always carries
the amino acid methionine (or a modified form of methionine, formylmethionine,
in bacteria) so that newly made proteins all have methionine
as the first amino acid at their N-terminal end, the end of a protein that
is synthesized first. This methionine is usually removed later by a specific
protease. The initiator tRNA is distinct from the tRNA that normally carries methionine.
In eucaryotes, the initiator tRNA, coupled to methionine, is first loaded into
a small ribosomal subunit, along with additional proteins called translation
initiation factors. Of all the charged tRNAs in the cell,
only the charged initiator tRNA is capable of binding tightly to the P-site
of the small ribosomal subunit. Next, the loaded ribosomal subunit binds
to the 5` end of an mRNA molecule, which is signaled by the cap that is
present on eucaryotic mRNA. The small ribosomal sub-
unit then moves forward (5` to 3`) along the mRNA searching for the first
AUG. When this AUG is encountered, several initiation factors dissociate
from the small ribosomal subunit to make way for the large ribosomal
subunit to assemble and complete the ribosome. Because the initiator
tRNA is bound to the P-site, protein synthesis is ready to begin with the
addition of the next charged tRNA to the A-site.

The end of the protein-coding message in both procaryotes and eucaryotes
is signaled by the presence of one of several codons called stop codons.
These special codons - UAA, UAG, and UGA - are not
recognized by a tRNA and do not specify an amino acid, but instead signal
to the ribosome to stop translation. Proteins known as release factors bind
to any stop codon that reaches the A-site on the ribosome, and this binding
alters the activity of the peptidyl transferase in the ribosome, causing
it to catalyze the addition of a water molecule instead of an amino acid
to the peptidyl-tRNA. This reaction frees the carboxyl end
of the polypeptide chain from its attachment to a tRNA molecule, and
because this is the only attachment that holds the growing polypeptide to
the ribosome, the completed protein chain is immediately released into
the cytosol. The ribosome releases the mRNA and dissociates into its two
separate subunits, which can then assemble on another mRNA molecule
to begin a new round of protein synthesis.

Many proteins can fold into their three-dimensional shape spontaneously,
and some do so as they are spun out of the
ribosome. Most proteins, however, require molecular chaperones to help
them fold correctly in the cell.

\subsection{Proteins Are Made on polyribosomes}

If the mRNA is being translated efficiently, a new ribosome hops onto the 5`
end of the mRNA molecule almost as soon as the preceding ribosome has
translated enough of the nucleotide sequence to move out of the way.
The mRNA molecules being translated are therefore usually found in
the form of polyribosomes (also known as polysomes), large cytoplasmic
assemblies made up of several ribosomes spaced as close as 80 nucleotides
apart along a single mRNA molecule.

\subsection{Inhibitors of procaryotic protein synthesis Are used as Antibiotics}

Many of our most effective antibiotics are compounds that act by inhibiting
bacterial, but not eucaryotic, protein synthesis.

\subsection{Carefully Controlled protein breakdown helps Regulate the Amount of each protein in a Cell}

After a protein is released from the ribosome, it becomes subject to a
number of controls by the cell. The number of copies of a protein in a cell
depends, like the human population, not only on how quickly new individuals
are made but also on how long they survive. How does the cell control these lifetimes?

Cells possess specialized pathways to enzymatically break proteins
down into their constituent amino acids (a process termed proteolysis).
The enzymes that degrade proteins, first to short peptides and finally to
individual amino acids, are known collectively as proteases. Proteases
act by cutting (hydrolyzing) the peptide bonds between amino acids.

Although most damaged proteins are broken down in the cytosol,
important protein degradation pathways also operate in other compartments
in eucaryotic cells, such as lysosomes.
In eucaryotic cells, most proteins are broken down by machines called
proteasomes. A proteasome contains a central cylinder formed from
proteases whose active sites into an inner chamber.

How does the proteasome select which proteins in the cell should enter
the cylinder and be degraded? Proteasomes act primarily on proteins that
have been marked for destruction by the covalent attachment of a small
protein called ubiquitin. Specialized enzymes tag selected proteins with
one or more ubiquitin molecules; these ubiquitylated proteins are then
recognized and sucked into the proteasome by proteins in the stopper.

\subsection{There Are Many steps between DNA and protein}

The final concentration of a protein in
a cell therefore depends on the efficiency with which each of the many
steps is carried out. In addition, many proteins - once they leave the
ribosome - require further attention before they are useful to the cell.
We will see in the next chapter that cells have the ability to change the
concentrations of most of their proteins according to their needs.

\section{RNA and the origins of Life}

If nucleic acids are required
to direct the synthesis of proteins, and proteins are required to synthesize
nucleic acids, how could this system of interdependent components have
arisen? One view is that an RNA world existed on Earth before modern
cells appeared. According to this hypothesis, RNA - which
today serves as an intermediate between genes and proteins - both stored
genetic information and catalyzed chemical reactions in primitive cells.
Only later in evolutionary time did DNA take over as the genetic material
and proteins become the major catalysts and structural components of
cells. If this idea is correct, then the transition out of the RNA world was
never completed; as we have seen in this chapter, RNA still catalyzes
several fundamental reactions in modern cells. These RNA catalysts,
including the ribosome and RNA-splicing machinery, can thus be viewed
as molecular fossils of an earlier world.

\subsection{Life require autocatalysis}

Catalysts with this special self-promoting property, once they had arisen
by chance, would reproduce themselves and would therefore divert raw
materials from the production of other substances. In this way one can
envisage the gradual development of an increasingly complex chemical
system of organic monomers and polymers that function together to generate
more molecules of the same types, fueled by a supply of simple
raw materials in the environment. Such an autocatalytic system would
have many of the properties we think of as characteristic of living matter.

\subsection{RNA Can both store information and Catalyze Chemical Reactions}

We have seen that complementary base-pairing enables one nucleic acid
to act as a template for the formation of another. Such
complementary templating mechanisms lie at the heart of DNA replication
and transcription in modern-day cells.

But the efficient synthesis of polynucleotides by such complementary
templating mechanisms also requires catalysts to promote the polymerization
reaction. The unique potential of RNA molecules to
act both as information carriers and as catalysts is thought to have enabled
them to play the central role in the origin of life.

As we saw previously, protein enzymes are able to catalyze biochemical
reactions because they have surfaces with unique contours and chemical
properties on which a given substrate can react. In the same way,
RNA molecules, with their unique folded shapes, can serve as enzymes,
although the fact that they are constructed of only four different
subunits limits their catalytic efficiency and their range of chemical
reactions compared with proteins. Nonetheless, ribozymes can catalyze
many types of chemical reactions.

RNA, therefore, has all the properties required of a molecule that could
catalyze its own synthesis.

\subsection{RNA is Thought to predate DNA in evolution}

Evidence that RNA arose before DNA in evolution can be found in the
chemical differences between them. Ribose, like glucose and other
simple carbohydrates, is readily formed from formaldehyde
(HCHO), which is one of the principal products of experiments simulating
conditions on the primitive Earth. The sugar deoxyribose is harder to
make, and in present-day cells it is produced from ribose in a reaction
catalyzed by a protein enzyme, suggesting that ribose predates deoxyribose
in cells. Presumably, DNA appeared on the scene later, and then
proved more suited than RNA as a permanent repository of genetic information.
In particular, the deoxyribose in its sugar-phosphate backbone
makes chains of DNA chemically much more stable than chains of RNA,
so that greater lengths of DNA can be maintained without breakage.

The other differences between RNA and DNA - the double-helical structure
of DNA and the use of thymine rather than uracil - further enhance
DNA stability by making the molecule easier to repair.
Furthermore, deamination, one of the most common unwanted chemical changes occurring in
polynucleotides, is easier to detect and repair in DNA than in RNA.

As cells more closely resembling present-day cells appeared, it is
believed that many of the functions originally performed by RNA were
taken over by molecules more specifically fitted to the tasks required.
Eventually DNA took over the primary genetic function, and proteins
became the major catalysts, while RNA remained primarily as the intermediary
connecting the two.

\section{Essential concepts}

\begin{itemize}
\item The flow of genetic information in all living cells is DNA $\rightarrow$ RNA $\rightarrow$
protein. The conversion of the genetic instructions in DNA into RNAs
and proteins is termed gene expression.
\item To express the genetic information carried in DNA, the nucleotide
sequence of a gene is first transcribed into RNA. Transcription is catalyzed by the enzyme RNA polymerase. Nucleotide sequences in the
DNA molecule indicate to the RNA polymerase where to start and
stop transcribing.
\item RNA differs in several respects from DNA. It contains the sugar ribose
instead of deoxyribose and the base uracil (U) instead of thymine (T).
RNAs in cells are synthesized as single-stranded molecules, which
often fold up into precise three-dimensional shapes.
\item Cells make several different functional types of RNAs, including
messenger RNA (mRNA), which carries the instructions for making
proteins; ribosomal RNA (rRNA), which is a component of ribosomes;
and transfer RNA (tRNA), which acts as an adaptor molecule in protein synthesis.
\item Transcription begins at DNA sites called promoters. To initiate tran-
scription, eucaryotic RNA polymerases require the assembly of a
complex of general transcription factors at the promoter, whereas
bacterial RNA polymerase requires only an additional subunit, called
sigma factor.
\item In eucaryotic DNA, most genes are composed of a number of smaller
coding regions (exons) interspersed with noncoding regions (introns).
When a eucaryotic gene is transcribed from DNA into RNA, both the
exons and introns are copied.
\item Introns are removed from the RNA transcripts in the nucleus by the
process of RNA splicing. In a reaction catalyzed by small ribonucleoprotein
complexes known as snRNPs, the introns are excised from
the RNA and the exons are joined together.
\item Eucaryotic mRNAs go through several additional RNA processing
steps before they leave the nucleus, including RNA capping and polyadenylation.
These reactions, along with splicing, take place as the
RNA is being transcribed. The mature mRNA is then transported to
the cytoplasm.
\item Translation of the nucleotide sequence of mRNA into a protein takes
place in the cytoplasm on large ribonucleoprotein assemblies called
ribosomes. As the mRNA is threaded through a ribosome, its message
is translated into protein.
\item The nucleotide sequence in mRNA is read in sets of three nucleotides
(codons), each codon corresponding to one amino acid.
\item The correspondence between amino acids and codons is specified
by the genetic code. The possible combinations of the 4 different
nucleotides in RNA give 64 different codons in the genetic code. Most
amino acids are specified by more than one codon.
\item tRNA acts as an adaptor molecule in protein synthesis. Enzymes
called aminoacyl-tRNA synthetases link amino acids to their appropriate
tRNAs. Each tRNA contains a sequence of three nucleotides,
the anticodon, which matches a codon in mRNA by complementary
base-pairing between codon and anticodon.
\item Protein synthesis begins when a ribosome assembles at an initiation
codon (AUG) in mRNA, a process that is regulated by proteins
called translation initiation factors. The completed protein chain is
released from the ribosome when a stop codon (UAA, UAG, or UGA)
is reached.
\item The stepwise linking of amino acids into a polypeptide chain is catalyzed
by an rRNA molecule in the large ribosomal subunit. Thus, the
ribosome is an example of a ribozyme, an RNA molecule that can
catalyze a chemical reaction.
\item The degradation of proteins in the cell is carefully controlled. Some
proteins are degraded in the cytosol by large protein complexes
called proteasomes.
\item From our knowledge of present-day organisms and the molecules
they contain, it seems likely that living systems began with the evolution
of RNA molecules that could catalyze their own replication.
\item It has been proposed that, as cells evolved, the DNA double helix
replaced RNA as a more stable molecule for storing genetic information,
and proteins replaced RNAs as major catalytic and structural
components. However, important reactions such as peptide bond
formation are still catalyzed by RNA; these are thought to provide a
glimpse into an ancient, RNA-based world.
\end{itemize}

\chapter{Control of Gene Expression}
