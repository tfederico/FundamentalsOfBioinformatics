\chapter{How cells obtain energy from food}

In this chapter we trace the major steps in the breakdown - or \textit{catabolism} -
of sugars and show how this oxidation produces ATP, NADH, and
other activated carrier molecules in cells. We concentrate on the break-
down of glucose because these reactions dominate energy production in
most animal cells. A very similar pathway operates in plants, fungi, and
many bacteria. Other molecules, such as fatty acids and proteins, can
also serve as energy sources if they are funneled through appropriate
enzymatic pathways. We also see how many of the molecules generated
from the breakdown of sugars and fats can be used to build the macro-
molecules in cells.

\section{The breakdown and utilization of sugars and fats}

Living cells use enzymes to carry out the oxidation of sugars in
a tightly controlled series of reactions. A
glucose molecule is degraded step by step, paying out energy in small
packets to activated carrier molecules by means of coupled reactions. In
this way, much of the energy released by oxidizing glucose is saved in
the high-energy bonds of ATP and other activated carrier molecules and
made available to do useful work for the cell.


Animal cells make ATP in two ways. First, certain steps in a series of
enzyme-catalyzed reactions are directly coupled to the energetically
unfavorable reaction $ADP + P_{i} \rightarrow ATP$. The oxidation of food molecules
provides the energy that allows this unfavorable reaction to proceed. Most
ATP synthesis, however, takes place in mitochondria and uses the energy
from activated carrier molecules to drive ATP production; this process
involves the mitochondrial membrane.

\subsection{Food Molecules are Broken down in Three Stages}

The proteins, lipids, and polysaccharides that make up most of the food
we eat must be broken down into smaller molecules before our cells can
use them - either as a source of energy or as building blocks for other
molecules. This breakdown process - which uses enzymes to degrade
complex molecules into simpler ones - is dubbed \textbf{catabolism}. Catabolism
must act on food taken in from outside, but not on the macromolecules
inside our own cells. Therefore stage 1 in the enzymatic breakdown of
food molecules - \textit{digestion} - occurs either outside cells (in our intestine) or
in a specialized organelle within cells called the lysosome. A membrane
that surrounds the lysosome keeps its digestive enzymes separated from
the cytosol.

Digestive enzymes reduce the large polymeric molecules in food into
their monomeric subunits. After digestion, the
small organic molecules derived from food enter the cytosol of a cell,
where their gradual oxidation begins. This
oxidation occurs in two further stages: stage 2 starts in the cytosol and
ends in mitochondria, while stage 3 is confined to the mitochondria.

In stage 2 of cellular catabolism, a chain of reactions called \textit{glycolysis}
converts each molecule of glucose into two smaller molecules of \textit{pyruvate}.
During the formation of pyruvate, two types of activated carrier molecules are produced:
ATP and NADH. The pyruvate is then transported from the cytosol into
the mitochondrion’s large, internal compartment or \textit{matrix}. There a giant
enzyme complex converts each pyruvate molecule into $CO_{2}$ plus \textbf{acetyl
CoA}, another of the activated carrier molecule. Large amounts of acetyl CoA are also produced by the
stepwise breakdown and oxidation of fatty acids derived from fats.

Stage 3 of the oxidative breakdown of food molecules takes place entirely
in mitochondria. The acetyl group in acetyl CoA is transferred to a molecule
called oxaloacetate to form citrate, which enters a series of reactions
called the \textit{citric acid cycle}. The transferred acetyl group is oxidized to $CO_2$
in these reactions, and large amounts of the high-energy electron carrier
NADH are generated. Finally, the high-energy electrons from NADH are
passed along a series of enzymes within the mitochondrial inner membrane
called an \textit{electron-transport chain}, where the energy released by
their transfer is used to drive a process that produces ATP and consumes
molecular oxygen ($O_2$ gas). It is in these final steps that most of the energy
released by oxidation is harnessed to produce most of the cell’s ATP.

Through the production of ATP, the energy derived from the breakdown
of sugars and fats is redistributed as packets of chemical energy in a
form convenient for use in the cell.

In total, nearly half of the energy that could, in theory, be derived from
the oxidation of glucose or fatty acids to $H_{2}O$ and $CO_2$ is captured and
used to drive the energetically unfavorable reaction $P_{i} + ADP \rightarrow ATP$.
The remaining energy is released as heat, which in animals
helps to make our bodies warm.

\subsection{Glycolysis is a Central ATP-producing Pathway}

The central process in stage 2 of the breakdown of food molecules is
the degradation of \textit{glucose} in the sequence of reactions known as \textbf{glycolysis}.
Glycolysis produces ATP without the involvement of $O_2$. It occurs in the cytosol of
most cells, including many anaerobic microorganisms.

During glycolysis, a glucose molecule, which has six carbon atoms, is
cleaved into two molecules of \textbf{pyruvate}, each of which contains three
carbon atoms. For each molecule of glucose, two molecules of ATP are
consumed to provide energy to drive the early steps, but four molecules
of ATP are produced in the later steps. Thus, at the end of glycolysis, there
is a net gain of two molecules of ATP for each glucose molecule broken
down.

Glycolysis involves a sequence of 10 separate reactions, each producing
a different sugar intermediate and each catalyzed by a different enzyme.
Like most enzymes, the enzymes that catalyze glycolysis all have names ending in \textit{-ase} - like
isomerase and dehydrogenase - which specify the type of reaction they
catalyze.

Although no molecular oxygen is involved in glycolysis, oxidation occurs:
electrons are removed from some of the carbons derived from glucose
by $NAD^{+}$, producing $NADH$. The stepwise nature of the process allows
the energy of oxidation to be released in small packets, so that much of
rather it can be stored in carrier molecules than all of it being released
as heat. Some of the energy released by this oxidation
drives the synthesis of ATP molecules from ADP and $P_{i}$. The synthesis
of ATP in glycolysis is known as \textit{substrate-level phosphorylation} because
it occurs by the transfer of a phosphate group directly from a substrate
molecule - a sugar intermediate - to ADP. The remainder of the energy
harnessed during glycolysis is stored in the electrons in NADH.

Two molecules of NADH are formed per molecule of glucose in the
course of glycolysis. In aerobic organisms these NADH molecules donate
their electrons to the electron-transport chain.
The electrons are passed along this chain to $O_2$ , forming
water, and the $NAD^{+}$ formed from the NADH is used again for glycolysis.

\subsection{Fermentations allow ATP to Be Produced in the absence of oxygen}

For many anaerobic microorganisms,
which do not use $O_2$ and can grow and divide in its absence, glycolysis
is the principal source of ATP. In these anaerobic conditions, the pyruvate and the NADH stay in the
cytosol. The pyruvate is converted into products that are excreted from
the cell: for example, lactate in muscle or ethanol and $CO_2$ in the yeasts
used in brewing and breadmaking. In the process, the NADH gives up its
electrons and is converted back into $NAD^+$. This regeneration of $NAD^+$ is
required to maintain the reactions of glycolysis. Anaerobic
energy-yielding pathways like these are called \textbf{fermentations}.

\subsection{Glycolysis illustrates how enzymes Couple oxidation to energy Storage}

These reactions - steps 6 and 7 - convert the three-carbon sugar
intermediate glyceraldehyde 3-phosphate (an aldehyde) into
3-phosphoglycerate (a carboxylic acid). This conversion entails the oxidation
of an aldehyde group to a carboxylic acid group, which occurs in
two steps. The overall reaction releases enough free energy to convert a
molecule of ADP to ATP and to transfer two electrons from the aldehyde
to $NAD^+$ to form NADH, while still releasing enough heat to the environment
to make the overall reaction energetically favorable.

The chemical reactions are precisely guided by two enzymes to which the
sugar intermediates are tightly bound. In fact,
the first enzyme (glyceraldehyde 3-phosphate dehydrogenase) forms a
short-lived covalent bond to the aldehyde through a reactive -SH group
on the enzyme, and catalyzes its oxidation in this attached state. The
reactive enzyme - substrate bond is then displaced by an inorganic phosphate
ion to produce a high-energy phosphate intermediate, which is
released from the enzyme. The intermediate, 1,3-bisphosphoglycerate,
binds to the second enzyme (phosphoglycerate kinase). This enzyme
then catalyzes the energetically favorable transfer of the intermediate’s
high-energy phosphate to ADP, forming ATP and completing the process
of oxidizing an aldehyde to a carboxylic acid.

These reactions (steps 6 and 7) are the only ones in glycolysis that create a high-energy phosphate
linkage directly from inorganic phosphate. As such, they account for the
net yield of two ATP molecules and two NADH molecules per molecule
of glucose. As we have mentioned, this NADH must be reoxidized to the
$NAD^+$ required for these coupled reactions. If $NAD^+$ is not available, gly-
colysis will stop

\subsection{Sugars and fats are Both degraded to acetyl Coa in Mitochondria}

In aerobic metabolism in eucaryotic cells, the pyruvate produced by
glycolysis is actively pumped into the mitochondrial matrix, the major
internal compartment of this organelle. There it is
rapidly decarboxylated by a giant complex of three enzymes, called the
\textit{pyruvate dehydrogenase complex}. The products of pyruvate decarboxylation
are a molecule of $CO_2$ (a waste product), a molecule of NADH, and a
molecule of acetyl CoA.

Fatty acids, derived from \textit{fat}, are an alternative fuel to sugars for energy
generation. Like the pyruvate derived from glycolysis, fatty acids are converted
into acetyl CoA in mitochondria. Each long molecule of fatty acid
(in the form of the activated molecule, fatty acyl CoA) is broken down
completely by a cycle of reactions that trims two carbons at a time from
its carboxyl end, generating one molecule of acetyl CoA for each turn of
the cycle. A molecule of NADH and a molecule of another electron carrier
$FADH_2$, are also produced in this process.

Sugars and fats provide the major energy sources for most nonphotosynthetic
organisms, including humans. In the course of their processing to
acetyl CoA, only a small part of the useful energy stored in these foodstuffs
is extracted and converted into ATP or NADH. Most of the energy is
still locked up in acetyl CoA. The next stage in respiration, in which the
acetyl group in acetyl CoA is oxidized to $CO_2$ and $H_{2}O$ in the citric acid
cycle, is therefore central to the energy metabolism of aerobic organisms.
In eucaryotes the citric acid cycle takes place in mitochondria, the
organelles to which pyruvate and fatty acids are directed for acetyl CoA
production.

In addition to pyruvate and fatty acids, some amino acids are transported
from the cytosol into mitochondria, where they are also converted into
acetyl CoA or one of the other intermediates of the citric acid cycle.
In aerobic bacteria, which have no mitochondria, all of these reactions
- glycolysis, acetyl CoA production, and the citric acid cycle - take
place in the single compartment of the cytosol.

\subsection{The Citric acid Cycle generates nadh by oxidizing acetyl groups to $CO_2$}

The third and final stage in the oxidative breakdown of food molecules
to generate energy requires abundant $O_2$ .

In the nineteenth century, biologists noticed that in the absence of air
(anaerobic conditions) cells produce lactic acid (for example, in muscle)
or ethanol (for example, in yeast), while in the presence of air (aerobic
conditions) these cells consume $O_2$ and produce $CO_2$ and $H_{2}O$. Intensive
efforts to define the pathways of aerobic metabolism eventually focused
on the oxidation of pyruvate and led in 1937 to the discovery of the \textbf{citric
acid cycle}, also known as the \textit{tricarboxylic acid cycle} or the
\textbf{Krebs cycle}.

The citric acid cycle accounts for about
two-thirds of the total oxidation of carbon compounds in most cells, and
its major end products are $CO_2$ and high-energy electrons in the form of
NADH. The $CO_2$ is released as a waste product, while the high-energy
electrons from NADH are passed to a series of membrane-bound enzymes
known collectively as the \textit{electron-transport chain}. At the end of the chain,
these electrons combine with $O_2$ to produce $H_{2}O$. The citric acid cycle
itself does not use $O_2$ . However, it requires $O_2$ to proceed because the
electron-transport chain allows NADH to get rid of its electrons and thus
regenerate the $NAD^+$ that is needed to keep the cycle going.

The citric acid cycle, which takes place in the mitochondrial matrix, catalyzes
the complete oxidation of the carbon atoms of the acetyl groups in
acetyl CoA, converting them into $CO_2$ . The acetyl group is not oxidized
directly, however. Instead, it is transferred from acetyl CoA to a larger
four-carbon molecule, \textit{oxaloacetate}, to form the six-carbon tricarboxylic
acid \textit{citric acid}, for which the subsequent cycle of reactions is named.
The citric acid molecule is then gradually oxidized, and the energy of this
oxidation is harnessed to produce energy-rich carrier molecules, in much
the same manner as we described for glycolysis. The chain of eight reactions
forms a cycle, because the oxaloacetate that began the process is
regenerated at the end.

We have so far discussed only one of the three types of activated carrier
molecules that are produced by the citric acid cycle - NADH. In addition
to three molecules of NADH, each turn of the cycle also produces one
molecule of $FADH_2$ (reduced flavin adenine dinucleotide) from FAD and
one molecule of the ribonucleotide GTP (guanosine triphosphate) from
GDP. GTP is a close relative of ATP, and
the transfer of its terminal phosphate group to ADP produces one ATP
molecule in each cycle. Like NADH, $FADH_2$ is a carrier of high-energy
electrons and hydrogen. As we discuss shortly, the energy that is stored
in the readily transferred high-energy electrons of NADH and $FADH_2$ are
subsequently used to produce ATP through the process of \textit{oxidative phosphorylation},
which occurs in the mitochondrial membrane. Oxidative
phosphorylation is the only step in the oxidative catabolism of foodstuffs
that directly requires $O_2$ from the atmosphere.

\subsection{Many Biosynthetic Pathways Begin with glycolysis or the Citric acid Cycle}

So far we have emphasized
energy production rather than the provision of starting materials for biosynthesis.
But many of the intermediates formed in glycolysis and the
citric acid cycle are siphoned off by biosynthetic, or \textit{anabolic}, pathways,
where they are converted by series of enzyme-catalyzed reactions into
amino acids, nucleotides, lipids, and other small organic molecules that
the cell needs.

\subsection{Electron Transport drives the Synthesis of the Majority of the ATP in Most Cells}

We now return to the last stage in the oxidation of a food molecule -
the stage in which most of its chemical energy is released. In this final
process, the electron carriers NADH and $FADH_2$ transfer the electrons
they have gained by oxidizing other molecules to the \textbf{electron-transport chain}.
This specialized chain of electron carriers is embedded in
the inner membrane of the mitochondrion in eucaryotic cells (in the
plasma membrane of bacteria). As the electrons pass through the series
of electron acceptor and donor molecules that form the chain, they fall to
successively lower energy states. The energy released is used to drive $H^+$
ions (protons) across the membrane, from the inner mitochondrial compartment to the outside.
This generates a transmembrane gradient of $H^+$
ions that serves as a source of energy (like a battery) that can be tapped
to drive a variety of energy-requiring reactions. In mitochondria,
the most prominent of these reactions is the phosphorylation
of ADP to generate ATP.

At the end of the transport chain, the electrons are added to molecules
of $O_2$ that have diffused into the mitochondrion; the resulting reduced
$O_2$ molecules simultaneously combine with protons ($H^+$ ) from the surrounding
solution to produce water. The electrons have now reached
their lowest energy level, and all the available energy has been extracted
from the food molecule being oxidized. The oxygen-requiring generation
of ATP is termed \textbf{oxidative phosphorylation}.

In total, the complete oxidation of a molecule of glucose to $H_{2}O$ and $CO_2$
produces about 30 molecules of ATP. In contrast, only two molecules
of ATP are produced per molecule of glucose by glycolysis alone.

\section{Regulation of Metabolism}

Many sets of reactions need to be carefully controlled. For example, to
maintain order within their cells, all organisms need to constantly replenish
their ATP pools through sugar or fat oxidation.
Yet animals have only periodic access to food, and plants need to survive
overnight without sunlight, when they are unable to produce sugar
through photosynthesis. Plants and animals have evolved several ways
to get around this problem. One is to synthesize food reserves in times of
plenty that can be later consumed when other energy sources are scarce.
Thus, a cell must control whether key metabolites will be routed into
anabolic or catabolic pathways - in other words, whether they will be
commissioned to build other molecules or burned to provide immediate
energy.

\subsection{Catabolic and anabolic reactions are organized and regulated}

the metabolic balance of a cell is amazingly stable. Whenever the balance is perturbed,
the cell reacts so as to restore the initial state: cells
can adapt and continue to function during starvation or disease. This
resilience is made possible by an elaborate network of \textit{control mechanisms}
that act on enzymes to regulate and coordinate the rates of the
many metabolic reactions in a cell.

\subsection{Feedback regulation allows Cells to Switch from glucose degradation to glucose Biosynthesis}

The body needs a continuous supply of glucose to meet its metabolic
needs. One way to replenish blood glucose is to synthesize it from
small non-carbohydrate organic molecules such as lactate, pyruvate, or
amino acids in a process called \textbf{gluconeogenesis}. An intricate pattern of
feedback regulation enables cells to switch from breaking down glucose
through glycolysis to synthesizing it through gluconeogenesis.

Most of the reactions involved in the breakdown of glucose to pyruvate
are readily reversible. However, three of the reactions - steps 1, 3, and 10
are effectively irreversible. In fact, it is the large negative
free-energy change that occurs in these reactions that normally drives the
breakdown of glucose. For the pathway to go in the opposite direction - to
make glucose from pyruvate - these three reactions must be bypassed.
This detour is achieved by substituting a set of alternative, enzyme-catalyzed
“bypass reactions” that require an input of chemical energy.
The reactions that synthesize 	a molecule of glucose in gluconeogenesis thus require the hydrolysis of
four ATP and two GTP molecules, compared with the overall generation
of two molecules of ATP for each molecule of glucose consumed during
glycolysis.

One of the key control points in the breakdown of glucose lies in step 3
of glycolysis, the production of fructose 1,6-bisphosphate by the enzyme
\textit{phosphofructokinase}. This is one of the reactions that must be bypassed
in gluconeogenesis. Phosphofructokinase is allosterically activated by AMP, ADP, and
inorganic phosphate - the byproducts of ATP hydrolysis; it is allosterically
inhibited by ATP, citrate, and alternative fuels for respiration, such as
fatty acids, which can be liberated from stored fat when glucose is not
available. Thus, when energy reserves are low and the products of ATP
hydrolysis accumulate, phosphofructokinase is activated and glycolysis
proceeds. On the other hand, when ATP or fuel sources - represented
by citrate and fatty acids - are abundant, phosphofructokinase is turned
off, favoring gluconeogenesis and, ultimately, the storage of food molecules.
To add an additional level of control, the enzyme that catalyzes
the reverse reaction (fructose 1, 6-bisphosphatase)
is regulated by the same molecules - but in the opposite
direction. Thus, this enzyme is activated when phosphofructokinase
is turned off.

\subsection{Cells Store food Molecules in Special reservoirs to Prepare for Periods of need}

To compensate for long periods when food
is unavailable, animals store food reserves within their cells. Glucose is
stored as the subunits of the large, branched polysaccharide \textbf{glycogen},
which is present as small granules in the cytoplasm of many cells, mainly
liver and muscle. The syn thesis and degradation of glycogen occur by quite separate metabolic
pathways, which can be rapidly and coordinately regulated according
to need. When more ATP is needed than can be generated from food
molecules taken in from the bloodstream, cells break down glycogen in
a reaction that produces \textit{glucose 1-phosphate}, which is then converted to
the glucose 6-phosphate that feeds into the glycolytic pathway.

The glycogen synthetic and degradative pathways are coordinated by
enzymes in each pathway that are allosterically regulated by glucose
6-phosphate, but in opposite directions: \textit{glycogen synthase} in the synthetic
pathway is activated by glucose 6-phosphate, whereas the \textit{glycogen
phosphorylase} that catalyzes the breakdown of glycogen is inhibited by
both glucose 6-phosphate and ATP. This regulation helps to prevent the
breakdown of glycogen when ATP is plentiful and favors its synthesis
when glucose 6-phosphate concentration is high. The balance between
glycogen synthesis and breakdown is also regulated by intracellular signaling
pathways that are controlled by the hormones insulin, adrenaline,
and glucagon.

Quantitatively, fat is a far more important storage material than glycogen,
in part because the oxidation of a gram of fat releases about twice as
much energy as the oxidation of a gram of glycogen. Moreover, glycogen
binds a great deal of water, producing a sixfold difference in the actual
mass of glycogen required to store the same amount of energy as fat.
An average adult human stores enough glycogen for only about a day
of normal activity, but enough fat to last nearly a month.

Most of our fat is stored as droplets of water-insoluble triacylglycerols in
specialized adipose tissue.
In response to hormonal signals, fatty acids can be released from these
depots into the bloodstream for other cells to use as required. Such a
need arises after a period of not eating; even a normal overnight fast
results in the mobilization of fat. In the morning, most of the acetyl CoA
that enters the citric acid cycle is derived from fatty acids rather than from
glucose. After a meal, however, most of the acetyl CoA entering the citric
acid cycle comes from glucose derived from food, and any excess glucose
is used to replenish depleted glycogen stores or to synthesize fats.

The food reserves in both animals and plants form a vital part of the
human diet. Plants convert some of the sugars that they make through
photosynthesis during daylight into fats and into \textit{starch}, a branched polymer
of glucose very similar to the glycogen of animals. The fats in plants
are triacylglycerols, just like the fats in animals, and they differ only in the
types of fatty acids that predominate.

\section{Essential concepts}

\begin{itemize}
\item Glucose and other food molecules are broken down by controlled
stepwise oxidation to provide useful chemical energy in the form of
the activated carriers ATP and NADH.
\item Sugars derived from food are broken down by distinct sets of reactions:
glycolysis (which occurs in the cytosol), the citric acid cycle
(in the mitochondrial matrix), and oxidative phosphorylation (in the
inner mitochondrial membrane).
\item The reactions of glycolysis degrade the six-carbon sugar glucose to
two molecules of the three-carbon sugar pyruvate, producing a relatively
small amount of ATP and NADH.
\item In the presence of oxygen, pyruvate is converted to acetyl CoA plus
$CO_2$ . The citric acid cycle then converts the acetyl group in acetyl
CoA to $CO_2$ and $H_{2}O$. Much of the energy released in these oxidation
reactions is stored as high-energy electrons in the activated carriers
NADH and $FADH_2$ . In eucaryotic cells, all these reactions occur in
mitochondria.
\item The other major energy source in foods is fat. The fatty acids produced
from the digestion of fats are imported into mitochondria and
converted to acetyl CoA molecules. These acetyl CoA molecules are
then further oxidized through the citric acid cycle, producing NADH
and $FADH_2$ , just like the acetyl CoA derived from pyruvate.
\item NADH and $FADH_2$ pass their high-energy electrons to an electron-transport
chain in the inner mitochondrial membrane, where a series
of electron transfers is used to drive the formation of ATP. Most of
the energy captured during the breakdown of food molecules is harvested
during this process of oxidative phosphorylation (described in
detail in Chapter 14).
\item The food we eat is not only a source of metabolic energy but also of
raw materials for biosynthesis. Many intermediates of glycolysis and
the citric acid cycle are starting points for pathways that lead to the
synthesis of proteins, nucleic acids, and the many other specialized
molecules of the cell.
\item The thousands of different reactions carried out simultaneously by a
cell are closely coordinated, enabling the cell to adapt and continue
to function under a wide range of external conditions.
\item During periods when food is scarce, regulation of the activities of a
few key enzymes allows the cell to switch from glucose breakdown
to glucose biosynthesis (gluconeogenesis).
\item Cells store food molecules in special reserves. Glucose subunits are
stored as glycogen in animals and as starch in plants; both animals
and plants store fatty acids as fats. The food reserves stored by plants
are major sources of food for animals, including humans.
\end{itemize}

\chapter{Energy generation in Mithocondria and Chloroplast}
