\chapter{Cell Communication}

In this chapter, we examine some of the most important methods by which
cells communicate, and we discuss how cells send signals and interpret
the signals they receive. Although we concentrate on the mechanisms of
signal reception and interpretation in animal cells, we also present a brief
review of what is known about cell-to-cell signaling in plants. We begin
our discussion with an overview of the general principles of cell signaling
and then consider two of the main systems animal cells use to receive
and interpret signals.

\section{General principles of cell signaling}

The signals that pass between living cells are simpler than the sorts of
messages that humans ordinarily exchange. In a typical communication
between cells, the \textit{signaling cell} produces a particular type of
signal molecule that is detected by the \textit{target cell}.
Target cells possess receptor proteins
that recognize and respond specifically to the signal molecule. Signal
transduction begins when the receptor protein on a target cell receives
an incoming extracellular signal and converts it to the intracellular sig-
nals that alter cell behavior. Most of this chapter is concerned with signal
reception and transduction - the events that cell biologists have in mind
when they refer to \textbf{cell signaling}.

\subsection{Signals can act over a long or short range}

In multicellular organisms, the most `public' style of communication
involves broadcasting the signal throughout the whole body by secreting
it into the bloodstream (in an animal) or the sap (in a plant). Signal molecules
used in this way are called \emph{hormones}, and, in animals, the cells
that produce hormones are called endocrine cells.

Somewhat less public is the process known as \textit{paracrine signaling}. In this
case, rather than entering the bloodstream, the signal molecules diffuse
locally through the extracellular fluid, remaining in the neighborhood of
the cell that secretes them. Thus, they act as \textbf{local mediators} on nearby
cells. In some cases, cells can respond to
the local mediators that they themselves produce, a form of paracrine
communication called \textit{autocrine signaling}.

\textit{Neuronal signaling} is a third form of cell communication. Like endocrine
cells, nerve cells (neurons) can deliver messages over long distances.
In the case of neuronal signaling, however, a message is not broadcast
widely but is instead delivered quickly and specifically to individual target
cells through private lines. The axon of a neuron terminates at specialized
junctions (synapses) on target cells that can lie far from the neuronal cell body.
When activated by signals from the environment or from other
nerve cells, a neuron sends electrical impulses racing along its axon at
speeds of up to 100 m/sec. On reaching the axon terminal, these electrical
signals are converted into a chemical form: each electrical impulse
stimulates the nerve terminal to release a pulse of an extracellular signal
molecule called a neurotransmitter. The neurotransmitter then diffuses
across the narrow (< 100 nm) gap between the axon-terminal membrane
and the membrane of the target cell, reaching the target cell receptors in
less than 1 msec.

A fourth style of signal-mediated cell–cell communication - the most intimate
and short-range of all - does not require the release of a secreted
molecule. Instead, the cells make direct physical contact through signal
molecules lodged in the plasma membrane of the signaling cell and
receptor proteins embedded in the plasma membrane of the target cell.

\subsection{Each cell responds to a limited set of signals, depending on its History and its current state}

Whether a cell responds to a signal molecule depends first of all on
whether it possesses a \textbf{receptor protein}, or receptor, for that signal.
Each receptor is usually activated by only one type of signal. Without the
appropriate receptor, a cell will be deaf to the signal and will not respond
to it. By producing only a limited set of receptors out of the thousands
that are possible, a cell restricts the types of signals that can affect it.

The signal from a cell-surface
receptor is generally conveyed into the target cell interior via a set of
\textit{intracellular signaling molecules}, which act in sequence and ultimately
alter the activity of effector proteins, which then affect the behavior of the
cell. This intracellular relay system and the intracellular \textit{effector proteins}
on which it acts vary from one type of specialized cell to another, so that
different types of cells respond to the same signal in different ways.

\subsection{A cell’s response to a signal can Be fast or slow}

The length of time a cell takes to respond to an extracellular signal can
vary greatly, depending on what needs to happen once the message has
been received.

\subsection{Some Hormones cross the Plasma membrane and Bind to intracellular receptors}

\textbf{Extracellular signal molecules} generally fall into two classes. The first
and largest class consists of molecules that are too large or too hydrophilic
to cross the plasma membrane of the target cell. They rely on receptors
on the surface of the target cell to relay their message across the membrane.
The second, and smaller, class of signals consists
of molecules that are small enough or hydrophobic enough to slip easily
across the plasma membrane. Once inside, these signal molecules
usually activate intracellular enzymes or bind to intracellular receptor
proteins that regulate gene expression.

One important class of signal molecules that rely on intracellular recep-
tor proteins is the \textbf{steroid hormones} - including cortisol, estradiol, and
testosterone - and the thyroid hormones such as thyroxine.
All of these hydrophobic molecules pass through the plasma membrane
of the target cell and bind to receptor proteins located in either the cytosol
or the nucleus. Both the cytosolic and nuclear receptors are referred to
as \textit{nuclear receptors}, because, when activated by hormone binding, they
act as transcription regulators in the nucleu. In
unstimulated cells, nuclear receptors are typically present in an inactive
form. When a hormone binds, the receptor undergoes a large conformational
change that activates the protein, allowing it to promote or inhibit
the transcription of specific target genes.
Each hormone binds to a different receptor protein, and each receptor acts
at a different set of regulatory sites in DNA.

Nuclear receptors and the hormones that activate them play an essential
role in human physiology. Loss of these signaling
systems can have dramatic consequences.

\subsection{Some dissolved Gases cross the Plasma membrane and activate intracellular enzymes directly}

Steroid hormones and thyroid hormones are not the only extracellular
signal molecules that can pass through the plasma membrane. Some
dissolved gases can slip across the membrane to the cell interior and
directly regulate the activity of specific intracellular proteins.
\textbf{Nitric oxide} (NO) acts in this way. This gas diffuses readily out of
the cell that generates it and enters neighboring cells. NO is synthesized
from the amino acid arginine and operates as a local mediator in many
tissues. The gas acts only locally because it is quickly converted to
nitrates and nitrites by reaction with oxygen and water outside cells.

Endothelial cells - the flattened cells that line every blood vessel - release
NO in response to stimulation by nerve endings. This NO signal causes
smooth muscle cells in the vessel wall to relax, allowing the vessel to
dilate, so that blood flows through it more freely.

Inside many target cells, NO binds to and activates the enzyme \textit{guanylyl
cyclase}, stimulating the formation of \textit{cyclic GMP} from the nucleotide GTP.
Cyclic GMP is itself a small intracellular signaling
molecule that forms the next link in the NO signaling chain that leads to
the cell’s ultimate response. Cyclic GMP is very similar in its structure and mechanism
of action to \textit{cyclic AMP}, a much more commonly used intracellular
messenger molecule.

\subsection{Cell-surface receptors relay extracellular signals via intracellular signaling Pathways}

In contrast to NO and the steroid and thyroid hormones, the vast majority
of signal molecules are too large or hydrophilic to cross the plasma
membrane of the target cell. These proteins, peptides, and small, highly
water-soluble molecules bind to cell-surface receptor proteins that span
the plasma membrane. These transmembrane receptors
detect a signal on the outside and relay the message, in a new form,
across the membrane into the interior of the cell.

The receptor protein performs the primary signal transduction step: it
binds to the extracellular signal and generates new intracellular signals
in response. The resulting intracellular signaling process
usually works like a molecular relay race in which the message
is passed ‘downstream’ from one \textbf{intracellular signaling molecule} to
another, each activating or generating the next signaling molecule in the
pathway, until a metabolic enzyme is kicked into action, the cytoskeleton
is tweaked into a new configuration, or a gene is switched on or off. This
final outcome is called the \textit{response} of the cell.

The components of these \textbf{intracellular signaling pathways} perform one
or more crucial functions:

\begin{enumerate}
\item They can simply \textit{relay} the signal onward and thereby help spread it
through the cell.
\item They can \textit{amplify} the signal received, making it stronger, so that
a few extracellular signal molecules are enough to evoke a large
intracellular response.
\item They can receive signals from more than one intracellular signaling
pathway and \textit{integrate} them before relaying a signal onward.
\item They can \textit{distribute} the signal to more than one signaling pathway or
effector protein, creating branches in the information flow diagram
and evoking a complex response.
\end{enumerate}

As part of the integration function, many steps in a signaling pathway
are open to \textit{modulation} by other factors, including both intracellular and
extracellular factors, so that the effects of each signal are tailored to the
conditions prevailing inside and outside the cell.

\subsection{Some intracellular signaling Proteins act as molecular switches}

Many of the key intracellular signaling proteins behave as \textbf{molecular
switches}: receipt of a signal causes them to toggle from an inactive to
an active state. Once activated, these proteins can turn on other proteins
in the signaling pathway. They then persist in an active state until some
other process switches them off again. The importance of the switching-
off process is often underappreciated. The two are equally important for the
signaling process.

Proteins that act as molecular switches fall mostly into one of two classes.
The first and by far the largest class consists of proteins that are activated
or inactivated by phosphorylation. For these molecules, the switch is thrown in
one direction by a \textbf{protein kinase}, which tacks a phosphate group onto
the switch protein, and in the other direction by a \textbf{protein phosphatase},
which plucks the phosphate off again. The activity of any protein that is regulated
by phosphorylation depends - moment by moment - on the balance between the activities
of the kinases that phosphorylate it and the phosphatases that dephosphorylate it.

Many of the switch proteins controlled by phosphorylation are themselves
protein kinases, and these are often organized into phosphorylation cascades:
one protein kinase, activated by phosphorylation, phosphorylates
the next protein kinase in the sequence, and so on, transmitting the signal
onward and, in the process, amplifying, distributing, and modulating
it. Two main types of protein kinases operate in intracellular signaling
pathways: the most common are \textbf{serine/threonine kinases}, which - as
the name implies - phosphorylate proteins on serines or threonines; oth-
ers are \textbf{tyrosine kinases}, which phosphorylate proteins on tyrosines.

The other main class of switch proteins involved in intracellular signaling
pathways is the \textbf{GTP-binding proteins}. These switch between an
active and an inactive state depending on whether they have GTP or
GDP bound to them, respectively. Once activated by GTP
binding, these proteins have intrinsic GTP-hydrolyzing (\textit{GTPase}) activity,
and they shut themselves off by hydrolyzing their bound GTP to GDP.
One class of GTP-activated switch proteins contains the large, trimeric
GTP-binding proteins (also called \textit{G proteins}) that relay messages from
\textit{G-protein-coupled receptors}.

\subsection{Cell-surface receptors fall into three main classes}

All cell-surface receptor proteins bind to an extracellular signal mole-
cule and transduce its message into one or more intracellular signaling
molecules that alter the cell’s behavior. These receptors, however, are
divided into three large families that differ in the transduction mecha-
nism they use.

\begin{enumerate}
\item \textit{Ion-channel-coupled} receptors allow a flow of ions
across the plasma membrane, which changes the membrane potential
and produces an electrical current;
\item \textit{G-protein–coupled} receptors activate membrane-bound, trimeric GTP-binding proteins (G
proteins), which then activate either an enzyme or an ion channel in the
plasma membrane, initiating a cascade of other effects;
\item \textit{Enzyme-coupled receptors} either act as enzymes or associate with
enzymes inside the cell; when stimulated, the enzymes
activate a variety of intracellular signaling pathways.
\end{enumerate}

\subsection{Ion-channel-coupled receptors convert chemical signals into electrical ones}

Of all the types of cell-surface receptors, \textbf{ion-channel-coupled receptors}
(also known as transmitter-gated ion channels) function in the
simplest and most direct way. These receptors are responsible for the
rapid transmission of signals across synapses in the nervous system.
They transduce a chemical signal, in the form of a pulse of neurotransmitter
delivered to the outside of the target cell, directly into an electrical
signal, in the form of a change in voltage across the target cell’s plasma
membrane. When the neurotransmitter binds, this type
of receptor alters its conformation so as to open or close an ion channel
in the plasma membrane, allowing the flow of specific types of ions.
Driven by their electrochemical gradients, the ions rush into or out of the cell,
creating a change in the membrane potential within a millisecond or so.
This change in potential may trigger a nerve impulse or make it easier
(or harder) for other neurotransmitters to do so.
Whereas ion-channel–coupled receptors are a specialty of the nervous
system and of other electrically excitable cells such as muscle cells,
G-protein-coupled receptors and enzyme-coupled receptors are used
by practically every cell type in the body.

\section{G-Protein-coupled receptors}

\textbf{G-protein-coupled receptors} (GPCRs) form the largest family of cell-surface
receptors. These receptors mediate responses to an enormous diversity of extracellular signal
molecules, including hormones, local mediators, and neurotransmitters.
The signal molecules are as varied in structure as they are in function:
they can be proteins, small peptides, or derivatives of amino acids or
fatty acids, and for each one of them there is a different receptor or set of
receptors. Because GPCRs are involved in such a large variety of cellular
processes, they are an attractive target for the development of drugs to
treat a variety of disorders. About half of all known drugs work through
GPCRs.

Despite the diversity of the signal molecules that bind to them, all GPCRs
that have been analyzed have a similar structure: each is made of a single
polypeptide chain that threads back and forth across the lipid bilayer
seven times. This superfamily of seven-pass transmembrane
receptor proteins includes rhodopsin (the light-activated photoreceptor
protein in the vertebrate eye), the olfactory (smell) receptors in the vertebrate
nose, and the receptors that participate in the mating rituals of
single-celled yeasts.

\subsection{Stimulation of GPCRs activates G-Protein subunits}

When an extracellular signal molecule binds to a GPCR, the receptor
protein undergoes a conformational change that enables it to activate a
G protein located on the underside of the plasma membrane. To explain
how this activation leads to the transmission of a signal, we must first
consider how G proteins are constructed and how they function.

There are several varieties of G proteins. Each is specific for a particular
set of receptors and a particular set of target enzymes or ion channels
in the plasma membrane. All of these G proteins, however, have a similar
general structure and operate in a similar way. They are composed
of three protein subunits - $\alpha$, $\beta$, and $\gamma$ - two of which are tethered to the
plasma membrane by short lipid tails. In the unstimulated state, the $\alpha$
subunit has GDP bound to it, and the G protein is idle.
When an extracellular ligand binds to its receptor, the altered receptor
activates a G protein by causing the $\alpha$ subunit to decrease its affinity for
GDP, which is then exchanged for a molecule of GTP. In some cases, this
activation is thought to break up the G-protein subunits, so that the activated
$\alpha$ subunit, clutching its GTP, detaches from the $\beta\gamma$ complex, which is
also activated. Regardless of whether they dissociate, the
two activated parts of a G protein - the $\alpha$ subunit and the $\beta\gamma$ complex -
can both interact directly with target proteins in the plasma membrane,
which in turn may relay the signal to yet other destinations in the cell.
The longer these target proteins have an $\alpha$ or a $\beta\gamma$ subunit bound to them,
the stronger and more prolonged the relayed signal will be.

The amount of time that the $\alpha$ and $\beta\gamma$ subunits remain `switched on' -
and hence available to relay signals - is limited by the behavior of the $\alpha$
subunit. The a subunit has an intrinsic GTPase activity, and it eventually
hydrolyzes its bound GTP back to GDP, returning the whole G protein
to its original, inactive conformation. GTP hydrolysis and
inactivation occur within seconds after the G protein has been activated.
The inactive G protein is now ready to be reactivated by another activated receptor.

The G-protein switch demonstrates a general principle of cell signaling
mentioned earlier: the mechanisms that shut a signal off are as important
as the mechanisms that turn it on. The shut-off
mechanisms also offer as many opportunities for control, and as many
dangers for mishap.

\subsection{Some G Proteins directly regulate ion channels}

The target proteins recognized by G-protein subunits are either enzymes
or ion channels in the plasma membrane. There are about 20 types of
mammalian G proteins, each activated by a particular set of cell-surface
receptors and dedicated to activating a particular set of target proteins.
In this way, binding of an extracellular signal molecule to a GPCR leads to
changes in the activities of a specific subset of the possible target proteins
in the plasma membrane, leading to a response that is appropriate for
that signal and that type of cell.

We look first at an example of direct G-protein regulation of ion channels.
The heartbeat in animals is controlled by two sets of nerves: one speeds
the heart up, the other slows it down. The nerves that signal a slowdown
in heartbeat do so by releasing acetylcholine, which binds to a GPCR on
the surface of the heart muscle cells. This GPCR activates the G protein,
$G_i$ . In this case, the $\beta\gamma$ complex is the active signaling component: it binds
to the intracellular face of a $K^+$ channel in the plasma membrane of the
heart muscle cell, forcing the ion channel into an open conformation.
This allows $K^+$ to flow out of the cell, thereby inhibiting
the cell’s electrical excitability. The signal is shut off -
and the $K^+$ channel recloses - when the a subunit inactivates itself by
hydrolyzing its bound GTP, returning the G protein to its inactive state.

\subsection{Some G Proteins activate membrane-bound enzymes}

When G proteins interact with ion channels, they cause an immediate
change in the state and behavior of the cell. Their interactions with
enzymes have more complex consequences, leading to the production
of additional intracellular signaling molecules. The two most frequent
target enzymes for G proteins are \textit{adenylyl cyclase}, the enzyme responsible
for production of the small intracellular signaling molecule \textit{cyclic
AMP}, and \textit{phospholipase C}, the enzyme responsible for production of the
small intracellular signaling molecules \textit{inositol trisphosphate} and \textit{diacyl-glycerol}.
The two enzymes are activated by different types of G proteins,
so that cells are able to couple the production of the small intracellular
signaling molecules to different extracellular signals. We concentrate
here on G proteins that stimulate enzyme activity. The small intracellular
signaling molecules generated in these cascades are often called \textbf{small
messengers}, or \textbf{second messengers} (the “first messengers” being the
extracellular signals); they are produced in large numbers when a membrane-bound
enzyme - such as adenylyl cyclase or phospholipase C - is
activated, and they rapidly diffuse away from their source, spreading the
signal.

Different small messenger molecules, of course, produce different
responses.

\subsection{The cyclic AMP Pathway can activate enzymes and turn on Genes}

Many extracellular signals acting via GPCRs affect the activity of the
enzyme \textbf{adenylyl cyclase} and thus alter the concentration of the small
messenger molecule \textbf{cyclic AMP} inside the cell. Most commonly, the
activated G-protein $\alpha$ subunit switches on the adenylyl cyclase, causing
a dramatic and sudden increase in the synthesis of cyclic AMP from ATP
(which is always present in the cell). Because it stimulates the cyclase,
this G protein is called $G_s$. To help terminate the signal, a second enzyme,
called \textit{cyclic AMP phosphodiesterase}, rapidly converts cyclic AMP to ordinary AMP.

Cyclic AMP phosphodiesterase is continuously active inside the cell.
Because it breaks cyclic AMP down so quickly, the concentration of this
small messenger can change rapidly in response to extracellular signals,
rising or falling tenfold in a matter of seconds. Cyclic AMP
is a water-soluble molecule, so it can, in some cases, carry the signal
throughout the cell, traveling from the site on the membrane where it is
synthesized to interact with proteins located in the cytosol, in the nucleus,
or on other organelles.

Cyclic AMP exerts most of its effects by activating the enzyme \textbf{cyclic-AMP–dependent protein kinase}
(PKA). This enzyme is normally
held inactive in a complex with another protein. The binding of cyclic
AMP forces a conformational change that unleashes the active kinase.
Activated PKA then catalyzes the phosphorylation of particular serines or
threonines on certain intracellular proteins, thus altering the activity of
the proteins. In different cell types, different sets of proteins are available
to be phosphorylated, which largely explains why the effects of cyclic
AMP vary with the type of target cell.

Different target cells respond very differently to extracellular signals that
change intracellular cyclic AMP concentrations.

In some cases, the effects of activating a cyclic AMP cascade are rapid;
in other cases, cyclic AMP responses involve changes in gene expression that take
minutes or hours to develop.

\subsection{The inositol Phospholipid Pathway triggers a rise in intracellular $Ca^{2+}$}

Some GPCRs exert their effects via G proteins that activate the membrane-bound
enzyme \textbf{phospholipase C} instead of adenylyl cyclase.

Once activated, phospholipase C propagates the signal by cleaving a lipid
molecule that is a component of the plasma membrane. The molecule is
an \textbf{inositol phospholipid} (a phospholipid with the sugar inositol attached
to its head) that is present in small quantities in the cytosolic half of the
membrane lipid bilayer. Because of the involvement of
this phospholipid, the signaling pathway that begins with the activation
of phospholipase C is often referred to as the inositol \textit{phospholipid pathway}.
It operates in almost all eucaryotic cells and can regulate a host of
different effector proteins.

The pathway works in the following way. When phospholipase C chops
the sugar-phosphate head off the inositol phospholipid, it generates two
small signaling molecules: \textbf{inositol 1,4,5-trisphosphate} ($IP_3$) and \textbf{diacylglycerol}
(DAG). $IP_3$, a water-soluble sugar phosphate, diffuses into the
cytosol, while the lipid diacylglycerol remains embedded in the plasma
membrane. Both molecules play a crucial part in relaying the signal, and
we will consider them in turn.

The $IP_3$ released into the cytosol rapidly encounters the endoplasmic
reticulum; there it binds to and opens $Ca^{2+}$ channels that are embedded in
the endoplasmic reticulum membrane. $Ca^{2+}$ stored inside the endoplasmic
reticulum rushes out into the cytosol through these open channels,
causing a sharp rise in the cytosolic concentration of free
$Ca^{2+}$, which is normally kept very low. This $Ca^{2+}$ in turn signals to other
proteins, as discussed below.

The diacylglycerol that is generated along with $IP_3$ helps recruit and acti-
vate a protein kinase, which translocates from the cytosol to the plasma
membrane. This enzyme is called protein kinase C (PKC) because it also
needs to bind $Ca^{2+}$ to become active. Once activated,
PKC phosphorylates a set of intracellular proteins that varies depending
on the cell type. PKC operates on the same principle as PKA, although
most of the proteins it phosphorylates are different.

\subsection{A $Ca^{2+}$ signal triggers many Biological Processes}

$Ca^{2+}$ has such an important and widespread role as a small intracellular
messenger that we must digress to consider its functions more generally.
A surge in the cytosolic concentration of free $Ca^{2+}$ is triggered by many
kinds of stimuli, not only those that act through GPCRs.

The concentration of free $Ca^{2+}$ in the cytosol of an unstimulated cell is
extremely low ($10^{–7}$ M) compared with its concentration in the extra-cellular
fluid and in the endoplasmic reticulum. These differences are
maintained by membrane-embedded pumps that actively pump $Ca^{2+}$
out of the cytosol - either into the endoplasmic reticulum or across the
plasma membrane and out of the cell. As a result, a steep electrochemical
gradient of $Ca^{2+}$ exists across both the endoplasmic reticulum membrane
and the plasma membrane. When a signal transiently opens $Ca^{2+}$ channels
in either of these membranes, $Ca^{2+}$ rushes
into the cytosol down its electrochemical gradient, triggering changes in
$Ca^{2+}$-responsive proteins in the cytosol. The same pumps that normally
operate to keep cytosolic $Ca^{2+}$ concentrations low also help to terminate
the $Ca^{2+}$ signal.

The effects of $Ca^{2+}$ in the cytosol are largely indirect, in that they are mediated
through the interaction of $Ca^{2+}$ with various kinds of \textit{$Ca^{2+}$-responsive
proteins}. The most widespread and common of these is \textbf{calmodulin},
which is present in the cytosol of all eucaryotic cells that have been examined,
including those of plants, fungi, and protozoa. When $Ca^{2+}$ binds to
calmodulin, the protein undergoes a conformational change that enables
it to wrap around a wide range of target proteins in the cell, altering their
activities. One particularly important class of targets for
calmodulin is the \textbf{$Ca^{2+}$/calmodulin-dependent protein kinases} (CaM-kinases).
When these kinases are activated by binding to calmodulin
complexed with $Ca^{2+}$, they influence other processes in the cell by phosphorylating
selected proteins. In the mammalian brain, for example, a
neuron-specific CaM-kinase is abundant at synapses, where it is thought
to play a part in some forms of learning and memory. This CaM-kinase is
activated by the pulses of $Ca^{2+}$ signals that occur during neural activity,
and mutant mice that lack the kinase show a marked inability to remember
where things are.

\subsection{Intracellular signaling cascades can achieve astonishing speed, sensitivity, and adaptability}

The steps in the signaling cascades associated with GPCRs take a long
time to describe, but they often take only seconds to execute.
Among the fastest of all responses
mediated by a GPCR, however, is the response of the eye to bright light:
it takes only 20 msec for the most quickly responding photoreceptor cells
of the retina (the cone photoreceptors, which are responsible for color
vision in bright light) to produce their electrical response to a sudden
flash of light.

This speed is achieved in spite of the necessity to relay the signal over
several steps of an intracellular signaling cascade. But photoreceptors
also provide a beautiful illustration of the positive advantages of signaling
cascades: in particular, such cascades allow spectacular amplification
of the incoming signal and also allow cells to adapt so as to be able to
detect signals of widely varying intensity. The quantitative details have
been most thoroughly analyzed for the rod photoreceptor cells in the eye,
which are responsible for non-color vision in dim light. In
this cell, light is sensed by \textit{rhodopsin}, a G-protein-coupled light receptor.
Light-activated rhodopsin activates a G protein called \textit{transducin}. The
activated a subunit of transducin then activates an intracellular signaling
cascade that causes cation channels to close in the plasma membrane
of the photoreceptor cell. This produces a change in the voltage across
the cell membrane, which alters neurotransmitter release and ultimately
leads to a nerve impulse being sent to the brain.

The signal is repeatedly amplified as it is relayed along this intracellular
signaling pathway. When lighting conditions are dim, as
on a moonless night, the amplification is enormous: as few as a dozen
photons absorbed in the entire retina will cause a perceptible signal to
be delivered to the brain. In bright sunlight, when photons flood through
each photoreceptor cell at a rate of billions per second, the signaling
cascade \textit{adapts}, stepping down the amplification more than 10,000-fold
so that the photoreceptor cells are not overwhelmed and can still register
increases and decreases in the strong light. The adaptation depends on
negative feedback: an intense response in the photoreceptor cell generates
an intracellular signal (a change in $Ca^{2+}$ concentration) that inhibits
the enzymes responsible for signal amplification.

\textbf{Adaptation} frequently occurs in signaling pathways that respond to
chemical signals; again, it allows cells to remain sensitive to changes of
signal intensity over a wide range of background levels of stimulation.
Adaptation, in other words, allows a cell to respond to both messages
that are whispered and those that are shouted.

\section{Enzyme-coupled receptors}

Like GPCRs, \textbf{enzyme-coupled receptors} are transmembrane proteins
that display their ligand-binding domains on the outer surface of the
plasma membrane. Instead of associating with a G protein, however, the
cytoplasmic domain of the receptor either acts as an enzyme itself or
forms a complex with another protein that acts as an enzyme.

Enzyme-coupled receptors, however, can also mediate direct, rapid
reconfigurations of the cytoskeleton, controlling the way a cell changes
its shape and moves. The extracellular signals for these architectural
alterations are often not diffusible signal proteins, but proteins attached
to the surfaces over which a cell is crawling. Disorders of cell growth,
proliferation, differentiation, survival, and migration are fundamental
to cancer, and abnormalities in signaling via enzyme-coupled receptors
have a major role in the development of this class of diseases.

The largest class of enzyme-coupled receptors is made up of those with
a cytoplasmic domain that functions as a tyrosine protein kinase, phosphorylating
specific tyrosines on selected intracellular proteins. Such
receptors are called \textbf{receptor tyrosine kinases} (RTKs), and we will focus
on these receptors here. Note that all of the other protein kinases we
have discussed so far - including PKA, PKC, and CaM-kinases - are serine/threonine kinases.

\subsection{Activated RTKs recruit a complex of intracellular signaling Proteins}

To do its job as a signal transducer, an enzyme-coupled receptor has to
switch on the enzyme activity of its intracellular domain (or of an associated
enzyme) when an external signal molecule binds to its extracellular
domain. Unlike the seven-pass GPCRs, enzyme-coupled receptor proteins
usually have only one transmembrane segment, which is thought
to span the lipid bilayer as a single a helix. Because a single $\alpha$ helix is
poorly suited to transmit a conformational change across the bilayer,
enzyme-coupled receptors have a different strategy for transducing the
extracellular signal. In many cases, the binding of a signal molecule
causes two receptor molecules to come together in the membrane, forming
a dimer. Contact between the two adjacent intracellular receptor tails
activates their kinase function, with the result that each receptor phosphorylates
the other. In the case of RTKs, the phosphorylations occur on
specific tyrosines located on the cytosolic tail of the receptors.

Tyrosine phosphorylation then triggers the assembly of an elaborate
intracellular signaling complex on the receptor tails. The newly phosphorylated
tyrosines serve as binding sites for a whole zoo of intracellular
signaling proteins - perhaps as many as 10 or 20 different molecules.
Some of these proteins become phosphorylated and activated
on binding to the receptor, and they then propagate the signal;
others function solely as adaptors, which couple the receptor to other signaling
proteins, thereby helping to build the active signaling complex.

While they last, the protein complexes assembled on the cytosolic tails
of the RTKs can transmit the signal along several routes simultaneously
to many destinations inside the cell, thus activating and coordinating the
numerous biochemical changes that are required to trigger a complex
response, such as cell proliferation. To help terminate the response, the
tyrosine phosphorylations are reversed by \textit{protein tyrosine phosphatases},
which remove the phosphates that were added to the tyrosines of the
RTKs and other signaling proteins in response to the extracellular signal.
In some cases, activated RTKs (and GPCRs) are inactivated in a more brutal
way: they are dragged into the interior of the cell by endocytosis and
then destroyed by digestion in lysosomes.

\subsection{Most RTKs activate the monomeric GTPase Ras}

As we have seen, activated RTKs recruit many kinds of intracellular sig-
naling proteins and form large signaling complexes. One of the key players
in these signaling complexes is \textbf{Ras} - a small GTP-binding protein that is
bound by a lipid tail to the cytoplasmic face of the plasma membrane.
Virtually all RTKs activate Ras, including platelet-derived growth factor
(PDGF) receptors, which mediate cell proliferation in wound healing, and
nerve growth factor (NGF) receptors, which prevent certain neurons from
dying in the developing nervous system.

The Ras protein is a member of a large family of small GTP-binding
proteins, often called \textbf{monomeric GTPases} to distinguish them from
the trimeric G proteins that we encountered earlier. Ras resembles the
$\alpha$ subunit of a G protein and functions as a molecular switch in much
the same way. It cycles between two distinct conformational states -
active when GTP is bound and inactive when GDP is bound.
Interaction with an activated signaling protein encourages Ras
to exchange its GDP for GTP, thus switching Ras to its activated state.
After a delay, Ras switches itself off again by hydrolyzing
its bound GTP to GDP.

In its active state, Ras promotes the activation of a phosphorylation cascade
in which a series of serine/threonine protein kinases phosphorylate
and activate one another in sequence, like an intracellular game of dominoes.
This relay system, which carries the signal from
the plasma membrane to the nucleus, includes a three-kinase protein
module called the \textbf{MAP-kinase signaling module}, in honor of the final
kinase in the chain, the mitogen-activated protein kinase or \textbf{MAP kinase}.
(\textit{Mitogens} are extracellular signal molecules that stimulate cell proliferation).
In this pathway, MAP kinase is phosphorylated and activated by an
enzyme called, logically enough, \textit{MAP kinase kinase}. And this protein is
itself switched on by a \textit{MAP kinase kinase kinase} (which is activated by
Ras). At the end of the MAP-kinase cascade, MAP kinase phosphorylates
various effector proteins, including certain transcription regulators, altering
their ability to control gene transcription. This change in the pattern
of gene expression may stimulate cell proliferation, promote cell survival,
or induce cell differentiation: the precise outcome will depend on which
other genes are active in the cell and what other signals the cell receives.

\subsection{RTKs activate Pi 3-Kinase to Produce lipid docking sites in the Plasma membrane}

Many of the extracellular signal proteins that stimulate animal cells to
survive, grow, and proliferate act through RTKs. These include signal
proteins belonging to the insulin-like growth factor (IGF) family. One
crucially important signaling pathway that RTKs activate to promote cell
growth and survival relies on the enzyme \textbf{phosphoinositide 3-kinase}
(PI 3-kinase), which phosphorylates inositol phospholipids in the plasma
membrane. These phosphorylated lipids become docking sites for specific
intracellular signaling proteins, which relocate from the cytosol to
the plasma membrane, where they can activate one another.

One of the most important of these relocated signaling proteins is the
serine/threonine protein kinase \textit{Akt}, which is also called \textit{protein kinase B},
or PKB. Akt promotes the growth and survival of many cell
types, often by inactivating the signaling proteins it phosphorylates.

In addition to promoting cell survival, the \textit{PI-3-kinase-Akt signaling pathway}
also stimulates cells to grow in size. It does so by indirectly activating
a large serine/threonine kinase called \textit{Tor}.

\subsection{Some receptors activate a fast track to the nucleus}

A few hormones and many local mediators called \textbf{cytokines} bind to
receptors that can activate transcription regulators that are held in a
latent, inactive state near the plasma membrane. Once turned on, these
regulatory proteins - called STATs (for signal transducers and activators
of transcription) - head straight for the nucleus, where they stimulate the
transcription of specific genes. This direct signaling pathway is used, for
example, by \textit{interferons}, which are cytokines produced by infected cells
that instruct other cells to produce proteins that make them more resistant
to viral infection.

Unlike the RTKs that stimulate elaborate signaling cascades, the cytokine
and hormone receptors that rely on STATs have no intrinsic enzyme activity.
Instead, they associate with cytoplasmic tyrosine kinases called JAKs,
which are activated when a cytokine or hormone binds to the receptor.
Once activated, the JAKs phosphorylate and activate STATs, which then
migrate to the nucleus, where they stimulate the transcription of specific
target genes. For example, the hormone prolactin, which stimulates
breast cells to make milk, acts by binding to a receptor that is associated
with a specific pair of JAKs. These JAKs activate a particular STAT
that then turns on the transcription of the genes encoding milk proteins.

Different cytokine and hormone receptors evoke different cell responses
by activating different STATs. Like any pathway that is turned on by
phosphorylation, these signals are shut off by protein phosphatases that
remove the phosphate groups from the activated signaling proteins.

\subsection{Multicellularity and cell communication evolved independently in Plants and animals}

Plants and animals have been evolving independently for more than a
billion years, the last common ancestor being a single-celled eucaryote
that most likely lived on its own. Because these kingdoms diverged so
long ago - when it was still "every cell for itself" - each has evolved its
own molecular solutions to the complex problem of becoming multicellular.
Thus, the mechanisms for cell-cell communication in plants and
animals evolved separately and might be expected to be quite different.
At the same time, however, plants and animals started with a common
set of eucaryotic genes - including some used by single-celled organisms
to communicate among themselves - and so their signaling systems
should show some similarities

\subsection{Protein Kinase networks integrate information to control complex cell Behaviors}

In this chapter, we have outlined several pathways for conveying a signal
from the cell surface to the cell interior. Each pathway differs from the others, yet they use
some common components to transmit their signals. Because all these
pathways eventually activate protein kinases, it seems that each is capable
in principle of regulating practically any process in the cell.

In fact, the complexity of cell signaling is much greater than we have
described. First, we have not discussed many of the intracellular signaling
pathways available to cells.
Perhaps more importantly, the major signaling pathways we have discussed
interact in ways that we have not described. They are connected
by interactions of many sorts, but the most extensive links are those
mediated by the protein kinases present in each of the pathways. These
kinases often phosphorylate, and hence regulate, components in other
signaling pathways, in addition to components in the pathway to which
they themselves primarily belong. Thus, a certain amount of cross-talk
occurs between the different pathways and, indeed,
between virtually all of the control systems of the cell.

A cell receives messages from many sources, and it must integrate this
information to generate an appropriate response: to live or die, to divide
or differentiate, to change shape, to move, to send out a chemical message
of its own, and so on. Through the cross-talk
between signaling pathways, the cell is able to put together multiple bits
of information and react to the combination. Thus, some intracellular
signaling proteins act as integrating devices, usually by having several
potential phosphorylation sites, each of which can be phosphorylated by
a different protein kinase. Information received from different sources
can converge on such proteins, which then convert the input to a single
outgoing signal. The integrating
proteins in turn can deliver a signal to many downstream targets. In this
way, the intracellular signaling system may act like a network of nerve
cells in the brain - or like a collection of microprocessors in a computer -
interpreting complex information and generating complex responses.

Our exploration of the pathways that cells use to process signals from
their environment has led us from receptors on the cell surface to the
proteins that form the elaborate control systems that operate deep within
the cell’s interior.

\section{Essential concepts}

\begin{itemize}
\item Cells in multicellular organisms communicate through a large variety
of extracellular chemical signals.
\item In animals, hormones are carried in the blood to distant target cells,
but most other extracellular signal molecules act over only a short
range. Neighboring cells often communicate through direct cell-cell
contact.
\item Extracellular signal molecules stimulate a target cell when they bind
to and activate receptor proteins. Each receptor protein recognizes a
particular signal molecule.
\item Small hydrophobic extracellular signal molecules, such as steroid
hormones and nitric oxide, can diffuse directly across the plasma
membrane; they activate intracellular proteins, which are usually
either transcription regulators or enzymes.
\item Most extracellular signal molecules cannot pass through the plasma
membrane; they bind to cell-surface receptor proteins that convert
(transduce) the extracellular signal into different intracellular
signals.
\item There are three main classes of cell-surface receptors: (1) ion-channel-coupled
receptors, (2) G-protein-coupled receptors (GPCRs), and
(3) enzyme-coupled receptors.
\item GPCRs and enzyme-coupled receptors respond to extracellular signals
by activating one or more intracellular signaling pathways that
alter the behavior of the cell.
\item Turning off signaling pathways is as important as turning them on.
Each activated component in a signaling pathway must be subsequently
inactivated or removed for the pathway to function again.
\item GPCRs activate a class of trimeric GTP-binding proteins called G proteins;
these act as molecular switches, transmitting the signal onward
for a short period and then switching themselves off by hydrolyzing
their bound GTP to GDP.
\item Some G proteins directly regulate ion channels in the plasma membrane.
Others directly activate (or inactivate) the enzyme adenylyl
cyclase, increasing (or decreasing) the intracellular concentration
of the small messenger molecule cyclic AMP. Still other G proteins
directly activate the enzyme phospholipase C, which generates
the small messenger molecules inositol trisphosphate ($IP_3$) and
diacylglycerol.
\item $IP_3$ opens $Ca^{2+}$ channels in the membrane of the endoplasmic reticulum,
releasing a flood of free $Ca^{2+}$ ions into the cytosol. $Ca^{2+}$ itself
acts as a small intracellular messenger, altering the activity of a
wide range of $Ca^{2+}$-responsive proteins, including calmodulin, which
activates various target proteins such as such as $Ca^{2+}$/calmodulin-dependent
protein kinases (CaM-kinases) .
\item A rise in cyclic AMP activates protein kinase A (PKA), while $Ca^{2+}$ and
diacylglycerol in combination activate protein kinase C (PKC).
\item PKA, PKC, and CaM-kinases phosphorylate selected target proteins
on serines and threonines, thereby altering protein activity. Different
cell types contain different sets of signaling and effector target proteins
and are therefore affected in different ways.
\item Many enzyme-coupled receptors have intracellular protein domains
that function as enzymes; many are receptor tyrosine kinases (RTKs),
which phosphorylate themselves and selected intracellular signaling
proteins on tyrosines.
\item The phosphotyrosines on RTKs serve as docking sites for various intracellular
signaling proteins, usually including the small GTP-binding
protein Ras. Ras activates a three-protein MAP-kinase signaling
module that helps relay the signal from the plasma membrane to the
nucleus.
\item Mutations that stimulate cell proliferation by making Ras constantly
active are a common feature of many human cancers.
\item Some RTKs stimulate cell growth and cell survival by activating PI
3-kinase, which phosphorylates specific inositol phospholipids to
produce lipid docking sites in the plasma membrane that allow certain
signaling proteins to congregate and activate one another.
\item Some receptors, including Notch and cytokine receptors, activate
a direct pathway to the nucleus. Instead of activating signaling
cascades, they turn on transcription regulators at the plasma membrane,
which then migrate to the nucleus where they activate specific
genes.
\item Plants, like animals, use enzyme-coupled cell-surface receptors to
control their growth and development.
\item Extracellular signals in plants often act by relieving the transcriptional
repression of signal-responsive genes.
\item Different intracellular signaling pathways interact, enabling cells to
produce an appropriate response to a complex combination of signals.
Some combinations of signals enable a cell to survive; other combinations
signal a cell to proliferate; and, in the absence of any signals,
most animal cells will kill themselves by undergoing apoptosis.
\end{itemize}

\chapter{Cytoskeleton}

The ability of eucaryotic cells to adopt a variety of shapes, organize
the many components in their interior, interact mechanically with the
environment, and carry out coordinated movements depends on the
cytoskeleton - an intricate network of protein filaments that extends
throughout the cytoplasm.
The cytoskeleton is built on a framework of three types of protein filaments:
intermediate filaments, microtubules, and actin filaments.
each type of filament has distinct mechanical
properties and is formed from a different protein subunit. A family of
fibrous proteins form the intermediate filaments; tubulin is the subunit
in microtubules; and actin is the subunit in actin filaments. In each case,
thousands of subunits assemble into a fine thread of protein that sometimes
extends across the entire cell.

\section{Intermediate filaments}

\textbf{Intermediate filaments} have great tensile strength, and their main function
is to enable cells to withstand the mechanical stress that occurs when
cells are stretched. Intermediate filaments are the toughest
and most durable of the three types of cytoskeletal filaments: when cells
are treated with concentrated salt solutions and nonionic detergents, the
intermediate filaments survive while most of the rest of the cytoskeleton
is destroyed.

Intermediate filaments are found in the cytoplasm of most animal cells.
They typically form a network throughout the cytoplasm, surrounding
the nucleus and extending out to the cell periphery. There they are often
anchored to the plasma membrane at cell-cell junctions such as desmosomes,
where the external face of the membrane is connected to that of another cell.
They are also found within the nucleus; a mesh of intermediate filaments, the
nuclear lamina, underlies and strengthens the nuclear envelope in all
eucaryotic cells.

\subsection{Intermediate Filaments are strong and ropelike}

Intermediate filaments are like ropes with many long strands twisted
together to provide tensile strength. The strands of this
rope—the subunits of intermediate filaments - are elongated fibrous
proteins, each composed of an N-terminal globular head, a C-terminal
globular tail, and a central elongated rod domain. The
rod domain consists of an extended $\alpha$-helical region that enables pairs of
intermediate filament proteins to form stable dimers by wrapping around
each other in a coiled-coil configuration. Two of these coiled-coil dimers
then associate by noncovalent bonding to form a tetramer, and the tetramers then
bind to one another end-to-end and side-by-side, and also by noncovalent
bonding, to generate the final ropelike intermediate filament.

The central rod domains of different intermediate filament proteins are all
similar in size and amino acid sequence, so that when they pack together
they always form filaments of similar diameter and internal structure. By
contrast, the globular head and tail regions, which are exposed on the
surface of the filament, allow it to interact with other components of the
cytoplasm.

\subsection{Intermediate Filaments strengthen cells against mechanical stress}

Intermediate filaments are particularly prominent in the cytoplasm of cells
that are subject to mechanical stress.

Intermediate filaments can be grouped into four classes:

\begin{enumerate}
\item \textit{keratin filaments} in epithelial cells;
\item \textit{vimentin} and \textit{vimentin-related filaments} in
connective-tissue cells, muscle cells, and supporting cells of the nervous
system (glial cells);
\item \textit{neurofilaments} in nerve cells;
\item \textit{nuclear lamins}, which strengthen the nuclear membrane of all animal cells.
\end{enumerate}

The first three filament types are found in the cytoplasm, the fourth
in the cell nucleus. Filaments of each class are formed by polymerization
of their corresponding protein subunits.

The keratins are the most diverse class of intermediate filament.
Keratin filaments typically span the interiors of epithelial
cells from one side of the cell to the other, and filaments in adjacent
epithelial cells are indirectly connected through cell-cell junctions
called \textit{desmosomes}. The ends of the keratin filaments
are anchored to the desmosomes, and they associate laterally
with other cell components through their globular head and tail domains,
which project from the surface of the assembled filament. This cabling
of high tensile strength, formed by the filaments throughout the epithelial
sheet, distributes the stress that occurs when the skin is stretched.

Many of the intermediate filaments are further stabilized and reinforced
by accessory proteins, such as plectin, that cross-link the filament bundles
into strong arrays. In addition to holding together bundles of intermediate
filaments (particularly vimentin), these proteins link intermediate filaments
to microtubules, to actin filaments, and to adhesive structures in
the desmosomes.

\subsection{The nuclear envelope Is supported by a meshwork of Intermediate Filaments}

Whereas cytoplasmic intermediate filaments form ropelike structures,
the intermediate filaments lining and strengthening the inside surface of
the inner nuclear membrane are organized as a two-dimensional mesh.
The intermediate filaments within this tough \textbf{nuclear lamina} are constructed
from a class of intermediate filament proteins called
\textit{lamins} (not to be confused with \textit{laminin}, which is an extracellular matrix
protein). In contrast to the very stable cytoplasmic intermediate filaments
found in many cells, the intermediate filaments of the nuclear lamina
disassemble and re-form at each cell division, when the nuclear envelope
breaks down during mitosis and then re-forms in each daughter cell.

Disassembly and reassembly of the nuclear lamina are controlled by
the phosphorylation and dephosphorylation of
the lamins by protein kinases. When the lamins are phosphorylated, the
consequent conformational change weakens the binding between the
tetramers and causes the filament to fall apart. Dephosphorylation at the
end of mitosis causes the lamins to reassemble.

\section{Microtubules}

\textbf{Microtubules} have a crucial organizing role in all eucaryotic cells. They
are long and relatively stiff hollow tubes of protein that can rapidly disassemble
in one location and reassemble in another. In a typical animal
cell, microtubules grow out from a small structure near the center of the
cell called the \textit{centrosome}. Extending out toward the cell
periphery, they create a system of tracks within the cell, along which vesi-
cles, organelles, and other cell components are moved. These and other
systems of cytoplasmic microtubules are the part of the cytoskeleton
mainly responsible for anchoring membrane-enclosed organelles within
the cell and for guiding intracellular transport.

When a cell enters mitosis, the cytoplasmic microtubules disassemble
and then reassemble into an intricate structure called the \textit{mitotic spindle}.
The mitotic spindle provides the machinery that will segregate the chromosomes equally
into the two daughter cells just before a cell divides. Microtubules can also form
permanent structures, as exemplified by the rhythmically beating hairlike
structures called \textit{cilia} and \textit{flagella}. These extend from the
surface of many eucaryotic cells, which use them either as a means of
propulsion or to sweep fluid over the cell surface. The core of a eucaryotic
cilium or flagellum consists of a highly organized and stable bundle of
microtubules.

\subsection{Microtubules are Hollow tubes with structurally distinct ends}

Microtubules are built from subunits - molecules of tubulin - each of
which is itself a dimer composed of two very similar globular proteins
called \textit{$\alpha$-tubulin} and \textit{$\beta$-tubulin}, bound tightly together by noncovalent
bonding. The tubulin dimers stack together, again by noncovalent bonding,
to form the wall of the hollow cylindrical microtubule. This tubelike
structure is made of 13 parallel protofilaments, each a linear chain of tubulin
dimers with $\alpha$- and $\beta$-tubulin alternating along its length.
Each \textit{protofilament} has a structural \textbf{polarity}, with $\alpha$-tubulin exposed at
one end and $\beta$-tubulin at the other, and this polarity - the directional
arrow embodied in the structure - is the same for all the protofilaments,
giving a structural polarity to the microtubule as a whole. One end of the
microtubule, thought to be the $\beta$-tubulin end, is called its \textit{plus end}, and
the other, the $\alpha$-tubulin end, its \textit{minus end}.

\subsection{The centrosome is the major microtubule-organizing center in animal cells}

Microtubules in cells are formed by outgrowth from specialized organ-
izing centers, which control the number of microtubules formed, their
location and their orientation in the cytoplasm. In animal cells, for exam-
ple, the \textbf{centrosome}, which is typically close to the cell nucleus when the
cell is not in mitosis, organizes the array of microtubules that radiates
outward from it through the cytoplasm.
Centrosomes contain hundreds of ring-shaped structures formed from another type of
tubulin, $\gamma$-tubulin, and each $\gamma$-tubulin ring serves as the starting point, or
nucleation site, for the growth of one microtubule. The
$\alpha\beta$-tubulin dimers add to the $\gamma$-tubulin ring in a specific orientation, with
the result that the minus end of each microtubule is embedded in the
centrosome and growth occurs only at the plus end - that is, at the outward-facing end.

In addition to its $\gamma$-tubulin rings, the centrosome in most animal cells also
contains a pair of \textbf{centrioles}, curious structures each made of a cylindrical
array of short microtubules. The centrioles have no role in the nucleation
of microtubules in the centrosome (the $\gamma$-tubulin rings alone are sufficient),
and their function remains something of a mystery, especially as
plant cells lack them. Centrioles are, however, similar, if not identical, to
the basal bodies that form the organizing centers for the microtubules in
cilia and flagella.

Microtubules need nucleating sites such as those provided by the $\gamma$-tubulin
rings in the centrosome because it is much harder to start a new
microtubule from scratch, by first assembling a ring of $\alpha\beta$-tubulin dimers,
than to add such dimers to a preexisting microtubule structure.

\subsection{Growing microtubules showing dynamic instability}

Once a microtubule has been nucleated, its plus end typically grows outward
from the organizing center by the addition of $\alpha\beta$-tubulin subunits
for many minutes. Then, without warning, the microtubule suddenly
undergoes a transition that causes it to shrink rapidly inward by losing
subunits from its free end. It may shrink partially and then,
no less suddenly, start growing again, or it may disappear completely, to
be replaced by a new microtubule from the same $\gamma$-tubulin ring.

This remarkable behavior, known as \textbf{dynamic instability}, stems from the
intrinsic capacity of tubulin molecules to hydrolyze GTP. Each free tubulin
dimer contains one tightly bound GTP molecule that is hydrolyzed to GDP
(still tightly bound) shortly after the subunit is added to a growing microtubule.
The GTP-associated tubulin molecules pack efficiently together
in the wall of the microtubule, whereas tubulin molecules carrying GDP
have a different conformation and bind less strongly to each other.

When polymerization is proceeding rapidly, tubulin molecules add to the
end of the microtubule faster than the GTP they carry is hydrolyzed. The
end of a growing microtubule is therefore composed entirely of GTP-tubulin
subunits, forming what is known as a \textit{GTP cap}. In this situation,
the growing microtubule will continue to grow. Because
of the randomness of chemical processes, however, it will occasionally
happen that tubulin at the free end of the microtubule hydrolyzes its GTP
before the next tubulin has been added, so that the free ends of protofilaments
are now composed of GDP-tubulin subunits. This change tips
the balance in favor of disassembly. Because the rest of
the microtubule is composed of GDP-tubulin, once depolymerization has
started, it will tend to continue, often at a catastrophic rate; the microtubule
starts to shrink rapidly and continuously, and may even disappear.

The GDP-containing tubulin molecules that are freed as the microtubule
depolymerizes join the unpolymerized tubulin molecules already in the
cytosol.

\subsection{Microtubules are maintained by a balance of assembly and disassembly}

The relative instability of microtubules allows them to undergo rapid
remodeling, and this is crucial for microtubule function. In a normal cell,
the centrosome (or other organizing center) is continually shooting out
new microtubules in an exploratory fashion in different directions and
retracting them. A microtubule growing out from the centrosome can,
however, be prevented from disassembling if its plus end is somehow
permanently stabilized by attachment to another molecule or cell structure
so as to prevent tubulin depolymerization. If stabilized by attachment
to a structure in a more distant region of the cell, the microtubule will
establish a relatively stable link between that structure and the centrosome.
The centrosome can be compared to a fisherman
casting a line: if there is no bite at the end of the line, the line is quickly
withdrawn and a new cast is made; but if a fish bites, the line remains in
place, tethering the fish to the fisherman. This simple strategy of random
exploration and selective stabilization enables the centrosome and other
nucleating centers to set up a highly organized system of microtubules
linking selected parts of the cell. It is also used to position organelles
relative to one another.

Drugs that prevent the polymerization or depolymerization of tubulin can
have a rapid and profound effect on the organization of the cytoskeleton - and
the behavior of the cell. Consider the mitotic spindle, the microtubule
framework that guides the chromosomes during mitosis.
If a cell in mitosis is exposed to the drug \textit{colchicine}, which binds
tightly to free tubulin and prevents its polymerization into microtubules,
the mitotic spindle rapidly disappears and the cell stalls in the middle of
mitosis, unable to partition its chromosomes into two groups.

The drug \textit{taxol} has the opposite action at the molecular level. It binds
tightly to microtubules and prevents them from losing subunits. Because
new subunits can still be added, the microtubules can grow but cannot
shrink. However, despite the differences in molecular detail, taxol has
the same overall effect on the cell as colchicine: it also arrests dividing
cells in mitosis.

The inactivation or destruction of the mitotic spindle eventually kills
dividing cells.

\subsection{Microtubules organize the Interior of the cell}

Cells are able to modify the dynamic instability of their microtubules for
particular purposes.

Most differentiated animal cells are \textit{polarized}; that is, one end of the cell
is structurally or functionally different from the other. The cell’s polarity is a reflection of
the polarized systems of microtubules in its interior, which help to position
organelles in their required location within the cell and to guide the
streams of traffic moving between one part of the cell and another.
Movement along microtubules is immeasurably faster and more efficient than free diffusion.

But the microtubules in living cells do not act alone. Their activity, like
those of other cytoskeletal filaments, depends on a large variety of acces-
sory proteins that bind to them.

\subsection{Motor Proteins drive Intracellular transport}

Mitochondria and the smaller
membrane-enclosed organelles and vesicles move in small, jerky steps - that
is, they move for a short period, stop, and then start again. This
\textit{saltatory movement} is much more sustained and directional than the continual,
small Brownian movements caused by random thermal motions.
Both microtubules and actin filaments are involved in saltatory and other
directed intracellular movements in eucaryotic cells. In both cases the
movements are generated by \textbf{motor proteins}, which use the energy
derived from repeated cycles of ATP hydrolysis to travel steadily along the
actin filament or the microtubule in a single direction. At the same time,
these motor proteins also attach to other cell components and thus transport
this cargo along the filaments.

The motor proteins that move along cytoplasmic microtubules, such as
those in the axon of a nerve cell, belong to two families: the \textbf{kinesins}
generally move toward the plus end of a microtubule (away from the centrosome),
while the \textbf{dyneins} move toward the minus end (toward the centrosome).
These kinesins and dyneins are both dimers with two globular
ATP-binding heads and a single tail. The heads interact
with microtubules in a stereospecific manner, so that the motor protein
will attach to a microtubule in only one direction. The tail of a motor
protein generally binds stably to some cell component, such as a vesicle
or an organelle, and thereby determines the type of cargo that the motor
protein can transport. The globular heads of kinesin and
dynein are enzymes with ATP-hydrolyzing (ATPase) activity. This reac-
tion provides the energy for a cycle of conformational changes in the
head that enable it to move along the microtubule by a cycle of binding,
release, and rebinding to the microtubule.

\subsection{Organelles move along microtubules}

Microtubules and motor proteins play an important part in positioning
membrane-enclosed organelles within a eucaryotic cell. In most animal
cells, for example, the tubules of the endoplasmic reticulum reach
almost to the edge of the cell. The Golgi apparatus, in contrast,
is located in the interior of the cell near the centrosome.
Both the endoplasmic reticulum and the Golgi apparatus depend
on microtubules for their alignment and positioning. The membranes of
the endoplasmic reticulum extend out from their points of connection
with the nuclear envelope, aligning with microtubules
that extend from the centrosome out to the plasma membrane. As the
cell develops and the endoplasmic reticulum grows, kinesins attached to
the outside of the endoplasmic reticulum membrane (via receptor proteins)
pull it outward along microtubules, stretching it like a net. Dyneins,
similarly attached to the Golgi membranes, pull the Golgi apparatus the
other way along microtubules, inward toward the cell center. In this way
the regional differences in internal membranes, on which the successful
function of the cell depends, are created and maintained.

When cells are treated with a drug such as colchicine
that causes microtubules to disassemble, both of these organelles change
their location dramatically.

\subsection{Cilia and Flagella contain stable microtubules moved by dynein}

Earlier in this chapter we mentioned that many microtubules in cells are
stabilized through their association with other proteins, and therefore no
longer show dynamic instability. Stable microtubules are employed by
cells as stiff supports on which to construct a variety of polarized struc-
tures, including the remarkable cilia and flagella that allow eucaryotic
cells to move water over their surface. \textbf{Cilia} are hairlike structures about
0.25 $\mu$m in diameter, covered by plasma membrane, that extend from
the surface of many kinds of eucaryotic cells. A single
cilium contains a core of stable microtubules, arranged in a bundle, that
grow from a \textit{basal body} in the cytoplasm; the basal body serves as the
organizing center for the cilium.
Cilia move fluid over the surface of a cell or propel single cells through a
fluid.

The \textbf{flagella} (singular flagellum) that propel sperm and many protozoa
are much like cilia in their internal structure but are usually very much
longer. They are designed to move the entire cell, and instead of generating
a current, they propagate regular waves along their length that drive
the cell through liquid.

The movement of a cilium or a flagellum is produced by the bending of
its core as the microtubules slide against each other. The microtubules
are associated with numerous proteins, which project at
regular positions along the length of the microtubule bundle. Some serve
as cross-links to hold the bundle of microtubules together; others generate
the force that causes the cilium to bend.

The most important of these accessory proteins is the motor protein \textit{ciliary dynein},
which generates the bending motion of the core. It closely
resembles cytoplasmic dynein and functions in much the same way.

\section{Actin filaments}

\textbf{Actin filaments} are found in all eucaryotic cells and are essential for
many of their movements, especially those involving the cell surface.
Without actin filaments, for example, an animal cell could not crawl
along a surface, engulf a large particle by phagocytosis, or divide in two.
Like microtubules, many actin filaments are unstable, but by associating
with other proteins they can also form stable structures in cells, such as
the contractile apparatus of muscle. Actin filaments interact with a large
number of \textit{actin-binding proteins} that enable the filaments to serve a variety
of functions in cells. Depending on their association with different
proteins, actin filaments can form stiff and relatively permanent structures,
such as the \textit{microvilli} on the brush-border cells lining the intestine
or small \textit{contractile bundles} in the cytoplasm that can
contract and act like the "muscles" of a cell; they can also
form temporary structures, such as the dynamic protrusions formed at
the leading edge of a crawling fibroblast or the \textit{contractile
ring} that pinches the cytoplasm in two when an animal cell divides.

\subsection{Actin Filaments are thin and Flexible}

Each filament is a twisted chain of identical globular actin
molecules, all of which “point” in the same direction along the axis of the
chain. Like a microtubule, therefore, an actin filament has a structural
polarity, with a plus end and a minus end.

Actin filaments are thinner, more flexible, and usually shorter than microtubules.
There are, however, many more of them, so that the total length
of all the actin filaments in a cell is generally many times greater than
the total length of all of the microtubules. Actin filaments rarely occur in
isolation in the cell; they are generally found in cross-linked bundles and
networks, which are much stronger than the individual filaments.

\subsection{Actin and tubulin Polymerize by similar mechanisms}

Actin filaments can grow by the addition of actin monomers at either end,
but the rate of growth is faster at the plus end than at the minus end. A
naked actin filament, like a microtubule without associated proteins, is
inherently unstable, and it can disassemble from both ends. Each free
actin monomer carries a tightly bound nucleoside triphosphate, in this
case ATP, which is hydrolyzed to ADP soon after the incorporation of
the actin monomer into the filament. As with the GTP bound to tubulin,
hydrolysis of ATP to ADP in an actin filament reduces the strength of
binding between monomers and decreases the stability of the polymer.
Nucleotide hydrolysis thereby promotes depolymerization, helping the
cell to disassemble filaments after they have formed.

As with microtubules, the ability to assemble and disassemble is required
for many of the functions performed by actin filaments, such as their role
in cell locomotion. Actin filament function can be perturbed experimentally
by certain toxins produced by fungi or marine sea sponges. Some,
such as the \textit{cytochalasins}, prevent actin polymerization; others, such as
\textit{phalloidin}, stabilize actin filaments against depolymerization.
Addition of these toxins to the medium bathing cells or
tissues, even in low concentrations, instantaneously freezes cell movements
such as the crawling motion of a fibroblast. Thus, the function of
actin filaments depends on a dynamic equilibrium between the actin filaments
and the pool of actin monomers.

\subsection{Many Proteins bind to actin and modify Its Properties}

What keeps the actin monomers in cells from polymerizing totally into filaments? The answer is that cells
contain small proteins, such as \textit{thymosin} and \textit{profilin}, that bind to actin
monomers in the cytosol, preventing them from adding to the ends of
actin filaments. By keeping actin monomers in reserve until they are
required, these proteins play a crucial role in regulating actin polymerization.
When actin filaments are needed, other actin-binding proteins
promote their assembly. Proteins called \textit{formins actin-related proteins} and
(ARPs) both control actin assembly at the advancing front of a migrating
cell.

There are a great many actin-binding proteins in cells. Most of these bind
to assembled actin filaments rather than to actin monomers and control
the behavior of the intact filaments. Actin-bundling proteins,
for example, hold actin filaments together in parallel bundles in
microvilli; other cross-linking proteins hold actin filaments together in
a gel-like meshwork within the cell \textit{cortex} - the layer of cytoplasm just
beneath the plasma membrane; filament-severing proteins, such as \textit{gelsolin},
fragment actin filaments into shorter lengths and thus can convert
an actin gel to a more fluid state.

\subsection{An actin-rich cortex underlies the Plasma membrane of most eucaryotic cells}

Although actin is found throughout the cytoplasm of a eucaryotic cell,
in most cells it is highly concentrated in a layer just beneath the plasma
membrane. In this region, called the \textbf{cell cortex}, actin filaments are
linked by actin-binding proteins into a meshwork that supports the outer
surface of the cell and gives it mechanical strength. The cell cortex of
other animal cells is thicker and more complex and supports
a far richer repertoire of shapes and movements. Like the cortex in a red
cell, it contains spectrin and ankyrin; however, it also includes a dense
network of actin filaments that project into the cytoplasm, where they
become cross-linked into a three-dimensional meshwork. This cortical
actin mesh governs the shape and mechanical properties of the plasma
membrane and the cell surface: the rearrangement of actin filaments
within the cortex provides the molecular basis for changes in cell shape
and cell locomotion.

\subsection{Cell crawling depends on actin}

Many cells move by crawling over surfaces, rather than by swimming
by means of cilia or flagella. White blood cells known
as neutrophils migrate through tissues when they ‘smell’ small diffus-
ing molecules released by bacteria, which the neutrophils seek out and
destroy. For these immune hunters, chemotactic molecules binding to
receptors on the cell surface trigger changes in actin filament assembly
that drive the cells toward their prey.

The molecular mechanisms of these and other forms of cell crawling
entail coordinated changes of many molecules in different regions of
the cell, and no single, easily identifiable locomotory organelle, such as
a flagellum, is responsible. In broad terms, however, three interrelated
processes are known to be essential:

\begin{enumerate}
\item the cell pushes out protrusions at its “front,” or leading edge;
\item these protrusions adhere to the surface over which the cell is crawling;
\item the rest of the cell drags itself forward by traction on these anchorage points.
\end{enumerate}

All three processes involve actin, but in different ways. The first step, the
pushing forward of the cell surface, is driven by actin polymerization. The
leading edge of a crawling fibroblast in culture regularly extends thin,
sheetlike \textbf{lamellipodia}, which contain a dense meshwork of actin fila-
ments, oriented so that most of the filaments have their plus ends close
to the plasma membrane. Many cells also extend thin, stiff
protrusions called \textbf{filopodia}, both at the leading edge and elsewhere on
their surface. Both lamellipodia and filopodia are exploratory, motile structures
that form and retract with great speed, moving at around 1 $\mu$m per
second.

The formation and growth of actin filaments at the leading edge of a
cell are assisted by various actin-binding accessory proteins.

When the lamellipodia and filopodia touch down on a favorable patch of
surface, they stick: transmembrane proteins in their plasma membrane,
known as \textit{integrins}, adhere to molecules in the extracellular matrix that
surrounds cells or on the surface of a neighboring cell over which the
moving cell is crawling. Meanwhile, on the intracellular face of the crawling
cell’s membrane, integrins capture actin filaments, thereby creating a
robust anchorage for the system of actin filaments inside the crawling cell.
To use this anchorage to drag its body forward, the
cell now makes use of internal contractions to exert a pulling force.
These too depend on actin, but in a different way - through
the interaction of actin filaments with motor proteins known as \textit{myosins}.

\subsection{Actin associates with myosin to Form contractile structures}

All actin-dependent motor proteins belong to the \textbf{myosin} family. They
bind to and hydrolyze ATP, which provides the energy for their movement
along actin filaments from the minus end of the filament toward the plus
end. Myosin, along with actin, was first discovered in skeletal muscle,
and much of what we know about the interaction of these two proteins
was learned from studies of muscle. There are several different types
of myosins in cells, of which the \textit{myosin-I} and \textit{myosin-II} subfamilies are
most abundant. Myosin-II is the major myosin found in muscle. Myosin-I
is found in all types of cells, and because it is simpler in structure and
mechanism of action we shall discuss it first.

Myosin-I molecules have only one head domain and a tail.
The head domain interacts with actin filaments and has an ATP-hydrolyzing
motor activity that enables it to move along the filament in
a cycle of binding, detachment, and rebinding. The tail varies
among the different types of myosin-I, and it determines what cell
components will be dragged along by the motor.

\subsection{extracellular signals control the arrangement of actin Filaments}

We have seen that myosin and other actin-binding proteins can regulate
the location, organization, and behavior of actin filaments. But the
activity of these accessory proteins is, in turn, controlled by extracellular
signals, allowing the cell to rearrange its cytoskeleton in response to the
environment.

For the actin cytoskeleton, such structural rearrangements are triggered
by activation of a variety of receptor proteins embedded in the plasma
membrane. All of these signals then seem to converge inside the cell on
a group of closely related GTP-binding proteins called the \textbf{Rho protein
family}. Proteins of this kind behave as molecular switches that control
cellular processes by cycling between an active, GTP-bound state and an
inactive, GDP-bound state. In the case of the cytoskeleton, activation of different members of the Rho
family affects the organization of actin filaments in different ways.

These dramatic and complex structural changes occur because the GTP-binding
proteins, together with the protein kinases and accessory proteins
with which they interact, act like a computational network to control
actin organization and dynamics. This network receives external signals
from nutrients, growth factors, and contacts with neighboring cells,
along with ‘inside information’ regarding the cell’s nutritional state, size,
and readiness for division. The Rho network then processes these inputs
and produces signals that shape the actin cytoskeleton.

One of the most tightly regulated rearrangements of cytoskeletal ele-
ments occurs when actin associates with myosin in muscle fibers in
response to signals from the nervous system. We now discuss how this
molecular interaction generates the rapid, repetitive, forceful movements
characteristic of the contraction of vertebrate muscles.

\section{Muscle contraction}

Muscle contraction is the most familiar and best understood of animal
cell movements. In vertebrates, running, walking, swimming, and flying
all depend on the ability of \textit{skeletal muscle} to contract strongly and move
various bones. Involuntary movements such as heart pumping and gut
peristalsis depend on \textit{cardiac muscle} and \textit{smooth muscle}, respectively,
which are formed from muscle cells that differ in structure from skeletal
muscle but use actin and myosin in a similar way to contract. Although
muscle cells are highly specialized, many cell movements - from the
locomotion of whole cells down to the motion of components inside
cells - depend on the interaction of actin and myosin.

\subsection{Muscle contraction depends on bundles of actin and myosin}

Muscle myosin belongs to the myosin-II subfamily of myosins, all of which
have two ATPase heads and a long, rodlike tail. Each
myosin-II molecule is a dimer composed of a pair of identical myosin
molecules held together by their tails; it has two globular ATPase heads
at one end and a single coiled-coil tail at the other. Clusters of myosin-II
molecules bind to each other through their coiled-coil tails, forming a
bipolar \textit{myosin filament} in which the heads project from the sides.

The myosin filament is like a double-headed arrow, with the two sets of
heads pointing in opposite directions away from the middle. One set of
heads binds to actin filaments in one orientation and moves them one
way; the other set of heads binds to other actin filaments in the opposite
orientation and moves them in the opposite direction. The
overall effect is to slide sets of oppositely oriented actin filaments past
one another. We can see how, therefore, if actin filaments and myosin
filaments are organized together in a bundle, the bundle can generate a
contractile force. This is seen most clearly in muscle contraction, but it
also occurs in the contractile bundles of actin filaments and myosin-II filaments
that assemble transiently in nonmuscle cells,
and in the contractile ring that pinches a dividing cell in two by contracting
and pulling inward on the plasma membrane.

\subsection{During muscle contraction actin Filaments slide against myosin Filaments}

The long fibers of skeletal muscle are huge single cells formed by the
fusion of many separate smaller cells. The individual nuclei of the contributing
cells are retained in the muscle fiber and lie just beneath the
plasma membrane. The bulk of the cytoplasm is made up of \textbf{myofibrils},
the contractile elements of the muscle cell.

A myofibril consists of a chain of identical tiny contractile units, or \textbf{sarcomeres}.
Each sarcomere is about 2.5 $\mu$m long, and the repeating pattern
of sarcomeres gives the vertebrate myofibril a striped, or striated, appearance.
Sarcomeres are highly organized assemblies of
two types of filaments - actin filaments and filaments of muscle-specific
myosin-II. Myosin filaments (the \textit{thick filaments}) are centrally positioned
in each sarcomere, whereas the more slender actin filaments (the \textit{thin
filaments}) extend inward from each end of the sarcomere (where they
are anchored by their plus ends to a structure known as the \textit{Z disc}) and
overlap the ends of the myosin filaments.

The contraction of a muscle cell is caused by a simultaneous shortening
of all the sarcomeres, which in turn is caused by the actin filaments sliding
past the myosin filaments, with no change in the length of either type
of filament The sliding motion is generated by myosin
heads that project from the sides of the myosin filament and interact with
adjacent actin filaments. When a muscle is stimulated to contract, the
myosin heads start to walk along the actin filament in repeated cycles
of attachment and detachment. During each cycle, a myosin head binds
and hydrolyzes one molecule of ATP. This causes a series of conformational
changes in the myosin molecule that move the tip of the head by
about 5 nm along the actin filament toward the plus end. This movement,
repeated with each round of ATP hydrolysis, propels the myosin molecule
unidirectionally along the actin filament. In so doing, the
myosin heads pull against the actin filament, causing it to slide against
the myosin filament. The concerted action of many myosin heads pulling
the actin and myosin filaments past each other causes the sarcomere to
contract. After a contraction is completed, the myosin heads lose contact
with the actin filaments completely, and the muscle relaxes.

\subsection{Muscle contraction Is triggered by a sudden rise in $Ca^{2+}$}

The force-generating molecular interaction between myosin and actin filaments
place only when the skeletal muscle receives a signal from takes
the nervous system. The signal from a nerve terminal triggers an action
potential in the muscle cell plasma membrane.
This electrical excitation spreads in a matter of milliseconds into a series
of membranous tubes, called \textit{transverse} (or T) \textit{tubules}, that extend inward
from the plasma membrane around each myofibril. The electrical signal
is then relayed to the \textit{sarcoplasmic reticulum}, an adjacent sheath of
interconnected flattened vesicles that surrounds each myofibril like a net
stocking.

The sarcoplasmic reticulum is a specialized region of the endoplasmic
reticulum in muscle cells. It contains a very high concentration of $Ca^{2+}$,
and in response to the incoming electrical excitation, much of this $Ca^{2+}$
is released into the cytosol through ion channels that open in the sarcoplasmic
reticulum membrane in response to the change in voltage across
the plasma membrane. $Ca^{2+}$ is widely used as an intracellular signal to relay a message from the exterior
to the internal machinery of the cell. In muscle, the $Ca^{2+}$ interacts with a
molecular switch made of specialized accessory proteins closely associated
with the actin filaments. One of these proteins is
\textit{tropomyosin}, a rigid, rod-shaped molecule that binds in the groove of the
and actin helix, overlapping seven actin monomers, prevents the myosin
heads from associating with the actin filament. The other is \textit{troponin}, a
protein complex that includes a $Ca^{2+}$-sensitive protein associated with
the end of a tropomyosin molecule. When the level of $Ca^{2+}$ rises in the
cytosol, $Ca^{2+}$ binds to troponin and induces a change in its shape. This
in turn causes the tropomyosin molecules to shift their position slightly,
allowing myosin heads to bind to the actin filament and initiating contraction.

Because the signal from the plasma membrane is passed within milliseconds
(via the transverse tubules and sarcoplasmic reticulum) to every
sarcomere in the cell, all the myofibrils in the cell contract at the same
time. The increase in $Ca^{2+}$ in the cytosol ceases as soon as the nerve signal
stops, because the $Ca^{2+}$ is rapidly pumped back into the sarcoplasmic
reticulum by abundant $Ca^{2+}$ pumps in its membrane.
As soon as $Ca^{2+}$ concentrations have returned to their resting level,
troponin and tropomyosin molecules move back to their original positions,
where they block myosin binding and thus end contraction.

\subsection{Muscle cells Perform Highly specialized Functions in the body}

The highly specialized contractile machinery in muscle cells is thought
to have evolved from the simpler contractile bundles of myosin and actin
filaments found in all eucaryotic cells. The myosin-II in nonmuscle cells
is also activated by a rise in cytosolic $Ca^{2+}$, but the mechanism of activation
is quite different. An increase in $Ca^{2+}$ leads to the phosphorylation of
myosin-II, which alters the myosin conformation and enables it to interact
with actin. A similar activation mechanism operates in \textit{smooth muscle},
which lies in the walls of the stomach, intestine, uterus, and arteries, and
in many other structures in which slow and sustained contractions are
needed. Contractions produced by this second mode are slower because
time is needed for enzyme molecules to diffuse to the myosin heads
and carry out the phosphorylation or dephosphorylation. However, this
mechanism has the advantage that it is less specialized and can be driven
by a variety of incoming signals.

In addition to skeletal and smooth muscle, other forms of muscle each
perform a specific mechanical function in the body. Perhaps the most
familiar is the heart, or \textit{cardiac}, muscle that drives the circulation of
blood.

\section{Essential concepts}

\begin{itemize}
\item The cytoplasm of a eucaryotic cell is supported and spatially organized
by a cytoskeleton of intermediate filaments, microtubules, and
actin filaments.
\item Intermediate filaments are stable, ropelike polymers of fibrous proteins
that give cells mechanical strength. Some types underlie the
nuclear membrane to form the nuclear lamina; others are distributed
throughout the cytoplasm.
\item Microtubules are stiff, hollow tubes formed by the polymerization of
tubulin dimer subunits. They are polarized structures with a slow-growing
“minus” end and a fast-growing “plus” end.
\item Microtubules are nucleated in, and grow out from, organizing centers
such as the centrosome. The minus ends of the microtubules are
embedded in the organizing center.
\item Many of the microtubules in a cell are in a labile, dynamic state in
which they alternate between a growing state and a shrinking state.
These transitions, known as dynamic instability, are controlled by the
hydrolysis of GTP bound to tubulin dimers.
\item Each tubulin dimer has a tightly bound GTP molecule that is hydrolyzed
to GDP after the tubulin has assembled into a microtubule. GTP
hydrolysis reduces the affinity of the subunit for its neighbors and
decreases the stability of the polymer, causing it to disassemble.
\item Microtubules can be stabilized by proteins that capture the plus
end - a process that influences the position of microtubule arrays in
a cell. Cells contain many microtubule-associated proteins that stabilize
microtubules, bind them to other cell components, and harness
them for specific functions.
\item Kinesins and dyneins are motor proteins that use the energy of ATP
hydrolysis to move unidirectionally along microtubules. They carry
specific membrane vesicles and other cargoes and in this way help to
maintain the spatial organization of the cytoplasm.
\item Eucaryotic cilia and flagella contain a bundle of stable microtubules.
Their beating is caused by bending of the microtubules, driven by a
motor protein called ciliary dynein.
\item Actin filaments are helical polymers of actin molecules. They are
more flexible than microtubules and are generally found in bundles
or networks.
\item Actin filaments are polarized structures with a fast- and a slow-growing
end, and their assembly and disassembly are controlled by the
hydrolysis of ATP tightly bound to each actin monomer.
\item The varied forms and functions of actin filaments in cells depend on
multiple actin-binding proteins. These control the polymerization of
actin filaments, cross-link the filaments into loose networks or stiff
bundles, attach them to membranes, or move them relative to one
another.
\item A concentrated network of actin filaments underneath the plasma
membrane forms the cell cortex and is responsible for the shape and
movement of the cell surface, including the movements involved
when a cell crawls along a surface.
\item Myosins are motor proteins that use the energy of ATP hydrolysis to
move along actin filaments: they can carry organelles along actin-filament
tracks or cause adjacent actin filaments to slide past each
other in contractile bundles.
\item In muscle, huge regular arrays of overlapping actin filaments and
myosin filaments generate contractions by sliding over one another.
\item Muscle contraction is initiated by a sudden rise in cytosolic $Ca^{2+}$,
which delivers a signal to the contractile apparatus via $Ca^{2+}$-binding
proteins.
\end{itemize}

\chapter{The Cell Division Cycle}

A cell reproduces by carrying out an orderly sequence of events in which
it duplicates its contents and then divides in two. This cycle of duplication
and division, known as the \textbf{cell cycle}, is the essential mechanism by
which all living things reproduce. To produce two
genetically identical daughter cells, the DNA in each chromosome must
be faithfully replicated, and the replicated chromosomes must then be
accurately distributed, or \textit{segregated}, into the two daughter cells, so that
each cell receives a complete copy of the entire genome.

\section{Overview of the cell cycle}

The most basic function of the cell cycle is to duplicate accurately the
vast amount of DNA in the chromosomes and then distribute the copies
into genetically identical daughter cells. The duration of the cell cycle
varies greatly from one cell type to another.

\subsection{The eucaryotic cell cycle is divided into four phases}

Seen under a microscope, the two most dramatic events in the cycle are
when the nucleus divides, a process called \textit{mitosis}, and when the cell later
splits in two, a process called \textit{cytokinesis}. These two processes together
constitute the \textbf{M phase} of the cell cycle. In a typical mammalian cell, the
whole of M phase takes about an hour, which is only a small fraction of
the total cell-cycle time.

The period between one M phase and the next is called \textbf{interphase}. Under
the microscope, it appears, deceptively, as an uneventful interlude during
which the cell simply increases in size. Interphase, however, is a very
busy time for the cell, and it encompasses the remaining three phases
of the cell cycle. During \textbf{S phase} (S = synthesis), the cell replicates its
nuclear DNA, an essential prerequisite for cell division. S phase is flanked
by two phases in which the cell continues to grow. The \textbf{$G_1$ phase} (G =
gap) is the interval between the completion of M phase and the beginning
of S phase. The \textbf{$G_2$ phase} is the interval between the end of S phase and
the beginning of M phase. During these gap phases, the cell
monitors the internal and external environments to ensure that conditions
are suitable and its preparations are complete before it commits
itself to the major upheavals of S phase and mitosis. At particular points
in $G_1$ and $G_2$ , the cell decides whether to proceed to the next phase or
pause to allow more time to prepare.

During all of interphase, a cell generally continues to transcribe genes,
synthesize proteins, and grow in mass. Together, $G_1$ and $G_2$ phases provide
additional time for the cell to grow and duplicate its cytoplasmic
organelles: if interphase lasted only long enough for DNA replication, the
cell would not have time to double its mass before it divided and would
consequently shrink with each division. Indeed, in some special circumstances
that is just what happens. In some animal embryos, for example,
the first cell divisions after fertilization (called \textit{cleavage divisions}) serve to
subdivide a giant egg cell into many smaller cells as quickly as possible.

Following DNA replication in S phase, the two copies of each chromosome
remain tightly bound together. The first visible sign that a cell is
about to enter M phase is the progressive \textit{condensation} of its chromosomes.
As condensation proceeds, the replicated chromosomes first
become visible in the light microscope as long threads, which gradually
get shorter and thicker. This condensation makes the chromosomes less
likely to get entangled, so that they are easier to segregate to the two
forming daughter cells during mitosis.

\subsection{A cell-cycle control system triggers the major processes of the cell cycle}

To ensure that they replicate all their DNA and organelles, and divide
in an orderly manner, eucaryotic cells possess a complex network of
regulatory proteins known as the \textit{cell-cycle control system}. This system
guarantees that the events of the cell cycle - DNA replication, mitosis,
and so on - occur in a set sequence and that each process has been completed
before the next one begins. To accomplish this, the control system
is itself regulated at certain critical points of the cycle by feedback from
the process being performed. Without such feedback, an interruption or
a delay in any of the processes could be disastrous. The cell-cycle
control system achieves all of this by means of molecular brakes that
can stop the cycle at various \textbf{checkpoints}. In this way, the control system
does not trigger the next step in the cycle unless the cell is properly
prepared.

One checkpoint operates in $G_1$ and allows
the cell to confirm that the environment is favorable for cell proliferation
before committing to S phase. Cell proliferation in animals requires
both sufficient nutrients and specific signal molecules in the extracellular
environment; if extracellular conditions are unfavorable, cells can delay
progress through $G_1$ and may even enter a specialized resting state known
as $G_0$ (G zero). Another checkpoint operates in $G_2$ and ensures that
cells do not enter mitosis until damaged DNA
has been repaired and DNA replication is complete. A third checkpoint
operates during mitosis and ensures that the replicated chromosomes
are properly attached to a cytoskeletal machine, called the \textit{mitotic spindle},
before the spindle pulls the chromosomes apart and distributes them into
the two daughter cells.

The checkpoint in $G_1$ is especially important as a point in the cell cycle
where the control system can be regulated by signals from other cells.

\subsection{Cell-cycle control is similar in all eucaryotes}

Some features of the cell cycle, including the time required to complete
certain events, vary greatly from one cell type to another, even within the
same organism. The basic organization of the cycle, however, is essentially
the same in all eucaryotic cells, and all eucaryotes appear to use
similar machinery and control mechanisms to drive and regulate cell-cycle events.

\section{The cell-cycle control system}

Two types of machinery are involved in cell division: one manufactures
the new components of the growing cell, and another hauls the components
into their correct places and partitions them appropriately when
the cell divides in two. The \textbf{cell-cycle control system} switches all this
machinery on and off at the correct times and thereby coordinates the
various steps of the cycle. The core of the cell-cycle control system is a
series of biochemical switches that operate in a defined sequence and
orchestrate the main events of the cycle, including DNA replication and
segregation of the duplicated chromosomes.

\subsection{The cell-cycle control system depends on cyclically activated protein kinases called Cdks}

The cell-cycle control system governs the cell-cycle machinery by cyclically
activating and then inactivating the key proteins and protein complexes
that initiate or regulate DNA replication, mitosis, and cytokinesis.
The phosphorylation reactions that control
the cell cycle are carried out by a specific set of protein kinases, while
dephosphorylation is performed by a set of protein phosphatases.

Switching these kinases on and off at the appropriate times is partly
the responsibility of another set of proteins in the control system - the
\textbf{cyclins}. Cyclins have no enzymatic activity themselves, but they have to
bind to the cell-cycle kinases before the kinases can become enzymatically
active. The kinases of the cell-cycle control system are therefore
known as \textbf{cyclin-dependent protein kinases}, or Cdks.
Cyclins are so-named because, unlike the Cdks, their concentrations
vary in a cyclical fashion during the cell cycle. The cyclical changes in
cyclin concentrations help drive the cyclic assembly and activation of
the cyclin-Cdk complexes; activation of these complexes in turn triggers
various cell-cycle events, such as entry into S phase or M phase.

\subsection{The activity of cdks is also regulated by phosphorylation and dephosphorylation}

For a cyclin-Cdk to be maximally active, the Cdk has to be
phosphorylated at one site by a specific protein kinase and dephosphorylated
at other sites by a specific protein phosphatase.

\subsection{Different cyclin-Cdk complexes trigger different steps in the cell cycle}

There are several types of cyclins and, in most eucaryotes, several types
of Cdks involved in cell-cycle control. Different cyclin-Cdk complexes
trigger different steps of the cell cycle. The cyclin that acts in $G_2$ to trigger
entry into M phase is called M cyclin, and the active complex it forms
with its Cdk is called M-Cdk. Distinct cyclins, called S cyclins and $G_1$/S
cyclins, bind to a distinct Cdk protein late in $G_1$ to form S-Cdk and $G_1$/S-Cdk,
respectively, and trigger S phase. Other cyclins, called $G_1$ cyclins, act earlier
in $G_1$ and bind to other Cdk proteins to form $G_1$-Cdks, which help drive
the cell through $G_1$ toward S phase.

As previously explained, the different Cdks also have to be phosphory-
lated and dephosphorylated in order to act. Each of
these activated cyclin-Cdk complexes in turn phosphorylates a different
set of target proteins in the cell. As a result, each type of complex triggers
a different transition step in the cycle.

\subsection{The cell-cycle control system also depends on cyclical proteolysis}

The concentration of each type of cyclin rises gradually but then falls
sharply at a specific time in the cell cycle. This abrupt
fall results from the targeted degradation of the cyclin protein. Specific
enzyme complexes add ubiquitin chains to the appropriate cyclin, which
is then directed to the proteasome for destruction. This
rapid elimination of the cyclin returns the Cdk to its inactive state.

\subsection{Proteins that inhibit cdks can arrest the cell cycle at specific checkpoints}

Some of the previous cited molecular brakes rely on \textbf{Cdk inhibitor proteins} that
block the assembly or activity of one or more cyclin-Cdk complexes.
Certain Cdk inhibitor proteins, for example, help maintain Cdks in an
inactive state during the $G_1$ phase of the cycle, thus delaying progression
into S phase. Pausing at this checkpoint gives the cell more time to
grow, or allows it to wait until extracellular conditions are favorable for
division. As a general rule, mammalian cells will multiply only if they are
stimulated to do so by extracellular signals called \textit{mitogens} produced by
other cells. If deprived of such signals, the cell cycle arrests at a \textit{$G_1$
check-point}; and, if the cell is deprived for long enough, it will
withdraw from the cell cycle and enter the non-proliferating state $G_0$, in
which the cell can remain for days or weeks or even for the lifetime of the
organism.

Most of the diversity in cell-division rates in the adult body lies in the variation
in the time that cells spend in $G_0$ or in $G_1$. Many of our cells fall somewhere in between: they
can divide if the need arises but normally do so infrequently. Escape from
the $G_1$ checkpoint or from $G_0$ requires the accumulation of $G_1$ cyclins, and
mitogens function by stimulating this accumulation.

Once past the $G_1$ checkpoint, a cell usually proceeds all the way through
the rest of the cell cycle quickly - typically within 12-24 hours in mammals.
The $G_1$ checkpoint is therefore sometimes called \textit{Start}, because
passing it represents a commitment to complete a full division cycle,
although a better name might be Stop.

The most radical decision that the cell-cycle control system can make is
to withdraw the cell from the cell cycle permanently. This is different from
withdrawing from the cell cycle temporarily, to wait for more favorable
conditions, and it has a special importance in multicellular organisms.
They enter an irreversible $G_0$ state, in which the cell-cycle control system is largely dismantled:
many of the Cdks and cyclins disappear, and the cyclin-Cdk complexes
that are still present are inhibited by Cdk inhibitor proteins.

\section{S phase}

Before a cell divides, it must duplicate its DNA.
This replication must occur with extreme accuracy to minimize the risk of
mutations in the next cell generation. Of equal importance, every nucleotide
in the genome must be copied once - and only once - to prevent
the damaging effects of gene amplification.

\subsection{S-Cdk initiates dna replication and helps block re-replication}

DNA replication begins at origins of replication, nucleotide sequences that are scattered along each chromosome.
These sequences recruit specific proteins that control the initiation and
completion of DNA replication. One multiprotein complex, the \textbf{origin
recognition complex} (ORC), remains bound to origins of replication
throughout the cell cycle, where it serves as a sort of landing pad for
additional regulatory proteins that bind before the start of S phase.

One of these regulatory proteins, called Cdc6, is present at low levels
during most of the cell cycle, but its concentration increases transiently
in early $G_1$. When Cdc6 binds to ORCs in $G_1$, it promotes the binding of
additional proteins to form a \textit{pre-replicative complex}. Once the pre-replicative
complex has been assembled, the replication origin is ready to
`fire'. The activation of S-Cdk in late $G_1$ then `pulls the trigger,' initiating
DNA replication.

S-Cdk does not only initiate origin firing; it
also helps prevent re-replication of the DNA. Activated S-Cdk helps phosphorylate
Cdc6, causing it and the other proteins in the pre-replicative
complex to dissociate from the ORC after an origin has fired. This disassembly
prevents replication from occurring again at the same origin. In
addition to promoting dissociation, phosphorylation of Cdc6 by S-Cdk
(and by M-Cdk, which becomes active at the start of M phase) marks it
for degradation, ensuring that DNA replication is not reinitiated later in
the same cell cycle.

\subsection{Cohesins help hold the sister chromatids of each replicated chromosome together}

After the chromosomes have been duplicated in S phase, the two copies
of each replicated chromosome remain tightly bound together as
identical \textbf{sister chromatids}. The sister chromatids are held together by
protein complexes called \textbf{cohesins}, which assemble along the length of
each sister chromatid as the DNA is replicated in S phase. The cohesins
form protein rings that surround the two sister chromatids, keeping them
united. This cohesion between sister chromatids is crucial
for proper chromosome segregation, and it is broken completely only
in late mitosis to allow the sister chromatids to be pulled apart by the
mitotic spindle.

\subsection{DNA damage checkpoints help prevent the replication of damaged DNA}

The cell-cycle control system uses several distinct checkpoint mecha-
nisms to halt progress through the cell cycle if DNA is damaged. DNA
damage checkpoints in $G_1$ and S phase prevent the cell from starting or
completing S phase and replicating damaged DNA. Another checkpoint
operates in $G_2$ to prevent the cell from entering M phase with damaged
or incompletely replicated DNA.

The $G_1$ checkpoint mechanism is especially well understood. DNA damage
causes an increase in both the concentration and activity of a protein
called \textbf{p53}, which is a transcription regulator that activates the transcription
of a gene encoding a Cdk inhibitor protein called p21. The p21
protein binds to $G_1$/S-Cdk and S-Cdk, preventing them from driving the
cell into S phase. The arrest of the cell cycle in $G_1$ gives
the cell time to repair the damaged DNA before replicating it. If the DNA
damage is too severe to be repaired, p53 can induce the cell to kill itself
by undergoing apoptosis. If p53 is missing or defective, the unrestrained
replication of damaged DNA leads to a high rate of mutation and the production
of cells that tend to become cancerous.

Once DNA replication has begun, another type of checkpoint mechanism
operates to prevent a cell entering M phase with damaged or
incompletely replicated DNA.

The activity of cyclin-Cdk complexes is inhibited by phosphorylation at particular sites.
For the cell to progress into mitosis, M-Cdk has to be activated by the
removal of these inhibitory phosphates by a specific protein phosphatase.
When DNA is damaged (or incompletely replicated), this activating protein
phosphatase is itself inhibited, so the inhibitory phosphates are not
removed from M-Cdk. As a result, M-Cdk remains inactive and M phase
cannot be initiated until DNA replication is complete and any DNA damage
is repaired.

Once a cell has passed through these checkpoints and has successfully
replicated its DNA in S phase and progressed through $G_2$, it is ready to
enter M phase, in which it divides its nucleus (the process of mitosis) and
then its cytoplasm.

\section{M phase}

Although M phase (mitosis plus cytokinesis) occurs over a relatively short
amount of time, it is by far the most dramatic phase of the
cell cycle. During this brief period, the cell reorganizes virtually all of its
components and distributes them equally into the two daughter cells.
The earlier phases of the cell cycle, in effect, serve to set the stage for the
drama of M phase.

The central problem for a cell in M phase is to accurately segregate its
chromosomes, which were replicated in the preceding S phase, so that
each new daughter cell receives an identical copy of the genome. With
minor variations, all eucaryotes solve this problem in a similar way: they
assemble two specialized cytoskeletal machines, one that pulls the duplicated
chromosome sets apart (during mitosis) and another that divides
the cytoplasm in two halfs (cytokinesis).

\subsection{M-Cdk drives entry into M phase and mitosis}

One of the most remarkable features of cell-cycle control is that a single
protein complex, M-Cdk, brings about all of the diverse and intricate
rearrangements that occur in the early stages of mitosis. M-Cdk triggers
the condensation of the replicated chromosomes into compact, rod-like
structures, readying them for segregation, and it also induces the assembly
of the mitotic spindle that will separate the condensed chromosomes
and segregate them into the two daughter cells.

As discussed earlier, M-Cdk activation begins with the accumulation of
M cyclin. Synthesis of M cyclin starts immediately after
S phase; its concentration then rises gradually and helps time the onset
of M phase. The increase in M cyclin protein leads to a corresponding
accumulation of M-Cdk complexes. But those complexes, when they first
form, are inactive. The sudden activation of the M-Cdk stockpile at the
end of $G_2$ is triggered by the activation of a protein phosphatase (Cdc25)
that removes the inhibitory phosphates holding M-Cdk activity in check.

Once activated, each M-Cdk complex can indirectly activate more M-Cdk,
by phosphorylating and activating more Cdc25. In addition, activated
M-Cdk also inhibits the inhibitory kinase Wee1, further promoting the activation of M-Cdk.
The overall consequence is that, once the activation of M-Cdk begins, there is
an explosive increase in M-Cdk activity that drives the cell abruptly from
$G_2$ into M phase.

\subsection{Condensins help configure duplicated chromosomes for separation}

When the cell is about to enter M phase, the replicated chromosomes
condense, becoming visible as threadlike structures. Protein complexes,
called \textbf{condensins}, help carry out this \textbf{chromosome condensation}. The
M-Cdk that initiates entry into M phase triggers the assembly of condensin
complexes onto DNA by phosphorylating some of the condensin
subunits. Condensation makes the mitotic chromosomes more compact,
reducing them to small physical packets that can be more easily segregated
within the crowded confines of the dividing cell.

Condensins are structurally related to cohesins - the proteins that hold
sister chromatids together. Both cohesins and condensins
form ring structures, and, together, the two types of protein
rings help to configure the replicated chromosomes for mitosis. Cohesins
assemble on the DNA as it replicates in S phase and tie together two
parallel DNA molecules - the identical sister chromatids. Condensins, by
contrast, assemble on each individual chromatid at the start of M phase
and coil up the DNA to help each chromatid condense.

\subsection{The cytoskeleton carries out both mitosis and cytokinesis}

After the replicated chromosomes have condensed, two complex cytoskeletal
machines assemble in sequence to carry out the two mechanical
processes that occur in M phase. The \textit{mitotic spindle} carries out nuclear
division (mitosis), and, in animal cells and many unicellular eucaryotes,
the \textit{contractile ring} carries out cytoplasmic division (cytokinesis).
Both structures rapidly disassemble after they have performed
their tasks.

The mitotic spindle is composed of microtubules and the various proteins
that interact with them, including microtubule-associated motor proteins.
In all eucaryotic cells, the mitotic spindle is
responsible for separating the replicated chromosomes and allocating
one copy of each chromosome to each daughter cell.

The contractile ring consists mainly of actin filaments and myosin filaments
arranged in a ring around the equator of the cell.
It starts to assemble just beneath the plasma membrane
toward the end of mitosis. As the ring contracts, it pulls the membrane
inward, thereby dividing the cell in two.

\subsection{M phase is conventionally divided into six stages}

Although M phase proceeds as a continuous sequence of events, it is
traditionally divided into six stages. The first five stages of M phase -
prophase, prometaphase, metaphase, anaphase, and telophase - constitute
\textbf{mitosis}, which was originally defined as the period in which the chromosomes
are visible (because they have become condensed). \textit{Cytokinesis}
constitutes the sixth stage, and it overlaps in time with the end of mitosis.
Together, they form a dynamic sequence in which many independent
cycles - involving the chromosomes, cytoskeleton, and centrosomes -
are coordinated to produce two genetically identical daughter cells.

The five stages of mitosis occur in strict sequential order, whereas cytokinesis
begins in anaphase and continues through telophase. During
\textit{prophase}, the replicated chromosomes condense and the mitotic spindle
begins to assemble outside the nucleus. During \textit{prometaphase}, the
nuclear envelope breaks down, allowing the spindle microtubules to bind
to the chromosomes. During \textit{metaphase}, the mitotic spindle gathers all of
the chromosomes to the center (equator) of the spindle. During \textit{anaphase},
the two sister chromatids in each replicated chromosome synchronously
split apart, and the spindle draws them to opposite poles of the cell.
During \textit{telophase}, a nuclear envelope reassembles around each of the
two sets of separated chromosomes to form two nuclei.
Cytokinesis is complete by the end of telophase, when the
nucleus and cytoplasm of each of the daughter cells return to interphase,
signaling the end of M phase.

\section{Mitosis}

Before nuclear division, or mitosis, begins, each chromosome has been
replicated and consists of two identical sister chromatids, held together
along their length by cohesin proteins. During mitosis, the cohesin
proteins are cleaved, the sister chromatids split apart,
and the resulting daughter chromosomes are pulled to opposite poles of
the cell by the mitotic spindle.

\subsection{Centrosomes duplicate to help form the two poles of the mitotic spindle}

Before M phase begins, two critical events must be completed: DNA must
be fully replicated, and, in animal cells, the centrosome must be duplicated.
The \textbf{centrosome} is the principal \textit{microtubule-organizing} center
in animal cells. It duplicates so that it can help form the two poles of
the mitotic spindle and so that each daughter cell can receive its own
centrosome.

Centrosome duplication begins at the start of S phase and is triggered
by the same Cdks ($G_1$/S-Cdk and S-Cdk) that trigger DNA replication.
Initially, when the centrosome duplicates, both copies remain together
as a single complex on one side of the nucleus. As mitosis begins, however,
the two centrosomes separate, and each nucleates a radial array of
microtubules called an \textbf{aster}. The two asters move to opposite sides of
the nucleus to form the two poles of the mitotic spindle.
The process of centrosome duplication and separation is known as the
\textbf{centrosome cycle}.

\subsection{The mitotic spindle starts to assemble in prophase}

The mitotic spindle begins to form in \textbf{prophase}. This assembly of the
highly dynamic spindle depends on the remarkable properties of microtubules.
Microtubules continuously polymerize
and depolymerize by the addition and loss of their tubulin subunits, and
individual filaments alternate between growing and shrinking - a process
called \textit{dynamic instability}. At the start of mitosis, the
dynamic instability of microtubules increases, in part because M-Cdk
phosphorylates microtubule-associated proteins that influence the stability
of microtubule filaments. As a result, during prophase, rapidly growing
and shrinking microtubules extend in all directions from the two centrosomes,
exploring the interior of the cell. Some of the microtubules
growing from one centrosome interact with the microtubules from the
other centrosome. This interaction stabilizes the microtubules, preventing
them from depolymerizing, and it joins the two sets of microtubules
together to form the basic framework of the \textbf{mitotic spindle}, with its
characteristic bipolar shape. The two centrosomes that give
rise to these microtubules are now called \textbf{spindle poles}, and the interacting
microtubules are called \textit{interpolar microtubules}. The
assembly of the spindle is driven, in part, by motor proteins associated
with the interpolar microtubules that help to cross-link the two sets of
microtubules.

In the next stage of mitosis, the replicated chromosomes attach to the
spindle in such a way that, when the sister chromatids separate, they will
be drawn to opposite poles of the cell.

\subsection{Chromosomes attach to the mitotic spindle at prometaphase}

Prometaphase starts abruptly with the disassembly of the nuclear envelope,
which breaks up into small membrane vesicles. This process is
triggered by the phosphorylation and consequent disassembly of nuclear
pore proteins and the intermediate filament proteins of the nuclear lamina,
the network of fibrous proteins that underlies and stabilizes the nuclear
envelope. The spindle microtubules, which have been
lying in wait outside the nucleus, now gain access to the replicated chro-
mosomes and capture them.

The spindle microtubules end up attached to the chromosomes through
specialized protein complexes called \textbf{kinetochores}, which assemble on
the condensed chromosomes during late prophase. As
discussed earlier, each replicated chromosome consists of two sister
chromatids joined along their length, and each chromatid is constricted
at a region of specialized DNA sequence called the \textit{centromere}.
Just before prometaphase, kinetochore proteins assemble into
a large complex on each centromere. Each duplicated chromosome
therefore has two kinetochores (one on each sister chromatid), which face
in opposite directions. Kinetochore assembly depends on the presence
of the centromere DNA sequence: in the absence of this sequence,
kinetochores fail to assemble and, consequently, the chromosomes fail
to segregate properly during mitosis.

Once the nuclear envelope has broken down, a randomly probing microtubule
encountering a chromosome will bind to it, thereby capturing the
chromosome. The microtubule eventually attaches to the kinetochore,
and this \textit{kinetochore microtubule} links the chromosome to a spindle
pole. Because kinetochores
on sister chromatids face in opposite directions, they tend to attach to
microtubules from opposite poles of the spindle, so that each replicated
chromosome becomes linked to both spindle poles. The attachment to
opposite poles, called \textbf{bi-orientation}, generates tension on the kinetochores,
which are being pulled in opposite directions. This tension signals
to the sister kinetochores that they are attached correctly, and are ready
to be separated. The cell-cycle control system monitors this tension to
ensure correct chromosome attachment, constituting another important
cell-cycle checkpoint.

\subsection{Chromosomes aid in the assembly of the mitotic spindle}

Chromosomes are more than passive passengers in the process of spindle
assembly: they can stabilize and organize microtubules into functional
mitotic spindles.

\subsection{Chromosomes line up at the spindle equator at metaphase}

During prometaphase, the chromosomes, now attached to the mitotic
spindle, begin to move about, as if jerked first this way and then that.
Eventually, they align at the equator of the spindle, halfway between the
two spindle poles, thereby forming the metaphase plate. This defines the
beginning of metaphase. Although the forces that act to
bring the chromosomes to the equator are not well understood, both the
continual growth and shrinkage of the microtubules and the action of
microtubule motor proteins are thought to be involved. A continuous balanced
addition and loss of tubulin subunits is also required to maintain
the metaphase spindle.

The chromosomes gathered at the equator of the metaphase spindle
oscillate back and forth, continually adjusting their positions, indicat-
ing that the tug-of-war between the microtubules attached to opposite
poles of the spindle continues to operate after the chromosomes are all
aligned.

\subsection{Proteolysis triggers sister-chromatid separation and the completion of mitosis}

\textbf{Anaphase} begins abruptly with the release of the cohesin linkage that
holds the sister chromatids together. This allows each
chromatid to be pulled to the spindle pole to which it is attached.
This movement segregates the two identical sets of chromosomes
to opposite ends of the spindle.

The cohesin linkage is destroyed by a protease called \textit{separase}, which up
to the beginning of anaphase is held in an inactive state by binding to an
inhibitory protein called \textit{securin}. At the beginning of anaphase, securin
is targeted for destruction by a protein complex called the \textbf{anaphase-promoting complex}
(APC). Once securin has been removed, separase is
then free to break the cohesin linkages.

The APC not only triggers the degradation of cohesins, but also targets
M cyclin for destruction, thus rendering the M-Cdk complex inactive. This
rapid inactivation of M-Cdk helps to initiate the exit from mitosis.

\subsection{Chromosomes segregate during anaphase}

Once the sister chromatids separate, they are pulled to the spindle pole to
which they are attached. They all move at the same speed.
The movement is the consequence of two
independent processes that involve different parts of the mitotic spindle.
The two processes are called \textit{anaphase A} and \textit{anaphase B}, and they occur
more or less simultaneously. In anaphase A, the kinetochore microtubules
shorten by depolymerization, and the attached chromosomes
move poleward. In anaphase B, the spindle poles themselves move apart,
further contributing to the segregation of the two sets of chromosomes.

The driving force for the movements of anaphase A is thought to be provided
mainly by the microtubule-associated motor proteins operating at
the kinetochore, aided by the shortening of kinetochore microtubules.
The loss of tubulin subunits from the kinetochore microtubules depends
on a motor-like protein that is bound to both the microtubule and the
kinetochore and uses the energy of ATP hydrolysis to remove tubulin subunits
from the microtubule.

In anaphase B, the spindle poles and the two sets of chromosomes move
farther apart. The driving forces for this movement are thought to be provided
by two sets of motor proteins—members of the kinesin and dynein
families - operating on different types of spindle microtubules.
One set of motor proteins acts on the long, overlapping
interpolar microtubules that form the spindle itself; these motor proteins
slide the interpolar microtubules from opposite poles past one another at
the equator of the spindle, pushing the spindle poles apart. The other set
operates on the astral microtubules that extend from the spindle poles
and point away from the spindle equator and toward the cell periphery.
These motor proteins are thought to be associated with the cell cortex,
which underlies the plasma membrane, and they pull each pole toward
the adjacent cortex and away from the other pole.

\subsection{Unattached chromosomes block sister-chromatid separation}

If a dividing cell were to begin to segregate its chromosomes before all
the chromosomes were properly attached to the spindle, one daughter
would receive an incomplete set of chromosomes, while the other
daughter would receive a surplus. Both situations could be lethal for the
cell. Thus, a dividing cell must ensure that every last chromosome is
attached properly to the spindle before it completes mitosis. To monitor
chromosome attachment, the cell makes use of a negative signal: unattached
chromosomes send a `stop' signal to the cell-cycle control system.
Although the exact nature of the signal remains elusive, we know that it
inhibits further progress through mitosis by blocking the activation of the
APC. Without active APC, the sister chromatids remain glued together.
Thus, none of the duplicated chromosomes can be pulled apart until
every chromosome has been positioned correctly on the mitotic spindle.
This so-called \textit{spindle assembly checkpoint} controls exit from mitosis.

\subsection{The nuclear envelope re-forms at telophase}

By the end of anaphase, the daughter chromosomes have separated into
two equal groups, one at each pole of the spindle. During telophase, the
final stage of mitosis, the mitotic spindle disassembles, and a nuclear
envelope reassembles around each group of chromosomes to form the
two daughter nuclei. Vesicles of nuclear membrane first cluster around
individual chromosomes and then fuse to re-form the nuclear envelope.
During this process, the nuclear pore proteins
and nuclear lamins that were phosphorylated during prometaphase are
now dephosphorylated, which allows them to re-assemble and form the
nuclear envelope and nuclear lamina, respectively. Once
the nuclear envelope has re-formed, the pores pump in nuclear proteins,
the nucleus expands, and the condensed mitotic chromosomes decondense
into their interphase state. As a consequence of decondensation,
gene transcription is able to resume. A new nucleus has been created,
and mitosis is complete. All that remains is for the cell to complete its
division into two separate daughter cells.

\section{Cytokinesis}

\textbf{Cytokinesis}, the process by which the cytoplasm is cleaved in two,
completes M phase. It usually begins in anaphase but is not completed
until the two daughter nuclei have formed in telophase. Whereas mitosis
depends on a transient microtubule-based structure, the mitotic spindle,
cytokinesis in animal cells depends on a transient structure based on
actin and myosin filaments, the \textit{contractile ring}. Both
the plane of cleavage and the timing of cytokinesis, however, are determined
by the mitotic spindle.

\subsection{The mitotic spindle determines the plane of cytoplasmic cleavage}

The first visible sign of cytokinesis in animal cells is a puckering and furrowing
of the plasma membrane that occurs during anaphase.
The furrowing invariably occurs in a plane that runs perpendicular
to the long axis of the mitotic spindle. This positioning ensures that
the cleavage furrow cuts between the two groups of segregated chromosomes
so that each daughter cell receives an identical and complete set of
chromosomes. If the mitotic spindle is deliberately displaced (using a fine
glass needle) as soon as the furrow appears, the furrow disappears and
a new one develops at a site corresponding to the new spindle location
and orientation. Once the furrowing process is well under way, however,
cleavage proceeds even if the mitotic spindle is artificially sucked out
of the cell or depolymerized using the drug colchicine. How the mitotic
spindle dictates the position of the cleavage furrow is still uncertain,
but it seems that, during anaphase, both the astral microtubules and
the interpolar microtubules (and their associated proteins) signal to the
cell cortex to initiate the assembly of the contractile ring at a position
midway between the spindle poles. Because these signals originate in
the anaphase spindle, this mechanism also contributes to the timing of
cytokinesis in late mitosis.

When the mitotic spindle is located centrally in the cell - the usual situation
in most dividing cells - the two daughter cells produced will be of
equal size. During embryonic development, however, there are some
instances in which the dividing cell moves its mitotic spindle to an asymmetrical
position, and, consequently, the furrow creates two daughter
cells that differ in size. In most cases, the daughters also differ in the
molecules they inherit, and they usually develop into different cell types.
Special mechanisms are required to position the mitotic spindle eccentrically
in such \textit{asymmetric divisions}.

\subsection{The contractile ring of animal cells is made of actin and myosin}

The contractile ring is composed mainly of an overlapping array of actin
filaments and myosin filaments. It assembles at anaphase
and is attached to membrane-associated proteins on the cytoplasmic face
of the plasma membrane. Once assembled, the contractile ring is capable
of exerting a force strong enough to bend a fine glass needle inserted into
the cell before cytokinesis. The sliding of the actin filaments against the
myosin filaments generates the force, much as it does
during muscle contraction. Unlike the contractile apparatus in muscle,
however, the contractile ring is a transient structure: it assembles to carry
out cytokinesis, gradually becomes smaller as cytokinesis progresses,
and disassembles completely once the cell has been cleaved in two.

Cell division in many animal cells is accompanied by large changes in
cell shape and a decrease in the adherence of the cell to its neighbors, to
the extracellular matrix, or to both. These changes result, in part, from
the reorganization of actin and myosin filaments in the cell cortex, only
one aspect of which is the assembly of the contractile ring. Mammalian
fibroblasts in culture, for example, spread out flat during interphase, as
a result of the strong adhesive contacts they make with the surface they
are growing on - called the substratum. As the cells enter M phase, however,
they round up. The cells change shape in part because some of
the plasma membrane proteins responsible for attaching the cells to the
substratum - the \textit{integrins} - become phosphorylated
and thus weaken their grip. Once cytokinesis is complete, the
daughter cells reestablish their strong contacts with the substratum and
flatten out again. When cells divide in an animal tissue,
this cycle of attachment and detachment presumably allows the cells to
rearrange their contacts with neighboring cells and with the extracellular
matrix, so that the new cells produced by cell division can be accommodated
within the tissue.

\subsection{Cytokinesis in plant cells involves the formation of a new cell wall}

The two daughter cells are
separated not by the action of a contractile ring at the cell surface but
instead by the construction of a new wall that forms inside the dividing
cell. The positioning of this new wall precisely determines the position of
the two daughter cells relative to neighboring cells. Thus, the planes of
cell division, together with cell enlargement, determine the final form of
the plant.

The new cell wall starts to assemble in the cytoplasm between the two
sets of segregated chromosomes at the start of telophase. The assembly
process is guided by a structure called the \textbf{phragmoplast}, which is formed
by the remains of the interpolar microtubules at the equator of the old
mitotic spindle. Small membrane-enclosed vesicles, largely derived from
the Golgi apparatus and filled with polysaccharides and glycoproteins
required for the cell-wall matrix, are transported along the microtubules
to the phragmoplast. Here, they fuse to form a disclike, membrane-enclosed
structure, which expands outward by further vesicle fusion
until it reaches the plasma membrane and original cell wall and divides
the cell in two. Later, cellulose microfibrils are laid down
within the matrix to complete the construction of the new cell wall.

\subsection{Membrane-enclosed organelles must be distributed to daughter cells when a cell divides}

Organelles such as mitochondria and chloroplasts are usually present in
large numbers and will be safely inherited if, on average, their numbers
simply double once each cell cycle. The ER in interphase cells is continuous
with the nuclear membrane and is organized by the microtubule
cytoskeleton. Upon entry into M phase, the reorganization
of the microtubules releases the ER; in most cells, the released
ER remains intact during mitosis and is cut in two during cytokinesis.
The Golgi apparatus fragments during mitosis; the fragments associate
with the spindle microtubules via motor proteins, thereby hitching a ride
into the daughter cells as the spindle elongates in anaphase. Other components
of the cell, including all of the soluble proteins, are inherited
randomly when the cell divides.

\section{Control of cell number and size}

Three fundamental processes largely determine organ and body size:
cell growth, cell division, and cell death. Each of these processes, in turn,
depends on programs intrinsic to the individual cell and is regulated by
signals from other cells in the body.

\subsection{Apoptosis helps regulate animal cell numbers}

The cells of a multicellular organism are members of a highly organized
community. The number of cells in this community is tightly regulated -
not simply by controlling the rate of cell division, but also by controlling
the rate of cell death. If cells are no longer needed, they can commit
suicide by activating an intracellular death program - a process called
\textbf{programmed cell death}. In animals, by far the most common form of
programmed cell death is called \textbf{apoptosis} (from a Greek word meaning
`falling off,' as leaves fall from a tree).

In adult tissues, cell death usually exactly balances cell division, unless
the tissue is growing or shrinking.

\subsection{Apoptosis is mediated by an intracellular proteolytic cascade}

Cells that die as a result of acute injury typically swell and burst, spilling
their contents all over their neighbors, a process called cell \textit{necrosis}.
This eruption triggers a potentially damaging inflammatory response.
By contrast, a cell that undergoes apoptosis dies neatly,
without damaging its neighbors. A cell in the throes of apoptosis shrinks
and condenses. The cytoskeleton collapses, the nuclear
envelope disassembles, and the nuclear DNA breaks up into fragments.
Most importantly, the cell surface is altered in such a manner
that it immediately attracts phagocytic cells, usually specialized
phagocytic cells called macrophages. These cells
engulf the apoptotic cell before it spills its contents. This
rapid removal of the dying cell avoids the damaging consequences of cell
necrosis, and also allows the organic components of the apoptotic cell to
be recycled by the cell that ingests it.

The machinery that is responsible for apoptosis seems to be similar in all
animal cells. It involves the \textbf{caspase} family of proteases, the members of
which are made as inactive precursors called \textit{procaspases}. Procaspases
are typically activated by proteolytic cleavage in response to signals that
induce apoptosis. The activated caspases cleave, and thereby activate,
other members of the procaspase family, resulting in an amplifying proteolytic
cascade. They also cleave other key proteins in the
cell.

Activation of the apoptotic program, like entry into a new stage of the cell
cycle, is usually triggered in an all-or-none fashion. The proteolytic cascade
is not only destructive and self-amplifying but also irreversible; once
a cell reaches a critical point along the path to destruction, it cannot turn
back. Thus, it is important that the decision to die is tightly controlled.

\subsection{The death program is regulated by the Bcl2 family of intracellular proteins}

The main proteins that regulate the activation of procaspases are members
of the \textbf{Bcl2 family} of intracellular proteins. Some members of this
protein family promote procaspase activation and cell death, whereas
others inhibit these processes. Two of the most important death-promoting
family members are proteins called \textit{Bax} and \textit{Bak}. These proteins
activate procaspases indirectly, by inducing the release of cytochrome c
from mitochondria into the cytosol. Cytochrome c promotes the assembly
of a large, seven-armed pinwheel-like structure that recruits specific procaspase
molecules, forming a protein complex called an apoptosome. The
procaspase molecules become activated within the apoptosome, triggering
a caspase cascade that leads to apoptosis. Bax and Bak
proteins are themselves activated by other death-promoting members of
the Bcl2 family, which are produced or activated by various insults to the
cell, such as DNA damage.

Other members of the Bcl2 family, including Bcl2 itself, act to inhibit,
rather than promote, procaspase activation and apoptosis. One way in
which they do so is by blocking the ability of Bax and Bak to release
cytochrome c from mitochondria. Some of the Bcl2 family members that
promote apoptosis, including a protein called Bad, do so by binding to
and blocking the activity of Bcl2 and other death-suppressing members
of the Bcl2 family. The balance between the activities
of pro-apoptotic and anti-apoptotic members of the Bcl2 family largely
determines whether a mammalian cell lives or dies by apoptosis.

The intracellular death program is also regulated by signals from other
cells, which can either activate or suppress the program. Indeed, cell
survival, cell division, and cell growth are all regulated by extracellular
signals, which together help multicellular organisms control cell number
and cell size, as we now discuss.

\subsection{Animal cells require extracellular signals to survive, Grow, and divide}

Most of the extracellular signal molecules that influence cell survival, cell
growth, and cell division are either soluble proteins secreted by other
cells or proteins bound to the surface of other cells or the extracellular
matrix. Although most act positively to stimulate one or more of these
cell processes, some act negatively to inhibit a particular process. The
positively acting signal proteins can be classified, on the basis of their
function, into three major categories:

\begin{enumerate}
\item \textbf{Survival factors} promote cell survival, largely by suppressing
apoptosis.
\item \textbf{Mitogens} stimulate cell division, primarily by overcoming the
intracellular braking mechanisms that tend to block progression
through the cell cycle.
\item \textbf{Growth factors} stimulate cell growth (an increase in cell size and
mass) by promoting the synthesis and inhibiting the degradation of
proteins and other macromolecules.
\end{enumerate}

These categories are not mutually exclusive, as many signal molecules
have more than one of these functions.

\subsection{Animal cells require survival factors to avoid apoptosis}

Animal cells need signals from other cells to survive. If deprived of such
survival factors, cells activate their intracellular suicide program and die
by apoptosis. This requirement for signals from other cells helps to ensure
that cells survive only when and where they are needed.

Survival factors usually act by binding to cell-surface receptors. These
activated receptors then turn on intracellular signaling pathways that
keep the death program suppressed, usually by regulating members of the
Bcl2 family of proteins. Some survival factors, for example, increase the
production of Bcl2, a protein that suppresses apoptosis.

\subsection{Mitogens stimulate cell division}

Most mitogens are secreted signal proteins that bind to cell-surface
receptors. When activated by mitogen binding, these receptors initiate
various intracellular signaling pathways that
stimulate cell division. These signaling pathways act mainly by releasing
the molecular brakes that block the transition from the $G_1$ phase of the
cell cycle into S phase.

An important example of such a molecular brake is the \textit{Retinoblastoma}
(Rb) protein, first identified through studies of a rare childhood eye tumor
called retinoblastoma, in which the Rb protein is missing or defective. The
Rb protein is abundant in the nucleus of all vertebrate cells. It binds to
particular transcription regulators, preventing them from stimulating the
transcription of genes required for cell proliferation. Mitogens release the
Rb brake in the following way. They activate intracellular signaling pathways
that lead to the activation of the $G_1$-Cdk and $G_1$/S-Cdk complexes
discussed earlier. These complexes phosphorylate the Rb protein, altering
its conformation so that it releases its bound transcription regulators,
which are then free to activate the genes required for cell proliferation.

Most mitogens have been identified and characterized by their effects
on cells in culture. One of the first mitogens identified in
this way was \textit{platelet-derived growth factor}, or PDGF, the effects of which
are typical of many others discovered since. When blood clots form (in
a wound, for example), blood platelets incorporated in the clots are triggered
to release PDGF. PDGF then binds to receptor tyrosine kinases
in surviving cells at the wound site, stimulating
them to proliferate and help heal the wound.

\subsection{Growth factors stimulate cells to Grow}

Like most survival factors and mitogens, most extracellular growth factors
bind to cell-surface receptors, which activate various intracellular
signaling pathways. These pathways lead to the accumulation of proteins
and other macromolecules, and they do so by both increasing the rate of
synthesis of these molecules, and decreasing their rate of degradation.
Some extracellular signal proteins, including PDGF, can
act as both growth factors and mitogens, stimulating both cell growth
and progression through the cell cycle. Such proteins help ensure that
cells maintain their appropriate size as they proliferate.

\subsection{Some extracellular signal proteins inhibit cell survival, division, or Growth}

Some extracellular signal proteins act to oppose these positive regulators and thereby inhibit tissue growth.
\textit{Myostatin}, for example, is a secreted signal protein that normally inhibits
the growth and proliferation of the myoblasts that fuse to form skeletal
muscle cells during mammalian development. When the gene that
encodes myostatin is deleted in mice, their muscles grow to be several
times larger than normal, because both the number and the size of muscle
cells is increased. Remarkably, two breeds of cattle that were bred for
large muscles turned out to have mutations in the gene encoding myostatin.

When we have discussed cell division, we have always
been referring to those ordinary divisions that produce two daughter
cells, each with a full and identical complement of the parent cell’s
genetic material. There is, however, a different and highly specialized
type of cell division called \textit{meiosis}, which is required for sexual reproduction
in eucaryotes.


















\section{Essential concepts}

\begin{itemize}
\item The eucaryotic cell cycle consists of several distinct phases. These
include S phase, during which the nuclear DNA is replicated, and M
phase, during which the nucleus divides (mitosis) and then the cytoplasm
divides (cytokinesis).
\item In most cells, there is one gap phase ($G_1$) after M phase and before
S phase, and another ($G_2$) after S phase and before M phase. These
gaps give the cell more time to grow and to prepare for the events of
S phase and M phase.
\item The cell-cycle control system coordinates the events of the cell cycle
by sequentially and cyclically switching on the appropriate parts of
the cell-cycle machinery and then switching them off.
\item The control system depends on a set of protein kinases, each composed
of a regulatory subunit called a cyclin and a catalytic subunit
called a cyclin-dependent protein kinase (Cdk).
\item Cyclin concentrations rise and fall at specific times in the cell cycle,
helping to trigger events of the cycle. The Cdks are cyclically activated
by both cyclin binding and the phosphorylation of some amino
acids and the dephosphorylation of others; when activated, Cdks
phosphorylate key proteins in the cell.
\item Different cyclin–Cdk complexes trigger different steps of the cell
cycle: M-Cdk drives the cell into mitosis; $G_1$-Cdk drives it through $G_1$;
$G_1$/S-Cdk and S-Cdk drive it into S phase.
\item The control system also uses protein complexes that trigger the proteolysis
of specific cell-cycle regulators at particular stages of the
cycle.
\item The cell-cycle control system can halt the cycle at specific checkpoints
to ensure that intracellular and extracellular conditions are favorable
and that the next step in the cycle does not begin before the previous
one has finished. Some of these checkpoints rely on Cdk inhibitors
that block the activity of one or more cyclin–Cdk complexes.
\item S-Cdk initiates DNA replication during S phase and helps ensure that
the genome is copied only once. Checkpoints in $G_1$, S phase, and $G_2$
prevent cells from replicating damaged DNA.
\item M-Cdk drives the cell into mitosis with the assembly of the microtubule-based
mitotic spindle, which will move daughter chromosomes
to opposite poles of the cell.
\item Microtubules grow out from the duplicated centrosomes, and some
interact with microtubules growing from the opposite pole, thereby
becoming the interpolar microtubules that form the spindle.
\item Centrosomes, microtubule-associated motor proteins, and the replicated
chromosomes themselves work together to assemble the
spindle.
\item When the nuclear envelope breaks down, the spindle microtubules
invade the nuclear area and capture the replicated chromosomes.
The microtubules bind to protein complexes, called kinetochores,
associated with the centromere of each sister chromatid.
\item Microtubules from opposite poles pull in opposite directions on each
replicated chromosome, bringing the chromosomes to the equator of
the mitotic spindle.
\item The sudden separation of sister chromatids allows the resulting
daughter chromosomes to be pulled to opposite poles by the spindle.
The two poles also move apart, further separating the two sets of
chromosomes.
\item The movement of chromosomes by the spindle is driven both by
microtubule motor proteins and by microtubule polymerization and
depolymerization.
\item A nuclear envelope re-forms around the two sets of segregated chromosomes
to form two new nuclei, thereby completing mitosis.
\item The Golgi apparatus breaks into many smaller fragments during
M phase, ensuring an even distribution between the daughter cells.
\item In animal cells, cytoplasmic division is mediated by a contractile ring
of actin filaments and myosin filaments, which assembles midway
between the spindle poles and contracts to divide the cytoplasm in
two; in plant cells, by contrast, cell division occurs by the formation
of a new cell wall inside the parent cell, which divides the cytoplasm
in two.
\item Animal cell numbers are regulated by a combination of intracellular
programs and extracellular signals that control cell survival, cell
growth, and cell proliferation.
\item Many normal cells die by apoptosis during the lifetime of an animal;
they do so by activating an internal suicide program and killing
themselves.
\item Apoptosis depends on a family of proteolytic enzymes called caspases,
which are made as inactive precursors (procaspases). The
procaspases are themselves often activated by proteolytic cleavage
mediated by caspases.
\item Most animal cells require continuous signaling from other cells to
avoid apoptosis; this may be a mechanism to ensure that cells sur-
vive only when and where they are needed.
\item Animal cells proliferate only if stimulated by extracellular mitogens
produced by other cells, ensuring that a cell divides only when
another cell is needed; the mitogens activate intracellular signaling
pathways to override the normal brakes that otherwise block cell-cycle
progression.
\item For an organism or an organ to grow, cells must grow as well as
divide. Animal cell growth depends on extracellular growth factors,
which stimulate protein synthesis and inhibit protein degradation.
\item Cell and tissue size can also be influenced by inhibitory extracellular
signal proteins that oppose the positive regulators of cell survival,
cell growth, and cell division.
\item Cancer cells fail to obey these normal ‘social’ controls on cell behavior
and therefore outgrow, out-divide, and out-survive their normal
neighbors.
\end{itemize}
