\chapter{Cell Communication}

In this chapter, we examine some of the most important methods by which
cells communicate, and we discuss how cells send signals and interpret
the signals they receive. Although we concentrate on the mechanisms of
signal reception and interpretation in animal cells, we also present a brief
review of what is known about cell-to-cell signaling in plants. We begin
our discussion with an overview of the general principles of cell signaling
and then consider two of the main systems animal cells use to receive
and interpret signals.

\section{General principles of cell signaling}

The signals that pass between living cells are simpler than the sorts of
messages that humans ordinarily exchange. In a typical communication
between cells, the \textit{signaling cell} produces a particular type of
signal molecule that is detected by the \textit{target cell}.
Target cells possess receptor proteins
that recognize and respond specifically to the signal molecule. Signal
transduction begins when the receptor protein on a target cell receives
an incoming extracellular signal and converts it to the intracellular sig-
nals that alter cell behavior. Most of this chapter is concerned with signal
reception and transduction - the events that cell biologists have in mind
when they refer to \textbf{cell signaling}.

\subsection{Signals can act over a long or short range}

In multicellular organisms, the most `public' style of communication
involves broadcasting the signal throughout the whole body by secreting
it into the bloodstream (in an animal) or the sap (in a plant). Signal molecules
used in this way are called \emph{hormones}, and, in animals, the cells
that produce hormones are called endocrine cells.

Somewhat less public is the process known as \textit{paracrine signaling}. In this
case, rather than entering the bloodstream, the signal molecules diffuse
locally through the extracellular fluid, remaining in the neighborhood of
the cell that secretes them. Thus, they act as \textbf{local mediators} on nearby
cells. In some cases, cells can respond to
the local mediators that they themselves produce, a form of paracrine
communication called \textit{autocrine signaling}.

\textit{Neuronal signaling} is a third form of cell communication. Like endocrine
cells, nerve cells (neurons) can deliver messages over long distances.
In the case of neuronal signaling, however, a message is not broadcast
widely but is instead delivered quickly and specifically to individual target
cells through private lines. The axon of a neuron terminates at specialized
junctions (synapses) on target cells that can lie far from the neuronal cell body.
When activated by signals from the environment or from other
nerve cells, a neuron sends electrical impulses racing along its axon at
speeds of up to 100 m/sec. On reaching the axon terminal, these electrical
signals are converted into a chemical form: each electrical impulse
stimulates the nerve terminal to release a pulse of an extracellular signal
molecule called a neurotransmitter. The neurotransmitter then diffuses
across the narrow (< 100 nm) gap between the axon-terminal membrane
and the membrane of the target cell, reaching the target cell receptors in
less than 1 msec.

A fourth style of signal-mediated cell–cell communication - the most intimate
and short-range of all - does not require the release of a secreted
molecule. Instead, the cells make direct physical contact through signal
molecules lodged in the plasma membrane of the signaling cell and
receptor proteins embedded in the plasma membrane of the target cell.

\subsection{Each cell responds to a limited set of signals, depending on its History and its current state}

Whether a cell responds to a signal molecule depends first of all on
whether it possesses a \textbf{receptor protein}, or receptor, for that signal.
Each receptor is usually activated by only one type of signal. Without the
appropriate receptor, a cell will be deaf to the signal and will not respond
to it. By producing only a limited set of receptors out of the thousands
that are possible, a cell restricts the types of signals that can affect it.

The signal from a cell-surface
receptor is generally conveyed into the target cell interior via a set of
\textit{intracellular signaling molecules}, which act in sequence and ultimately
alter the activity of effector proteins, which then affect the behavior of the
cell. This intracellular relay system and the intracellular \textit{effector proteins}
on which it acts vary from one type of specialized cell to another, so that
different types of cells respond to the same signal in different ways.

\subsection{A cell’s response to a signal can Be fast or slow}

The length of time a cell takes to respond to an extracellular signal can
vary greatly, depending on what needs to happen once the message has
been received.

\subsection{Some Hormones cross the Plasma membrane and Bind to intracellular receptors}

\textbf{Extracellular signal molecules} generally fall into two classes. The first
and largest class consists of molecules that are too large or too hydrophilic
to cross the plasma membrane of the target cell. They rely on receptors
on the surface of the target cell to relay their message across the membrane.
The second, and smaller, class of signals consists
of molecules that are small enough or hydrophobic enough to slip easily
across the plasma membrane. Once inside, these signal molecules
usually activate intracellular enzymes or bind to intracellular receptor
proteins that regulate gene expression.

One important class of signal molecules that rely on intracellular recep-
tor proteins is the \textbf{steroid hormones} - including cortisol, estradiol, and
testosterone - and the thyroid hormones such as thyroxine.
All of these hydrophobic molecules pass through the plasma membrane
of the target cell and bind to receptor proteins located in either the cytosol
or the nucleus. Both the cytosolic and nuclear receptors are referred to
as \textit{nuclear receptors}, because, when activated by hormone binding, they
act as transcription regulators in the nucleu. In
unstimulated cells, nuclear receptors are typically present in an inactive
form. When a hormone binds, the receptor undergoes a large conformational
change that activates the protein, allowing it to promote or inhibit
the transcription of specific target genes.
Each hormone binds to a different receptor protein, and each receptor acts
at a different set of regulatory sites in DNA.

Nuclear receptors and the hormones that activate them play an essential
role in human physiology. Loss of these signaling
systems can have dramatic consequences.

\subsection{Some dissolved Gases cross the Plasma membrane and activate intracellular enzymes directly}

Steroid hormones and thyroid hormones are not the only extracellular
signal molecules that can pass through the plasma membrane. Some
dissolved gases can slip across the membrane to the cell interior and
directly regulate the activity of specific intracellular proteins.
\textbf{Nitric oxide} (NO) acts in this way. This gas diffuses readily out of
the cell that generates it and enters neighboring cells. NO is synthesized
from the amino acid arginine and operates as a local mediator in many
tissues. The gas acts only locally because it is quickly converted to
nitrates and nitrites by reaction with oxygen and water outside cells.

Endothelial cells - the flattened cells that line every blood vessel - release
NO in response to stimulation by nerve endings. This NO signal causes
smooth muscle cells in the vessel wall to relax, allowing the vessel to
dilate, so that blood flows through it more freely.

Inside many target cells, NO binds to and activates the enzyme \textit{guanylyl
cyclase}, stimulating the formation of \textit{cyclic GMP} from the nucleotide GTP.
Cyclic GMP is itself a small intracellular signaling
molecule that forms the next link in the NO signaling chain that leads to
the cell’s ultimate response. Cyclic GMP is very similar in its structure and mechanism
of action to \textit{cyclic AMP}, a much more commonly used intracellular
messenger molecule.

\subsection{Cell-surface receptors relay extracellular signals via intracellular signaling Pathways}

In contrast to NO and the steroid and thyroid hormones, the vast majority
of signal molecules are too large or hydrophilic to cross the plasma
membrane of the target cell. These proteins, peptides, and small, highly
water-soluble molecules bind to cell-surface receptor proteins that span
the plasma membrane. These transmembrane receptors
detect a signal on the outside and relay the message, in a new form,
across the membrane into the interior of the cell.

The receptor protein performs the primary signal transduction step: it
binds to the extracellular signal and generates new intracellular signals
in response. The resulting intracellular signaling process
usually works like a molecular relay race in which the message
is passed ‘downstream’ from one \textbf{intracellular signaling molecule} to
another, each activating or generating the next signaling molecule in the
pathway, until a metabolic enzyme is kicked into action, the cytoskeleton
is tweaked into a new configuration, or a gene is switched on or off. This
final outcome is called the \textit{response} of the cell.

The components of these \textbf{intracellular signaling pathways} perform one
or more crucial functions:

\begin{enumerate}
\item They can simply \textit{relay} the signal onward and thereby help spread it
through the cell.
\item They can \textit{amplify} the signal received, making it stronger, so that
a few extracellular signal molecules are enough to evoke a large
intracellular response.
\item They can receive signals from more than one intracellular signaling
pathway and \textit{integrate} them before relaying a signal onward.
\item They can \textit{distribute} the signal to more than one signaling pathway or
effector protein, creating branches in the information flow diagram
and evoking a complex response.
\end{enumerate}

As part of the integration function, many steps in a signaling pathway
are open to \textit{modulation} by other factors, including both intracellular and
extracellular factors, so that the effects of each signal are tailored to the
conditions prevailing inside and outside the cell.

\subsection{Some intracellular signaling Proteins act as molecular switches}

Many of the key intracellular signaling proteins behave as \textbf{molecular
switches}: receipt of a signal causes them to toggle from an inactive to
an active state. Once activated, these proteins can turn on other proteins
in the signaling pathway. They then persist in an active state until some
other process switches them off again. The importance of the switching-
off process is often underappreciated. The two are equally important for the
signaling process.

Proteins that act as molecular switches fall mostly into one of two classes.
The first and by far the largest class consists of proteins that are activated
or inactivated by phosphorylation. For these molecules, the switch is thrown in
one direction by a \textbf{protein kinase}, which tacks a phosphate group onto
the switch protein, and in the other direction by a \textbf{protein phosphatase},
which plucks the phosphate off again. The activity of any protein that is regulated
by phosphorylation depends - moment by moment - on the balance between the activities
of the kinases that phosphorylate it and the phosphatases that dephosphorylate it.

Many of the switch proteins controlled by phosphorylation are themselves
protein kinases, and these are often organized into phosphorylation cascades:
one protein kinase, activated by phosphorylation, phosphorylates
the next protein kinase in the sequence, and so on, transmitting the signal
onward and, in the process, amplifying, distributing, and modulating
it. Two main types of protein kinases operate in intracellular signaling
pathways: the most common are \textbf{serine/threonine kinases}, which - as
the name implies - phosphorylate proteins on serines or threonines; oth-
ers are \textbf{tyrosine kinases}, which phosphorylate proteins on tyrosines.

The other main class of switch proteins involved in intracellular signaling
pathways is the \textbf{GTP-binding proteins}. These switch between an
active and an inactive state depending on whether they have GTP or
GDP bound to them, respectively. Once activated by GTP
binding, these proteins have intrinsic GTP-hydrolyzing (\textit{GTPase}) activity,
and they shut themselves off by hydrolyzing their bound GTP to GDP.
One class of GTP-activated switch proteins contains the large, trimeric
GTP-binding proteins (also called \textit{G proteins}) that relay messages from
\textit{G-protein-coupled receptors}.

\subsection{Cell-surface receptors fall into three main classes}

All cell-surface receptor proteins bind to an extracellular signal mole-
cule and transduce its message into one or more intracellular signaling
molecules that alter the cell’s behavior. These receptors, however, are
divided into three large families that differ in the transduction mecha-
nism they use.

\begin{enumerate}
\item \textit{Ion-channel-coupled} receptors allow a flow of ions
across the plasma membrane, which changes the membrane potential
and produces an electrical current;
\item \textit{G-protein–coupled} receptors activate membrane-bound, trimeric GTP-binding proteins (G
proteins), which then activate either an enzyme or an ion channel in the
plasma membrane, initiating a cascade of other effects;
\item \textit{Enzyme-coupled receptors} either act as enzymes or associate with
enzymes inside the cell; when stimulated, the enzymes
activate a variety of intracellular signaling pathways.
\end{enumerate}

\subsection{Ion-channel-coupled receptors convert chemical signals into electrical ones}

Of all the types of cell-surface receptors, \textbf{ion-channel-coupled receptors}
(also known as transmitter-gated ion channels) function in the
simplest and most direct way. These receptors are responsible for the
rapid transmission of signals across synapses in the nervous system.
They transduce a chemical signal, in the form of a pulse of neurotransmitter
delivered to the outside of the target cell, directly into an electrical
signal, in the form of a change in voltage across the target cell’s plasma
membrane. When the neurotransmitter binds, this type
of receptor alters its conformation so as to open or close an ion channel
in the plasma membrane, allowing the flow of specific types of ions.
Driven by their electrochemical gradients, the ions rush into or out of the cell,
creating a change in the membrane potential within a millisecond or so.
This change in potential may trigger a nerve impulse or make it easier
(or harder) for other neurotransmitters to do so.
Whereas ion-channel–coupled receptors are a specialty of the nervous
system and of other electrically excitable cells such as muscle cells,
G-protein-coupled receptors and enzyme-coupled receptors are used
by practically every cell type in the body.

\section{G-Protein-coupled receptors}

\textbf{G-protein-coupled receptors} (GPCRs) form the largest family of cell-surface
receptors. These receptors mediate responses to an enormous diversity of extracellular signal
molecules, including hormones, local mediators, and neurotransmitters.
The signal molecules are as varied in structure as they are in function:
they can be proteins, small peptides, or derivatives of amino acids or
fatty acids, and for each one of them there is a different receptor or set of
receptors. Because GPCRs are involved in such a large variety of cellular
processes, they are an attractive target for the development of drugs to
treat a variety of disorders. About half of all known drugs work through
GPCRs.

Despite the diversity of the signal molecules that bind to them, all GPCRs
that have been analyzed have a similar structure: each is made of a single
polypeptide chain that threads back and forth across the lipid bilayer
seven times. This superfamily of seven-pass transmembrane
receptor proteins includes rhodopsin (the light-activated photoreceptor
protein in the vertebrate eye), the olfactory (smell) receptors in the vertebrate
nose, and the receptors that participate in the mating rituals of
single-celled yeasts.

\subsection{Stimulation of GPCRs activates G-Protein subunits}

When an extracellular signal molecule binds to a GPCR, the receptor
protein undergoes a conformational change that enables it to activate a
G protein located on the underside of the plasma membrane. To explain
how this activation leads to the transmission of a signal, we must first
consider how G proteins are constructed and how they function.

There are several varieties of G proteins. Each is specific for a particular
set of receptors and a particular set of target enzymes or ion channels
in the plasma membrane. All of these G proteins, however, have a similar
general structure and operate in a similar way. They are composed
of three protein subunits - $\alpha$, $\beta$, and $\gamma$ - two of which are tethered to the
plasma membrane by short lipid tails. In the unstimulated state, the $\alpha$
subunit has GDP bound to it, and the G protein is idle.
When an extracellular ligand binds to its receptor, the altered receptor
activates a G protein by causing the $\alpha$ subunit to decrease its affinity for
GDP, which is then exchanged for a molecule of GTP. In some cases, this
activation is thought to break up the G-protein subunits, so that the activated
$\alpha$ subunit, clutching its GTP, detaches from the $\beta\gamma$ complex, which is
also activated. Regardless of whether they dissociate, the
two activated parts of a G protein - the $\alpha$ subunit and the $\beta\gamma$ complex -
can both interact directly with target proteins in the plasma membrane,
which in turn may relay the signal to yet other destinations in the cell.
The longer these target proteins have an $\alpha$ or a $\beta\gamma$ subunit bound to them,
the stronger and more prolonged the relayed signal will be.

The amount of time that the $\alpha$ and $\beta\gamma$ subunits remain `switched on' -
and hence available to relay signals - is limited by the behavior of the $\alpha$
subunit. The a subunit has an intrinsic GTPase activity, and it eventually
hydrolyzes its bound GTP back to GDP, returning the whole G protein
to its original, inactive conformation. GTP hydrolysis and
inactivation occur within seconds after the G protein has been activated.
The inactive G protein is now ready to be reactivated by another activated receptor.

The G-protein switch demonstrates a general principle of cell signaling
mentioned earlier: the mechanisms that shut a signal off are as important
as the mechanisms that turn it on. The shut-off
mechanisms also offer as many opportunities for control, and as many
dangers for mishap.

\subsection{Some G Proteins directly regulate ion channels}

The target proteins recognized by G-protein subunits are either enzymes
or ion channels in the plasma membrane. There are about 20 types of
mammalian G proteins, each activated by a particular set of cell-surface
receptors and dedicated to activating a particular set of target proteins.
In this way, binding of an extracellular signal molecule to a GPCR leads to
changes in the activities of a specific subset of the possible target proteins
in the plasma membrane, leading to a response that is appropriate for
that signal and that type of cell.

We look first at an example of direct G-protein regulation of ion channels.
The heartbeat in animals is controlled by two sets of nerves: one speeds
the heart up, the other slows it down. The nerves that signal a slowdown
in heartbeat do so by releasing acetylcholine, which binds to a GPCR on
the surface of the heart muscle cells. This GPCR activates the G protein,
$G_i$ . In this case, the $\beta\gamma$ complex is the active signaling component: it binds
to the intracellular face of a $K^+$ channel in the plasma membrane of the
heart muscle cell, forcing the ion channel into an open conformation.
This allows $K^+$ to flow out of the cell, thereby inhibiting
the cell’s electrical excitability. The signal is shut off -
and the $K^+$ channel recloses - when the a subunit inactivates itself by
hydrolyzing its bound GTP, returning the G protein to its inactive state.

\subsection{Some G Proteins activate membrane-bound enzymes}

When G proteins interact with ion channels, they cause an immediate
change in the state and behavior of the cell. Their interactions with
enzymes have more complex consequences, leading to the production
of additional intracellular signaling molecules. The two most frequent
target enzymes for G proteins are \textit{adenylyl cyclase}, the enzyme responsible
for production of the small intracellular signaling molecule \textit{cyclic
AMP}, and \textit{phospholipase C}, the enzyme responsible for production of the
small intracellular signaling molecules \textit{inositol trisphosphate} and \textit{diacyl-glycerol}.
The two enzymes are activated by different types of G proteins,
so that cells are able to couple the production of the small intracellular
signaling molecules to different extracellular signals. We concentrate
here on G proteins that stimulate enzyme activity. The small intracellular
signaling molecules generated in these cascades are often called \textbf{small
messengers}, or \textbf{second messengers} (the “first messengers” being the
extracellular signals); they are produced in large numbers when a membrane-bound
enzyme - such as adenylyl cyclase or phospholipase C - is
activated, and they rapidly diffuse away from their source, spreading the
signal.

Different small messenger molecules, of course, produce different
responses.

\subsection{The cyclic AMP Pathway can activate enzymes and turn on Genes}

Many extracellular signals acting via GPCRs affect the activity of the
enzyme \textbf{adenylyl cyclase} and thus alter the concentration of the small
messenger molecule \textbf{cyclic AMP} inside the cell. Most commonly, the
activated G-protein $\alpha$ subunit switches on the adenylyl cyclase, causing
a dramatic and sudden increase in the synthesis of cyclic AMP from ATP
(which is always present in the cell). Because it stimulates the cyclase,
this G protein is called $G_s$. To help terminate the signal, a second enzyme,
called \textit{cyclic AMP phosphodiesterase}, rapidly converts cyclic AMP to ordinary AMP.

Cyclic AMP phosphodiesterase is continuously active inside the cell.
Because it breaks cyclic AMP down so quickly, the concentration of this
small messenger can change rapidly in response to extracellular signals,
rising or falling tenfold in a matter of seconds. Cyclic AMP
is a water-soluble molecule, so it can, in some cases, carry the signal
throughout the cell, traveling from the site on the membrane where it is
synthesized to interact with proteins located in the cytosol, in the nucleus,
or on other organelles.

Cyclic AMP exerts most of its effects by activating the enzyme \textbf{cyclic-AMP–dependent protein kinase}
(PKA). This enzyme is normally
held inactive in a complex with another protein. The binding of cyclic
AMP forces a conformational change that unleashes the active kinase.
Activated PKA then catalyzes the phosphorylation of particular serines or
threonines on certain intracellular proteins, thus altering the activity of
the proteins. In different cell types, different sets of proteins are available
to be phosphorylated, which largely explains why the effects of cyclic
AMP vary with the type of target cell.

Different target cells respond very differently to extracellular signals that
change intracellular cyclic AMP concentrations.

In some cases, the effects of activating a cyclic AMP cascade are rapid;
in other cases, cyclic AMP responses involve changes in gene expression that take
minutes or hours to develop.

\subsection{The inositol Phospholipid Pathway triggers a rise in intracellular $Ca^{2+}$}

Some GPCRs exert their effects via G proteins that activate the membrane-bound
enzyme \textbf{phospholipase C} instead of adenylyl cyclase.

Once activated, phospholipase C propagates the signal by cleaving a lipid
molecule that is a component of the plasma membrane. The molecule is
an \textbf{inositol phospholipid} (a phospholipid with the sugar inositol attached
to its head) that is present in small quantities in the cytosolic half of the
membrane lipid bilayer. Because of the involvement of
this phospholipid, the signaling pathway that begins with the activation
of phospholipase C is often referred to as the inositol \textit{phospholipid pathway}.
It operates in almost all eucaryotic cells and can regulate a host of
different effector proteins.

The pathway works in the following way. When phospholipase C chops
the sugar-phosphate head off the inositol phospholipid, it generates two
small signaling molecules: \textbf{inositol 1,4,5-trisphosphate} ($IP_3$) and \textbf{diacylglycerol}
(DAG). $IP_3$, a water-soluble sugar phosphate, diffuses into the
cytosol, while the lipid diacylglycerol remains embedded in the plasma
membrane. Both molecules play a crucial part in relaying the signal, and
we will consider them in turn.

The $IP_3$ released into the cytosol rapidly encounters the endoplasmic
reticulum; there it binds to and opens $Ca^{2+}$ channels that are embedded in
the endoplasmic reticulum membrane. $Ca^{2+}$ stored inside the endoplasmic
reticulum rushes out into the cytosol through these open channels,
causing a sharp rise in the cytosolic concentration of free
$Ca^{2+}$, which is normally kept very low. This $Ca^{2+}$ in turn signals to other
proteins, as discussed below.

The diacylglycerol that is generated along with $IP_3$ helps recruit and acti-
vate a protein kinase, which translocates from the cytosol to the plasma
membrane. This enzyme is called protein kinase C (PKC) because it also
needs to bind $Ca^{2+}$ to become active. Once activated,
PKC phosphorylates a set of intracellular proteins that varies depending
on the cell type. PKC operates on the same principle as PKA, although
most of the proteins it phosphorylates are different.

\subsection{A $Ca^{2+}$ signal triggers many Biological Processes}

$Ca^{2+}$ has such an important and widespread role as a small intracellular
messenger that we must digress to consider its functions more generally.
A surge in the cytosolic concentration of free $Ca^{2+}$ is triggered by many
kinds of stimuli, not only those that act through GPCRs.

The concentration of free $Ca^{2+}$ in the cytosol of an unstimulated cell is
extremely low ($10^{–7}$ M) compared with its concentration in the extra-cellular
fluid and in the endoplasmic reticulum. These differences are
maintained by membrane-embedded pumps that actively pump $Ca^{2+}$
out of the cytosol - either into the endoplasmic reticulum or across the
plasma membrane and out of the cell. As a result, a steep electrochemical
gradient of $Ca^{2+}$ exists across both the endoplasmic reticulum membrane
and the plasma membrane. When a signal transiently opens $Ca^{2+}$ channels
in either of these membranes, $Ca^{2+}$ rushes
into the cytosol down its electrochemical gradient, triggering changes in
$Ca^{2+}$-responsive proteins in the cytosol. The same pumps that normally
operate to keep cytosolic $Ca^{2+}$ concentrations low also help to terminate
the $Ca^{2+}$ signal.

The effects of $Ca^{2+}$ in the cytosol are largely indirect, in that they are mediated
through the interaction of $Ca^{2+}$ with various kinds of \textit{$Ca^{2+}$-responsive
proteins}. The most widespread and common of these is \textbf{calmodulin},
which is present in the cytosol of all eucaryotic cells that have been examined,
including those of plants, fungi, and protozoa. When $Ca^{2+}$ binds to
calmodulin, the protein undergoes a conformational change that enables
it to wrap around a wide range of target proteins in the cell, altering their
activities. One particularly important class of targets for
calmodulin is the \textbf{$Ca^{2+}$/calmodulin-dependent protein kinases} (CaM-kinases).
When these kinases are activated by binding to calmodulin
complexed with $Ca^{2+}$, they influence other processes in the cell by phosphorylating
selected proteins. In the mammalian brain, for example, a
neuron-specific CaM-kinase is abundant at synapses, where it is thought
to play a part in some forms of learning and memory. This CaM-kinase is
activated by the pulses of $Ca^{2+}$ signals that occur during neural activity,
and mutant mice that lack the kinase show a marked inability to remember
where things are.

\subsection{Intracellular signaling cascades can achieve astonishing speed, sensitivity, and adaptability}

The steps in the signaling cascades associated with GPCRs take a long
time to describe, but they often take only seconds to execute.
Among the fastest of all responses
mediated by a GPCR, however, is the response of the eye to bright light:
it takes only 20 msec for the most quickly responding photoreceptor cells
of the retina (the cone photoreceptors, which are responsible for color
vision in bright light) to produce their electrical response to a sudden
flash of light.

This speed is achieved in spite of the necessity to relay the signal over
several steps of an intracellular signaling cascade. But photoreceptors
also provide a beautiful illustration of the positive advantages of signaling
cascades: in particular, such cascades allow spectacular amplification
of the incoming signal and also allow cells to adapt so as to be able to
detect signals of widely varying intensity. The quantitative details have
been most thoroughly analyzed for the rod photoreceptor cells in the eye,
which are responsible for non-color vision in dim light. In
this cell, light is sensed by \textit{rhodopsin}, a G-protein-coupled light receptor.
Light-activated rhodopsin activates a G protein called \textit{transducin}. The
activated a subunit of transducin then activates an intracellular signaling
cascade that causes cation channels to close in the plasma membrane
of the photoreceptor cell. This produces a change in the voltage across
the cell membrane, which alters neurotransmitter release and ultimately
leads to a nerve impulse being sent to the brain.

The signal is repeatedly amplified as it is relayed along this intracellular
signaling pathway. When lighting conditions are dim, as
on a moonless night, the amplification is enormous: as few as a dozen
photons absorbed in the entire retina will cause a perceptible signal to
be delivered to the brain. In bright sunlight, when photons flood through
each photoreceptor cell at a rate of billions per second, the signaling
cascade \textit{adapts}, stepping down the amplification more than 10,000-fold
so that the photoreceptor cells are not overwhelmed and can still register
increases and decreases in the strong light. The adaptation depends on
negative feedback: an intense response in the photoreceptor cell generates
an intracellular signal (a change in $Ca^{2+}$ concentration) that inhibits
the enzymes responsible for signal amplification.

\textbf{Adaptation} frequently occurs in signaling pathways that respond to
chemical signals; again, it allows cells to remain sensitive to changes of
signal intensity over a wide range of background levels of stimulation.
Adaptation, in other words, allows a cell to respond to both messages
that are whispered and those that are shouted.

\section{Enzyme-coupled receptors}

Like GPCRs, \textbf{enzyme-coupled receptors} are transmembrane proteins
that display their ligand-binding domains on the outer surface of the
plasma membrane. Instead of associating with a G protein, however, the
cytoplasmic domain of the receptor either acts as an enzyme itself or
forms a complex with another protein that acts as an enzyme.

Enzyme-coupled receptors, however, can also mediate direct, rapid
reconfigurations of the cytoskeleton, controlling the way a cell changes
its shape and moves. The extracellular signals for these architectural
alterations are often not diffusible signal proteins, but proteins attached
to the surfaces over which a cell is crawling. Disorders of cell growth,
proliferation, differentiation, survival, and migration are fundamental
to cancer, and abnormalities in signaling via enzyme-coupled receptors
have a major role in the development of this class of diseases.

The largest class of enzyme-coupled receptors is made up of those with
a cytoplasmic domain that functions as a tyrosine protein kinase, phosphorylating
specific tyrosines on selected intracellular proteins. Such
receptors are called \textbf{receptor tyrosine kinases} (RTKs), and we will focus
on these receptors here. Note that all of the other protein kinases we
have discussed so far - including PKA, PKC, and CaM-kinases - are serine/threonine kinases.

\subsection{Activated RTKs recruit a complex of intracellular signaling Proteins}

To do its job as a signal transducer, an enzyme-coupled receptor has to
switch on the enzyme activity of its intracellular domain (or of an associated
enzyme) when an external signal molecule binds to its extracellular
domain. Unlike the seven-pass GPCRs, enzyme-coupled receptor proteins
usually have only one transmembrane segment, which is thought
to span the lipid bilayer as a single a helix. Because a single $\alpha$ helix is
poorly suited to transmit a conformational change across the bilayer,
enzyme-coupled receptors have a different strategy for transducing the
extracellular signal. In many cases, the binding of a signal molecule
causes two receptor molecules to come together in the membrane, forming
a dimer. Contact between the two adjacent intracellular receptor tails
activates their kinase function, with the result that each receptor phosphorylates
the other. In the case of RTKs, the phosphorylations occur on
specific tyrosines located on the cytosolic tail of the receptors.

Tyrosine phosphorylation then triggers the assembly of an elaborate
intracellular signaling complex on the receptor tails. The newly phosphorylated
tyrosines serve as binding sites for a whole zoo of intracellular
signaling proteins - perhaps as many as 10 or 20 different molecules.
Some of these proteins become phosphorylated and activated
on binding to the receptor, and they then propagate the signal;
others function solely as adaptors, which couple the receptor to other signaling
proteins, thereby helping to build the active signaling complex.

While they last, the protein complexes assembled on the cytosolic tails
of the RTKs can transmit the signal along several routes simultaneously
to many destinations inside the cell, thus activating and coordinating the
numerous biochemical changes that are required to trigger a complex
response, such as cell proliferation. To help terminate the response, the
tyrosine phosphorylations are reversed by \textit{protein tyrosine phosphatases},
which remove the phosphates that were added to the tyrosines of the
RTKs and other signaling proteins in response to the extracellular signal.
In some cases, activated RTKs (and GPCRs) are inactivated in a more brutal
way: they are dragged into the interior of the cell by endocytosis and
then destroyed by digestion in lysosomes.

\subsection{Most RTKs activate the monomeric GTPase Ras}

As we have seen, activated RTKs recruit many kinds of intracellular sig-
naling proteins and form large signaling complexes. One of the key players
in these signaling complexes is \textbf{Ras} - a small GTP-binding protein that is
bound by a lipid tail to the cytoplasmic face of the plasma membrane.
Virtually all RTKs activate Ras, including platelet-derived growth factor
(PDGF) receptors, which mediate cell proliferation in wound healing, and
nerve growth factor (NGF) receptors, which prevent certain neurons from
dying in the developing nervous system.

The Ras protein is a member of a large family of small GTP-binding
proteins, often called \textbf{monomeric GTPases} to distinguish them from
the trimeric G proteins that we encountered earlier. Ras resembles the
$\alpha$ subunit of a G protein and functions as a molecular switch in much
the same way. It cycles between two distinct conformational states -
active when GTP is bound and inactive when GDP is bound.
Interaction with an activated signaling protein encourages Ras
to exchange its GDP for GTP, thus switching Ras to its activated state.
After a delay, Ras switches itself off again by hydrolyzing
its bound GTP to GDP.

In its active state, Ras promotes the activation of a phosphorylation cascade
in which a series of serine/threonine protein kinases phosphorylate
and activate one another in sequence, like an intracellular game of dominoes.
This relay system, which carries the signal from
the plasma membrane to the nucleus, includes a three-kinase protein
module called the \textbf{MAP-kinase signaling module}, in honor of the final
kinase in the chain, the mitogen-activated protein kinase or \textbf{MAP kinase}.
(\textit{Mitogens} are extracellular signal molecules that stimulate cell proliferation).
In this pathway, MAP kinase is phosphorylated and activated by an
enzyme called, logically enough, \textit{MAP kinase kinase}. And this protein is
itself switched on by a \textit{MAP kinase kinase kinase} (which is activated by
Ras). At the end of the MAP-kinase cascade, MAP kinase phosphorylates
various effector proteins, including certain transcription regulators, altering
their ability to control gene transcription. This change in the pattern
of gene expression may stimulate cell proliferation, promote cell survival,
or induce cell differentiation: the precise outcome will depend on which
other genes are active in the cell and what other signals the cell receives.

\subsection{RTKs activate Pi 3-Kinase to Produce lipid docking sites in the Plasma membrane}

Many of the extracellular signal proteins that stimulate animal cells to
survive, grow, and proliferate act through RTKs. These include signal
proteins belonging to the insulin-like growth factor (IGF) family. One
crucially important signaling pathway that RTKs activate to promote cell
growth and survival relies on the enzyme \textbf{phosphoinositide 3-kinase}
(PI 3-kinase), which phosphorylates inositol phospholipids in the plasma
membrane. These phosphorylated lipids become docking sites for specific
intracellular signaling proteins, which relocate from the cytosol to
the plasma membrane, where they can activate one another.

One of the most important of these relocated signaling proteins is the
serine/threonine protein kinase \textit{Akt}, which is also called \textit{protein kinase B},
or PKB. Akt promotes the growth and survival of many cell
types, often by inactivating the signaling proteins it phosphorylates.

In addition to promoting cell survival, the \textit{PI-3-kinase-Akt signaling pathway}
also stimulates cells to grow in size. It does so by indirectly activating
a large serine/threonine kinase called \textit{Tor}.

\subsection{Some receptors activate a fast track to the nucleus}

A few hormones and many local mediators called \textbf{cytokines} bind to
receptors that can activate transcription regulators that are held in a
latent, inactive state near the plasma membrane. Once turned on, these
regulatory proteins - called STATs (for signal transducers and activators
of transcription) - head straight for the nucleus, where they stimulate the
transcription of specific genes. This direct signaling pathway is used, for
example, by \textit{interferons}, which are cytokines produced by infected cells
that instruct other cells to produce proteins that make them more resistant
to viral infection.

Unlike the RTKs that stimulate elaborate signaling cascades, the cytokine
and hormone receptors that rely on STATs have no intrinsic enzyme activity.
Instead, they associate with cytoplasmic tyrosine kinases called JAKs,
which are activated when a cytokine or hormone binds to the receptor.
Once activated, the JAKs phosphorylate and activate STATs, which then
migrate to the nucleus, where they stimulate the transcription of specific
target genes. For example, the hormone prolactin, which stimulates
breast cells to make milk, acts by binding to a receptor that is associated
with a specific pair of JAKs. These JAKs activate a particular STAT
that then turns on the transcription of the genes encoding milk proteins.

Different cytokine and hormone receptors evoke different cell responses
by activating different STATs. Like any pathway that is turned on by
phosphorylation, these signals are shut off by protein phosphatases that
remove the phosphate groups from the activated signaling proteins.

\subsection{Multicellularity and cell communication evolved independently in Plants and animals}

Plants and animals have been evolving independently for more than a
billion years, the last common ancestor being a single-celled eucaryote
that most likely lived on its own. Because these kingdoms diverged so
long ago - when it was still "every cell for itself" - each has evolved its
own molecular solutions to the complex problem of becoming multicellular.
Thus, the mechanisms for cell-cell communication in plants and
animals evolved separately and might be expected to be quite different.
At the same time, however, plants and animals started with a common
set of eucaryotic genes - including some used by single-celled organisms
to communicate among themselves - and so their signaling systems
should show some similarities

\subsection{Protein Kinase networks integrate information to control complex cell Behaviors}

In this chapter, we have outlined several pathways for conveying a signal
from the cell surface to the cell interior. Each pathway differs from the others, yet they use
some common components to transmit their signals. Because all these
pathways eventually activate protein kinases, it seems that each is capable
in principle of regulating practically any process in the cell.

In fact, the complexity of cell signaling is much greater than we have
described. First, we have not discussed many of the intracellular signaling
pathways available to cells.
Perhaps more importantly, the major signaling pathways we have discussed
interact in ways that we have not described. They are connected
by interactions of many sorts, but the most extensive links are those
mediated by the protein kinases present in each of the pathways. These
kinases often phosphorylate, and hence regulate, components in other
signaling pathways, in addition to components in the pathway to which
they themselves primarily belong. Thus, a certain amount of cross-talk
occurs between the different pathways and, indeed,
between virtually all of the control systems of the cell.

A cell receives messages from many sources, and it must integrate this
information to generate an appropriate response: to live or die, to divide
or differentiate, to change shape, to move, to send out a chemical message
of its own, and so on. Through the cross-talk
between signaling pathways, the cell is able to put together multiple bits
of information and react to the combination. Thus, some intracellular
signaling proteins act as integrating devices, usually by having several
potential phosphorylation sites, each of which can be phosphorylated by
a different protein kinase. Information received from different sources
can converge on such proteins, which then convert the input to a single
outgoing signal. The integrating
proteins in turn can deliver a signal to many downstream targets. In this
way, the intracellular signaling system may act like a network of nerve
cells in the brain - or like a collection of microprocessors in a computer -
interpreting complex information and generating complex responses.

Our exploration of the pathways that cells use to process signals from
their environment has led us from receptors on the cell surface to the
proteins that form the elaborate control systems that operate deep within
the cell’s interior.

\section{Essential concepts}

\begin{itemize}
\item Cells in multicellular organisms communicate through a large variety
of extracellular chemical signals.
\item In animals, hormones are carried in the blood to distant target cells,
but most other extracellular signal molecules act over only a short
range. Neighboring cells often communicate through direct cell-cell
contact.
\item Extracellular signal molecules stimulate a target cell when they bind
to and activate receptor proteins. Each receptor protein recognizes a
particular signal molecule.
\item Small hydrophobic extracellular signal molecules, such as steroid
hormones and nitric oxide, can diffuse directly across the plasma
membrane; they activate intracellular proteins, which are usually
either transcription regulators or enzymes.
\item Most extracellular signal molecules cannot pass through the plasma
membrane; they bind to cell-surface receptor proteins that convert
(transduce) the extracellular signal into different intracellular
signals.
\item There are three main classes of cell-surface receptors: (1) ion-chan-
nel-coupled receptors, (2) G-protein-coupled receptors (GPCRs), and
(3) enzyme-coupled receptors.
\item GPCRs and enzyme-coupled receptors respond to extracellular signals
by activating one or more intracellular signaling pathways that
alter the behavior of the cell.
\item Turning off signaling pathways is as important as turning them on.
Each activated component in a signaling pathway must be subse-
quently inactivated or removed for the pathway to function again.
\item GPCRs activate a class of trimeric GTP-binding proteins called G proteins;
these act as molecular switches, transmitting the signal onward
for a short period and then switching themselves off by hydrolyzing
their bound GTP to GDP.
\item Some G proteins directly regulate ion channels in the plasma membrane.
Others directly activate (or inactivate) the enzyme adenylyl
cyclase, increasing (or decreasing) the intracellular concentration
of the small messenger molecule cyclic AMP. Still other G proteins
directly activate the enzyme phospholipase C, which generates
the small messenger molecules inositol trisphosphate (IP 3 ) and
diacylglycerol.
\item IP 3 opens $Ca^{2+}$ channels in the membrane of the endoplasmic reticu-
lum, releasing a flood of free $Ca^{2+}$ ions into the cytosol. $Ca^{2+}$ itself
acts as a small intracellular messenger, altering the activity of a
wide range of $Ca^{2+}$-responsive proteins, including calmodulin, which
activates various target proteins such as such as $Ca^{2+}$/calmodulin-
dependent protein kinases (CaM-kinases) .
\item A rise in cyclic AMP activates protein kinase A (PKA), while $Ca^{2+}$ and
diacylglycerol in combination activate protein kinase C (PKC).
\item PKA, PKC, and CaM-kinases phosphorylate selected target proteins
on serines and threonines, thereby altering protein activity. Different
cell types contain different sets of signaling and effector target pro-
teins and are therefore affected in different ways.
\item Many enzyme-coupled receptors have intracellular protein domains
that function as enzymes; many are receptor tyrosine kinases (RTKs),
which phosphorylate themselves and selected intracellular signaling
proteins on tyrosines.
\item The phosphotyrosines on RTKs serve as docking sites for various intracellular
signaling proteins, usually including the small GTP-binding
protein Ras. Ras activates a three-protein MAP-kinase signaling
module that helps relay the signal from the plasma membrane to the
nucleus.
\item Mutations that stimulate cell proliferation by making Ras constantly
active are a common feature of many human cancers.
\item Some RTKs stimulate cell growth and cell survival by activating PI
3-kinase, which phosphorylates specific inositol phospholipids to
produce lipid docking sites in the plasma membrane that allow cer-
tain signaling proteins to congregate and activate one another.
\item Some receptors, including Notch and cytokine receptors, activate
a direct pathway to the nucleus. Instead of activating signaling
cascades, they turn on transcription regulators at the plasma membrane,
which then migrate to the nucleus where they activate specific
genes.
\item Plants, like animals, use enzyme-coupled cell-surface receptors to
control their growth and development.
\item Extracellular signals in plants often act by relieving the transcriptional
repression of signal-responsive genes.
\item Different intracellular signaling pathways interact, enabling cells to
produce an appropriate response to a complex combination of signals.
Some combinations of signals enable a cell to survive; other combinations
signal a cell to proliferate; and, in the absence of any signals,
most animal cells will kill themselves by undergoing apoptosis.
\end{itemize}

\chapter{Cytoskeleton}



\section{Essential concepts}

\begin{itemize}
\item The cytoplasm of a eucaryotic cell is supported and spatially organized
by a cytoskeleton of intermediate filaments, microtubules, and
actin filaments.
\item Intermediate filaments are stable, ropelike polymers of fibrous proteins
that give cells mechanical strength. Some types underlie the
nuclear membrane to form the nuclear lamina; others are distributed
throughout the cytoplasm.
\item Microtubules are stiff, hollow tubes formed by the polymerization of
tubulin dimer subunits. They are polarized structures with a slow-growing
“minus” end and a fast-growing “plus” end.
\item Microtubules are nucleated in, and grow out from, organizing centers
such as the centrosome. The minus ends of the microtubules are
embedded in the organizing center.
\item Many of the microtubules in a cell are in a labile, dynamic state in
which they alternate between a growing state and a shrinking state.
These transitions, known as dynamic instability, are controlled by the
hydrolysis of GTP bound to tubulin dimers.
\item Each tubulin dimer has a tightly bound GTP molecule that is hydrolyzed
to GDP after the tubulin has assembled into a microtubule. GTP
hydrolysis reduces the affinity of the subunit for its neighbors and
decreases the stability of the polymer, causing it to disassemble.
\item Microtubules can be stabilized by proteins that capture the plus
end - a process that influences the position of microtubule arrays in
a cell. Cells contain many microtubule-associated proteins that stabilize
microtubules, bind them to other cell components, and harness
them for specific functions.
\item Kinesins and dyneins are motor proteins that use the energy of ATP
hydrolysis to move unidirectionally along microtubules. They carry
specific membrane vesicles and other cargoes and in this way help to
maintain the spatial organization of the cytoplasm.
\item Eucaryotic cilia and flagella contain a bundle of stable microtubules.
Their beating is caused by bending of the microtubules, driven by a
motor protein called ciliary dynein.
\item Actin filaments are helical polymers of actin molecules. They are
more flexible than microtubules and are generally found in bundles
or networks.
\item Actin filaments are polarized structures with a fast- and a slow-growing
end, and their assembly and disassembly are controlled by the
hydrolysis of ATP tightly bound to each actin monomer.
\item The varied forms and functions of actin filaments in cells depend on
multiple actin-binding proteins. These control the polymerization of
actin filaments, cross-link the filaments into loose networks or stiff
bundles, attach them to membranes, or move them relative to one
another.
\item A concentrated network of actin filaments underneath the plasma
membrane forms the cell cortex and is responsible for the shape and
movement of the cell surface, including the movements involved
when a cell crawls along a surface.
\item Myosins are motor proteins that use the energy of ATP hydrolysis to
move along actin filaments: they can carry organelles along actin-filament
tracks or cause adjacent actin filaments to slide past each
other in contractile bundles.
\item In muscle, huge regular arrays of overlapping actin filaments and
myosin filaments generate contractions by sliding over one another.
\item Muscle contraction is initiated by a sudden rise in cytosolic $Ca^{2+}$,
which delivers a signal to the contractile apparatus via $Ca^{2+}$-binding
proteins.
\end{itemize}

\chapter{The Cell Division Cycle}




































\section{Essential concepts}

\begin{itemize}
\item The eucaryotic cell cycle consists of several distinct phases. These
include S phase, during which the nuclear DNA is replicated, and M
phase, during which the nucleus divides (mitosis) and then the cytoplasm
divides (cytokinesis).
\item In most cells, there is one gap phase ($G_1$) after M phase and before
S phase, and another ($G_2$) after S phase and before M phase. These
gaps give the cell more time to grow and to prepare for the events of
S phase and M phase.
\item The cell-cycle control system coordinates the events of the cell cycle
by sequentially and cyclically switching on the appropriate parts of
the cell-cycle machinery and then switching them off.
\item The control system depends on a set of protein kinases, each composed
of a regulatory subunit called a cyclin and a catalytic subunit
called a cyclin-dependent protein kinase (Cdk).
\item Cyclin concentrations rise and fall at specific times in the cell cycle,
helping to trigger events of the cycle. The Cdks are cyclically activated
by both cyclin binding and the phosphorylation of some amino
acids and the dephosphorylation of others; when activated, Cdks
phosphorylate key proteins in the cell.
\item Different cyclin–Cdk complexes trigger different steps of the cell
cycle: M-Cdk drives the cell into mitosis; $G_1$-Cdk drives it through $G_1$;
$G_1$/S-Cdk and S-Cdk drive it into S phase.
\item The control system also uses protein complexes that trigger the proteolysis
of specific cell-cycle regulators at particular stages of the
cycle.
\item The cell-cycle control system can halt the cycle at specific checkpoints
to ensure that intracellular and extracellular conditions are favorable
and that the next step in the cycle does not begin before the previous
one has finished. Some of these checkpoints rely on Cdk inhibitors
that block the activity of one or more cyclin–Cdk complexes.
\item S-Cdk initiates DNA replication during S phase and helps ensure that
the genome is copied only once. Checkpoints in $G_1$, S phase, and $G_2$
prevent cells from replicating damaged DNA.
\item M-Cdk drives the cell into mitosis with the assembly of the microtubule-based
mitotic spindle, which will move daughter chromosomes
to opposite poles of the cell.
\item Microtubules grow out from the duplicated centrosomes, and some
interact with microtubules growing from the opposite pole, thereby
becoming the interpolar microtubules that form the spindle.
\item Centrosomes, microtubule-associated motor proteins, and the replicated
chromosomes themselves work together to assemble the
spindle.
\item When the nuclear envelope breaks down, the spindle microtubules
invade the nuclear area and capture the replicated chromosomes.
The microtubules bind to protein complexes, called kinetochores,
associated with the centromere of each sister chromatid.
\item Microtubules from opposite poles pull in opposite directions on each
replicated chromosome, bringing the chromosomes to the equator of
the mitotic spindle.
\item The sudden separation of sister chromatids allows the resulting
daughter chromosomes to be pulled to opposite poles by the spindle.
The two poles also move apart, further separating the two sets of
chromosomes.
\item The movement of chromosomes by the spindle is driven both by
microtubule motor proteins and by microtubule polymerization and
depolymerization.
\item A nuclear envelope re-forms around the two sets of segregated chromosomes
to form two new nuclei, thereby completing mitosis.
\item The Golgi apparatus breaks into many smaller fragments during
M phase, ensuring an even distribution between the daughter cells.
\item In animal cells, cytoplasmic division is mediated by a contractile ring
of actin filaments and myosin filaments, which assembles midway
between the spindle poles and contracts to divide the cytoplasm in
two; in plant cells, by contrast, cell division occurs by the formation
of a new cell wall inside the parent cell, which divides the cytoplasm
in two.
\item Animal cell numbers are regulated by a combination of intracellular
programs and extracellular signals that control cell survival, cell
growth, and cell proliferation.
\item Many normal cells die by apoptosis during the lifetime of an animal;
they do so by activating an internal suicide program and killing
themselves.
\item Apoptosis depends on a family of proteolytic enzymes called caspases,
which are made as inactive precursors (procaspases). The
procaspases are themselves often activated by proteolytic cleavage
mediated by caspases.
\item Most animal cells require continuous signaling from other cells to
avoid apoptosis; this may be a mechanism to ensure that cells sur-
vive only when and where they are needed.
\item Animal cells proliferate only if stimulated by extracellular mitogens
produced by other cells, ensuring that a cell divides only when
another cell is needed; the mitogens activate intracellular signaling
pathways to override the normal brakes that otherwise block cell-cycle
progression.
\item For an organism or an organ to grow, cells must grow as well as
divide. Animal cell growth depends on extracellular growth factors,
which stimulate protein synthesis and inhibit protein degradation.
\item Cell and tissue size can also be influenced by inhibitory extracellular
signal proteins that oppose the positive regulators of cell survival,
cell growth, and cell division.
\item Cancer cells fail to obey these normal ‘social’ controls on cell behavior
and therefore outgrow, out-divide, and out-survive their normal
neighbors.
\end{itemize}
